\documentclass[12pt]{cdblatex}
\usepackage{fancyhdr}
\usepackage{footer}

\begin{document}

% =================================================================================================
% create checkpoint file

\bgroup
\CdbSetup{action=hide}
\begin{cadabra}
   import cdblib
   checkpoint_file = 'tests/semantic/output/rnc2rnc.json'
   cdblib.create (checkpoint_file)
   checkpoint = []
\end{cadabra}
\egroup

% =================================================================================================
\section*{From one RNC to another}

Consider an RNC frame with RNC cooridnates $x^{a}$.

In the {\tts geodesic-bvp} code the two point boundary value problem (for the geodesic connecting
two points) was solved. There is a bonus in that calculation -- it can be trivaly adapted to the
case of transforming form one RNC into another.

The starting point is the basic equation for the geodesic connecting $P$ (with coordinaties
$x^{a}$) to Q (with coordinates $x^{a} + Dx^{a}$)
\begin{equation*}
   x^a(s) = x^a_i + s y^a - \sum_{k=2}^\infty\>\frac{1}{k!}\>\Gamma^{a}{}_{\ubk}y^{.\ubk} s^k
\end{equation*}
The affine parameter $s$ varies form 0 (at $P$) to 1 (at $Q$).

A new RNC frame, with origin at $P$, can be defined via the $y^{a}$ with the coordinates of $Q$ in
the new RNC frame defined by $y^{a}$ (since $s=1$ at $Q$). Recall that in an RNC all geodesics
through the origin are described by $y^{a}(s) = s y^{a}$. Thus the transformation from $x^a$ to
$y^a$ satisfies
\begin{equation*}
   x^a = x^a_i + y^a - \sum_{k=2}^\infty\>\frac{1}{k!}\>\Gamma^{a}{}_{\ubk}y^{.\ubk}
\end{equation*}
where the $\Gamma^{a}{}_{\ubk}$ are the generalised connections of the $x^a$ frame evaluated at
$x^a=0$. This equation can be inverted to express $y^a$ in terms of $x^a$. This computation is
done in the {\tts geodesic-bvp} code -- we only quote the results here (at the end).

The new $y^a$ frame has origin at $P$. Its coordinate axes are aligned with those (at $P$) of the
origianl RNC frame. To see this just note that $\partial x^a/\partial y^b = \delta^a_b$ at $P$.
Thus the metric at $P$ in the new frame has values $g_{ab}(x)$ (i.e., exactly those of the
original RNC frame). Note that this means that the coordinate axes of the new frame are not
ncessarily orthogonal.

The calculations in this code are trivial. It uses the $y^{a}$ found in {\tts geodesic-bvp} as
the basis of the transformation from $x^{a}$ to $y^{a}$. Most of the code involves reformatting
the $y^{a}$.

\clearpage

\begin{cadabra}
   {a,b,c,d,e,f,g,h,i,j,k,l,m,n,o,p,q,r,s,t,u,v,w#}::Indices(position=independent).

   \nabla{#}::Derivative.

   g_{a b}::Metric.
   g^{a b}::InverseMetric.

   R_{a b c d}::RiemannTensor.
   R^{a}_{b c d}::RiemannTensor.

   # Dx{#}::LaTeXForm{"{\Dx}"}.  # LCB: currently causes a bug, it kills ::KeepWeight for Dx

   import cdblib

   Y5 = cdblib.get ('y5','geodesic-bvp.json')

   Y50 = cdblib.get ('y50','geodesic-bvp.json')
   Y52 = cdblib.get ('y52','geodesic-bvp.json')
   Y53 = cdblib.get ('y53','geodesic-bvp.json')
   Y54 = cdblib.get ('y54','geodesic-bvp.json')
   Y55 = cdblib.get ('y55','geodesic-bvp.json')

   # this copies y5* from geodesic-bvp.json to rnc2rnc.json

   cdblib.create ('rnc2rnc.json')

   cdblib.put ('rnc2rnc',Y5,'rnc2rnc.json')

   cdblib.put ('rnc2rnc0',Y50,'rnc2rnc.json')
   cdblib.put ('rnc2rnc2',Y52,'rnc2rnc.json')
   cdblib.put ('rnc2rnc3',Y53,'rnc2rnc.json')
   cdblib.put ('rnc2rnc4',Y54,'rnc2rnc.json')
   cdblib.put ('rnc2rnc5',Y55,'rnc2rnc.json')

\end{cadabra}

% =================================================================================================
% the remaining code is just for pretty printing

\clearpage

\begin{cadabra}
   # note: keeping numbering as is (out of order) to ensure R appears before \nabla R etc.
   def product_sort (obj):
       substitute (obj,$ x^{a}                            -> A001^{a}               $)
       substitute (obj,$ Dx^{a}                           -> A002^{a}               $)
       substitute (obj,$ g^{a b}                          -> A003^{a b}             $)
       substitute (obj,$ \nabla_{e f g h}{R_{a b c d}}    -> A008_{a b c d e f g h} $)
       substitute (obj,$ \nabla_{e f g}{R_{a b c d}}      -> A007_{a b c d e f g}   $)
       substitute (obj,$ \nabla_{e f}{R_{a b c d}}        -> A006_{a b c d e f}     $)
       substitute (obj,$ \nabla_{e}{R_{a b c d}}          -> A005_{a b c d e}       $)
       substitute (obj,$ R_{a b c d}                      -> A004_{a b c d}         $)
       sort_product   (obj)
       rename_dummies (obj)
       substitute (obj,$ A001^{a}                  -> x^{a}                         $)
       substitute (obj,$ A002^{a}                  -> Dx^{a}                        $)
       substitute (obj,$ A003^{a b}                -> g^{a b}                       $)
       substitute (obj,$ A004_{a b c d}            -> R_{a b c d}                   $)
       substitute (obj,$ A005_{a b c d e}          -> \nabla_{e}{R_{a b c d}}       $)
       substitute (obj,$ A006_{a b c d e f}        -> \nabla_{e f}{R_{a b c d}}     $)
       substitute (obj,$ A007_{a b c d e f g}      -> \nabla_{e f g}{R_{a b c d}}   $)
       substitute (obj,$ A008_{a b c d e f g h}    -> \nabla_{e f g h}{R_{a b c d}} $)

       return obj

   def get_xDxterm (obj,n,m):

       x^{a}::Weight(label=numx,value=1).
       Dx^{a}::Weight(label=numDx,value=1).

       tmp := @(obj).
       distribute  (tmp)

       foo = Ex("numx = " + str(n))
       bah = Ex("numDx = " + str(m))
       keep_weight (tmp, foo)
       keep_weight (tmp, bah)

       return tmp

   def reformat (obj,scale):
       foo  = Ex(str(scale))
       bah := @(foo) @(obj).
       distribute     (bah)
       bah = product_sort (bah)
       rename_dummies (bah)
       canonicalise   (bah)
       substitute     (bah,$Dx^{b}->zzz^{b}$)
       factor_out     (bah,$x^{a?},zzz^{b?}$)
       substitute     (bah,$zzz^{b}->Dx^{b}$)
       ans := @(bah) / @(foo).
       return ans

   def rescale (obj,scale):
       foo  = Ex(str(scale))
       bah := @(foo) @(obj).
       distribute  (bah)
       substitute  (bah,$Dx^{b}->zzz^{b}$)
       factor_out  (bah,$x^{a?},zzz^{b?}$)
       substitute  (bah,$zzz^{b}->Dx^{b}$)
       return bah

   term0 := @(Y50).  # cdb (term0.101,term0)
   term2 := @(Y52).  # cdb (term2.101,term2)
   term3 := @(Y53).  # cdb (term3.101,term3)
   term4 := @(Y54).  # cdb (term4.101,term4)
   term5 := @(Y55).  # cdb (term5.101,term5)

   term0 = reformat (term0,1)  # cdb (term0.102,term0)
   term2 = reformat (term2,1)  # cdb (term2.102,term2)
   term3 = reformat (term3,1)  # cdb (term3.102,term3)
   term4 = reformat (term4,1)  # cdb (term4.102,term4)
   term5 = reformat (term5,1)  # cdb (term5.102,term5)

   xDxterm12 = get_xDxterm (term2,1,2)   # cdb(xDxterm12.101,xDxterm12)

   xDxterm13 = get_xDxterm (term3,1,3)   # cdb(xDxterm13.101,xDxterm13)
   xDxterm22 = get_xDxterm (term3,2,2)   # cdb(xDxterm22.101,xDxterm22)

   xDxterm14 = get_xDxterm (term4,1,4)   # cdb(xDxterm14.101,xDxterm14)
   xDxterm23 = get_xDxterm (term4,2,3)   # cdb(xDxterm23.101,xDxterm23)
   xDxterm32 = get_xDxterm (term4,3,2)   # cdb(xDxterm32.101,xDxterm32)

   xDxterm15 = get_xDxterm (term5,1,5)   # cdb(xDxterm15.101,xDxterm15)
   xDxterm24 = get_xDxterm (term5,2,4)   # cdb(xDxterm24.101,xDxterm24)
   xDxterm33 = get_xDxterm (term5,3,3)   # cdb(xDxterm33.101,xDxterm33)
   xDxterm42 = get_xDxterm (term5,4,2)   # cdb(xDxterm42.101,xDxterm42)


   xDxterm12 = rescale ( reformat (xDxterm12,    3),     3 )   # cdb(xDxterm12.102,xDxterm12)

   xDxterm13 = rescale ( reformat (xDxterm13,   12),   -12 )   # cdb(xDxterm13.102,xDxterm13)
   xDxterm22 = rescale ( reformat (xDxterm22,   24),   -24 )   # cdb(xDxterm22.102,xDxterm22)

   xDxterm14 = rescale ( reformat (xDxterm14,  180),  -180 )   # cdb(xDxterm14.102,xDxterm14)
   xDxterm23 = rescale ( reformat (xDxterm23,  720),  -720 )   # cdb(xDxterm23.102,xDxterm23)
   xDxterm32 = rescale ( reformat (xDxterm32,  720),  -720 )   # cdb(xDxterm32.102,xDxterm32)

   xDxterm15 = rescale ( reformat (xDxterm15,  360),  -360 )   # cdb(xDxterm15.102,xDxterm15)
   xDxterm24 = rescale ( reformat (xDxterm24, 2160), -2160 )   # cdb(xDxterm24.102,xDxterm24)
   xDxterm33 = rescale ( reformat (xDxterm33, 1080), -1080 )   # cdb(xDxterm33.102,xDxterm33)
   xDxterm42 = rescale ( reformat (xDxterm42,  360),  -360 )   # cdb(xDxterm42.102,xDxterm42)

   checkpoint.append (term0)
   checkpoint.append (term2)
   checkpoint.append (term3)
   checkpoint.append (term4)
   checkpoint.append (term5)

\end{cadabra}

\clearpage

% =================================================================================================
\section*{Tranformation between two RNC frames}

\begin{align*}
     y^{a} = \ny{0}^{a} + \ny{2}^{a} + \ny{3}^{a} + \ny{4}^{a} + \ny{5}^{a} + \BigO{\eps^6}
\end{align*}

\begin{dgroup*}
   \begin{dmath*} \ny{0}^{a} = \cdb{term0.102} \end{dmath*}
   \begin{dmath*} \ny{2}^{a} = \cdb{term2.102} \end{dmath*}
   \begin{dmath*} \ny{3}^{a} = \cdb{term3.102} \end{dmath*}
   \begin{dmath*} \ny{4}^{a} = \cdb{term4.102} \end{dmath*}
   \begin{dmath*} \ny{5}^{a} = \cdb{term5.102} \end{dmath*}
\end{dgroup*}

\clearpage

% =================================================================================================
\section*{Tranformation between two RNC frames}

Same as before but with an improved format (maybe) for the expressions.

\begin{align}
   y^{a} = \ny{0}^{a} + \ny{2}^{a} + \ny{3}^{a} + \ny{4}^{a} + \ny{5}^{a} + \BigO{\eps^6}
\end{align}

\begin{dgroup}
   \begin{dmath} \ny{0}^{a} = Dx^{a} \end{dmath}
\end{dgroup}

\begin{dgroup}
   \begin{dmath} \ny{2}^{a} = \ny{2}^{a}_1 \end{dmath}
   \begin{dmath}   3 \ny{2}^{a}_1 = \cdb{xDxterm12.102} \end{dmath}
\end{dgroup}

\begin{dgroup}
   \begin{dmath} \ny{3}^{a} = \ny{3}^{a}_1 + \ny{3}^{a}_2 \end{dmath}
   \begin{dmath} -12 \ny{3}^{a}_1 = \cdb{xDxterm13.102} \end{dmath}
   \begin{dmath} -24 \ny{3}^{a}_2 = \cdb{xDxterm22.102} \end{dmath}
\end{dgroup}

\begin{dgroup}
   \begin{dmath} \ny{4}^{a} = \ny{4}^{a}_1 + \ny{4}^{a}_2 + \ny{4}^{a}_3 \end{dmath}
   \begin{dmath} -180 \ny{4}^{a}_1 = \cdb{xDxterm14.102} \end{dmath}
   \begin{dmath} -720 \ny{4}^{a}_2 = \cdb{xDxterm23.102} \end{dmath}
   \begin{dmath} -720 \ny{4}^{a}_3 = \cdb{xDxterm32.102} \end{dmath}
\end{dgroup}

\begin{dgroup}
   \begin{dmath} \ny{5}^{a} = \ny{5}^{a}_1 + \ny{5}^{a}_2 + \ny{5}^{a}_3 + \ny{5}^{a}_4 \end{dmath}
   \begin{dmath}  -360 \ny{5}^{a}_1 = \cdb{xDxterm15.102} \end{dmath}
   \begin{dmath} -2160 \ny{5}^{a}_2 = \cdb{xDxterm24.102} \end{dmath}
   \begin{dmath} -1080 \ny{5}^{a}_3 = \cdb{xDxterm33.102} \end{dmath}
   \begin{dmath}  -360 \ny{5}^{a}_4 = \cdb{xDxterm42.102} \end{dmath}
\end{dgroup}

% =================================================================================================
% export checkpoints in json format

\bgroup
\CdbSetup{action=hide}
\begin{cadabra}
   for i in range( len(checkpoint) ):
      cdblib.put ('check{:03d}'.format(i),checkpoint[i],checkpoint_file)
\end{cadabra}
\egroup

\end{document}
