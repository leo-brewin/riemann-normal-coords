\documentclass[12pt]{cdblatex}

\begin{document}

% =================================================================================================
% create checkpoint file

\bgroup
\CdbSetup{action=hide}
\begin{cadabra}
   import cdblib
   checkpoint_file = 'tests/semantic/output/genGamma.json'
   cdblib.create (checkpoint_file)
   checkpoint = []
\end{cadabra}
\egroup

% =================================================================================================
\section*{The generalised connections}

The generalised connections may be computed recursively using
\begin{align}
   \label{eq:GenGamma}
\Gamma^{a}{}_{b\uc d} = \Gamma^{a}{}_{(b\uc,d)}
                - (n+1) \Gamma^{a}{}_{p(\uc}
                        \Gamma^{p}{}_{bd)}
\end{align}
where $\uc$ contains $n>0$ indices. The sequence begins with the standard metric compatible connection
\begin{align}
   \Gamma^{d}_{ab} = \frac{1}{2} g^{dc}\left( g_{cb,a} + g_{ac,b} - g_{ab,c} \right)
\end{align}

Here we will use the results of {\tt metric.tex} and {\tt metric-inv.tex} to compute the metric connection
$\Gamma^{d}_{ab}$. But since the $g_{ab}$ and $g^{ab}$ provided by those codes are truncated at a
particular order in the curvatures (and thus are only approximations to the $g_{ab}$ and $g^{ab}$) similar
truncations will arise in the $\Gamma^{a}{}_{b\uc d}$.

Approximations will be denoted by the addition of an overbar to an object. In this notation the metric
$g$ can be written as
\begin{align}
   g = {\bar g} + \BigO{\eps^n}
\end{align}
in which ${\bar g}$ is the truncated polynomial approximation to $g$ and $\BigO{\eps^n}$ is the error term
(containing terms no smaller than $\eps^n$). The polynomial structure of ${\bar g}$ can be expressed as
\begin{align}
   \gabBar = \ngabBar{0}
           + \ngabBar{1}
           + \ngabBar{2}
           + \dots
           + \ngabBar{p}
\end{align}
in which each terms like $\overset{m}{\bar g}$ contains only terms of order $m$. This notation will be applied
to other quantities in particular the generalised connections.

The notation $\BigO{\eps^n}$ denotes terms in the curvatures that are of order $\eps^n$. What does this actually mean?
Each term in $R$ is of order $\eps^2$ while each derivative of $R$ carries an extra power of $\eps$.
Thus $R\cdot R = \BigO{\eps^4}$, $R\cdot R\cdot\nabla R = \BigO{\eps^7}$ and $R\cdot R\cdot\nabla^2R = \BigO{\eps^8}$.

We will also adopt the convention that an object is said to be an $\BigO{\eps^{m}}$ approximation when the corresponding error term is $\BigO{\eps^{m+1}}$.

Consider the $\BigO{\eps^{m}}$ approximation of the generalised connection, namely,
\begin{align}
   \GammaBar^{a}{}_{b\ucn d}
      = \nGammaBar{0}^{a}{}_{b\ucn d}
      + \nGammaBar{1}^{a}{}_{b\ucn d}
      + \nGammaBar{2}^{a}{}_{b\ucn d}
      + \dots
      + \nGammaBar{m}^{a}{}_{b\ucn d}
\end{align}
where $\ucn$ denotes a set of indices such as $c_1c_2c_3\dots c_n$.

The first thing to note is that
\begin{align}
   0 = \nGammaBar{1+n}^{a}{}_{(b\ucn,d)}
\end{align}

There are two proofs of this claim. For the first proof, note (by inspection) that the order $\BigO{\eps^p}$
approximation for $\GammaBar^{a}{}_{b\ucn d}$ is a polynomial in $x$ of degree $p-n-1$. Thus
$\nGammaBar{1+n}^{a}{}_{(b\ucn,d)}$ is a polynomial in $x$ of degree
zero, i.e., a constant. However, we know that all generalised connections vanish at the origin of the RNC frame.
Thus this constant must be zero. The second proof makes explicit use of the first (and second?) Bianchi identity,
that is $0=R_{a(bcd)}$. The term $\nGammaBar{1+n}^{a}{}_{(b\ucn,d)}$ will
itself consist of a sum of terms built from combinations of $x$, $R$, $\nabla R$ etc. The $x^{a}$ will always
appear in a contraction with one of the indices on $R_{abcd}$ or one of its derivatives. Consider any one of
these terms, denoted by $A$, and assume for the moment that $1+n$ is an even number, say $1+n=2p$. The indices
$(b\ucn,d)$ must somehow be assigned to the factors that comprise $A$. Our aim is to show that at least one $R$
factor in $A$ will receive 3 of these indices and thus by the Bianchi identities will be zero. If there are too
many $R$ factors then the Bianchi identities will not come into play. So how many $R$ factors can we expect?
Since $A$ is a term in an $\BigO{\eps^{(n+1)}}$ approximation there can be no more than $(n+1)/2=p$ Riemann
factors. There will be at least one $x$ term contracted with one of the $p$ Riemann factors. However, we have
$n+2=2p+1$ indices to distribute amongst the $x$ term and $p$ Riemann factors. One of the indices is a derivative
index and will have nett effect of transferring that index from $x$ to one of the Riemann factors. The remaining
$2p$ indices must be distributed amongst the $p$ Riemann factors. It is not possible to avoid assigning three
indices to at least one of the Riemann factors. Thus, by the Bianchi identity, this $A$ term must vanish. Similar
arguments can be applied to the other cases where the $A$ terms consists of products of $R$ and its derivatives
and in the case where $n+1$ is an odd number. The analysis always comes down to the distribution of the indices
$(b\ucn,d)$ amongst the factors of a typical $A$ term. In all cases the Bianchi identity will enter the play and
force $A$ to be zero.

A corollary of the second proof is that for all $m<n+2$
\begin{align}
   0 = \nGammaBar{m}^{a}{}_{b\ucn d}
\end{align}
The proof follows exactly that of the second proof given above.

We can use the above results to streamline the computation of the generalised connections.
We begin with the formal expression for the $\BigO{\eps^m}$ approximations
\begin{align}
   \Gamma^{a}{}_{bc}
      &= \nGammaBar{2}^{a}{}_{bc}
       + \nGammaBar{3}^{a}{}_{bc}
       + \nGammaBar{4}^{a}{}_{bc}
       + \dots
       + \nGammaBar{m}^{a}{}_{bc}\\
   \Gamma^{a}{}_{b\uc}
      &= \nGammaBar{n+1}^{a}{}_{b\uc}
       + \nGammaBar{n+2}^{a}{}_{b\uc}
       + \nGammaBar{n+3}^{a}{}_{b\uc}
       + \dots
       + \nGammaBar{m}^{a}{}_{b\uc}\\
   \Gamma^{a}{}_{b\uc d}
      &= \nGammaBar{n+2}^{a}{}_{b\uc d}
       + \nGammaBar{n+3}^{a}{}_{b\uc d}
       + \nGammaBar{n+4}^{a}{}_{b\uc d}
       + \dots
       + \nGammaBar{m}^{a}{}_{b\uc d}\label{eq:GenGammaA}
\end{align}
These can be substituted into equation (\ref{eq:GenGamma}) with the result
\def\m{\hskip 4pt}
\begin{align}
   \Gamma^{a}{}_{b\uc d}
      &= \nGammaBar{n+1}^{a}{}_{(b\uc,d)}
       + \nGammaBar{n+2}^{a}{}_{(b\uc,d)}
       + \nGammaBar{n+3}^{a}{}_{(b\uc,d)}
       + \dots
       + \nGammaBar{m}^{a}{}_{(b\uc,d)}
       -(n+1)\left(\m \nGammaBar{n+1}^{a}{}_{p\uc}
                    + \nGammaBar{n+2}^{a}{}_{p\uc}
                    + \nGammaBar{n+3}^{a}{}_{p\uc}
                    + \dots
                    + \nGammaBar{m}^{a}{}_{p\uc}\right)
             \left(   \nGammaBar{2}^{p}{}_{bd}
                    + \nGammaBar{3}^{p}{}_{bd}
                    + \nGammaBar{4}^{p}{}_{bd}
                    + \dots
                    + \nGammaBar{m}^{p}{}_{bd}\right)\label{eq:GenGammaB}
\end{align}
where it is understood that in expanding the pair of bracketed terms in the last result the terms should be
symmetrised over $b\uc d$ and also truncated to terms of order $\BigO{\eps^m}$. Note that the first term
on the right hand side of this equation vanishes by way of the results described above.

Comparing the order $m$ terms in equation (\ref{eq:GenGammaA}) and (\ref{eq:GenGammaB}) leads to the
following equation
\begin{align}
   \nGammaBar{m}^{a}{}_{b\uc d}
   = \nGammaBar{m}^{a}{}_{(b\uc,d)}
   - (n+1)\left(\m \nGammaBar{m-2}^{a}{}_{p(\uc}
                   \nGammaBar{  2}^{p}{}_{bd)}
                  +\nGammaBar{m-3}^{a}{}_{p(\uc}
                   \nGammaBar{  3}^{p}{}_{bd)}
                  +\nGammaBar{m-4}^{a}{}_{p(\uc}
                   \nGammaBar{  4}^{p}{}_{bd)}
                  + \dots
                  + \nGammaBar{  n+1}^{a}{}_{p(\uc}
                    \nGammaBar{m-n-1}^{p}{}_{bd)}
   \right)
   \label{eq:GenGammaC}
\end{align}
This one equation is all that is needed to compute all of the
$\nGammaBar{p}^{a}{}_{b\uc d}$ for $p=3,4,5,\dots m$ given just
the $\nGammaBar{p}^{a}{}_{bd}$ for $p=2,3,4,\dots m$. For example,
suppose $m=5$ and suppose that we are given
$\nGammaBar{p}^{a}{}_{bd}$ for $p=2,3,4,5$. Then with $n=1$ we can
use equation (\ref{eq:GenGammaC}) to compute in turn,
$\nGammaBar{p}^{a}{}_{bc_1d}$ for $p=3,4,5$. Then with $n=2$ we
compute
$\nGammaBar{p}^{a}{}_{bc_1c_2d}$ for $p=4,5$ and finally with $n=3$
we compute $\nGammaBar{p}^{a}{}_{bc_1c_2c_3d}$ for $p=5$. There
are no terms like
$\nGammaBar{p}^{a}{}_{bc_1c_2c_3c_4d}$ for $p\le5$ due to the
corollary given earlier.

\clearpage

The explicit computations for $m=5$ are as follows.

For $n=1$,
\begin{align}
   \nGammaBar{3}^{a}{}_{bc_1d}
   &=
   \nGammaBar{3}^{a}{}_{(bc_1,d)}\\
   %-----------------------------------------------------------------------
   \nGammaBar{4}^{a}{}_{bc_1d}
   &=
   \nGammaBar{4}^{a}{}_{(bc_1,d)}
   - 2 \nGammaBar{2}^{a}{}_{p(c_1}
       \nGammaBar{2}^{p}{}_{bd)}\\
   %-----------------------------------------------------------------------
   \nGammaBar{5}^{a}{}_{bc_1d}
   &=
   \nGammaBar{5}^{a}{}_{(bc_1,d)}
   - 2 \nGammaBar{3}^{a}{}_{p(c_1}
       \nGammaBar{2}^{p}{}_{bd)}
   - 2 \nGammaBar{2}^{a}{}_{p(c_1}
       \nGammaBar{3}^{p}{}_{bd)}
\end{align}

For $n=2$,
\begin{align}
   \nGammaBar{4}^{a}{}_{bc_1c_2d}
   &=
   \nGammaBar{4}^{a}{}_{(bc_1c_2,d)}\\
   %-----------------------------------------------------------------------
   \nGammaBar{5}^{a}{}_{bc_1c_2d}
   &=
   \nGammaBar{5}^{a}{}_{(bc_1c_2,d)}
   - 3 \nGammaBar{2}^{a}{}_{p(c_1c_2}
       \nGammaBar{2}^{p}{}_{bd)}
\end{align}

For $n=3$,
\begin{align}
   \nGammaBar{5}^{a}{}_{bc_1c_2c_3d}
   &=
   \nGammaBar{5}^{a}{}_{(bc_1c_2c_3,d)}
\end{align}


\clearpage

\begin{cadabra}
   {a,b,c,d,e,f,g,h,i,j,k,l,m,n,o,p,q,r,s,t,u,v,c1,c2,c3,c4,c5,w#}::Indices(position=independent).

   D{#}::Derivative.
   \nabla{#}::Derivative.
   \partial{#}::PartialDerivative.

   g_{a b}::Metric.
   g^{a b}::InverseMetric.
   g_{a}^{b}::KroneckerDelta.
   g^{a}_{b}::KroneckerDelta.
   \delta^{a}_{b}::KroneckerDelta.
   \delta_{a}^{b}::KroneckerDelta.

   R_{a b c d}::RiemannTensor.
   R^{a}_{b c d}::RiemannTensor.
   R_{a b c}^{d}::RiemannTensor.

   \Gamma^{a}_{b c}::TableauSymmetry(shape={2}, indices={1,2}).

   x^{a}::Depends(D{#}).

   g_{a b}::Depends(\partial{#}).
   R_{a b c d}::Depends(\partial{#}).
   R^{a}_{b c d}::Depends(\partial{#}).
   \Gamma^{a}_{b c}::Depends(\partial{#}).

   R_{a b c d}::Depends(\nabla{#}).
   R^{a}_{b c d}::Depends(\nabla{#}).

   import cdblib

   term0 = cdblib.get ('GammaRterm0','connection.json')
   term1 = cdblib.get ('GammaRterm2','connection.json')
   term2 = cdblib.get ('GammaRterm3','connection.json')
   term3 = cdblib.get ('GammaRterm4','connection.json')
   term4 = cdblib.get ('GammaRterm5','connection.json')

   # LCB: these terms were not computed in connection.tex so set them to zero
   #      maybe in the future I will compute down to term6.

   term5 := 0.
   term6 := 0.

   # genGmn : m = eps order of Rabcd terms
   #          n = number of c indices

   # --------------------------------------------------------------------------
   # rules for building the genGmn

   # note: after applying each rule, must symmetrise over (b c1 c2 ... cn d)

   # n = 0

   genG20 := genG2^{a}_{b d}.
   genG30 := genG3^{a}_{b d}.
   genG40 := genG4^{a}_{b d}.
   genG50 := genG5^{a}_{b d}.

   defG20 := genG2^{d}_{a b} -> @(term1).
   defG30 := genG3^{d}_{a b} -> @(term2).
   defG40 := genG4^{d}_{a b} -> @(term3).
   defG50 := genG5^{d}_{a b} -> @(term4).

   # LCB: rncGamma in connection.json limited to "term4" (ie. to 4th order in x)
   #      so can only compute genG3*, genG4* and genG5* (at this stage)
   #      but it doesn't hurt to provide the definitions for genG6*, genG7* etc. we just won't use them (at this atage)

   defG60 := genG6^{d}_{a b} -> @(term5).
   defG70 := genG7^{d}_{a b} -> @(term6).

   # n = 1

   defG31 := genG3^{a}_{b c1 d} -> D_{d}{genG3^{a}_{b c1}}.

   defG41 := genG4^{a}_{b c1 d} -> D_{d}{genG4^{a}_{b c1}}
                                   - 2 genG2^{a}_{p c1} genG2^{p}_{b d}.

   defG51 := genG5^{a}_{b c1 d} -> D_{d}{genG5^{a}_{b c1}}
                                   - 2 genG3^{a}_{p c1} genG2^{p}_{b d}
                                   - 2 genG2^{a}_{p c1} genG3^{p}_{b d}.

   defG61 := genG6^{a}_{b c1 d} -> D_{d}{genG6^{a}_{b c1}}
                                   - 2 genG4^{a}_{p c1} genG2^{p}_{b d}
                                   - 2 genG3^{a}_{p c1} genG3^{p}_{b d}
                                   - 2 genG3^{a}_{p c1} genG4^{p}_{b d}.

   defG71 := genG7^{a}_{b c1 d} -> D_{d}{genG7^{a}_{b c1}}
                                   - 2 genG5^{a}_{p c1} genG2^{p}_{b d}
                                   - 2 genG4^{a}_{p c1} genG3^{p}_{b d}
                                   - 2 genG3^{a}_{p c1} genG4^{p}_{b d}
                                   - 2 genG2^{a}_{p c1} genG5^{p}_{b d}.

   # n = 2

   defG42 := genG4^{a}_{b c1 c2 d} -> D_{d}{genG4^{a}_{b c1 c2}}.

   defG52 := genG5^{a}_{b c1 c2 d} -> D_{d}{genG5^{a}_{b c1 c2}}
                                      - 3 genG3^{a}_{p c1 c2} genG2^{p}_{b d}.

   defG62 := genG6^{a}_{b c1 c2 d} -> D_{d}{genG6^{a}_{b c1 c2}}
                                      - 3 genG4^{a}_{p c1 c2} genG2^{p}_{b d}
                                      - 3 genG3^{a}_{p c1 c2} genG3^{p}_{b d}.

   defG72 := genG7^{a}_{b c1 c2 d} -> D_{d}{genG7^{a}_{b c1 c2}}
                                      - 3 genG5^{a}_{p c1 c2} genG2^{p}_{b d}
                                      - 3 genG4^{a}_{p c1 c2} genG3^{p}_{b d}
                                      - 3 genG3^{a}_{p c1 c2} genG4^{p}_{b d}.

   # n = 3

   defG53 := genG5^{a}_{b c1 c2 c3 d} -> D_{d}{genG5^{a}_{b c1 c2 c3}}.

   defG63 := genG6^{a}_{b c1 c2 c3 d} -> D_{d}{genG6^{a}_{b c1 c2 c3}}
                                         - 4 genG3^{a}_{p c1 c2 c3} genG3^{p}_{b d}.

   defG73 := genG7^{a}_{b c1 c2 c3 d} -> D_{d}{genG7^{a}_{b c1 c2 c3}}
                                         - 4 genG4^{a}_{p c1 c2 c3} genG3^{p}_{b d}
                                         - 4 genG3^{a}_{p c1 c2 c3} genG4^{p}_{b d}.

   # n = 4

   defG64 := genG6^{a}_{b c1 c2 c3 c4 d} -> D_{d}{genG6^{a}_{b c1 c2 c3 c4}}.

   defG74 := genG7^{a}_{b c1 c2 c3 c4 d} -> D_{d}{genG7^{a}_{b c1 c2 c3 c4}}
                                            - 5 genG5^{a}_{p c1 c2 c3 c4} genG2^{p}_{b d}.

   # n = 5

   defG75 := genG7^{a}_{b c1 c2 c3 c4 c5 d} -> D_{d}{genG7^{a}_{b c1 c2 c3 c4 c5}}.

   # --------------------------------------------------------------------------
   # build the genGmn

   # ==========================================================================
   # n = 1

   genG31 := genG3^{a}_{b c1 d}.                              # cdb (genG31.000,genG31)
   genG41 := genG4^{a}_{b c1 d}.                              # cdb (genG41.000,genG41)
   genG51 := genG5^{a}_{b c1 d}.
   # genG61 := genG6^{a}_{b c1 d}.
   # genG71 := genG7^{a}_{b c1 d}.

   # --------------------------------------------------------------------------
   substitute     (genG20,defG20)                             # cdb (genG20.001,genG20)
   substitute     (genG30,defG30)                             # cdb (genG30.001,genG30)
   substitute     (genG40,defG40)                             # cdb (genG40.001,genG40)
   substitute     (genG50,defG50)                             # cdb (genG50.001,genG50)

   # --------------------------------------------------------------------------
   substitute     (genG31,defG31)                             # cdb (genG31.001,genG31)
   substitute     (genG31,defG30)                             # cdb (genG31.002,genG31)

   distribute     (genG31)                                    # cdb (genG31.002,genG31)
   unwrap         (genG31)                                    # cdb (genG31.003,genG31)
   product_rule   (genG31)                                    # cdb (genG31.004,genG31)
   distribute     (genG31)                                    # cdb (genG31.005,genG31)
   substitute     (genG31,$D_{a}{x^b}->\delta_{a}^{b}$)       # cdb (genG31.006,genG31)
   eliminate_kronecker (genG31)                               # cdb (genG31.007,genG31)
   sym            (genG31,$_{b}, _{c1}, _{d}$)
   sort_product   (genG31)                                    # cdb (genG31.008,genG31)
   rename_dummies (genG31)                                    # cdb (genG31.009,genG31)
   canonicalise   (genG31)                                    # cdb (genG31.010,genG31)

   # --------------------------------------------------------------------------
   substitute     (genG41,defG41)                             # cdb (genG41.001,genG41)
   substitute     (genG41,defG40)                             # cdb (genG41.002,genG41)
   substitute     (genG41,defG20,repeat=True)                 # cdb (genG41.003,genG41)

   distribute     (genG41)                                    # cdb (genG41.004,genG41)
   unwrap         (genG41)                                    # cdb (genG41.005,genG41)
   product_rule   (genG41)                                    # cdb (genG41.006,genG41)
   distribute     (genG41)                                    # cdb (genG41.007,genG41)
   substitute     (genG41,$D_{a}{x^b}->\delta_{a}^{b}$)       # cdb (genG41.008,genG41)
   eliminate_kronecker (genG41)                               # cdb (genG41.009,genG41)
   sym            (genG41,$_{b}, _{c1}, _{d}$)
   sort_product   (genG41)                                    # cdb (genG41.010,genG41)
   rename_dummies (genG41)                                    # cdb (genG41.011,genG41)
   canonicalise   (genG41)                                    # cdb (genG41.012,genG41)

   # --------------------------------------------------------------------------
   substitute     (genG51,defG51)
   substitute     (genG51,defG50)
   substitute     (genG51,defG30,repeat=True)
   substitute     (genG51,defG20,repeat=True)

   distribute     (genG51)
   unwrap         (genG51)
   product_rule   (genG51)
   distribute     (genG51)
   substitute     (genG51,$D_{a}{x^b}->\delta_{a}^{b}$)
   eliminate_kronecker (genG51)
   sym            (genG51,$_{b}, _{c1}, _{d}$)
   sort_product   (genG51)
   rename_dummies (genG51)
   canonicalise   (genG51)

   # update the rules

   defG31 := genG3^{a}_{b c1 d} -> @(genG31).
   defG41 := genG4^{a}_{b c1 d} -> @(genG41).
   defG51 := genG5^{a}_{b c1 d} -> @(genG51).

   # ==========================================================================
   # n = 2

   genG42 := genG4^{a}_{b c1 c2 d}.                           # cdb (genG42.000,genG42)
   genG52 := genG5^{a}_{b c1 c2 d}.
   # genG62 := genG6^{a}_{b c1 c2 d}.
   # genG72 := genG7^{a}_{b c1 c2 d}.

   # --------------------------------------------------------------------------
   substitute     (genG42,defG42)                             # cdb (genG42.001,genG42)
   substitute     (genG42,defG41)                             # cdb (genG42.002,genG42)

   distribute     (genG42)                                    # cdb (genG42.003,genG42)
   unwrap         (genG42)                                    # cdb (genG42.004,genG42)
   product_rule   (genG42)                                    # cdb (genG42.005,genG42)
   distribute     (genG42)                                    # cdb (genG42.006,genG42)
   substitute     (genG42,$D_{a}{x^b}->\delta_{a}^{b}$)       # cdb (genG42.007,genG42)
   eliminate_kronecker (genG42)                               # cdb (genG42.008,genG42)
   sym            (genG42,$_{b}, _{c1}, _{c2}, _{d}$)
   sort_product   (genG42)                                    # cdb (genG42.009,genG42)
   rename_dummies (genG42)                                    # cdb (genG42.010,genG42)
   canonicalise   (genG42)                                    # cdb (genG42.011,genG42)

   # --------------------------------------------------------------------------
   substitute     (genG52,defG52)
   substitute     (genG52,defG51)
   substitute     (genG52,defG31,repeat=True)
   substitute     (genG52,defG20,repeat=True)

   distribute     (genG52)
   unwrap         (genG52)
   product_rule   (genG52)
   distribute     (genG52)
   substitute     (genG52,$D_{a}{x^b}->\delta_{a}^{b}$)
   eliminate_kronecker (genG52)
   sym            (genG52,$_{b}, _{c1}, _{c2}, _{d}$)
   sort_product   (genG52)
   rename_dummies (genG52)
   canonicalise   (genG52)                                    # cdb (genG52.001,genG52)

   # update the rules

   defG42 := genG4^{a}_{b c1 c2 d} -> @(genG42).
   defG52 := genG5^{a}_{b c1 c2 d} -> @(genG52).

   # ==========================================================================
   # n = 3

   genG53 := genG5^{a}_{b c1 c2 c3 d}.
   # genG63 := genG6^{a}_{b c1 c2 c3 d}.
   # genG73 := genG7^{a}_{b c1 c2 c3 d}.

   # --------------------------------------------------------------------------
   substitute     (genG53,defG53)
   substitute     (genG53,defG52)

   distribute     (genG53)
   unwrap         (genG53)
   product_rule   (genG53)
   distribute     (genG53)
   substitute     (genG53,$D_{a}{x^b}->\delta_{a}^{b}$)
   eliminate_kronecker (genG53)
   sym            (genG53,$_{b}, _{c1}, _{c2}, _{c3}, _{d}$)
   sort_product   (genG53)
   rename_dummies (genG53)
   canonicalise   (genG53)                                    # cdb (genG53.001,genG53)

   # update the rules

   defG53 := genG5^{a}_{b c1 c2 c3 d} -> @(genG53).

\end{cadabra}

\clearpage

\clearpage

\begin{dgroup*}
   \begin{dmath*} \cdb*{genG31.000} \end{dmath*}
   \begin{dmath*} \cdb*{genG31.001} \end{dmath*}
   \begin{dmath*} \cdb*{genG31.002} \end{dmath*}
   \begin{dmath*} \cdb*{genG31.003} \end{dmath*}
   \begin{dmath*} \cdb*{genG31.004} \end{dmath*}
   \begin{dmath*} \cdb*{genG31.005} \end{dmath*}
   \begin{dmath*} \cdb*{genG31.006} \end{dmath*}
   \begin{dmath*} \cdb*{genG31.007} \end{dmath*}
   \begin{dmath*} \cdb*{genG31.008} \end{dmath*}
   \begin{dmath*} \cdb*{genG31.009} \end{dmath*}
   \begin{dmath*} \cdb*{genG31.010} \end{dmath*}
\end{dgroup*}

\clearpage

\begin{dgroup*}
   \begin{dmath*} \cdb*{genG41.000} \end{dmath*}
   \begin{dmath*} \cdb*{genG41.001} \end{dmath*}
   \begin{dmath*} \cdb*{genG41.002} \end{dmath*}
   \begin{dmath*} \cdb*{genG41.003} \end{dmath*}
   \begin{dmath*} \cdb*{genG41.004} \end{dmath*}
   \begin{dmath*} \cdb*{genG41.005} \end{dmath*}
   \begin{dmath*} \cdb*{genG41.006} \end{dmath*}
   \begin{dmath*} \cdb*{genG41.007} \end{dmath*}
   \begin{dmath*} \cdb*{genG41.008} \end{dmath*}
   \begin{dmath*} \cdb*{genG41.009} \end{dmath*}
   % \begin{dmath*} \cdb*{genG41.010} \end{dmath*}
   % \begin{dmath*} \cdb*{genG41.011} \end{dmath*}
   \begin{dmath*} \cdb*{genG41.012} \end{dmath*}
\end{dgroup*}

\clearpage

\begin{dgroup*}
   \begin{dmath*} \cdb*{genG42.000} \end{dmath*}
   \begin{dmath*} \cdb*{genG42.001} \end{dmath*}
   \begin{dmath*} \cdb*{genG42.002} \end{dmath*}
   \begin{dmath*} \cdb*{genG42.003} \end{dmath*}
   \begin{dmath*} \cdb*{genG42.004} \end{dmath*}
   \begin{dmath*} \cdb*{genG42.005} \end{dmath*}
   % \begin{dmath*} \cdb*{genG42.006} \end{dmath*}
   % \begin{dmath*} \cdb*{genG42.007} \end{dmath*}
   % \begin{dmath*} \cdb*{genG42.008} \end{dmath*}
   % \begin{dmath*} \cdb*{genG42.009} \end{dmath*}
   % \begin{dmath*} \cdb*{genG42.010} \end{dmath*}
   \begin{dmath*} \cdb*{genG42.011} \end{dmath*}
\end{dgroup*}

\clearpage

\begin{cadabra}
   # note: keeping numbering as is (out of order) to ensure R appears before \nabla R etc.
   def product_sort (obj):
       substitute (obj,$ A^{a}                            -> A001^{a}               $)
       substitute (obj,$ x^{a}                            -> A002^{a}               $)
       substitute (obj,$ g^{a b}                          -> A003^{a b}             $)
       substitute (obj,$ \nabla_{e f g h}{R_{a b c d}}    -> A008_{a b c d e f g h} $)
       substitute (obj,$ \nabla_{e f g}{R_{a b c d}}      -> A007_{a b c d e f g}   $)
       substitute (obj,$ \nabla_{e f}{R_{a b c d}}        -> A006_{a b c d e f}     $)
       substitute (obj,$ \nabla_{e}{R_{a b c d}}          -> A005_{a b c d e}       $)
       substitute (obj,$ R_{a b c d}                      -> A004_{a b c d}         $)
       sort_product   (obj)
       rename_dummies (obj)
       substitute (obj,$ A001^{a}                  -> A^{a}                         $)
       substitute (obj,$ A002^{a}                  -> x^{a}                         $)
       substitute (obj,$ A003^{a b}                -> g^{a b}                       $)
       substitute (obj,$ A004_{a b c d}            -> R_{a b c d}                   $)
       substitute (obj,$ A005_{a b c d e}          -> \nabla_{e}{R_{a b c d}}       $)
       substitute (obj,$ A006_{a b c d e f}        -> \nabla_{e f}{R_{a b c d}}     $)
       substitute (obj,$ A007_{a b c d e f g}      -> \nabla_{e f g}{R_{a b c d}}   $)
       substitute (obj,$ A008_{a b c d e f g h}    -> \nabla_{e f g h}{R_{a b c d}} $)

       return obj

   # --------------------------------------------------------------------------
   symG20 := @(genG20) A^{b} A^{d}.                           # cdb (symG20.100,symG20)

   distribute            (symG20)                             # cdb (symG20.101,symG20)
   symG20 = product_sort (symG20)                             # cdb (symG20.102,symG20)
   rename_dummies        (symG20)                             # cdb (symG20.103,symG20)
   canonicalise          (symG20)                             # cdb (symG20.104,symG20)

   # --------------------------------------------------------------------------
   symG30 := @(genG30) A^{b} A^{d}.                           # cdb (symG30.100,symG30)

   distribute            (symG30)                             # cdb (symG30.101,symG30)
   symG30 = product_sort (symG30)                             # cdb (symG30.102,symG30)
   rename_dummies        (symG30)                             # cdb (symG30.103,symG30)
   canonicalise          (symG30)                             # cdb (symG30.104,symG30)

   # --------------------------------------------------------------------------
   symG40 := @(genG40) A^{b} A^{d}.                           # cdb (symG40.100,symG40)

   distribute            (symG40)                             # cdb (symG40.101,symG40)
   symG40 = product_sort (symG40)                             # cdb (symG40.102,symG40)
   rename_dummies        (symG40)                             # cdb (symG40.103,symG40)
   canonicalise          (symG40)                             # cdb (symG40.104,symG40)

   # --------------------------------------------------------------------------
   symG50 := @(genG50) A^{b} A^{d}.                           # cdb (symG50.100,symG50)

   distribute            (symG50)                             # cdb (symG50.101,symG50)
   symG50 = product_sort (symG50)                             # cdb (symG50.102,symG50)
   rename_dummies        (symG50)                             # cdb (symG50.103,symG50)
   canonicalise          (symG50)                             # cdb (symG50.104,symG50)

   # --------------------------------------------------------------------------
   symG31 := @(genG31) A^{b} A^{c1} A^{d}.                    # cdb (symG31.100,symG31)

   distribute            (symG31)                             # cdb (symG31.101,symG31)
   symG31 = product_sort (symG31)                             # cdb (symG31.102,symG31)
   rename_dummies        (symG31)                             # cdb (symG31.103,symG31)
   canonicalise          (symG31)                             # cdb (symG31.104,symG31)

   # --------------------------------------------------------------------------
   symG41 := @(genG41) A^{b} A^{c1} A^{d}.                    # cdb (symG41.100,symG41)

   distribute            (symG41)                             # cdb (symG41.101,symG41)
   symG41 = product_sort (symG41)                             # cdb (symG41.102,symG41)
   rename_dummies        (symG41)                             # cdb (symG41.103,symG41)
   canonicalise          (symG41)                             # cdb (symG41.104,symG41)

   # --------------------------------------------------------------------------
   symG51 := @(genG51) A^{b} A^{c1} A^{d}.                    # cdb (symG51.100,symG51)

   distribute            (symG51)                             # cdb (symG51.101,symG51)
   symG51 = product_sort (symG51)                             # cdb (symG51.102,symG51)
   rename_dummies        (symG51)                             # cdb (symG51.103,symG51)
   canonicalise          (symG51)                             # cdb (symG51.104,symG51)

   # --------------------------------------------------------------------------
   symG42 := @(genG42) A^{b} A^{c1} A^{c2} A^{d}.             # cdb (symG42.100,symG42)

   distribute            (symG42)                             # cdb (symG42.101,symG42)
   symG42 = product_sort (symG42)                             # cdb (symG42.102,symG42)
   rename_dummies        (symG42)                             # cdb (symG42.103,symG42)
   canonicalise          (symG42)                             # cdb (symG42.104,symG42)

   # --------------------------------------------------------------------------
   symG52 := @(genG52) A^{b} A^{c1} A^{c2} A^{d}.             # cdb (symG52.100,symG52)

   distribute            (symG52)                             # cdb (symG52.101,symG52)
   symG52 = product_sort (symG52)                             # cdb (symG52.102,symG52)
   rename_dummies        (symG52)                             # cdb (symG52.103,symG52)
   canonicalise          (symG52)                             # cdb (symG52.104,symG52)

   # --------------------------------------------------------------------------
   symG53 := @(genG53) A^{b} A^{c1} A^{c2} A^{c3} A^{d}.      # cdb (symG53.100,symG53)

   distribute            (symG53)                             # cdb (symG53.101,symG53)
   symG53 = product_sort (symG53)                             # cdb (symG53.102,symG53)
   rename_dummies        (symG53)                             # cdb (symG53.103,symG53)
   canonicalise          (symG53)                             # cdb (symG53.104,symG53)

\end{cadabra}

\clearpage

\begin{dgroup*}
   \begin{dmath*} \cdb*{symG31.100} \end{dmath*}
   \begin{dmath*} \cdb*{symG31.101} \end{dmath*}
   \begin{dmath*} \cdb*{symG31.102} \end{dmath*}
   \begin{dmath*} \cdb*{symG31.103} \end{dmath*}
   \begin{dmath*} \cdb*{symG31.104} \end{dmath*}
\end{dgroup*}

\clearpage

\begin{dgroup*}
   \begin{dmath*} \cdb*{symG41.100} \end{dmath*}
   \begin{dmath*} \cdb*{symG41.101} \end{dmath*}
   \begin{dmath*} \cdb*{symG41.102} \end{dmath*}
   \begin{dmath*} \cdb*{symG41.103} \end{dmath*}
   \begin{dmath*} \cdb*{symG41.104} \end{dmath*}
\end{dgroup*}

\clearpage

\begin{dgroup*}
   \begin{dmath*} \cdb*{symG51.104} \end{dmath*}
\end{dgroup*}

\clearpage

\begin{dgroup*}
   \begin{dmath*} \cdb*{symG42.104} \end{dmath*}
   \begin{dmath*} \cdb*{symG52.104} \end{dmath*}
\end{dgroup*}

\clearpage

\begin{dgroup*}
   \begin{dmath*} \cdb*{symG53.104} \end{dmath*}
\end{dgroup*}

\clearpage

\begin{cadabra}
   def reformat (obj,scale):
       foo  = Ex(str(scale))
       bah := @(foo) @(obj).
       distribute (bah)
       factor_out (bah,$A^{a?},x^{b?}$)
       ans := @(bah) / @(foo).
       return ans

   fooG20 = reformat (symG20,3)
   fooG30 = reformat (symG30,12)
   fooG40 = reformat (symG40,360)
   fooG50 = reformat (symG50,180)

   fooG31 = reformat (symG31,2)
   fooG41 = reformat (symG41,120)
   fooG51 = reformat (symG51,180)

   fooG42 = reformat (symG42,15)
   fooG52 = reformat (symG52,90)

   fooG53 = reformat (symG53,3)

   genGamma0 := @(fooG20) + @(fooG30) + @(fooG40) + @(fooG50).  # cdb (genGamma0.000,genGamma0)
   genGamma1 := @(fooG31) + @(fooG41) + @(fooG51).              # cdb (genGamma1.000,genGamma1)
   genGamma2 := @(fooG42) + @(fooG52).                          # cdb (genGamma2.000,genGamma2)
   genGamma3 := @(fooG53).                                      # cdb (genGamma3.000,genGamma3)

   cdblib.create ('genGamma.json')

   cdblib.put ('genGamma0',genGamma0,'genGamma.json')
   cdblib.put ('genGamma1',genGamma1,'genGamma.json')
   cdblib.put ('genGamma2',genGamma2,'genGamma.json')
   cdblib.put ('genGamma3',genGamma3,'genGamma.json')

   cdblib.put ('genGamma01',fooG20,'genGamma.json')
   cdblib.put ('genGamma02',fooG30,'genGamma.json')
   cdblib.put ('genGamma03',fooG40,'genGamma.json')
   cdblib.put ('genGamma04',fooG50,'genGamma.json')

   cdblib.put ('genGamma11',fooG31,'genGamma.json')
   cdblib.put ('genGamma12',fooG41,'genGamma.json')
   cdblib.put ('genGamma13',fooG51,'genGamma.json')

   cdblib.put ('genGamma21',fooG42,'genGamma.json')
   cdblib.put ('genGamma22',fooG52,'genGamma.json')

   cdblib.put ('genGamma31',fooG53,'genGamma.json')

\end{cadabra}

\clearpage

% =================================================================================================
\section*{The generalised connection in Riemann normal coordinates}

\begin{dgroup*}
   \begin{dmath*} A^b A^c \Gamma^{a}_{b c}(x) = \cdb{genGamma0.000} \end{dmath*}
   \begin{dmath*} A^b A^c A^d \Gamma^{a}_{b c d}(x) = \cdb{genGamma1.000} \end{dmath*}
   \begin{dmath*} A^b A^c A^d A^e \Gamma^{a}_{b c d e}(x) = \cdb{genGamma2.000} \end{dmath*}
   \begin{dmath*} A^b A^c A^d A^e A^f \Gamma^{a}_{b c d e f}(x) = \cdb{genGamma3.000} \end{dmath*}
\end{dgroup*}

\clearpage

\begin{cadabra}
   scaledGamma0 := 360 @(genGamma0).  # cdb (scaledGamma0.001,scaledGamma0)
   scaledGamma1 := 360 @(genGamma1).  # cdb (scaledGamma1.001,scaledGamma1)
   scaledGamma2 :=  90 @(genGamma2).  # cdb (scaledGamma2.001,scaledGamma2)
   scaledGamma3 :=   3 @(genGamma3).  # cdb (scaledGamma3.001,scaledGamma3)

\end{cadabra}

\clearpage

% =================================================================================================
\section*{The generalised connection in Riemann normal coordinates}

This is the same as the previous page but with a small change in the format to avoid fractions.

\begin{dgroup*}
   \begin{dmath*} 360 A^b A^c \Gamma^{a}_{b c}(x) = \cdb{scaledGamma0.001} \end{dmath*}
   \begin{dmath*} 360 A^b A^c A^d \Gamma^{a}_{b c d}(x) = \cdb{scaledGamma1.001} \end{dmath*}
   \begin{dmath*}  90 A^b A^c A^d A^e \Gamma^{a}_{b c d e}(x) = \cdb{scaledGamma2.001} \end{dmath*}
   \begin{dmath*}   3 A^b A^c A^d A^e A^f \Gamma^{a}_{b c d e f}(x) = \cdb{scaledGamma3.001} \end{dmath*}
\end{dgroup*}

\clearpage

% =================================================================================================
% export selected objects, these will later be imported into a library
% these are the objects that will appear in the paper

\begin{cadabra}
   tmp0 := @(fooG20) + @(fooG30).
   tmp1 := @(fooG31).

   alt0 := @(genGamma0).
   alt1 := @(genGamma1).
   alt2 := @(genGamma2).
   alt3 := @(genGamma3).

   alt0scaled := @(scaledGamma0).
   alt1scaled := @(scaledGamma1).
   alt2scaled := @(scaledGamma2).
   alt3scaled := @(scaledGamma3).

   substitute (tmp0, $A^{a}->1$)
   substitute (tmp1, $A^{a}->1$)

   substitute (alt0, $A^{a}->1$)
   substitute (alt1, $A^{a}->1$)
   substitute (alt2, $A^{a}->1$)
   substitute (alt3, $A^{a}->1$)

   substitute (alt0scaled, $A^{a}->1$)
   substitute (alt1scaled, $A^{a}->1$)
   substitute (alt2scaled, $A^{a}->1$)
   substitute (alt3scaled, $A^{a}->1$)

   cdblib.create ('genGamma.export')

   # 4th order gen gamma
   cdblib.put ('gen_gamma_0_4th',tmp0,'genGamma.export')
   cdblib.put ('gen_gamma_1_4th',tmp1,'genGamma.export')

   # 6th order gen gamma
   cdblib.put ('gen_gamma_0',alt0,'genGamma.export')
   cdblib.put ('gen_gamma_1',alt1,'genGamma.export')
   cdblib.put ('gen_gamma_2',alt2,'genGamma.export')
   cdblib.put ('gen_gamma_3',alt3,'genGamma.export')

   # 6th order gen gamma scaled
   cdblib.put ('gen_gamma_0_scaled',alt0scaled,'genGamma.export')
   cdblib.put ('gen_gamma_1_scaled',alt1scaled,'genGamma.export')
   cdblib.put ('gen_gamma_2_scaled',alt2scaled,'genGamma.export')
   cdblib.put ('gen_gamma_3_scaled',alt3scaled,'genGamma.export')

   checkpoint.append (tmp0)
   checkpoint.append (tmp1)

   checkpoint.append (alt0)
   checkpoint.append (alt1)
   checkpoint.append (alt2)
   checkpoint.append (alt3)

   checkpoint.append (alt0scaled)
   checkpoint.append (alt1scaled)
   checkpoint.append (alt2scaled)
   checkpoint.append (alt3scaled)

\end{cadabra}

% =================================================================================================
% export checkpoints in json format

\bgroup
\CdbSetup{action=hide}
\begin{cadabra}
   for i in range( len(checkpoint) ):
      cdblib.put ('check{:03d}'.format(i),checkpoint[i],checkpoint_file)
\end{cadabra}
\egroup

\end{document}
