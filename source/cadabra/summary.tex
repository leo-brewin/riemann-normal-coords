\documentclass[a4paper,12pt]{article}
\usepackage{cdblatex}
% \usepackage{amsmath}% part of cdblatex
% \usepackage{amssymb}% ditto
% \usepackage{breqn}%   ditto
\usepackage{hyperref}
\usepackage{geometry}
\usepackage{summary}
\usepackage{config}

\geometry{a4paper,landscape,margin=2cm}
% \geometry{a4paper,portrait,margin=2cm}
\hypersetup{colorlinks=true}% use false for journals and paper
\numberwithin{equation}{section}% requires amsmath

\begin{document}

% =================================================================================================
\section*{Notes}

The convention for the curvature used in these notes conforms to that of Misner-Thorne-Wheeler
(MTW, eq. 11.12) , namely
\begin{align*}
   V^{a}{}_{;bc} - V^{a}{}_{;cb} = - R^{a}{}_{dbc} V^{d}
\end{align*}

Also, note the following shorthand for mixed covariant derivatives
\begin{align*}
   \nabla_a\left(\nabla_b\right) &= \nabla_{ab}\\
   \nabla_a\left(\nabla_b\left(\nabla_c\right)\right) &= \nabla_{abc}\\
   \nabla_a\left(\nabla_b\left(\nabla_c\left(\nabla_d\right)\right)\right) &= \nabla_{abcd}
\end{align*}
and so on.

In terms of $\nabla$ the above MTW definition of $R^{a}{}_{bcd}$ can written as
\begin{align*}
   \left(\nabla_{cb}-\nabla_{bc}\right) V^{a} = - R^{a}{}_{dbc} V^{d}
\end{align*}

% -------------------------------------------------------------------------------------------------
\subsection*{Symmetrisation}
\documentclass[12pt]{cdblatex}

\begin{document}

% =================================================================================================
% create checkpoint file

\bgroup
\CdbSetup{action=hide}
\begin{cadabra}
   import cdblib
   checkpoint_file = 'tests/semantic/output/dGamma.json'
   cdblib.create (checkpoint_file)
   checkpoint = []
\end{cadabra}
\egroup

% =================================================================================================
\section*{Symmetrized partial derivatives of the connection}

Here we calculate the recursive sequences
\begin{align*}
(n+3)\Gamma^a{}_{d(b,c\uen)}
   &= (n+1)\left(R^a{}_{(bc\Dot d,\uen)}
    - \left(\Gamma^a{}_{f(c}\Gamma^f{}_{b{\Dot d}}\right){\vphantom{\Gamma}}_{,\uen)}\right)
\end{align*}
for $n=1,2,3,\cdots$. Note that the (extended) index $\uen$ contains $n$ normal indices.

The result will be expressions for the $\Gamma^a{}_{d(b,c\uen)}$ in terms of the Riemann tensor and its
partial derivatives.

% =================================================================================================
\section*{Stage 1: Compute symmetrised derivatives}

In the first stage we simply apply the above recursive equation using a simple trick to impose
the symmetries. Start with the original equation and dot out the symmetric indices with $A^a$ then
factor out the partial derivatives. This leads to
\begin{equation}
(n+3)\Gamma^a{}_{db,c\uen} A^b A^c A^{\cdot\uen}
    = (n+1)\left(R^a{}_{bcd}
    - \Gamma^a{}_{fc}\Gamma^f{}_{bd}\right)_{,\uen} A^b A^c A^{\cdot\uen}
\end{equation}
Thus we also have (for the next iteration)
\begin{equation}
(n+4)\Gamma^a{}_{db,c\uenp} A^b A^c A^{\cdot\uenp}
    = (n+2)\left(R^a{}_{bcd}
    - \Gamma^a{}_{fc}\Gamma^f{}_{bd}\right)_{,\uenp} A^b A^c A^{\cdot\uenp}
\end{equation}
The $A^a$ can be freely chosen so choose $A^a$ to be a constant (i.e., zero derivative). Now
define $P_n$ by
\begin{equation}
   P_n = \Gamma^a{}_{db,c\uen} A^b A^c A^{\cdot\uen}
\end{equation}
then the above pair of equations can be combinded to give
\begin{equation}
   P_{n+1} = \frac{(n+2)(n+3)}{(n+4)(n+1)} A^f\partial_f\left(P_n\right)
\end{equation}
This is a very easy equation to compute as it just requires successive rounds of differentiation.

The first term in the sequence is $P_0$ given by
\begin{equation}
   P_0 = \cdb{dGamma01.101}
\end{equation}

The first few results are

\begin{dgroup*}
   \begin{dmath*} P_0 \hiderel{=} A^{b} A^{c} \Gamma^a{}_{d(b,c)} = \cdb{dGamma01.101} \end{dmath*}
   \begin{dmath*} P_1 \hiderel{=} A^{b} A^{c} A^{e} \Gamma^a{}_{d(b,ce)} = \cdb{dGamma02.105} \end{dmath*}
   \begin{dmath*} P_2 \hiderel{=} A^{b} A^{c} A^{e} A^{f} \Gamma^a{}_{d(b,cef)} = \cdb{dGamma03.105} \end{dmath*}
\end{dgroup*}

% =================================================================================================
\section*{Stage 2: Impose Riemann normal coordinates}

Here we impose the RNC condition by setting the $\Gamma^{a}{}_{bc}$ to zero (but not their derivatives).

\begin{dgroup*}
   \begin{dmath*} A^{b} A^{c} \Gamma^a{}_{d(b,c)} = \cdb{dGamma01.201} \end{dmath*}
   \begin{dmath*} A^{b} A^{c} A^{e} \Gamma^a{}_{d(b,ce)} = \cdb{dGamma02.202} \end{dmath*}
   \begin{dmath*} A^{b} A^{c} A^{e} A^{f} \Gamma^a{}_{d(b,cef)} = \cdb{dGamma03.203} \end{dmath*}
\end{dgroup*}

% =================================================================================================
\section*{Stage 3: Replace partial derivatives of $\Gamma$ with partial derivatives of $R$}

The key point to note is that the partial derivatives of $\Gamma$ on the right hand side are all
symmetrized in exactly the same manner as the partial derivatives on the left hand side. Thus
results from the lower order equations can be fed into the later equations to completely eliminate
the partial derivatives of $\Gamma$.

\begin{dgroup*}
   \begin{dmath*} A^{b} A^{c} \Gamma^a{}_{d(b,c)} = \cdb{dGamma01.201} \end{dmath*}
   \begin{dmath*} A^{b} A^{c} A^{e} \Gamma^a{}_{d(b,ce)} = \cdb{dGamma02.202} \end{dmath*}
   \begin{dmath*} A^{b} A^{c} A^{e} A^{f} \Gamma^a{}_{d(b,cef)} = \cdb{dGamma03.402} \end{dmath*}
\end{dgroup*}

% =================================================================================================
\section*{Stage 4: Reformatting}

This is just simple reformatting.

\begin{dgroup*}
   \begin{dmath*}   3 A^b A^c\Gamma^a{}_{d(b,c)} = \cdb{scaled1.002} \end{dmath*}
   \begin{dmath*}   6 A^b A^c A^e \Gamma^a{}_{d(b,ce)} = \cdb{scaled2.002} \end{dmath*}
   \begin{dmath*}  15 A^b A^c A^e A^f \Gamma^a{}_{d(b,cef)} = \cdb{scaled3.002} \end{dmath*}
\end{dgroup*}

\clearpage

% =================================================================================================
\section*{Stage 1: Compute symmetrised derivatives}

\begin{cadabra}
   {a,b,c,d,e,f,g,h,i,j,k,l,m,n,o,p,q,r,s,t,u,v,w#}::Indices(position=independent).

   \nabla{#}::Derivative.
   \partial{#}::PartialDerivative.

   g_{a b}::Metric.
   g^{a b}::InverseMetric.
   g_{a}^{b}::KroneckerDelta.
   g^{a}_{b}::KroneckerDelta.

   R_{a b c d}::RiemannTensor.
   R^{a}_{b c d}::RiemannTensor.
   R_{a b c}^{d}::RiemannTensor.

   \Gamma^{a}_{b c}::TableauSymmetry(shape={2}, indices={1,2}).

   g_{a b}::Depends(\partial{#}).
   R_{a b c d}::Depends(\partial{#}).
   R^{a}_{b c d}::Depends(\partial{#}).
   \Gamma^{a}_{b c}::Depends(\partial{#}).

   # symmetrized partial derivatives of \Gamma

   dGamma01:= (1/3) A^{b} A^{c} ( R^{a}_{b c d} - \Gamma^{a}_{c e}\Gamma^{e}_{b d} ).
                                                        # cdb (dGamma01.101,dGamma01)

   dGamma02:= (6/4) A^{a}\partial_{a}{ @(dGamma01) }.   # cdb (dGamma02.101,dGamma02)
   distribute   (dGamma02)                              # cdb (dGamma02.102,dGamma02)
   product_rule (dGamma02)                              # cdb (dGamma02.103,dGamma02)
   unwrap       (dGamma02)                              # cdb (dGamma02.104,dGamma02)
   distribute   (dGamma02)                              # cdb (dGamma02.105,dGamma02)

   dGamma03:= (12/10) A^{a}\partial_{a}{ @(dGamma02) }. # cdb (dGamma03.101,dGamma03)
   distribute   (dGamma03)                              # cdb (dGamma03.102,dGamma03)
   product_rule (dGamma03)                              # cdb (dGamma03.103,dGamma03)
   unwrap       (dGamma03)                              # cdb (dGamma03.104,dGamma03)
   distribute   (dGamma03)                              # cdb (dGamma03.105,dGamma03)

   dGamma04:= (20/18) A^{a}\partial_{a}{ @(dGamma03) }. # cdb (dGamma04.101,dGamma04)
   distribute   (dGamma04)                              # cdb (dGamma04.102,dGamma04)
   product_rule (dGamma04)                              # cdb (dGamma04.103,dGamma04)
   unwrap       (dGamma04)                              # cdb (dGamma04.104,dGamma04)
   distribute   (dGamma04)                              # cdb (dGamma04.105,dGamma04)

   dGamma05:= (30/28) A^{a}\partial_{a}{ @(dGamma04) }. # cdb (dGamma05.101,dGamma05)
   distribute   (dGamma05)                              # cdb (dGamma05.102,dGamma05)
   product_rule (dGamma05)                              # cdb (dGamma05.103,dGamma05)
   unwrap       (dGamma05)                              # cdb (dGamma05.104,dGamma05)
   distribute   (dGamma05)                              # cdb (dGamma05.105,dGamma05)

\end{cadabra}

\clearpage

\begin{dgroup*}
   \begin{dmath*} \cdb*{dGamma01.101} \end{dmath*}
\end{dgroup*}

\begin{dgroup*}
   \begin{dmath*} \cdb*{dGamma02.101} \end{dmath*}
   \begin{dmath*} \cdb*{dGamma02.102} \end{dmath*}
   \begin{dmath*} \cdb*{dGamma02.103} \end{dmath*}
   \begin{dmath*} \cdb*{dGamma02.104} \end{dmath*}
   \begin{dmath*} \cdb*{dGamma02.105} \end{dmath*}
\end{dgroup*}

\begin{dgroup*}
   \begin{dmath*} \cdb*{dGamma03.101} \end{dmath*}
   \begin{dmath*} \cdb*{dGamma03.102} \end{dmath*}
   \begin{dmath*} \cdb*{dGamma03.103} \end{dmath*}
   \begin{dmath*} \cdb*{dGamma03.104} \end{dmath*}
   \begin{dmath*} \cdb*{dGamma03.105} \end{dmath*}
\end{dgroup*}

\begin{dgroup*}
   \begin{dmath*} \cdb*{dGamma04.101} \end{dmath*}
   \begin{dmath*} \cdb*{dGamma04.102} \end{dmath*}
   \begin{dmath*} \cdb*{dGamma04.103} \end{dmath*}
   \begin{dmath*} \cdb*{dGamma04.104} \end{dmath*}
   \begin{dmath*} \cdb*{dGamma04.105} \end{dmath*}
\end{dgroup*}

% too long for pdflatex

% \begin{dgroup*}
%    \begin{dmath*} \cdb*{dGamma05.101} \end{dmath*}
%    \begin{dmath*} \cdb*{dGamma05.102} \end{dmath*}
%    \begin{dmath*} \cdb*{dGamma05.103} \end{dmath*}
%    \begin{dmath*} \cdb*{dGamma05.104} \end{dmath*}
%    \begin{dmath*} \cdb*{dGamma05.105} \end{dmath*}
% \end{dgroup*}

\clearpage

% =================================================================================================
\section*{Stage 2: Impose Riemann normal coordinates}

\begin{cadabra}
   def impose_rnc (obj):
       # hide the derivatives of Gamma
       substitute (obj,$\partial_{d}{\Gamma^{a}_{b c}} -> zzz_{d}^{a}_{b c}$,repeat=True)
       substitute (obj,$\partial_{d e}{\Gamma^{a}_{b c}} -> zzz_{d e}^{a}_{b c}$,repeat=True)
       substitute (obj,$\partial_{d e f}{\Gamma^{a}_{b c}} -> zzz_{d e f}^{a}_{b c}$,repeat=True)
       substitute (obj,$\partial_{d e f g}{\Gamma^{a}_{b c}} -> zzz_{d e f g}^{a}_{b c}$,repeat=True)
       substitute (obj,$\partial_{d e f g h}{\Gamma^{a}_{b c}} -> zzz_{d e f g h}^{a}_{b c}$,repeat=True)
       # set Gamma to zero
       substitute (obj,$\Gamma^{a}_{b c} -> 0$,repeat=True)
       # recover the derivatives Gamma
       substitute (obj,$zzz_{d}^{a}_{b c} -> \partial_{d}{\Gamma^{a}_{b c}}$,repeat=True)
       substitute (obj,$zzz_{d e}^{a}_{b c} -> \partial_{d e}{\Gamma^{a}_{b c}}$,repeat=True)
       substitute (obj,$zzz_{d e f}^{a}_{b c} -> \partial_{d e f}{\Gamma^{a}_{b c}}$,repeat=True)
       substitute (obj,$zzz_{d e f g}^{a}_{b c} -> \partial_{d e f g}{\Gamma^{a}_{b c}}$,repeat=True)
       substitute (obj,$zzz_{d e f g h}^{a}_{b c} -> \partial_{d e f g h}{\Gamma^{a}_{b c}}$,repeat=True)
       return obj

   # switch to RNC

   dGamma01 = impose_rnc (dGamma01)   # cdb (dGamma01.201,dGamma01)
   dGamma02 = impose_rnc (dGamma02)   # cdb (dGamma02.202,dGamma02)
   dGamma03 = impose_rnc (dGamma03)   # cdb (dGamma03.203,dGamma03)
   dGamma04 = impose_rnc (dGamma04)   # cdb (dGamma04.204,dGamma04)
   dGamma05 = impose_rnc (dGamma05)   # cdb (dGamma05.205,dGamma05)

\end{cadabra}

\begin{dgroup*}
   \begin{dmath*} \cdb*{dGamma01.201} \end{dmath*}
   \begin{dmath*} \cdb*{dGamma02.202} \end{dmath*}
   \begin{dmath*} \cdb*{dGamma03.203} \end{dmath*}
   \begin{dmath*} \cdb*{dGamma04.204} \end{dmath*}
   \begin{dmath*} \cdb*{dGamma05.205} \end{dmath*}
\end{dgroup*}

\clearpage

% =================================================================================================
\section*{Stage 3: Replace partial derivatives of $\Gamma$ with partial derivatives of $R$}

\begin{cadabra}
   # use lower equations to eliminate partial derivs of Gamma from rhs

   # this produces experssions for the partial derivs of the Gamma's in terms of the Rabcd and its partial derivs

   substitute (dGamma03,$A^{c}A^{b}\partial_{c}{\Gamma^{a}_{b d}} -> @(dGamma01)$,repeat=True)                 # cdb(dGamma03.301,dGamma03)
   substitute (dGamma03,$A^{c}A^{b}\partial_{c}{\Gamma^{a}_{d b}} -> @(dGamma01)$,repeat=True)                 # cdb(dGamma03.302,dGamma03)
   distribute (dGamma03)                                                                                       # cdb(dGamma03.303,dGamma03)

   substitute (dGamma04,$A^{c}A^{b}A^{e}\partial_{c e}{\Gamma^{a}_{d b}} -> @(dGamma02)$,repeat=True)          # cdb(dGamma04.301,dGamma04)
   substitute (dGamma04,$A^{c}A^{b}A^{e}\partial_{c e}{\Gamma^{a}_{b d}} -> @(dGamma02)$,repeat=True)          # cdb(dGamma04.302,dGamma04)
   substitute (dGamma04,$A^{c}A^{b}\partial_{c}{\Gamma^{a}_{b d}} -> @(dGamma01)$,repeat=True)                 # cdb(dGamma04.303,dGamma04)
   substitute (dGamma04,$A^{c}A^{b}\partial_{c}{\Gamma^{a}_{d b}} -> @(dGamma01)$,repeat=True)                 # cdb(dGamma04.304,dGamma04)
   distribute (dGamma04)                                                                                       # cdb(dGamma04.305,dGamma04)

   substitute (dGamma05,$A^{c}A^{b}A^{e}A^{f}\partial_{c e f}{\Gamma^{a}_{d b}} -> @(dGamma03)$,repeat=True)   # cdb(dGamma05.301,dGamma05)
   substitute (dGamma05,$A^{c}A^{b}A^{e}A^{f}\partial_{c e f}{\Gamma^{a}_{b d}} -> @(dGamma03)$,repeat=True)   # cdb(dGamma05.302,dGamma05)
   substitute (dGamma05,$A^{c}A^{b}A^{e}\partial_{c e}{\Gamma^{a}_{d b}} -> @(dGamma02)$,repeat=True)          # cdb(dGamma05.303,dGamma05)
   substitute (dGamma05,$A^{c}A^{b}A^{e}\partial_{c e}{\Gamma^{a}_{b d}} -> @(dGamma02)$,repeat=True)          # cdb(dGamma05.304,dGamma05)
   substitute (dGamma05,$A^{c}A^{b}\partial_{c}{\Gamma^{a}_{b d}} -> @(dGamma01)$,repeat=True)                 # cdb(dGamma05.305,dGamma05)
   substitute (dGamma05,$A^{c}A^{b}\partial_{c}{\Gamma^{a}_{d b}} -> @(dGamma01)$,repeat=True)                 # cdb(dGamma05.306,dGamma05)
   distribute (dGamma05)                                                                                       # cdb(dGamma05.307,dGamma05)

\end{cadabra}

\clearpage

\begin{dgroup*}
   \begin{dmath*} \cdb*{dGamma03.301} \end{dmath*}
   \begin{dmath*} \cdb*{dGamma03.302} \end{dmath*}
   \begin{dmath*} \cdb*{dGamma03.303} \end{dmath*}
\end{dgroup*}

\begin{dgroup*}
   \begin{dmath*} \cdb*{dGamma04.301} \end{dmath*}
   \begin{dmath*} \cdb*{dGamma04.302} \end{dmath*}
   \begin{dmath*} \cdb*{dGamma04.303} \end{dmath*}
   \begin{dmath*} \cdb*{dGamma04.304} \end{dmath*}
   \begin{dmath*} \cdb*{dGamma04.305} \end{dmath*}
\end{dgroup*}

\begin{dgroup*}
   \begin{dmath*} \cdb*{dGamma05.301} \end{dmath*}
   \begin{dmath*} \cdb*{dGamma05.302} \end{dmath*}
   \begin{dmath*} \cdb*{dGamma05.303} \end{dmath*}
   \begin{dmath*} \cdb*{dGamma05.304} \end{dmath*}
   \begin{dmath*} \cdb*{dGamma05.305} \end{dmath*}
   \begin{dmath*} \cdb*{dGamma05.306} \end{dmath*}
   \begin{dmath*} \cdb*{dGamma05.307} \end{dmath*}
\end{dgroup*}

\clearpage

\begin{cadabra}
   # note:
   # canonicalise must not be used here because it may make changes like
   #    R^{a}_{b c d} -> - R_{b}^{a}_{c d}
   # these changes can not be applied inside a \partial, must defer use
   # of canocialise until we have \nabla acting on curvatures

   sort_product   (dGamma03) # cdb(dGamma03.401,dGamma03)
   rename_dummies (dGamma03) # cdb(dGamma03.402,dGamma03)
   # canonicalise   (dGamma03) # cdb(dGamma03.403,dGamma03)

   sort_product   (dGamma04) # cdb(dGamma04.401,dGamma04)
   rename_dummies (dGamma04) # cdb(dGamma04.402,dGamma04)
   # canonicalise   (dGamma04) # cdb(dGamma04.403,dGamma04)

   sort_product   (dGamma05) # cdb(dGamma05.401,dGamma05)
   rename_dummies (dGamma05) # cdb(dGamma05.402,dGamma05)
   # canonicalise   (dGamma05) # cdb(dGamma05.403,dGamma05)

\end{cadabra}

\clearpage

\begin{dgroup*}
   \begin{dmath*} \cdb*{dGamma03.401} \end{dmath*}
   \begin{dmath*} \cdb*{dGamma03.402} \end{dmath*}
   % \begin{dmath*} \cdb*{dGamma03.403} \end{dmath*}
\end{dgroup*}

\begin{dgroup*}
   \begin{dmath*} \cdb*{dGamma04.401} \end{dmath*}
   \begin{dmath*} \cdb*{dGamma04.402} \end{dmath*}
   % \begin{dmath*} \cdb*{dGamma04.403} \end{dmath*}
\end{dgroup*}

\begin{dgroup*}
   \begin{dmath*} \cdb*{dGamma05.401} \end{dmath*}
   \begin{dmath*} \cdb*{dGamma05.402} \end{dmath*}
   % \begin{dmath*} \cdb*{dGamma05.403} \end{dmath*}
\end{dgroup*}

\clearpage

\begin{cadabra}
   import cdblib

   cdblib.create ('dGamma.json')

   cdblib.put ('dGamma01',dGamma01,'dGamma.json')
   cdblib.put ('dGamma02',dGamma02,'dGamma.json')
   cdblib.put ('dGamma03',dGamma03,'dGamma.json')
   cdblib.put ('dGamma04',dGamma04,'dGamma.json')
   cdblib.put ('dGamma05',dGamma05,'dGamma.json')

\end{cadabra}

\clearpage

% =================================================================================================
\section*{Stage 4: Reformatting}

\begin{cadabra}
   # note: keeping numbering as is (out of order) to ensure R appears before \nabla R etc.
   def product_sort (obj):
       substitute (obj,$ A^{a}                              -> A001^{a}                 $)
       substitute (obj,$ x^{a}                              -> A002^{a}                 $)
       substitute (obj,$ g^{a b}                            -> A003^{a b}               $)
       substitute (obj,$ \partial_{e f g h}{R^{a}_{b c d}}  -> A008^{a}_{b c d e f g h} $)
       substitute (obj,$ \partial_{e f g}{R^{a}_{b c d}}    -> A007^{a}_{b c d e f g}   $)
       substitute (obj,$ \partial_{e f}{R^{a}_{b c d}}      -> A006^{a}_{b c d e f}     $)
       substitute (obj,$ \partial_{e}{R^{a}_{b c d}}        -> A005^{a}_{b c d e}       $)
       substitute (obj,$ R^{a}_{b c d}                      -> A004^{a}_{b c d}         $)
       sort_product   (obj)
       rename_dummies (obj)
       substitute (obj,$ A001^{a}                  -> A^{a}                             $)
       substitute (obj,$ A002^{a}                  -> x^{a}                             $)
       substitute (obj,$ A003^{a b}                -> g^{a b}                           $)
       substitute (obj,$ A004^{a}_{b c d}          -> R^{a}_{b c d}                     $)
       substitute (obj,$ A005^{a}_{b c d e}        -> \partial_{e}{R^{a}_{b c d}}       $)
       substitute (obj,$ A006^{a}_{b c d e f}      -> \partial_{e f}{R^{a}_{b c d}}     $)
       substitute (obj,$ A007^{a}_{b c d e f g}    -> \partial_{e f g}{R^{a}_{b c d}}   $)
       substitute (obj,$ A008^{a}_{b c d e f g h}  -> \partial_{e f g h}{R^{a}_{b c d}} $)

       return obj

   def reformat (obj,scale):
       bah  = Ex(str(scale))
       tmp := @(bah) @(obj).
       distribute     (tmp)
       tmp = product_sort (tmp)
       rename_dummies (tmp)
       factor_out     (tmp,$A^{a?}$)
       return tmp

   def get_term (obj,n):

       A^{a}::Weight(label=numA).

       foo := @(obj).
       bah  = Ex("numA = " + str(n))
       distribute  (foo)
       keep_weight (foo, bah)

       return foo

   Gterm01 := @(dGamma01).
   Gterm02 := @(dGamma02).
   Gterm03 := @(dGamma03).
   Gterm04 := @(dGamma04).
   Gterm05 := @(dGamma05).

   scaled1 = reformat (Gterm01,   3)   # cdb (scaled1.002,scaled1)
   scaled2 = reformat (Gterm02,   6)   # cdb (scaled2.002,scaled2)
   scaled3 = reformat (Gterm03,  15)   # cdb (scaled3.002,scaled3)
   scaled4 = reformat (Gterm04,   9)   # cdb (scaled4.002,scaled4)
   scaled5 = reformat (Gterm05, 252)   # cdb (scaled5.002,scaled5)

\end{cadabra}

\clearpage

% =================================================================================================
\section*{Symmetrised partial derivatives of the connection}

\begin{dgroup*}
   \begin{dmath*}   3 A^b A^c\Gamma^a{}_{d(b,c)} = \cdb{scaled1.002} \end{dmath*}
   \begin{dmath*}   6 A^b A^c A^e \Gamma^a{}_{d(b,ce)} = \cdb{scaled2.002} \end{dmath*}
   \begin{dmath*}  15 A^b A^c A^e A^f \Gamma^a{}_{d(b,cef)} = \cdb{scaled3.002} \end{dmath*}
   \begin{dmath*}   9 A^b A^c A^e A^f A^g \Gamma^a{}_{d(b,cefg)} = \cdb{scaled4.002} \end{dmath*}
   \begin{dmath*} 252 A^b A^c A^e A^f A^g A^h\Gamma^a{}_{d(b,cefgh)} = \cdb{scaled5.002} \end{dmath*}
\end{dgroup*}

\clearpage

% =================================================================================================
% export selected objects, these will later be imported into a library
% these are the objects that will appear in the paper

\begin{cadabra}
   substitute (scaled1,$A^{a}->1$)
   substitute (scaled2,$A^{a}->1$)
   substitute (scaled3,$A^{a}->1$)
   substitute (scaled4,$A^{a}->1$)
   substitute (scaled5,$A^{a}->1$)

   cdblib.create ('dGamma.export')

   # 6th order dGamma, scaled
   cdblib.put ('dGamma61scaled',scaled1,'dGamma.export')
   cdblib.put ('dGamma62scaled',scaled2,'dGamma.export')
   cdblib.put ('dGamma63scaled',scaled3,'dGamma.export')
   cdblib.put ('dGamma64scaled',scaled4,'dGamma.export')
   cdblib.put ('dGamma65scaled',scaled5,'dGamma.export')

   checkpoint.append (scaled1)
   checkpoint.append (scaled2)
   checkpoint.append (scaled3)
   checkpoint.append (scaled4)
   checkpoint.append (scaled5)

\end{cadabra}

% =================================================================================================
% export checkpoints in json format

\bgroup
\CdbSetup{action=hide}
\begin{cadabra}
   for i in range( len(checkpoint) ):
      cdblib.put ('check{:03d}'.format(i),checkpoint[i],checkpoint_file)
\end{cadabra}
\egroup

\end{document}


In the following pages there will be frequent constructions of the form
\begin{dgroup*}
   \begin{dmath*}  3 A^b A^c\Gamma^a{}_{d(b,c)} = \cdb{scaled1.002} \end{dmath*}
   \begin{dmath*}  6 A^b A^c A^e \Gamma^a{}_{d(b,ce)} = \cdb{scaled2.002} \end{dmath*}
   \begin{dmath*} 15 A^b A^c A^e A^f \Gamma^a{}_{d(b,cef)} = \cdb{scaled3.002} \end{dmath*}
\end{dgroup*}
The vector $A^{a}$ has no special meaning. Its purpose is to indicate that the
associciated tensor is symmetric over a selection of its indices. If the $A^{a}$ were not included
then the right hand side would either need to be spelt out in full or some other device would
be needed to denote the symmetries. The symmetrisation brackets are included on the left hand
side though they are redundant (in the presence of the $A{a}$).

\clearpage

% =================================================================================================
\section*{The metric in RNC}
\documentclass[12pt]{cdblatex}

\begin{document}

\section*{\jobname}

\CdbSetup{action=hide}

\begin{cadabra}
   import shared

   import cdblib

   term00A = cdblib.get ('check000','expected/metric.json')
   term01A = cdblib.get ('check001','expected/metric.json')
   term02A = cdblib.get ('check002','expected/metric.json')
   term03A = cdblib.get ('check003','expected/metric.json')
   term04A = cdblib.get ('check004','expected/metric.json')
   term05A = cdblib.get ('check005','expected/metric.json')
   term06A = cdblib.get ('check005','expected/metric.json')
   term07A = cdblib.get ('check005','expected/metric.json')

   term00B = cdblib.get ('check000','output/metric.json')
   term01B = cdblib.get ('check001','output/metric.json')
   term02B = cdblib.get ('check002','output/metric.json')
   term03B = cdblib.get ('check003','output/metric.json')
   term04B = cdblib.get ('check004','output/metric.json')
   term05B = cdblib.get ('check005','output/metric.json')
   term06B = cdblib.get ('check005','output/metric.json')
   term07B = cdblib.get ('check005','output/metric.json')

   # bug: can't push this function into shared.py
   #      no synatx error, but cadabra doesn't cancel equal terms
   # see ~/cadabra/bugs/bug02

   def check (objA,objB):
       tmp := @(objA) - @(objB).
       distribute         (tmp)
       tmp = shared.standard_indices (tmp)
       tmp = shared.product_sort (tmp)
       rename_dummies     (tmp)
       canonicalise       (tmp)

       return tmp

   diff000 = shared.check (term00A,term00B)   # cdb (diff000,diff000)
   diff001 = shared.check (term01A,term01B)   # cdb (diff001,diff001)
   diff002 = shared.check (term02A,term02B)   # cdb (diff002,diff002)
   diff003 = shared.check (term03A,term03B)   # cdb (diff003,diff003)
   diff004 = shared.check (term04A,term04B)   # cdb (diff004,diff004)
   diff005 = shared.check (term05A,term05B)   # cdb (diff005,diff005)
   diff006 = shared.check (term06A,term06B)   # cdb (diff006,diff006)
   diff007 = shared.check (term07A,term07B)   # cdb (diff007,diff007)

\end{cadabra}

\begin{dgroup*}
   \Dmath*{ \cdb*{diff000} }
   \Dmath*{ \cdb*{diff001} }
   \Dmath*{ \cdb*{diff002} }
   \Dmath*{ \cdb*{diff003} }
   \Dmath*{ \cdb*{diff004} }
   \Dmath*{ \cdb*{diff005} }
   \Dmath*{ \cdb*{diff006} }
   \Dmath*{ \cdb*{diff007} }
\end{dgroup*}

\end{document}


\begin{dgroup*}
   \begin{dmath*} g_{a b}(x) = \cdb{Metric.601}+\BigO{\eps^6} \end{dmath*}
\end{dgroup*}

% =================================================================================================
\section*{Curvature expansion of the metric}
\begin{align*}
     g_{a b}(x) =
     \ngab{0}_{a b}
   + \ngab{2}_{a b}
   + \ngab{3}_{a b}
   + \ngab{4}_{a b}
   + \ngab{5}_{a b}+\BigO{\eps^6}
\end{align*}
\begin{dgroup*}
   \begin{dmath*}     \ngab{0}_{a b} = \cdb{scaled0.601} \end{dmath*}
   \begin{dmath*}   3 \ngab{2}_{a b} = \cdb{scaled2.601} \end{dmath*}
   \begin{dmath*}   6 \ngab{3}_{a b} = \cdb{scaled3.601} \end{dmath*}
   \begin{dmath*} 180 \ngab{4}_{a b} = \cdb{scaled4.601} \end{dmath*}
   \begin{dmath*}  90 \ngab{5}_{a b} = \cdb{scaled5.601} \end{dmath*}
\end{dgroup*}

\clearpage

% =================================================================================================
\section*{The inverse metric in RNC}
\def\Date{30 Jul 2024}
% \def\FileID{file:}

\documentclass[12pt]{cdblatex}

\begin{document}

% =================================================================================================
% create checkpoint file

\bgroup
\CdbSetup{action=hide}
\begin{cadabra}
   import cdblib
   checkpoint_file = 'tests/semantic/output/metric-inv.json'
   cdblib.create (checkpoint_file)
   checkpoint = []
\end{cadabra}
\egroup

% =================================================================================================
\section*{The inverse metric tensor in Riemann normal coordinates}

Here we calculate the Riemann normal expansion of the inverse metric, $g^{ab}$, by developing
the recursive sequences
\begin{align}
\label{eq:pdgab}
g^{ab}{}_{,d\ue} &= - \left(g^{cb}\Gamma^{a}{}_{c(d}\right){}_{,\ue)}
                    - \left(g^{ac}\Gamma^{b}{}_{c(d}\right){}_{,\ue)}\\[10pt]
\label{eq:pdGamma}
(n+3)\Gamma^a{}_{d(b,c\ue)} &= (n+1)\left(R^a{}_{(bc\Dot d,\ue)} - \left(\Gamma^a{}_{f(c}\Gamma^f{}_{b{\Dot d}}\right){}_{,\ue)}\right)
\end{align}
for $n=1,2,3,\cdots$. Note in these equations that the (extended) index $\ue$ contains $n$
normal indices.

We then construct a Taylor series for the metric using
\begin{dmath*}[spread=5pt]
g^{ab}(x) = g^{ab} + g^{ab}{}_{,c}x^c + \frac{1}{2!} g^{ab}{}_{,cd}x^cx^d + \frac{1}{3!} g^{ab}{}_{,cde}x^cx^dx^e + \cdots
          = g^{ab} + \sum_{n=1}^\infty\> \frac{1}{n!}\>g^{ab}{}_{,\uc}\>x^{.\uc}
\end{dmath*}

% =================================================================================================
\section*{Stage 1: Symmetrised partial derivatives of $g^{ab}$}

In this stage, equation (\ref{eq:pdgab}) is used to express the symmetrised partial derivatives
of the metric in terms of the symmetrised partial derivatives of the connection.

\begin{dgroup*}
   \begin{dmath*} g^{ab}{}_{,c} A^{c} = \cdb{term1.200} \end{dmath*}
   \begin{dmath*} g^{ab}{}_{,cd} A^{c} A^{d} = \cdb{term2.200} \end{dmath*}
   \begin{dmath*} g^{ab}{}_{,cde} A^{c} A^{d} A^{e} = \cdb{term3.200} \end{dmath*}
\end{dgroup*}

% =================================================================================================
\section*{Stage 2: Replace derivatives of $\Gamma$ with partial derivs of $R$}

Now we use the results from {\verb|dGamma|} to replace derivatives of $\Gamma$ with
partial derivatives of $R$. These were computed in {\verb|dGamma|} using equation
(\ref{eq:pdGamma}) above.

\begin{dgroup*}
   \begin{dmath*} g^{ab}{}_{,c} A^{c} = \cdb{term1.200} \end{dmath*}
   \begin{dmath*} g^{ab}{}_{,cd} A^{c} A^{d} = \cdb{term2.303} \end{dmath*}
   \begin{dmath*} g^{ab}{}_{,cde} A^{c} A^{d} A^{e} = \cdb{term3.305} \end{dmath*}
\end{dgroup*}

% =================================================================================================
\section*{Stage 3: Replace partial derivs of $R$ with covariant derivs of $R$}

Next we use the results from {\verb|dRabcd|} to replace the partial derivatives of $R$ with
covariant deriavtives.

\begin{dgroup*}
   \begin{dmath*} g^{ab}{}_{,c} A^{c} = \cdb{term1.404} \end{dmath*}
   \begin{dmath*} g^{ab}{}_{,cd} A^{c} A^{d} = \cdb{term2.404} \end{dmath*}
   \begin{dmath*} g^{ab}{}_{,cde} A^{c} A^{d} A^{e} = \cdb{term3.403} \end{dmath*}
\end{dgroup*}

% =================================================================================================
\section*{Stage 4: Build the Taylor series for $g_{ab}$, reformatting and output}

Each of the above expressions constitutues one term in the Taylor series for the metric.
We also make the trivial change $A\rightarrow x$. Then we do some trivial reformatting.

\begin{align*}
   g_{ab}(x) &=   g^{ab}
                + g^{ab}{}_{,c} x^c
                + \frac{1}{2!} g^{ab}{}_{,cd} x^c x^d
                + \frac{1}{3!} g^{ab}{}_{,cde} x^c x^d x^e +  \BigO{\eps^4}\\
             &= \cdb{metric4.501} + \BigO{\eps^4}
\end{align*}

\clearpage

% =================================================================================================
\section*{Shared properties}

\begin{cadabra}
   import time

   def flatten_Rabcd (obj):
       substitute (obj,$R^{a}_{b c d}   -> g^{a e} R_{e b c d}$)
       substitute (obj,$R_{a}^{b}_{c d} -> g^{b e} R_{a e c d}$)
       substitute (obj,$R_{a b}^{c}_{b} -> g^{c e} R_{a b e d}$)
       substitute (obj,$R_{a b c}^{d}   -> g^{d e} R_{a b c e}$)
       unwrap     (obj)
       sort_product   (obj)
       rename_dummies (obj)
       return obj

   def impose_rnc (obj):
       # hide the derivatives of Gamma
       substitute (obj,$\partial_{d}{\Gamma^{a}_{b c}} -> zzz_{d}^{a}_{b c}$,repeat=True)
       substitute (obj,$\partial_{d e}{\Gamma^{a}_{b c}} -> zzz_{d e}^{a}_{b c}$,repeat=True)
       substitute (obj,$\partial_{d e f}{\Gamma^{a}_{b c}} -> zzz_{d e f}^{a}_{b c}$,repeat=True)
       substitute (obj,$\partial_{d e f g}{\Gamma^{a}_{b c}} -> zzz_{d e f g}^{a}_{b c}$,repeat=True)
       substitute (obj,$\partial_{d e f g h}{\Gamma^{a}_{b c}} -> zzz_{d e f g h}^{a}_{b c}$,repeat=True)
       # set Gamma to zero
       substitute (obj,$\Gamma^{a}_{b c} -> 0$,repeat=True)
       # recover the derivatives Gamma
       substitute (obj,$zzz_{d}^{a}_{b c} -> \partial_{d}{\Gamma^{a}_{b c}}$,repeat=True)
       substitute (obj,$zzz_{d e}^{a}_{b c} -> \partial_{d e}{\Gamma^{a}_{b c}}$,repeat=True)
       substitute (obj,$zzz_{d e f}^{a}_{b c} -> \partial_{d e f}{\Gamma^{a}_{b c}}$,repeat=True)
       substitute (obj,$zzz_{d e f g}^{a}_{b c} -> \partial_{d e f g}{\Gamma^{a}_{b c}}$,repeat=True)
       substitute (obj,$zzz_{d e f g h}^{a}_{b c} -> \partial_{d e f g h}{\Gamma^{a}_{b c}}$,repeat=True)
       return obj

   def get_xterm (obj,n):

       x^{a}::Weight(label=numx).

       foo := @(obj).
       bah  = Ex("numx = " + str(n))
       keep_weight (foo,bah)

       return foo

   # note: keeping numbering as is (out of order) to ensure R appears before \nabla R etc.
   def product_sort (obj):
       substitute (obj,$ A^{a}                             -> A001^{a}                  $)
       substitute (obj,$ x^{a}                             -> A002^{a}                  $)
       substitute (obj,$ g_{a b}                           -> A003_{a b}                $)
       substitute (obj,$ g^{a b}                           -> A004^{a b}                $)
       substitute (obj,$ \nabla_{e f g h}{R_{a b c d}}     -> A010_{a b c d e f g h}    $)
       substitute (obj,$ \nabla_{e f g}{R_{a b c d}}       -> A009_{a b c d e f g}      $)
       substitute (obj,$ \nabla_{e f}{R_{a b c d}}         -> A008_{a b c d e f}        $)
       substitute (obj,$ \nabla_{e}{R_{a b c d}}           -> A007_{a b c d e}          $)
       substitute (obj,$ \partial_{e f g h}{R_{a b c d}}   -> A014_{a b c d e f g h}    $)
       substitute (obj,$ \partial_{e f g}{R_{a b c d}}     -> A013_{a b c d e f g}      $)
       substitute (obj,$ \partial_{e f}{R_{a b c d}}       -> A012_{a b c d e f}        $)
       substitute (obj,$ \partial_{e}{R_{a b c d}}         -> A011_{a b c d e}          $)
       substitute (obj,$ \partial_{e f g h}{R^{a}_{b c d}} -> A018^{a}_{b c d e f g h}  $)
       substitute (obj,$ \partial_{e f g}{R^{a}_{b c d}}   -> A017^{a}_{b c d e f g}    $)
       substitute (obj,$ \partial_{e f}{R^{a}_{b c d}}     -> A016^{a}_{b c d e f}      $)
       substitute (obj,$ \partial_{e}{R^{a}_{b c d}}       -> A015^{a}_{b c d e}        $)
       substitute (obj,$ R_{a b c d}                       -> A005_{a b c d}            $)
       substitute (obj,$ R^{a}_{b c d}                     -> A006^{a}_{b c d}          $)
       sort_product   (obj)
       rename_dummies (obj)
       substitute (obj,$ A001^{a}                  -> A^{a}                             $)
       substitute (obj,$ A002^{a}                  -> x^{a}                             $)
       substitute (obj,$ A003_{a b}                -> g_{a b}                           $)
       substitute (obj,$ A004^{a b}                -> g^{a b}                           $)
       substitute (obj,$ A005_{a b c d}            -> R_{a b c d}                       $)
       substitute (obj,$ A006^{a}_{b c d}          -> R^{a}_{b c d}                     $)
       substitute (obj,$ A007_{a b c d e}          -> \nabla_{e}{R_{a b c d}}           $)
       substitute (obj,$ A008_{a b c d e f}        -> \nabla_{e f}{R_{a b c d}}         $)
       substitute (obj,$ A009_{a b c d e f g}      -> \nabla_{e f g}{R_{a b c d}}       $)
       substitute (obj,$ A010_{a b c d e f g h}    -> \nabla_{e f g h}{R_{a b c d}}     $)
       substitute (obj,$ A011_{a b c d e}          -> \partial_{e}{R_{a b c d}}         $)
       substitute (obj,$ A012_{a b c d e f}        -> \partial_{e f}{R_{a b c d}}       $)
       substitute (obj,$ A013_{a b c d e f g}      -> \partial_{e f g}{R_{a b c d}}     $)
       substitute (obj,$ A014_{a b c d e f g h}    -> \partial_{e f g h}{R_{a b c d}}   $)
       substitute (obj,$ A015^{a}_{b c d e}        -> \partial_{e}{R^{a}_{b c d}}       $)
       substitute (obj,$ A016^{a}_{b c d e f}      -> \partial_{e f}{R^{a}_{b c d}}     $)
       substitute (obj,$ A017^{a}_{b c d e f g}    -> \partial_{e f g}{R^{a}_{b c d}}   $)
       substitute (obj,$ A018^{a}_{b c d e f g h}  -> \partial_{e f g h}{R^{a}_{b c d}} $)

       return obj

   def reformat_xterm (obj,scale):
      foo  = Ex(str(scale))
      bah := @(foo) @(obj).
      distribute     (bah)
      bah = product_sort (bah)
      rename_dummies (bah)
      canonicalise   (bah)
      factor_out     (bah,$x^{a?}$)
      ans := @(bah) / @(foo).
      return ans

   def rescale_xterm (obj,scale):
      foo  = Ex(str(scale))
      bah := @(foo) @(obj).
      distribute  (bah)
      factor_out  (bah,$x^{a?}$)
      return bah

   def add_tags (obj,tag):
      n = 0
      ans = Ex('0')
      for i in obj.top().terms():
         foo = obj[i]
         bah = Ex(tag+'_{'+str(n)+'}')
         ans := @(ans) + @(bah) @(foo).
         n = n + 1
      return ans

   def clear_tags (obj,tag):
      ans := @(obj).
      foo  = Ex(tag+'_{a?} -> 1')
      substitute (ans,foo)
      return ans

   {a,b,c,d,e,f,g,h,i,j,k,l,m,n,o,p,q,r,s,t,u,v,w#}::Indices(position=independent).

   \nabla{#}::Derivative.
   \partial{#}::PartialDerivative.

   g_{a b}::Metric.
   g^{a b}::InverseMetric.
   g_{a}^{b}::KroneckerDelta.
   g^{a}_{b}::KroneckerDelta.

   R_{a b c d}::RiemannTensor.
   R^{a}_{b c d}::RiemannTensor.
   R_{a b c}^{d}::RiemannTensor.

   \Gamma^{a}_{b c}::TableauSymmetry(shape={2}, indices={1,2}).

   g_{a b}::Depends(\partial{#}).
   R_{a b c d}::Depends(\partial{#}).
   R^{a}_{b c d}::Depends(\partial{#}).
   \Gamma^{a}_{b c}::Depends(\partial{#}).

   R_{a b c d}::Depends(\nabla{#}).
   R^{a}_{b c d}::Depends(\nabla{#}).

\end{cadabra}

\clearpage

% =================================================================================================
\section*{Stage 1: Symmetrised partial derivatives of $g^{ab}$}

\begin{cadabra}
   beg_stage_1 = time.time()

   # symmetrised partial derivatives of g^{ab}

   gab00:=g^{a b}.                                              # cdb (gab00.101,gab00)

   gab01:= - g^{c b}\Gamma^{a}_{c d} - g^{a c}\Gamma^{b}_{c d}. # cdb (gab01.101,gab01)

   gab02:=\partial_{e}{ @(gab01) }.                             # cdb (gab02.101,gab02)
   distribute   (gab02)                                         # cdb (gab02.102,gab02)
   product_rule (gab02)                                         # cdb (gab02.103,gab02)
   substitute   (gab02, $\partial_{d}{g^{a b}} -> @(gab01)$)    # cdb (gab02.104,gab02)
   distribute   (gab02)                                         # cdb (gab02.105,gab02)

   gab03:=\partial_{f}{ @(gab02) }.                             # cdb (gab03.101,gab03)
   distribute   (gab03)                                         # cdb (gab03.102,gab03)
   product_rule (gab03)                                         # cdb (gab03.103,gab03)
   substitute   (gab03, $\partial_{d}{g^{a b}} -> @(gab01)$)    # cdb (gab03.104,gab03)
   distribute   (gab03)                                         # cdb (gab03.105,gab03)

   gab04:=\partial_{g}{ @(gab03) }.                             # cdb (gab04.101,gab04)
   distribute   (gab04)                                         # cdb (gab04.102,gab04)
   product_rule (gab04)                                         # cdb (gab04.103,gab04)
   substitute   (gab04, $\partial_{d}{g^{a b}} -> @(gab01)$)    # cdb (gab04.104,gab04)
   distribute   (gab04)                                         # cdb (gab04.105,gab04)

   gab05:=\partial_{h}{ @(gab04) }.                             # cdb (gab05.101,gab05)
   distribute   (gab05)                                         # cdb (gab05.102,gab05)
   product_rule (gab05)                                         # cdb (gab05.103,gab05)
   substitute   (gab05, $\partial_{d}{g^{a b}} -> @(gab01)$)    # cdb (gab05.104,gab05)
   distribute   (gab05)                                         # cdb (gab05.105,gab05)

   gab00 = impose_rnc (gab00)   # cdb (gab00.102,gab00)
   gab01 = impose_rnc (gab01)   # cdb (gab01.102,gab01)
   gab02 = impose_rnc (gab02)   # cdb (gab02.106,gab02)
   gab03 = impose_rnc (gab03)   # cdb (gab03.106,gab03)
   gab04 = impose_rnc (gab04)   # cdb (gab04.106,gab04)
   gab05 = impose_rnc (gab05)   # cdb (gab05.106,gab05)

\end{cadabra}

\clearpage

\begin{dgroup*}
   \begin{dmath*} \cdb*{gab00.101} \end{dmath*}
   \begin{dmath*} \cdb*{gab00.102} \end{dmath*}
   \begin{dmath*} \cdb*{gab01.101} \end{dmath*}
   \begin{dmath*} \cdb*{gab01.102} \end{dmath*}
\end{dgroup*}

\begin{dgroup*}
   \begin{dmath*} \cdb*{gab02.101} \end{dmath*}
   \begin{dmath*} \cdb*{gab02.102} \end{dmath*}
   \begin{dmath*} \cdb*{gab02.103} \end{dmath*}
   \begin{dmath*} \cdb*{gab02.104} \end{dmath*}
   \begin{dmath*} \cdb*{gab02.105} \end{dmath*}
   \begin{dmath*} \cdb*{gab02.106} \end{dmath*}
\end{dgroup*}

\begin{dgroup*}
   \begin{dmath*} \cdb*{gab03.101} \end{dmath*}
   \begin{dmath*} \cdb*{gab03.102} \end{dmath*}
   \begin{dmath*} \cdb*{gab03.103} \end{dmath*}
   \begin{dmath*} \cdb*{gab03.104} \end{dmath*}
   \begin{dmath*} \cdb*{gab03.105} \end{dmath*}
   \begin{dmath*} \cdb*{gab03.106} \end{dmath*}
\end{dgroup*}

\begin{dgroup*}
   \begin{dmath*} \cdb*{gab04.101} \end{dmath*}
   \begin{dmath*} \cdb*{gab04.102} \end{dmath*}
   \begin{dmath*} \cdb*{gab04.103} \end{dmath*}
   \begin{dmath*} \cdb*{gab04.104} \end{dmath*}
   \begin{dmath*} \cdb*{gab04.105} \end{dmath*}
   \begin{dmath*} \cdb*{gab04.106} \end{dmath*}
\end{dgroup*}

\begin{cadabra}
   # prepare first six terms in the Taylor series expansion of g^{ab}(x)

   term0:= @(gab00).
   distribute (term0)                             # cdb(term0.200,term0)

   term1:= @(gab01) A^d.
   distribute (term1)                             # cdb(term1.200,term1)

   term2:= @(gab02) A^d A^e.
   distribute (term2)                             # cdb(term2.200,term2)

   term3:= @(gab03) A^d A^e A^f.
   distribute (term3)                             # cdb(term3.200,term3)

   term4:= @(gab04) A^d A^e A^f A^g.
   distribute (term4)                             # cdb(term4.200,term4)

   term5:= @(gab05) A^d A^e A^f A^g A^h.
   distribute (term5)                             # cdb(term5.200,term5)

   end_stage_1 = time.time()
\end{cadabra}

\begin{dgroup*}
   \begin{dmath*} \cdb*{term0.200} \end{dmath*}
   \begin{dmath*} \cdb*{term1.200} \end{dmath*}
   \begin{dmath*} \cdb*{term2.200} \end{dmath*}
   \begin{dmath*} \cdb*{term3.200} \end{dmath*}
   % \begin{dmath*} \cdb*{term4.200} \end{dmath*}
   % \begin{dmath*} \cdb*{term5.200} \end{dmath*}
\end{dgroup*}

\clearpage

% =================================================================================================
\section*{Stage 2: Replace derivatives of $\Gamma$ with partial derivs of $R$}

\begin{cadabra}
   import cdblib

   beg_stage_2 = time.time()

   dGamma01 = cdblib.get ('dGamma01','dGamma.json')  # cdb(dGamma01.300,dGamma01)
   dGamma02 = cdblib.get ('dGamma02','dGamma.json')  # cdb(dGamma02.300,dGamma02)
   dGamma03 = cdblib.get ('dGamma03','dGamma.json')  # cdb(dGamma03.300,dGamma03)
   dGamma04 = cdblib.get ('dGamma04','dGamma.json')  # cdb(dGamma04.300,dGamma04)
   dGamma05 = cdblib.get ('dGamma05','dGamma.json')  # cdb(dGamma05.300,dGamma05)

   # replace partial derivs of \Gamma with products and derivs of Riemann tensor

   substitute (term2,$\partial_{c}{\Gamma^{a}_{b d}}A^{c}A^{b} -> @(dGamma01)$,repeat=True)                       # cdb(term2.301,term2)
   substitute (term2,$\partial_{c}{\Gamma^{a}_{d b}}A^{c}A^{b} -> @(dGamma01)$,repeat=True)                       # cdb(term2.302,term2)
   distribute (term2)                                                                                             # cdb(term2.303,term2)

   substitute (term3,$\partial_{c e}{\Gamma^{a}_{d b}}A^{c}A^{b}A^{e} -> @(dGamma02)$,repeat=True)                # cdb(term3.301,term3)
   substitute (term3,$\partial_{c e}{\Gamma^{a}_{b d}}A^{c}A^{b}A^{e} -> @(dGamma02)$,repeat=True)                # cdb(term3.302,term3)
   substitute (term3,$\partial_{c}{\Gamma^{a}_{b d}}A^{c}A^{b} -> @(dGamma01)$,repeat=True)                       # cdb(term3.303,term3)
   substitute (term3,$\partial_{c}{\Gamma^{a}_{d b}}A^{c}A^{b} -> @(dGamma01)$,repeat=True)                       # cdb(term3.304,term3)
   distribute (term3)                                                                                             # cdb(term3.305,term3)

   substitute (term4,$\partial_{c e f}{\Gamma^{a}_{d b}}A^{c}A^{b}A^{e}A^{f} -> @(dGamma03)$,repeat=True)         # cdb(term4.301,term4)
   substitute (term4,$\partial_{c e f}{\Gamma^{a}_{b d}}A^{c}A^{b}A^{e}A^{f} -> @(dGamma03)$,repeat=True)         # cdb(term4.302,term4)
   substitute (term4,$\partial_{c e}{\Gamma^{a}_{d b}}A^{c}A^{b}A^{e} -> @(dGamma02)$,repeat=True)                # cdb(term4.303,term4)
   substitute (term4,$\partial_{c e}{\Gamma^{a}_{b d}}A^{c}A^{b}A^{e} -> @(dGamma02)$,repeat=True)                # cdb(term4.304,term4)
   substitute (term4,$\partial_{c}{\Gamma^{a}_{b d}}A^{c}A^{b} -> @(dGamma01)$,repeat=True)                       # cdb(term4.305,term4)
   substitute (term4,$\partial_{c}{\Gamma^{a}_{d b}}A^{c}A^{b} -> @(dGamma01)$,repeat=True)                       # cdb(term4.306,term4)
   distribute (term4)                                                                                             # cdb(term4.307,term4)

   substitute (term5,$\partial_{c e f g}{\Gamma^{a}_{d b}}A^{c}A^{b}A^{e}A^{f}A^{g} -> @(dGamma04)$,repeat=True)  # cdb(term5.301,term5)
   substitute (term5,$\partial_{c e f g}{\Gamma^{a}_{b d}}A^{c}A^{b}A^{e}A^{f}A^{g} -> @(dGamma04)$,repeat=True)  # cdb(term5.302,term5)
   substitute (term5,$\partial_{c e f}{\Gamma^{a}_{d b}}A^{c}A^{b}A^{e}A^{f} -> @(dGamma03)$,repeat=True)         # cdb(term5.303,term5)
   substitute (term5,$\partial_{c e f}{\Gamma^{a}_{b d}}A^{c}A^{b}A^{e}A^{f} -> @(dGamma03)$,repeat=True)         # cdb(term5.304,term5)
   substitute (term5,$\partial_{c e}{\Gamma^{a}_{d b}}A^{c}A^{b}A^{e} -> @(dGamma02)$,repeat=True)                # cdb(term5.305,term5)
   substitute (term5,$\partial_{c e}{\Gamma^{a}_{b d}}A^{c}A^{b}A^{e} -> @(dGamma02)$,repeat=True)                # cdb(term5.306,term5)
   substitute (term5,$\partial_{c}{\Gamma^{a}_{b d}}A^{c}A^{b} -> @(dGamma01)$,repeat=True)                       # cdb(term5.307,term5)
   substitute (term5,$\partial_{c}{\Gamma^{a}_{d b}}A^{c}A^{b} -> @(dGamma01)$,repeat=True)                       # cdb(term5.308,term5)
   distribute (term5)                                                                                             # cdb(term5.309,term5)

   # ------------------------------------------------------------------------------------
   # this block only produces formatted output, it is not part of the main computation
   # ------------------------------------------------------------------------------------

   # the metric in terms of partial derivatives of Rabcd

   metric:=@(term0)
         + (1/1) @(term1)
         + (1/2) @(term2)
         + (1/6) @(term3)
         + (1/24) @(term4)
         + (1/120) @(term5).  # cdb(metric.301,metric)

   substitute (metric,$A^{a} -> x^{a}$)  # cdb (metric.302,metric)

   # reformat and tidy up

   Xterm0 := @(term0).
   Xterm1 := (1/1) @(term1).     # zero
   Xterm2 := (1/2) @(term2).
   Xterm3 := (1/6) @(term3).
   Xterm4 := (1/24) @(term4).
   Xterm5 := (1/120) @(term5).

   substitute (Xterm0,$A^{a} -> x^{a}$)
   substitute (Xterm1,$A^{a} -> x^{a}$)
   substitute (Xterm2,$A^{a} -> x^{a}$)
   substitute (Xterm3,$A^{a} -> x^{a}$)
   substitute (Xterm4,$A^{a} -> x^{a}$)
   substitute (Xterm5,$A^{a} -> x^{a}$)

   # Manipulating these expressions is hampered by the presence of the partial derivative on Rabcd.
   # Thus we can't freely rasie/lower indices on the dRabcd terms. But we can do so on the first
   # derivatives (since these are evaluated at x=0 where the connection vanishes).

   substitute       (Xterm2,$g^{a b} R^{c}_{d e b} -> R^{c}_{d e}^{a}}$)  # cdb(Xterm2.301,Xterm2)
   substitute       (Xterm3,$g^{a b} R^{c}_{d e b} -> R^{c}_{d e}^{a}}$)  # cdb(Xterm3.301,Xterm3)
   substitute       (Xterm4,$g^{a b} R^{c}_{d e b} -> R^{c}_{d e}^{a}}$)  # cdb(Xterm4.301,Xterm4)
   substitute       (Xterm5,$g^{a b} R^{c}_{d e b} -> R^{c}_{d e}^{a}}$)  # cdb(Xterm5.301,Xterm5)

   substitute       (Xterm2,$g^{b a} R^{c}_{d e b} -> R^{c}_{d e}^{a}}$)  # cdb(Xterm2.302,Xterm2)
   substitute       (Xterm3,$g^{b a} R^{c}_{d e b} -> R^{c}_{d e}^{a}}$)  # cdb(Xterm3.302,Xterm3)
   substitute       (Xterm4,$g^{b a} R^{c}_{d e b} -> R^{c}_{d e}^{a}}$)  # cdb(Xterm4.302,Xterm4)
   substitute       (Xterm5,$g^{b a} R^{c}_{d e b} -> R^{c}_{d e}^{a}}$)  # cdb(Xterm5.302,Xterm5)

   substitute       (Xterm2,$g^{a b} \partial_{c}{R^{d}_{e f b}} -> \partial_{c}{R^{d}_{e f}^{a}$)  # cdb(Xterm2.303,Xterm2)
   substitute       (Xterm3,$g^{a b} \partial_{c}{R^{d}_{e f b}} -> \partial_{c}{R^{d}_{e f}^{a}$)  # cdb(Xterm3.303,Xterm3)
   substitute       (Xterm4,$g^{a b} \partial_{c}{R^{d}_{e f b}} -> \partial_{c}{R^{d}_{e f}^{a}$)  # cdb(Xterm4.303,Xterm4)
   substitute       (Xterm5,$g^{a b} \partial_{c}{R^{d}_{e f b}} -> \partial_{c}{R^{d}_{e f}^{a}$)  # cdb(Xterm5.303,Xterm5)

   substitute       (Xterm2,$g^{b a} \partial_{c}{R^{d}_{e f b}} -> \partial_{c}{R^{d}_{e f}^{a}$)  # cdb(Xterm2.304,Xterm2)
   substitute       (Xterm3,$g^{b a} \partial_{c}{R^{d}_{e f b}} -> \partial_{c}{R^{d}_{e f}^{a}$)  # cdb(Xterm3.304,Xterm3)
   substitute       (Xterm4,$g^{b a} \partial_{c}{R^{d}_{e f b}} -> \partial_{c}{R^{d}_{e f}^{a}$)  # cdb(Xterm4.304,Xterm4)
   substitute       (Xterm5,$g^{b a} \partial_{c}{R^{d}_{e f b}} -> \partial_{c}{R^{d}_{e f}^{a}$)  # cdb(Xterm5.304,Xterm5)

   sort_product     (Xterm2)  # cdb(Xterm2.305,Xterm2)
   sort_product     (Xterm3)  # cdb(Xterm3.305,Xterm3)
   sort_product     (Xterm4)  # cdb(Xterm4.305,Xterm4)
   sort_product     (Xterm5)  # cdb(Xterm5.305,Xterm5)

   rename_dummies   (Xterm2)  # cdb(Xterm2.306,Xterm2)
   rename_dummies   (Xterm3)  # cdb(Xterm3.306,Xterm3)
   rename_dummies   (Xterm4)  # cdb(Xterm4.306,Xterm4)
   rename_dummies   (Xterm5)  # cdb(Xterm5.306,Xterm5)

   canonicalise     (Xterm2)  # cdb(Xterm2.307,Xterm2)
   canonicalise     (Xterm3)  # cdb(Xterm3.307,Xterm3)
   canonicalise     (Xterm4)  # cdb(Xterm4.307,Xterm4)
   canonicalise     (Xterm5)  # cdb(Xterm5.307,Xterm5)

   # We can simplify Xterm2 and Xterm3 by careful juggling of the indices (swapping free indices on selected terms)

   tmp = add_tags (Xterm2,'\\mu')       # cdb (tmp.001,tmp)
   zoom (tmp, $\mu_{1} Q??$)            # cdb (tmp.002,tmp)
   substitute (tmp, $R^{b}_{c d}^{a} x^{c} x^{d} -> R^{a}_{c d}^{b} x^{c} x^{d}$)  # cdb (tmp.003,tmp)
   unzoom (tmp)
   Xterm2 = clear_tags (tmp,'\\mu')     # cdb (Xterm2.401,Xterm2)

   tmp = add_tags (Xterm3,'\\mu')       # cdb (tmp.011,tmp)
   zoom (tmp, $\mu_{1} Q??$)            # cdb (tmp.012,tmp)
   substitute (tmp, $\partial_{c}{R^{b}_{d e}^{a}} x^{c} x^{d} x^{e} -> \partial_{c}{R^{a}_{d e}^{b}} x^{c} x^{d} x^{e}$)  # cdb (tmp.013,tmp)
   unzoom (tmp)
   Xterm3 = clear_tags (tmp,'\\mu')     # cdb (Xterm3.401,Xterm3)

   Xterm0 = reformat_xterm (Xterm0,  1)    # cdb(Xterm0.308,Xterm0)
   Xterm2 = reformat_xterm (Xterm2,  3)    # cdb(Xterm2.308,Xterm2)
   Xterm3 = reformat_xterm (Xterm3,  6)    # cdb(Xterm3.308,Xterm3)
   Xterm4 = reformat_xterm (Xterm4,360)    # cdb(Xterm4.308,Xterm4)
   Xterm5 = reformat_xterm (Xterm5,360)    # cdb(Xterm5.308,Xterm5)

   # metric to 4th and 6th order terms in powers of x^a

   Metric3 := @(Xterm0) + @(Xterm2).                                      # cdb (Metric3.301,Metric3)
   Metric4 := @(Xterm0) + @(Xterm2) + @(Xterm3).                          # cdb (Metric4.301,Metric4)
   Metric5 := @(Xterm0) + @(Xterm2) + @(Xterm3) + @(Xterm4).              # cdb (Metric5.301,Metric5)
   Metric6 := @(Xterm0) + @(Xterm2) + @(Xterm3) + @(Xterm4) + @(Xterm5).  # cdb (Metric6.301,Metric6)

   # ------------------------------------------------------------------------------------
   # end of format block
   # ------------------------------------------------------------------------------------

   end_stage_2 = time.time()
\end{cadabra}

\clearpage

\begin{dgroup*}
   \begin{dmath*} \cdb*{term2.301} \end{dmath*}
   \begin{dmath*} \cdb*{term2.302} \end{dmath*}
   \begin{dmath*} \cdb*{term2.303} \end{dmath*}
\end{dgroup*}

\begin{dgroup*}
   \begin{dmath*} \cdb*{term3.301} \end{dmath*}
   \begin{dmath*} \cdb*{term3.302} \end{dmath*}
   \begin{dmath*} \cdb*{term3.303} \end{dmath*}
   \begin{dmath*} \cdb*{term3.304} \end{dmath*}
   \begin{dmath*} \cdb*{term3.305} \end{dmath*}
\end{dgroup*}

\begin{dgroup*}
   \begin{dmath*} \cdb*{term4.301} \end{dmath*}
   \begin{dmath*} \cdb*{term4.302} \end{dmath*}
   \begin{dmath*} \cdb*{term4.303} \end{dmath*}
   \begin{dmath*} \cdb*{term4.304} \end{dmath*}
   \begin{dmath*} \cdb*{term4.305} \end{dmath*}
   \begin{dmath*} \cdb*{term4.306} \end{dmath*}
   \begin{dmath*} \cdb*{term4.307} \end{dmath*}
\end{dgroup*}

\clearpage

\begin{dgroup*}
   \begin{dmath*} g^{ab}(x) = \cdb{Metric3.301} \end{dmath*}
   \begin{dmath*} g^{ab}(x) = \cdb{Metric4.301} \end{dmath*}
   \begin{dmath*} g^{ab}(x) = \cdb{Metric5.301} \end{dmath*}
   \begin{dmath*} g^{ab}(x) = \cdb{Metric6.301} \end{dmath*}
\end{dgroup*}

\clearpage

% =================================================================================================
\section*{Stage 3: Replace partial derivs of $R$ with covariant derivs of $R$}

\begin{cadabra}
   beg_stage_3 = time.time()

   # now convert partial derivs of Rabcd to covariant derivs

   dRabcd01 = cdblib.get ('dRabcd01','dRabcd.json')  # cdb(dRabcd01.400,dRabcd01)
   dRabcd02 = cdblib.get ('dRabcd02','dRabcd.json')  # cdb(dRabcd02.400,dRabcd02)
   dRabcd03 = cdblib.get ('dRabcd03','dRabcd.json')  # cdb(dRabcd03.400,dRabcd03)

   # term1 & term2 need no special care, just a bit of tidying

   eliminate_metric (term1)   # cdb(term1.401,term1)
   sort_product     (term1)   # cdb(term1.402,term1)
   rename_dummies   (term1)   # cdb(term1.403,term1)
   canonicalise     (term1)   # cdb(term1.404,term1)

   eliminate_metric (term2)   # cdb(term2.401,term2)
   sort_product     (term2)   # cdb(term2.402,term2)
   rename_dummies   (term2)   # cdb(term2.403,term2)
   canonicalise     (term2)   # cdb(term2.404,term2)

   # replace partial derivatives of Riemann tensor in term3, term4 etc. with covariant derivatives of Rabcd

   tmp01 := @(dRabcd01).      # cdb(tmp01.403,tmp01)
   tmp02 := @(dRabcd02).      # cdb(tmp02.403,tmp02)
   tmp03 := @(dRabcd03).      # cdb(tmp03.403,tmp03)

   substitute (term3,$A^{c}A^{d}A^{e}\partial_{e}{R^{a}_{c d b}} ->   @(tmp01)$,repeat=True)         # cdb(term3.401,term3)
   substitute (term3,$A^{c}A^{d}A^{e}\partial_{e}{R^{a}_{c b d}} -> - @(tmp01)$,repeat=True)         # cdb(term3.402,term3)
   distribute (term3)                                                                                # cdb(term3.403,term3)

   substitute (term4,$A^{c}A^{d}A^{e}A^{f}\partial_{e f}{R^{a}_{c d b}} ->   @(tmp02)$,repeat=True)  # cdb(term4.401,term4)
   substitute (term4,$A^{c}A^{d}A^{e}A^{f}\partial_{e f}{R^{a}_{c b d}} -> - @(tmp02)$,repeat=True)  # cdb(term4.402,term4)
   substitute (term4,$A^{c}A^{d}A^{e}\partial_{e}{R^{a}_{c d b}} ->   @(tmp01)$,repeat=True)         # cdb(term4.403,term4)
   substitute (term4,$A^{c}A^{d}A^{e}\partial_{e}{R^{a}_{c b d}} -> - @(tmp01)$,repeat=True)         # cdb(term4.404,term4)
   distribute (term4)                                                                                # cdb(term4.405,term4)

   substitute (term5,$A^{c}A^{d}A^{e}A^{f}A^{g}\partial_{e f g}{R^{a}_{c d b}} ->   @(tmp03)$,repeat=True)
   substitute (term5,$A^{c}A^{d}A^{e}A^{f}A^{g}\partial_{e f g}{R^{a}_{c b d}} -> - @(tmp03)$,repeat=True)
   substitute (term5,$A^{c}A^{d}A^{e}A^{f}\partial_{e f}{R^{a}_{c d b}} ->   @(tmp02)$,repeat=True)
   substitute (term5,$A^{c}A^{d}A^{e}A^{f}\partial_{e f}{R^{a}_{c b d}} -> - @(tmp02)$,repeat=True)
   substitute (term5,$A^{c}A^{d}A^{e}\partial_{e}{R^{a}_{c d b}} ->   @(tmp01)$,repeat=True)
   substitute (term5,$A^{c}A^{d}A^{e}\partial_{e}{R^{a}_{c b d}} -> - @(tmp01)$,repeat=True)
   distribute (term5)

   end_stage_3 = time.time()
\end{cadabra}

\begin{dgroup*}
   \begin{dmath*} \cdb*{tmp01.403} \end{dmath*}
   \begin{dmath*} \cdb*{tmp02.403} \end{dmath*}
   \begin{dmath*} \cdb*{tmp03.403} \end{dmath*}
\end{dgroup*}

\clearpage

\begin{dgroup*}
   \begin{dmath*} \cdb*{term1.401} \end{dmath*}
   \begin{dmath*} \cdb*{term1.402} \end{dmath*}
   \begin{dmath*} \cdb*{term1.403} \end{dmath*}
   \begin{dmath*} \cdb*{term1.404} \end{dmath*}
\end{dgroup*}

\begin{dgroup*}
   \begin{dmath*} \cdb*{term2.401} \end{dmath*}
   \begin{dmath*} \cdb*{term2.402} \end{dmath*}
   \begin{dmath*} \cdb*{term2.403} \end{dmath*}
   \begin{dmath*} \cdb*{term2.404} \end{dmath*}
\end{dgroup*}

\begin{dgroup*}
   \begin{dmath*} \cdb*{term3.401} \end{dmath*}
   \begin{dmath*} \cdb*{term3.402} \end{dmath*}
   \begin{dmath*} \cdb*{term3.403} \end{dmath*}
\end{dgroup*}

\begin{dgroup*}
   \begin{dmath*} \cdb*{term4.401} \end{dmath*}
   \begin{dmath*} \cdb*{term4.402} \end{dmath*}
   \begin{dmath*} \cdb*{term4.403} \end{dmath*}
   \begin{dmath*} \cdb*{term4.404} \end{dmath*}
   \begin{dmath*} \cdb*{term4.405} \end{dmath*}
\end{dgroup*}

\clearpage

% =================================================================================================
\section*{Stage 4: Build the Taylor series for $g_{ab}$, reformatting and output}

\begin{cadabra}
   beg_stage_4 = time.time()

   # final housekeeping

   # lower the ^{ab} indices to _{uv}

   tmp0 := g_{a u} g_{b v} @(term0).
   tmp1 := g_{a u} g_{b v} @(term1).
   tmp2 := g_{a u} g_{b v} @(term2).
   tmp3 := g_{a u} g_{b v} @(term3).
   tmp4 := g_{a u} g_{b v} @(term4).
   tmp5 := g_{a u} g_{b v} @(term5).

   distribute           (tmp1)  # cdb(tmp1.501,tmp1)
   eliminate_metric     (tmp1)  # cdb(tmp1.502,tmp1)
   eliminate_kronecker  (tmp1)  # cdb(tmp1.503,tmp1)
   tmp1 = flatten_Rabcd (tmp1)
   canonicalise         (tmp1)  # cdb(tmp1.506,tmp1)

   distribute           (tmp2)  # cdb(tmp2.501,tmp2)
   eliminate_metric     (tmp2)  # cdb(tmp2.502,tmp2)
   eliminate_kronecker  (tmp2)  # cdb(tmp2.503,tmp2)
   tmp2 = flatten_Rabcd (tmp2)
   canonicalise         (tmp2)  # cdb(tmp2.506,tmp2)

   distribute           (tmp3)  # cdb(tmp3.501,tmp3)
   eliminate_metric     (tmp3)  # cdb(tmp3.502,tmp3)
   eliminate_kronecker  (tmp3)  # cdb(tmp3.503,tmp3)
   tmp3 = flatten_Rabcd (tmp3)
   canonicalise         (tmp3)  # cdb(tmp3.506,tmp3)

   distribute           (tmp4)  # cdb(tmp4.501,tmp4)
   eliminate_metric     (tmp4)  # cdb(tmp4.502,tmp4)
   eliminate_kronecker  (tmp4)  # cdb(tmp4.503,tmp4)
   tmp4 = flatten_Rabcd (tmp4)
   canonicalise         (tmp4)  # cdb(tmp4.506,tmp4)

   distribute           (tmp5)  # cdb(tmp5.501,tmp5)
   eliminate_metric     (tmp5)  # cdb(tmp5.502,tmp5)
   eliminate_kronecker  (tmp5)  # cdb(tmp5.503,tmp5)
   tmp5 = flatten_Rabcd (tmp5)
   canonicalise         (tmp5)  # cdb(tmp5.506,tmp5)

   # this is out final answer

   # raise the _{uv} indices to ^{ab}

   metric:= g^{a u} g^{b v} (  @(tmp0)
                              + (1/1) @(tmp1)
                              + (1/2) @(tmp2)
                              + (1/6) @(tmp3)
                              + (1/24) @(tmp4)
                              + (1/120) @(tmp5) ).  # cdb(metric.500,metric)

   distribute             (metric)   # cdb(metric.501,metric)
   eliminate_metric       (metric)   # cdb(metric.502,metric)
   eliminate_kronecker    (metric)   # cdb(metric.503,metric)
   metric = flatten_Rabcd (metric)   # cdb(metric.504,metric)
   canonicalise           (metric)   # cdb(metric.505,metric)

   substitute          (metric,$g_{a b} g^{b c} -> g_{a}^{c}$)
   substitute          (metric,$g_{b a} g^{b c} -> g_{a}^{c}$)
   substitute          (metric,$g_{b a} g^{c b} -> g_{a}^{c}$)
   substitute          (metric,$g_{a b} g^{c b} -> g_{a}^{c}$)
   eliminate_kronecker (metric)   # cdb(metric.506,metric)
   canonicalise        (metric)   # cdb(metric.507,metric)

   substitute (metric,$A^{a} -> x^{a}$)  # cdb (metric.508,metric)

   cdblib.create ('metric-inv.json')

   cdblib.put ('g^ab',metric,'metric-inv.json')

   # extract the terms of the metric in powers of x

   term0 = get_xterm (metric,0)   # cdb(term0.501,term0)
   term1 = get_xterm (metric,1)   # cdb(term1.501,term1)
   term2 = get_xterm (metric,2)   # cdb(term2.501,term2)
   term3 = get_xterm (metric,3)   # cdb(term3.501,term3)
   term4 = get_xterm (metric,4)   # cdb(term4.501,term4)
   term5 = get_xterm (metric,5)   # cdb(term5.501,term5)

   cdblib.put ('g^ab_0',term0,'metric-inv.json')
   cdblib.put ('g^ab_1',term1,'metric-inv.json')
   cdblib.put ('g^ab_2',term2,'metric-inv.json')
   cdblib.put ('g^ab_3',term3,'metric-inv.json')
   cdblib.put ('g^ab_4',term4,'metric-inv.json')
   cdblib.put ('g^ab_5',term5,'metric-inv.json')

   # this version of "metric" is used only in the commentary at the start of this notebook

   metric4:=@(term0) + @(term1) + @(term2) + @(term3).  # cdb(metric4.501,metric4)

   # these versions of "metric" are created just to add to the metric.json library
   # note: term1 = 0, I could have used this fact above but ...

   metric2:=@(term0) + @(term2).
   metric3:=@(term0) + @(term2) + @(term3).
   metric4:=@(term0) + @(term2) + @(term3) + @(term4).
   metric5:=@(term0) + @(term2) + @(term3) + @(term4) + @(term5).

   cdblib.put ('g^ab2',metric2,'metric-inv.json')
   cdblib.put ('g^ab3',metric3,'metric-inv.json')
   cdblib.put ('g^ab4',metric4,'metric-inv.json')
   cdblib.put ('g^ab5',metric5,'metric-inv.json')
\end{cadabra}

\clearpage

\begin{dgroup*}
   \begin{dmath*} \cdb*{term0.501} \end{dmath*}
   \begin{dmath*} \cdb*{term1.501} \end{dmath*}
   \begin{dmath*} \cdb*{term2.501} \end{dmath*}
   \begin{dmath*} \cdb*{term3.501} \end{dmath*}
   \begin{dmath*} \cdb*{term4.501} \end{dmath*}
   \begin{dmath*} \cdb*{term5.501} \end{dmath*}
\end{dgroup*}

\clearpage

\begin{dgroup*}
   \begin{dmath*} \cdb*{tmp2.501} \end{dmath*}
   \begin{dmath*} \cdb*{tmp2.502} \end{dmath*}
   \begin{dmath*} \cdb*{tmp2.503} \end{dmath*}
   \begin{dmath*} \cdb*{tmp2.506} \end{dmath*}
\end{dgroup*}

\begin{dgroup*}
   \begin{dmath*} \cdb*{tmp3.501} \end{dmath*}
   \begin{dmath*} \cdb*{tmp3.502} \end{dmath*}
   \begin{dmath*} \cdb*{tmp3.503} \end{dmath*}
   \begin{dmath*} \cdb*{tmp3.506} \end{dmath*}
\end{dgroup*}

\clearpage

\begin{dgroup*}
   \begin{dmath*} \cdb*{tmp4.501} \end{dmath*}
   \begin{dmath*} \cdb*{tmp4.502} \end{dmath*}
   \begin{dmath*} \cdb*{tmp4.503} \end{dmath*}
   \begin{dmath*} \cdb*{tmp4.506} \end{dmath*}
\end{dgroup*}

\clearpage

\begin{dgroup*}
   \begin{dmath*} \cdb*{tmp5.506} \end{dmath*}
\end{dgroup*}

\clearpage

\begin{dgroup*}
   \begin{dmath*} \cdb*{metric.500} \end{dmath*}
   \begin{dmath*} \cdb*{metric.501} \end{dmath*}
   \begin{dmath*} \cdb*{metric.502} \end{dmath*}
   \begin{dmath*} \cdb*{metric.503} \end{dmath*}
   \begin{dmath*} \cdb*{metric.504} \end{dmath*}
   \begin{dmath*} \cdb*{metric.505} \end{dmath*}
   \begin{dmath*} \cdb*{metric.506} \end{dmath*}
   \begin{dmath*} \cdb*{metric.507} \end{dmath*}
   \begin{dmath*} \cdb*{metric.508} \end{dmath*}
\end{dgroup*}

% =================================================================================================
% the remaining code is just for pretty printing

\clearpage

\begin{cadabra}
   Rterm0 := @(term0).
   Rterm1 := @(term1).  # zero
   Rterm2 := @(term2).
   Rterm3 := @(term3).
   Rterm4 := @(term4).
   Rterm5 := @(term5).

   Rterm0 = reformat_xterm (Rterm0,  1)    # cdb(Rterm0.601,Rterm0)
   Rterm2 = reformat_xterm (Rterm2,  3)    # cdb(Rterm2.601,Rterm2)
   Rterm3 = reformat_xterm (Rterm3,  6)    # cdb(Rterm3.601,Rterm3)
   Rterm4 = reformat_xterm (Rterm4, 60)    # cdb(Rterm4.601,Rterm4)
   Rterm5 = reformat_xterm (Rterm5, 90)    # cdb(Rterm5.601,Rterm5)

   Metric := @(Rterm0) + @(Rterm2) + @(Rterm3) + @(Rterm4) + @(Rterm5).  # cdb (Metric.601,Metric)

   scaled0 = rescale_xterm (Rterm0,  1)    # cdb(scaled0.601,scaled0)
   scaled2 = rescale_xterm (Rterm2,  3)    # cdb(scaled2.601,scaled2)
   scaled3 = rescale_xterm (Rterm3,  6)    # cdb(scaled3.601,scaled3)
   scaled4 = rescale_xterm (Rterm4, 60)    # cdb(scaled4.601,scaled4)
   scaled5 = rescale_xterm (Rterm5, 90)    # cdb(scaled5.601,scaled5)

   end_stage_4 = time.time()
\end{cadabra}

\clearpage

% =================================================================================================
\section*{The inverse metric in Riemann normal coordinates}

\begin{dgroup*}
   \begin{dmath*} g^{a b}(x) = \cdb{Metric.601}+\BigO{\eps^6} \end{dmath*}
\end{dgroup*}

\clearpage

% =================================================================================================
\section*{Curvature expansion of the inverse metric}
\begin{align*}
     g^{a b}(x) =
     \ngab{0}^{a b}
   + \ngab{2}^{a b}
   + \ngab{3}^{a b}
   + \ngab{4}^{a b}
   + \ngab{5}^{a b}+\BigO{\eps^6}
\end{align*}
\begin{dgroup*}
   \begin{dmath*}    \ngab{0}^{a b} = \cdb{scaled0.601} \end{dmath*}
   \begin{dmath*}  3 \ngab{2}^{a b} = \cdb{scaled2.601} \end{dmath*}
   \begin{dmath*}  6 \ngab{3}^{a b} = \cdb{scaled3.601} \end{dmath*}
   \begin{dmath*} 60 \ngab{4}^{a b} = \cdb{scaled4.601} \end{dmath*}
   \begin{dmath*} 90 \ngab{5}^{a b} = \cdb{scaled5.601} \end{dmath*}
\end{dgroup*}

\clearpage

% =================================================================================================
% export selected objects, these will later be imported into a library
% these are the objects that will appear in the paper

\begin{cadabra}
   cdblib.create ('metric-inv.export')

   cdblib.put ('g^ab_3',Metric3,'metric-inv.export')  # R and \partial R
   cdblib.put ('g^ab_4',Metric4,'metric-inv.export')
   cdblib.put ('g^ab_5',Metric5,'metric-inv.export')
   cdblib.put ('g^ab_6',Metric6,'metric-inv.export')

   cdblib.put ('g^ab',  Metric, 'metric-inv.export')  # R and \nabla R

   cdblib.put ('g^ab_scaled0',scaled0,'metric-inv.export')
   cdblib.put ('g^ab_scaled2',scaled2,'metric-inv.export')
   cdblib.put ('g^ab_scaled3',scaled3,'metric-inv.export')
   cdblib.put ('g^ab_scaled4',scaled4,'metric-inv.export')
   cdblib.put ('g^ab_scaled5',scaled5,'metric-inv.export')

   checkpoint.append (Metric4)
   checkpoint.append (Metric6)

   checkpoint.append (Metric)

   checkpoint.append (scaled0)
   checkpoint.append (scaled2)
   checkpoint.append (scaled3)
   checkpoint.append (scaled4)
   checkpoint.append (scaled5)

   # cdbBeg (timing)
   print ("Stage 1: {:7.1f} secs\\hfill\\break".format(end_stage_1-beg_stage_1))
   print ("Stage 2: {:7.1f} secs\\hfill\\break".format(end_stage_2-beg_stage_2))
   print ("Stage 3: {:7.1f} secs\\hfill\\break".format(end_stage_3-beg_stage_3))
   print ("Stage 4: {:7.1f} secs".format(end_stage_4-beg_stage_4))
   # cdbEnd (timing)

\end{cadabra}

\clearpage

% =================================================================================================
\section*{Timing}

\cdb{timing}

% =================================================================================================
% export checkpoints in json format

\bgroup
\CdbSetup{action=hide}
\begin{cadabra}
   for i in range( len(checkpoint) ):
      cdblib.put ('check{:03d}'.format(i),checkpoint[i],checkpoint_file)
\end{cadabra}
\egroup

\end{document}


\begin{dgroup*}
   \begin{dmath*} g^{a b}(x) = \cdb{Metric.601}+\BigO{\eps^6} \end{dmath*}
\end{dgroup*}

% =================================================================================================
\section*{Curvature expansion of the inverse metric}
\begin{align*}
     g^{a b}(x) =
     \ngab{0}^{a b}
   + \ngab{2}^{a b}
   + \ngab{3}^{a b}
   + \ngab{4}^{a b}
   + \ngab{5}^{a b}+\BigO{\eps^6}
\end{align*}
\begin{dgroup*}
   \begin{dmath*}    \ngab{0}^{a b} = \cdb{scaled0.601} \end{dmath*}
   \begin{dmath*}  3 \ngab{2}^{a b} = \cdb{scaled2.601} \end{dmath*}
   \begin{dmath*}  6 \ngab{3}^{a b} = \cdb{scaled3.601} \end{dmath*}
   \begin{dmath*} 60 \ngab{4}^{a b} = \cdb{scaled4.601} \end{dmath*}
   \begin{dmath*} 90 \ngab{5}^{a b} = \cdb{scaled5.601} \end{dmath*}
\end{dgroup*}

\clearpage

% =================================================================================================
\section*{The metric determinant in RNC}
\documentclass[12pt]{cdblatex}

\begin{document}

% =================================================================================================
% create checkpoint file

\bgroup
\CdbSetup{action=hide}
\begin{cadabra}
   import cdblib
   checkpoint_file = 'tests/semantic/output/detg2.json'
   cdblib.create (checkpoint_file)
   checkpoint = []
\end{cadabra}
\egroup

% =================================================================================================
\section*{The determinant of the metric}

Our game here is to compute (the leading terms) in $\det g$ of the metric in RNC form
\begin{dgroup*}
   \begin{dmath*} g_{a b}(x) = \cdb{gab.001}+\BigO{\eps^5} \end{dmath*}
\end{dgroup*}
For the sake of simplicity let's assume that we are working in 3-dimensions. The following
analysis is easily generalsied to other dimensions (and the final answers for $\det g$ and
friends are unchanged).

Define $\eps^{abc}_{ijk}$ by
\begin{align}
   \eps^{abc}_{ijk} =
        \delta^a_i \delta^b_j \delta^c_k - \delta^b_i \delta^a_j \delta^c_k
      + \delta^c_i \delta^a_j \delta^b_k - \delta^c_i \delta^b_j \delta^a_k
      + \delta^b_i \delta^c_j \delta^a_k - \delta^a_i \delta^c_j \delta^b_k
\end{align}
It is easy to see that $\eps^{abc}_{ijk}$ is anti-symmetric in both its upper and lower
indices. A trivial computation shows that for any $3{}\times{}3$ square matrix $M_{ab}$,
\begin{align}
   \eps^{abc}_{123} M_{1a} M_{2b} M_{3c}
   = \left(
          \delta^a_1 \delta^b_2 \delta^c_3 - \delta^b_1 \delta^a_2 \delta^c_3
        + \delta^c_1 \delta^a_2 \delta^b_3 - \delta^c_1 \delta^b_2 \delta^a_3
        + \delta^b_1 \delta^c_2 \delta^a_3 - \delta^a_1 \delta^c_2 \delta^b_3
     \right)M_{1a} M_{2b} M_{3c}
   = \det M
\end{align}
This can be easily generalised to
\begin{align}
   \eps^{abc}_{ijk} M_{pa} M_{qb} M_{rc}
   =
   \begin{cases}
      \pm \det M &\text{when $(ijk)$ and $(pqr)$ are permutations of $(123)$}\\
      0 & \text{otherwise}
   \end{cases}
\end{align}
The $\pm$ sign in the above depends on the particular permutations of $(ijk)$ and $(pqr)$. If
both permutations are even or both odd then the sign is $+1$ otherwise the sign is $-1$.
The same arguments can also be applied to a matrix inverse $N^{-1}$ leading to
\begin{align}
   \eps^{ijk}_{uvw} N^{pu} N^{qv} M^{rw}
   =
   \begin{cases}
      \pm \det {N^{-1}} &\text{when $(ijk)$ and $(pqr)$ are permutations of $(123)$}\\
      0 & \text{otherwise}
   \end{cases}
\end{align}
Note that the $\pm$ in this case will match exactly that for the case of $\det M$. Thus,
multiplying both expressions and summing over all choices for $(ijk)$ and $(pqr)$ leads
to
\begin{align}
   \sum_{\substack{(ijk)\\(pqr)}}\left(\det N^{-1}\right) \det M
   = \eps^{ijk}_{uvw} N^{pu} N^{qv} M^{rw} \eps^{abc}_{ijk} M_{pa} M_{qb} M_{rc}
\end{align}
where the sum on the left hand side includes just those $(ijk)$ and $(prq)$ that are
permutations of $(123)$. There are $3!$ choices for $(ijk)$ and $3!$ choices for
$(pqr)$ and thus the left hand side is easily reduced to $(3!)^2 \det M/\det N$ where
$\det N = 1/\det N^{-1}$. For the right hand side notice that
\begin{align}
   \eps^{ijk}_{uvw} \eps^{abc}_{ijk} = 3! \eps^{abc}_{uvw}
\end{align}
which leads to
\begin{align}
   \det M = \frac{1}{3!} \det N \eps^{abc}_{uvw} M_{pa} M_{qb} M_{rc} N^{pu} N^{qv} N^{rw}
\end{align}

For our RNC metric we will set $N^{ab} = g^{ab}$ and $M_{ij} = g_{ij}(x)$. Since $g^{ab}$ is
of the form ${\rm diag}(-1,1,1,1)$ we have $\det g = -1$ and thus
\begin{align}
   \det g(x) = - \frac{1}{3!} \eps^{abc}_{ijk}\, g_{pa}(x)\, g_{qb}(x)\, g_{rc}(x)\, g^{ip} g^{jq} g^{kr}
\end{align}

The $\eps^{abc}_{ijk}$ can be constructed in Cadabra by applying the \verb|asym| algorithm
to the upper indices of $\delta^a_i \delta^b_j \delta^c_k$. Note that \verb|asym| will
include the $1/3!$ coeffcient as part of its output.

The following code computes $-\det g$ rather than $\det g$.

{\bf Note} that Calzetta etal. use an opposite sign for $R_{abcd}$ so when comparing the
following results against Calzetta do take note of this flipped sign in $R_{abcd}$.

\clearpage

\begin{cadabra}
   {a,b,c,d,e,f,g,h,i,j,k,l,m,n,o,p,q,r,s,t,u,v,w#}::Indices(position=independent).

   {a,b,c,d,e,f,g,h,i,j,k,l,m,n,o,p,q,r,s,t,u,v,w#}::Integer(1..2).

   \nabla{#}::Derivative.

   d{#}::KroneckerDelta.

   g^{a b}::Symmetric.
   g_{a b}::Symmetric.

   R_{a b c d}::RiemannTensor.

   x^{a}::Weight(label=numx,value=1).

   def truncate (obj,n):

       ans = Ex(0)

       for i in range (0,n+1):
          foo := @(obj).
          bah = Ex("numx = " + str(i))
          keep_weight (foo, bah)
          ans = ans + foo

       return ans

   import cdblib

   g0ab = cdblib.get('g_ab_0','metric.json')
   g1ab = cdblib.get('g_ab_1','metric.json')  # zero in RNC
   g2ab = cdblib.get('g_ab_2','metric.json')
   g3ab = cdblib.get('g_ab_3','metric.json')
   g4ab = cdblib.get('g_ab_4','metric.json')
   g5ab = cdblib.get('g_ab_5','metric.json')

   gab := @(g0ab) + @(g1ab) + @(g2ab) + @(g3ab) + @(g4ab) + @(g5ab).  # cdb (gab.001,gab)
   gxab := gx_{a b} -> @(gab).

   eps := d^{a}_{i} d^{b}_{j}.   # cdb(eps.001,eps)
   asym (eps,$^{a},^{b}$)        # cdb(eps.002,eps) # includes a factor of 1/2!

   # compute negative Ndetg rather than det g
   Ndetg := @(eps) gx_{p a} gx_{q b} g^{i p} g^{j q}.  # note 1/2! included in eps

   substitute       (Ndetg,gxab)
   distribute       (Ndetg)
   Ndetg = truncate (Ndetg,5)                                          # cdb (Ndetg.001,Ndetg)
   substitute       (Ndetg,$g^{a b} g_{b c} -> d^{a}_{c}$,repeat=True) # cdb (Ndetg.002,Ndetg)
   eliminate_kronecker (Ndetg)                                         # cdb (Ndetg.003,Ndetg)
   sort_product     (Ndetg)                                            # cdb (Ndetg.004,Ndetg)
   rename_dummies   (Ndetg)                                            # cdb (Ndetg.005,Ndetg)
   canonicalise     (Ndetg)                                            # cdb (Ndetg.006,Ndetg)

   # introduce the Ricci tensor

   substitute     (Ndetg,$R_{a b c d} g^{a c} -> R_{b d}$,repeat=True)                                  # cdb (Ndetg.101,Ndetg)
   substitute     (Ndetg,$\nabla_{a}{R_{b c d e}} g^{b d}  -> \nabla_{a}{R_{c e}}$,repeat=True)         # cdb (Ndetg.102,Ndetg)
   substitute     (Ndetg,$\nabla_{a b}{R_{c d e f}} g^{c e}  -> \nabla_{a b}{R_{d f}}$,repeat=True)     # cdb (Ndetg.103,Ndetg)
   substitute     (Ndetg,$\nabla_{a b c}{R_{d e f g}} g^{d f}  -> \nabla_{a b c}{R_{e g}}$,repeat=True) # cdb (Ndetg.104,Ndetg)

   # the following are based on sqrt-Ndetg.tex

   sqrtNdetg := 1/2 + (1/2) @(Ndetg)
               - (1/8) (1/9) R_{a b} R_{c d} x^{a} x^{b} x^{c} x^{d}
               - (1/4) (1/18) R_{a b} \nabla_{c}{R_{d e}} x^{a} x^{b} x^{c} x^{d} x^{e}.
               # cdb (sqrtNdetg.001,sqrtNdetg)

   sort_product   (sqrtNdetg)                                          # cdb (sqrtNdetg.002,sqrtNdetg)
   rename_dummies (sqrtNdetg)                                          # cdb (sqrtNdetg.003,sqrtNdetg)
   canonicalise   (sqrtNdetg)                                          # cdb (sqrtNdetg.004,sqrtNdetg)

   logNdetg := -1 + @(Ndetg)
               - (1/2) (1/9) R_{a b} R_{c d} x^{a} x^{b} x^{c} x^{d}
               - (1/18) R_{a b} \nabla_{c}{R_{d e}} x^{a} x^{b} x^{c} x^{d} x^{e}.
               # cdb (logNdetg.001,logNdetg)

   sort_product   (logNdetg)                                           # cdb (logNdetg.002,logNdetg)
   rename_dummies (logNdetg)                                           # cdb (logNdetg.003,logNdetg)
   canonicalise   (logNdetg)                                           # cdb (logNdetg.004,logNdetg)

\end{cadabra}

\clearpage

\begin{dgroup*}
   \begin{dmath*} \cdb*{eps.001} \end{dmath*}
   \begin{dmath*} \cdb*{eps.002} \end{dmath*}
   \begin{dmath*} \cdb*{Ndetg.001} \end{dmath*}
   \begin{dmath*} \cdb*{Ndetg.002} \end{dmath*}
   \begin{dmath*} \cdb*{Ndetg.003} \end{dmath*}
   \begin{dmath*} \cdb*{Ndetg.004} \end{dmath*}
   \begin{dmath*} \cdb*{Ndetg.005} \end{dmath*}
   \begin{dmath*} \cdb*{Ndetg.006} \end{dmath*}
   \begin{dmath*} \cdb*{Ndetg.104} \end{dmath*}
   \begin{dmath*} \cdb*{sqrtNdetg.004} \end{dmath*}
   \begin{dmath*} \cdb*{logNdetg.004} \end{dmath*}
\end{dgroup*}

% =================================================================================================
% the remaining code is just for pretty printing

\clearpage

\begin{cadabra}
   # note: keeping numbering as is (out of order) to ensure R appears before \nabla R etc.
   def product_sort (obj):
       substitute (obj,$ x^{a}                            -> A000^{a}               $)
       substitute (obj,$ g^{a b}                          -> A001^{a b}             $)
       substitute (obj,$ \nabla_{c d e f}{R_{a b}}        -> A007_{a b c d e f}     $)
       substitute (obj,$ \nabla_{c d e}{R_{a b}}          -> A006_{a b c d e}       $)
       substitute (obj,$ \nabla_{c d}{R_{a b}}            -> A005_{a b c d}         $)
       substitute (obj,$ \nabla_{c}{R_{a b}}              -> A004_{a b c}           $)
       substitute (obj,$ \nabla_{e f g h}{R_{a b c d}}    -> A011_{a b c d e f g h} $)
       substitute (obj,$ \nabla_{e f g}{R_{a b c d}}      -> A010_{a b c d e f g}   $)
       substitute (obj,$ \nabla_{e f}{R_{a b c d}}        -> A009_{a b c d e f}     $)
       substitute (obj,$ \nabla_{e}{R_{a b c d}}          -> A008_{a b c d e}       $)
       substitute (obj,$ R_{a b}                          -> A002_{a b}             $)
       substitute (obj,$ R_{a b c d}                      -> A003_{a b c d}         $)
       sort_product   (obj)
       rename_dummies (obj)
       substitute (obj,$ A000^{a}                 -> x^{a}                          $)
       substitute (obj,$ A001^{a b}               -> g^{a b}                        $)
       substitute (obj,$ A002_{a b}               -> R_{a b}                        $)
       substitute (obj,$ A003_{a b c d}           -> R_{a b c d}                    $)
       substitute (obj,$ A004_{a b c}             -> \nabla_{c}{R_{a b}}            $)
       substitute (obj,$ A005_{a b c d}           -> \nabla_{c d}{R_{a b}}          $)
       substitute (obj,$ A006_{a b c d e}         -> \nabla_{c d e}{R_{a b}}        $)
       substitute (obj,$ A007_{a b c d e f}       -> \nabla_{c d e f}{R_{a b}}      $)
       substitute (obj,$ A008_{a b c d e}         -> \nabla_{e}{R_{a b c d}}        $)
       substitute (obj,$ A009_{a b c d e f}       -> \nabla_{e f}{R_{a b c d}}      $)
       substitute (obj,$ A010_{a b c d e f g}     -> \nabla_{e f g}{R_{a b c d}}    $)
       substitute (obj,$ A011_{a b c d e f g h}   -> \nabla_{e f g h}{R_{a b c d}}  $)

       return obj

   def get_term (obj,n):

       x^{a}::Weight(label=numx).

       foo := @(obj).
       bah  = Ex("numx = " + str(n))
       keep_weight (foo,bah)

       return foo

   def reformat (obj,scale):
       foo  = Ex(str(scale))
       bah := @(foo) @(obj).
       distribute     (bah)
       bah = product_sort (bah)
       rename_dummies (bah)
       canonicalise   (bah)
       sort_sum       (bah)
       factor_out     (bah,$x^{a?}$)
       ans := @(bah) / @(foo).
       return ans

   def rescale (obj,scale):
       foo  = Ex(str(scale))
       bah := @(foo) @(obj).
       distribute  (bah)
       factor_out  (bah,$x^{a?}$)
       return bah

   # ---------------------------------------------------------------
   # reformat Ndetg

   Rterm0 = get_term (Ndetg,0)       # cdb(Rterm0.701,Rterm0)
   Rterm1 = get_term (Ndetg,1)       # cdb(Rterm1.701,Rterm1)
   Rterm2 = get_term (Ndetg,2)       # cdb(Rterm2.701,Rterm2)
   Rterm3 = get_term (Ndetg,3)       # cdb(Rterm3.701,Rterm3)
   Rterm4 = get_term (Ndetg,4)       # cdb(Rterm4.701,Rterm4)
   Rterm5 = get_term (Ndetg,5)       # cdb(Rterm5.701,Rterm5)

   Rterm0 = reformat (Rterm0,  1)    # cdb(Rterm0.702,Rterm0)
   Rterm1 = reformat (Rterm1,  1)    # cdb(Rterm1.702,Rterm1)
   Rterm2 = reformat (Rterm2,  3)    # cdb(Rterm2.702,Rterm2)
   Rterm3 = reformat (Rterm3,  6)    # cdb(Rterm3.702,Rterm3)
   Rterm4 = reformat (Rterm4,180)    # cdb(Rterm4.702,Rterm4)
   Rterm5 = reformat (Rterm5, 90)    # cdb(Rterm5.702,Rterm5)

   Ndetg := @(Rterm0) + @(Rterm1) + @(Rterm2) + @(Rterm3) + @(Rterm4) + @(Rterm5).  # cdb (Ndetg.701,Ndetg)

   # ---------------------------------------------------------------
   # reformat sqrtNdetg

   Rterm0 = get_term (sqrtNdetg,0)   # cdb(Rterm0.801,Rterm0)
   Rterm1 = get_term (sqrtNdetg,1)   # cdb(Rterm1.801,Rterm1)
   Rterm2 = get_term (sqrtNdetg,2)   # cdb(Rterm2.801,Rterm2)
   Rterm3 = get_term (sqrtNdetg,3)   # cdb(Rterm3.801,Rterm3)
   Rterm4 = get_term (sqrtNdetg,4)   # cdb(Rterm4.801,Rterm4)
   Rterm5 = get_term (sqrtNdetg,5)   # cdb(Rterm5.801,Rterm5)

   Rterm0 = reformat (Rterm0,  1)    # cdb(Rterm0.802,Rterm0)
   Rterm1 = reformat (Rterm1,  1)    # cdb(Rterm1.802,Rterm1)
   Rterm2 = reformat (Rterm2,  6)    # cdb(Rterm2.802,Rterm2)
   Rterm3 = reformat (Rterm3, 12)    # cdb(Rterm3.802,Rterm3)
   Rterm4 = reformat (Rterm4,360)    # cdb(Rterm4.802,Rterm4)
   Rterm5 = reformat (Rterm5,360)    # cdb(Rterm5.802,Rterm5)

   sqrtNdetg := @(Rterm0) + @(Rterm1) + @(Rterm2) + @(Rterm3) + @(Rterm4) + @(Rterm5).  # cdb (sqrtNdetg.801,sqrtNdetg)

   # ---------------------------------------------------------------
   # reformat logNdetg

   Rterm0 = get_term (logNdetg,0)    # cdb(Rterm0.901,Rterm0)
   Rterm1 = get_term (logNdetg,1)    # cdb(Rterm1.901,Rterm1)
   Rterm2 = get_term (logNdetg,2)    # cdb(Rterm2.901,Rterm2)
   Rterm3 = get_term (logNdetg,3)    # cdb(Rterm3.901,Rterm3)
   Rterm4 = get_term (logNdetg,4)    # cdb(Rterm4.901,Rterm4)
   Rterm5 = get_term (logNdetg,5)    # cdb(Rterm5.901,Rterm5)

   Rterm0 = reformat (Rterm0,  1)    # cdb(Rterm0.902,Rterm0)
   Rterm1 = reformat (Rterm1,  1)    # cdb(Rterm1.902,Rterm1)
   Rterm2 = reformat (Rterm2,  3)    # cdb(Rterm2.902,Rterm2)
   Rterm3 = reformat (Rterm3,  6)    # cdb(Rterm3.902,Rterm3)
   Rterm4 = reformat (Rterm4,180)    # cdb(Rterm4.902,Rterm4)
   Rterm5 = reformat (Rterm5, 90)    # cdb(Rterm5.902,Rterm5)

   logNdetg := @(Rterm0) + @(Rterm1) + @(Rterm2) + @(Rterm3) + @(Rterm4) + @(Rterm5).  # cdb (logNdetg.901,logNdetg)

\end{cadabra}

\clearpage

% =================================================================================================
\section*{The metric determinant in Riemann normal coordinates}

\begin{dgroup*}
   \Dmath*{-\det g(x) = \cdb{Ndetg.701}+\BigO{\eps^6}}
\end{dgroup*}

% =================================================================================================
\section*{The volume element in Riemann normal coordinates}

If $-\det g(x)$ is non-negative then we also have
%
\begin{dgroup*}
   \Dmath*{\sqrt{-\det g(x)} = \cdb{sqrtNdetg.801}+\BigO{\eps^6}}
\end{dgroup*}

% =================================================================================================
\section*{The log of -detg in Riemann normal coordinates}

Apart from the signs, this matches exactly the expression given by Calzetta etal. (eq. A14)

\begin{dgroup*}
   \Dmath*{\log\left(-\det g(x)\right) = \cdb{logNdetg.901}+\BigO{\eps^6}}
\end{dgroup*}

\clearpage

% =================================================================================================
% export selected objects, these will later be imported into a library
% these are the objects that will appear in the paper

\begin{cadabra}
   cdblib.create ('detg2.export')

   cdblib.put ('Ndetg',    Ndetg,    'detg2.export')
   cdblib.put ('sqrtNdetg',sqrtNdetg,'detg2.export')
   cdblib.put ('logNdetg', logNdetg, 'detg2.export')

   checkpoint.append (Ndetg)
   checkpoint.append (sqrtNdetg)
   checkpoint.append (logNdetg)

\end{cadabra}

% =================================================================================================
% export checkpoints in json format

\bgroup
\CdbSetup{action=hide}
\begin{cadabra}
   for i in range( len(checkpoint) ):
      cdblib.put ('check{:03d}'.format(i),checkpoint[i],checkpoint_file)
\end{cadabra}
\egroup

\end{document}


\begin{dgroup*}
   \begin{dmath*} -\det g(x) = \cdb{Ndetg.701}+\BigO{\eps^5} \end{dmath*}
\end{dgroup*}

% =================================================================================================
\section*{The metric Jacobian in RNC}

\begin{dgroup*}
   \begin{dmath*} \sqrt{-\det g(x)} = \cdb{sqrtNdetg.801}+\BigO{\eps^5} \end{dmath*}
\end{dgroup*}

% =================================================================================================
\section*{The log of detg in RNC}

\begin{dgroup*}
   \begin{dmath*} \log{-\det g(x)\vert} = \cdb{logNdetg.901}+\BigO{\eps^5} \end{dmath*}
\end{dgroup*}

\clearpage

% =================================================================================================
\section*{The connection in RNC}
\documentclass[12pt]{cdblatex}
\usepackage{fancyhdr}
\usepackage{footer}

\begin{document}

% =================================================================================================
% create checkpoint file

\bgroup
\CdbSetup{action=hide}
\begin{cadabra}
   import cdblib
   checkpoint_file = 'tests/semantic/output/connection.json'
   cdblib.create (checkpoint_file)
   checkpoint = []
\end{cadabra}
\egroup

% =================================================================================================
\section*{The connection}

Here we use the output from {\tt metric.tex} and {\tt metric-inv.tex} to compute the metric connection
$\Gamma^{d}_{ab}$. We use the standard metric compatible connection
\begin{align}
   \label{eqn:Gamma}
   \Gamma^{d}_{ab} = \frac{1}{2} g^{dc}\left( g_{cb,a} + g_{ac,b} - g_{ab,c} \right)
\end{align}

Since {\tt metric.tex} and {\tt metric-inv.tex} generate truncated expressions for $g_{ab}$ and
$g^{ab}$ a similar truncation must be applied to this computation of $\Gamma^{d}_{ab}$. The naive
choice is to truncate $\Gamma^{d}_{ab}$ \emph{after} it has been fully evaluated on the truncated
expersions for $g_{ab}$ and $g^{ab}$. This will work but it wastes time and memory (big time).

A better approach is to truncate $\Gamma^{d}_{ab}$ during its construction. That is, we take
careful note of how the terms in the finite series for $g_{ab}$ and $g^{ab}$ combine to produce
the terms of a particular order in the expansion of $\Gamma^{d}_{ab}$.

Suppose $g_{ab}$ and $g^{ab}$ are known to say fourth order. We can write each of these as follows
\begin{align}
   g_{ab} &= \ngab{0}_{ab} + \ngab{1}_{ab} + \ngab{2}_{ab} + \ngab{3}_{ab} + \ngab{4}_{ab}\\
   g^{ab} &= \ngab{0}^{ab} + \ngab{1}^{ab} + \ngab{2}^{ab} + \ngab{3}^{ab} + \ngab{4}^{ab}
\end{align}
where $\ngab{n}$ denotes a term of order $\BigO{\eps^n}$. A similar expansion applies for $\Gamma^{d}_{ab}$, that is
\begin{align}
   \Gamma^{d}_{ab} = \nGamma{0}^{d}_{ab}
                   + \nGamma{1}^{d}_{ab}
                   + \nGamma{2}^{d}_{ab}
                   + \nGamma{3}^{d}_{ab}
                   + \nGamma{4}^{d}_{ab}
\end{align}
After substituting these formal expansions into the equation \eqref{eqn:Gamma} and then matching
corresponing terms we obtain
\begin{align}
   \nGamma{n}^{d}_{ab}
   =
   \frac{1}{2} \sum_{i=0}^{i=n} \ngab{i}^{dc}\left( \ngab{n-i}_{cb,a} + \ngab{n-i}_{ac,b} - \ngab{n-i}_{ab,c} \right)
\end{align}

We use this equation to compute the successive terms in $\Gamma^{d}_{ab}$.

\clearpage

\begin{cadabra}
   {a,b,c,d,e,f,g,h,i,j,k,l,m,n,o,p,q,r,s,t,u,v,w#}::Indices(position=independent).

   D{#}::Derivative.
   \nabla{#}::Derivative.
   \partial{#}::PartialDerivative.

   g_{a b}::Metric.
   g^{a b}::InverseMetric.
   g_{a}^{b}::KroneckerDelta.
   g^{a}_{b}::KroneckerDelta.
   \delta^{a}_{b}::KroneckerDelta.
   \delta_{a}^{b}::KroneckerDelta.

   R_{a b c d}::RiemannTensor.
   R^{a}_{b c d}::RiemannTensor.

   x^{a}::Depends(D{#}).

   R_{a b c d}::Depends(\nabla{#}).
   R^{a}_{b c d}::Depends(\nabla{#}).

   import cdblib

   gab = cdblib.get ('g_ab','metric.json')      # cdb(gab.000,gab)
   iab = cdblib.get ('g^ab','metric-inv.json')  # cdb(iab.000,iab)

   defgab := g_{a b} -> @(gab).
   defiab := g^{a b} -> @(iab).

   dgab := D_{a}{g_{c b}} + D_{b}{g_{a c}} - D_{c}{g_{a b}}.  # cdb(dgab.001,dgab)

   substitute   (dgab,defgab)

   distribute   (dgab)              # cdb(dgab.002,dgab)
   unwrap       (dgab)              # cdb(dgab.003,dgab)
   product_rule (dgab)              # cdb(dgab.004,dgab)
   distribute   (dgab)              # cdb(dgab.005,dgab)
   substitute   (dgab,$D_{a}{x^{b}}->\delta^{b}_{a}$,repeat=True)  # cdb(dgab.006,dgab)
   eliminate_kronecker (dgab)       # cdb(dgab.007,dgab)
   sort_product   (dgab)            # cdb(dgab.008,dgab)
   rename_dummies (dgab)            # cdb(dgab.009,dgab)
   canonicalise   (dgab)            # cdb(dgab.010,dgab)

\end{cadabra}

\clearpage

\begin{dgroup*}
   \begin{dmath*} \cdb*{gab.000} \end{dmath*}
\end{dgroup*}

\begin{dgroup*}
   \begin{dmath*} \cdb*{dgab.001} \end{dmath*}
   \begin{dmath*} \cdb*{dgab.002} \end{dmath*}
   % \begin{dmath*} \cdb*{dgab.003} \end{dmath*} % too big for pdfLaTeX
   % \begin{dmath*} \cdb*{dgab.004} \end{dmath*}
   % \begin{dmath*} \cdb*{dgab.005} \end{dmath*}
   % \begin{dmath*} \cdb*{dgab.006} \end{dmath*}
   % \begin{dmath*} \cdb*{dgab.007} \end{dmath*}
   % \begin{dmath*} \cdb*{dgab.008} \end{dmath*}
   % \begin{dmath*} \cdb*{dgab.009} \end{dmath*}
   \begin{dmath*} \cdb*{dgab.010} \end{dmath*}
\end{dgroup*}

\clearpage

\begin{cadabra}
   # Note:
   # Computing Gamma directly by (1/2) iab dgab and *then* truncating to lower order
   # is not optimal. We only want the leading oder terms (to 4th order in x). But the direct
   # calculation would compute *all* terms before the truncation. This does work but it
   # is slower than the following code.
   #
   # The better approach (as adopted in this code) is to extract all of the terms of iab
   # and dgab then construct the leading order terms of Gamma (to fifth order) term by term.

   def get_Rterm (obj,n):

   # I would like to assign different weights to \nabla_{a}, \nabla_{a b}, \nabla_{a b c} etc. but no matter
   # what I do it appears that Cadabra assigns the same weight to all of these regardless of the number of subscripts.
   # It seems that the weight is assigned to the symbol \nabla alone. So I'm forced to use the following substitution trick.

       Q_{a b c d}::Weight(label=numR,value=2).
       Q_{a b c d e}::Weight(label=numR,value=3).
       Q_{a b c d e f}::Weight(label=numR,value=4).
       Q_{a b c d e f g}::Weight(label=numR,value=5).

       tmp := @(obj).

       distribute (tmp)

       substitute (tmp, $\nabla_{e f g}{R_{a b c d}} -> Q_{a b c d e f g}$)
       substitute (tmp, $\nabla_{e f}{R_{a b c d}} -> Q_{a b c d e f}$)
       substitute (tmp, $\nabla_{e}{R_{a b c d}} -> Q_{a b c d e}$)
       substitute (tmp, $R_{a b c d} -> Q_{a b c d}$)

       foo := @(tmp).
       bah = Ex("numR = " + str(n))
       keep_weight (foo, bah)

       substitute (foo, $Q_{a b c d e f g} -> \nabla_{e f g}{R_{a b c d}}$)
       substitute (foo, $Q_{a b c d e f} -> \nabla_{e f}{R_{a b c d}}$)
       substitute (foo, $Q_{a b c d e} -> \nabla_{e}{R_{a b c d}}$)
       substitute (foo, $Q_{a b c d} -> R_{a b c d}$)

       return foo

   # terms of the curvature expansion of dg_{ab}

   dgab00 = get_Rterm (dgab,0)   # cdb(dgab00.105,dgab00)  # zero
   dgab01 = get_Rterm (dgab,1)   # cdb(dgab01.105,dgab01)  # zero
   dgab02 = get_Rterm (dgab,2)   # cdb(dgab02.105,dgab02)
   dgab03 = get_Rterm (dgab,3)   # cdb(dgab03.105,dgab03)
   dgab04 = get_Rterm (dgab,4)   # cdb(dgab04.105,dgab04)
   dgab05 = get_Rterm (dgab,5)   # cdb(dgab05.105,dgab05)

   # Convert free indices on iab from ^{a b} to ^{d c}
   # This ensures we can later build products like @(iab) @(dgab) knowing that the indices are correctly ordered.
   # Without this step we would be using free indices ^{a b} and _{a b c}. Thus the product @(iab) @(dgab) would
   # have just one free index _{c}. This is clearly wrong.

   tmp := @(iab) \delta_{a}^{d} \delta_{b}^{c}.

   distribute     (tmp)
   eliminate_kronecker (tmp)
   sort_product   (tmp)
   rename_dummies (tmp)
   canonicalise   (tmp)

   idc := @(tmp).

   # terms of the curvature expansion of g^{ab}

   idc00 = get_Rterm (idc,0)   # cdb(idc00.105,idc00)
   idc01 = get_Rterm (idc,1)   # cdb(idc01.105,idc01)  # zero
   idc02 = get_Rterm (idc,2)   # cdb(idc02.105,idc02)
   idc03 = get_Rterm (idc,3)   # cdb(idc03.105,idc03)
   idc04 = get_Rterm (idc,4)   # cdb(idc04.105,idc04)
   idc05 = get_Rterm (idc,5)   # cdb(idc05.105,idc05)

\end{cadabra}

\clearpage

\begin{dgroup*}
   \begin{dmath*} \cdb*{dgab00.105} \end{dmath*}
   \begin{dmath*} \cdb*{dgab01.105} \end{dmath*}
   \begin{dmath*} \cdb*{dgab02.105} \end{dmath*}
   \begin{dmath*} \cdb*{dgab03.105} \end{dmath*}
   \begin{dmath*} \cdb*{dgab04.105} \end{dmath*}
   \begin{dmath*} \cdb*{dgab05.105} \end{dmath*}
\end{dgroup*}

\clearpage

\begin{dgroup*}
   \begin{dmath*} \cdb*{idc00.105} \end{dmath*}
   \begin{dmath*} \cdb*{idc01.105} \end{dmath*}
   \begin{dmath*} \cdb*{idc02.105} \end{dmath*}
   \begin{dmath*} \cdb*{idc03.105} \end{dmath*}
   \begin{dmath*} \cdb*{idc04.105} \end{dmath*}
   \begin{dmath*} \cdb*{idc05.105} \end{dmath*}
\end{dgroup*}

\clearpage

\begin{cadabra}
   # idc  = g^{d c}
   # dgab = D_{a}{g_{c b}} + D_{b}{g_{a c}} - D_{c}{g_{a b}}

   # terms of the curvature expansion of \Gamma^{d}_{a b}

   # term0 := (1/2)  @(idc00) @(dgab00).
   # term1 := (1/2) (@(idc01) @(dgab00) + @(idc00) @(dgab01)).
   # term2 := (1/2) (@(idc02) @(dgab00) + @(idc01) @(dgab01) + @(idc00) @(dgab02)).
   # term3 := (1/2) (@(idc03) @(dgab00) + @(idc02) @(dgab01) + @(idc01) @(dgab02) + @(idc00) @(dgab03)).
   # term4 := (1/2) (@(idc04) @(dgab00) + @(idc03) @(dgab01) + @(idc02) @(dgab02) + @(idc01) @(dgab03) + @(idc00) @(dgab04)).
   # term5 := (1/2) (@(idc05) @(dgab00) + @(idc04) @(dgab01) + @(idc03) @(dgab02) + @(idc02) @(dgab03) + @(idc01) @(dgab04) + @(idc00) @(dgab05)).

   # simplidied version of the above after noting dgab00 = dgab01 = 0

   term0 := 0.
   term1 := 0.
   term2 := (1/2) (@(idc00) @(dgab02)).
   term3 := (1/2) (@(idc01) @(dgab02) + @(idc00) @(dgab03)).
   term4 := (1/2) (@(idc02) @(dgab02) + @(idc01) @(dgab03) + @(idc00) @(dgab04)).
   term5 := (1/2) (@(idc03) @(dgab02) + @(idc02) @(dgab03) + @(idc01) @(dgab04) + @(idc00) @(dgab05)).

   def tidy_terms (obj):
       substitute     (obj,$x^{a}->AA^{a}$,repeat=True)  # will force AA to the left of all terms
       distribute     (obj)
       sort_product   (obj)
       rename_dummies (obj)
       canonicalise   (obj)
       substitute     (obj,$AA^{a}->x^{a}$,repeat=True)  # replace AA with x
       factor_out     (obj,$x^{a?}$)

       return obj

   term0 = tidy_terms (term0)   # cdb(term0.201,term0)  # zero
   term1 = tidy_terms (term1)   # cdb(term1.201,term1)  # zero
   term2 = tidy_terms (term2)   # cdb(term2.201,term2)
   term3 = tidy_terms (term3)   # cdb(term3.201,term3)
   term4 = tidy_terms (term4)   # cdb(term4.201,term4)
   term5 = tidy_terms (term5)   # cdb(term5.201,term5)

   Gamma := @(term0) + @(term1) + @(term2) + @(term3) + @(term4) + @(term5). # cdb(Gamma.200,Gamma)

\end{cadabra}

\clearpage

\begin{dgroup*}
   \begin{dmath*} \cdb*{term0.201} \end{dmath*}
   \begin{dmath*} \cdb*{term1.201} \end{dmath*}
   \begin{dmath*} \cdb*{term2.201} \end{dmath*}
   \begin{dmath*} \cdb*{term3.201} \end{dmath*}
   \begin{dmath*} \cdb*{term4.201} \end{dmath*}
   \begin{dmath*} \cdb*{term5.201} \end{dmath*}
\end{dgroup*}

\clearpage

\begin{dgroup*}
   \begin{dmath*} \cdb*{Gamma.200} \end{dmath*}
\end{dgroup*}

\clearpage

\begin{cadabra}
   cdblib.create ('connection.json')

   cdblib.put ('Gamma',Gamma,'connection.json')

   cdblib.put ('GammaRterm0',term0,'connection.json')
   cdblib.put ('GammaRterm1',term1,'connection.json')
   cdblib.put ('GammaRterm2',term2,'connection.json')
   cdblib.put ('GammaRterm3',term3,'connection.json')
   cdblib.put ('GammaRterm4',term4,'connection.json')
   cdblib.put ('GammaRterm5',term5,'connection.json')

   checkpoint.append (term0)
   checkpoint.append (term1)
   checkpoint.append (term2)
   checkpoint.append (term3)
   checkpoint.append (term4)
   checkpoint.append (term5)

\end{cadabra}

% =================================================================================================
% the remaining code is just for pretty printing

\clearpage

\begin{cadabra}
   # note: keeping numbering as is (out of order) to ensure R appears before \nabla R etc.
   def product_sort (obj):
       substitute (obj,$ A^{a}                            -> A001^{a}               $)
       substitute (obj,$ x^{a}                            -> A002^{a}               $)
       substitute (obj,$ g^{a b}                          -> A003^{a b}             $)
       substitute (obj,$ \nabla_{e f g h}{R_{a b c d}}    -> A008_{a b c d e f g h} $)
       substitute (obj,$ \nabla_{e f g}{R_{a b c d}}      -> A007_{a b c d e f g}   $)
       substitute (obj,$ \nabla_{e f}{R_{a b c d}}        -> A006_{a b c d e f}     $)
       substitute (obj,$ \nabla_{e}{R_{a b c d}}          -> A005_{a b c d e}       $)
       substitute (obj,$ R_{a b c d}                      -> A004_{a b c d}         $)
       sort_product   (obj)
       rename_dummies (obj)
       substitute (obj,$ A001^{a}                  -> A^{a}                         $)
       substitute (obj,$ A002^{a}                  -> x^{a}                         $)
       substitute (obj,$ A003^{a b}                -> g^{a b}                       $)
       substitute (obj,$ A008_{a b c d e f g h}    -> \nabla_{e f g h}{R_{a b c d}} $)
       substitute (obj,$ A007_{a b c d e f g}      -> \nabla_{e f g}{R_{a b c d}}   $)
       substitute (obj,$ A006_{a b c d e f}        -> \nabla_{e f}{R_{a b c d}}     $)
       substitute (obj,$ A005_{a b c d e}          -> \nabla_{e}{R_{a b c d}}       $)
       substitute (obj,$ A004_{a b c d}            -> R_{a b c d}                   $)

       return obj

   def reformat (obj,scale):
      foo  = Ex(str(scale))
      bah := @(foo) @(obj).
      distribute     (bah)
      bah = product_sort (bah)
      rename_dummies (bah)
      canonicalise   (bah)
      factor_out     (bah,$A^{a?},x^{b?}$)
      ans := @(bah) / @(foo).
      return ans

   def rescale (obj,scale):
      foo  = Ex(str(scale))
      bah := @(foo) @(obj).
      distribute  (bah)
      factor_out  (bah,$A^{a?},x^{b?}$)
      return bah

   Rterm2 := @(term2) A^{a} A^{b}.
   Rterm3 := @(term3) A^{a} A^{b}.
   Rterm4 := @(term4) A^{a} A^{b}.
   Rterm5 := @(term5) A^{a} A^{b}.

   Rterm2 = reformat (Rterm2,  3)    # cdb(Rterm2.301,Rterm2)
   Rterm3 = reformat (Rterm3, 12)    # cdb(Rterm3.301,Rterm3)
   Rterm4 = reformat (Rterm4,360)    # cdb(Rterm4.301,Rterm4)
   Rterm5 = reformat (Rterm5,180)    # cdb(Rterm5.301,Rterm5)

   Gamma  := @(Rterm2) + @(Rterm3) + @(Rterm4) + @(Rterm5).  # cdb (Gamma.301,Gamma)
   Scaled := 360 @(Gamma).                                   # cdb (Scaled.301,Scaled)

   scaled2 = rescale (Rterm2,   3)   # cdb (scaled2.301,scaled2)
   scaled3 = rescale (Rterm3,  12)   # cdb (scaled3.301,scaled3)
   scaled4 = rescale (Rterm4, 360)   # cdb (scaled4.301,scaled4)
   scaled5 = rescale (Rterm5, 180)   # cdb (scaled5.301,scaled5)
\end{cadabra}

\clearpage

% =================================================================================================
\section*{The connection in Riemann normal coordinates}

\begin{dgroup*}
   \begin{dmath*} A^a A^b \Gamma^{d}_{a b} = \cdb{Gamma.301} \end{dmath*}
\end{dgroup*}

\begin{dgroup*}
   \begin{dmath*} 360 A^a A^b \Gamma^{d}_{a b} = \cdb{Scaled.301} \end{dmath*}
\end{dgroup*}

\clearpage

% =================================================================================================
\section*{Curvature expansion of the connection}

\begin{align*}
     A^a A^b \Gamma^{d}_{a b} =
     A^a A^b \nGamma{2}^{d}{}_{a b}
   + A^a A^b \nGamma{3}^{d}{}_{a b}
   + A^a A^b \nGamma{4}^{d}{}_{a b}
   + A^a A^b \nGamma{5}^{d}{}_{a b}+\BigO{\eps^6}
\end{align*}
\begin{dgroup*}
   \begin{dmath*}   3 A^a A^b \nGamma{2}^{d}_{a b} = \cdb{scaled2.301} \end{dmath*}
   \begin{dmath*}  12 A^a A^b \nGamma{3}^{d}_{a b} = \cdb{scaled3.301} \end{dmath*}
   \begin{dmath*} 360 A^a A^b \nGamma{4}^{d}_{a b} = \cdb{scaled4.301} \end{dmath*}
   \begin{dmath*} 180 A^a A^b \nGamma{5}^{d}_{a b} = \cdb{scaled5.301} \end{dmath*}
\end{dgroup*}

% =================================================================================================
% export checkpoints in json format

\bgroup
\CdbSetup{action=hide}
\begin{cadabra}
   for i in range( len(checkpoint) ):
      cdblib.put ('check{:03d}'.format(i),checkpoint[i],checkpoint_file)
\end{cadabra}
\egroup

\end{document}


\begin{dgroup*}
   \begin{dmath*} A^a A^b \Gamma^{d}_{a b} = \cdb{Gamma.301} \end{dmath*}
\end{dgroup*}

\begin{dgroup*}
   \begin{dmath*} 360 A^a A^b \Gamma^{d}_{a b} = \cdb{Scaled.301} \end{dmath*}
\end{dgroup*}

\clearpage

% =================================================================================================
\section*{Curvature expansion of the connection}
\begin{align*}
     A^a A^b \Gamma^{d}_{a b} =
     A^a A^b \nGamma{2}^{d}{}_{a b}
   + A^a A^b \nGamma{3}^{d}{}_{a b}
   + A^a A^b \nGamma{4}^{d}{}_{a b}
   + A^a A^b \nGamma{5}^{d}{}_{a b}+\BigO{\eps^6}
\end{align*}
\begin{dgroup*}
   \begin{dmath*}   3 A^a A^b \nGamma{2}^{d}_{a b} = \cdb{scaled2.301} \end{dmath*}
   \begin{dmath*}  12 A^a A^b \nGamma{3}^{d}_{a b} = \cdb{scaled3.301} \end{dmath*}
   \begin{dmath*} 360 A^a A^b \nGamma{4}^{d}_{a b} = \cdb{scaled4.301} \end{dmath*}
   \begin{dmath*} 180 A^a A^b \nGamma{5}^{d}_{a b} = \cdb{scaled5.301} \end{dmath*}
\end{dgroup*}

\clearpage

% =================================================================================================
\section*{Symmetrised partial derivatives of the connection}
\documentclass[12pt]{cdblatex}

\begin{document}

% =================================================================================================
% create checkpoint file

\bgroup
\CdbSetup{action=hide}
\begin{cadabra}
   import cdblib
   checkpoint_file = 'tests/semantic/output/dGamma.json'
   cdblib.create (checkpoint_file)
   checkpoint = []
\end{cadabra}
\egroup

% =================================================================================================
\section*{Symmetrized partial derivatives of the connection}

Here we calculate the recursive sequences
\begin{align*}
(n+3)\Gamma^a{}_{d(b,c\uen)}
   &= (n+1)\left(R^a{}_{(bc\Dot d,\uen)}
    - \left(\Gamma^a{}_{f(c}\Gamma^f{}_{b{\Dot d}}\right){\vphantom{\Gamma}}_{,\uen)}\right)
\end{align*}
for $n=1,2,3,\cdots$. Note that the (extended) index $\uen$ contains $n$ normal indices.

The result will be expressions for the $\Gamma^a{}_{d(b,c\uen)}$ in terms of the Riemann tensor and its
partial derivatives.

% =================================================================================================
\section*{Stage 1: Compute symmetrised derivatives}

In the first stage we simply apply the above recursive equation using a simple trick to impose
the symmetries. Start with the original equation and dot out the symmetric indices with $A^a$ then
factor out the partial derivatives. This leads to
\begin{equation}
(n+3)\Gamma^a{}_{db,c\uen} A^b A^c A^{\cdot\uen}
    = (n+1)\left(R^a{}_{bcd}
    - \Gamma^a{}_{fc}\Gamma^f{}_{bd}\right)_{,\uen} A^b A^c A^{\cdot\uen}
\end{equation}
Thus we also have (for the next iteration)
\begin{equation}
(n+4)\Gamma^a{}_{db,c\uenp} A^b A^c A^{\cdot\uenp}
    = (n+2)\left(R^a{}_{bcd}
    - \Gamma^a{}_{fc}\Gamma^f{}_{bd}\right)_{,\uenp} A^b A^c A^{\cdot\uenp}
\end{equation}
The $A^a$ can be freely chosen so choose $A^a$ to be a constant (i.e., zero derivative). Now
define $P_n$ by
\begin{equation}
   P_n = \Gamma^a{}_{db,c\uen} A^b A^c A^{\cdot\uen}
\end{equation}
then the above pair of equations can be combinded to give
\begin{equation}
   P_{n+1} = \frac{(n+2)(n+3)}{(n+4)(n+1)} A^f\partial_f\left(P_n\right)
\end{equation}
This is a very easy equation to compute as it just requires successive rounds of differentiation.

The first term in the sequence is $P_0$ given by
\begin{equation}
   P_0 = \cdb{dGamma01.101}
\end{equation}

The first few results are

\begin{dgroup*}
   \begin{dmath*} P_0 \hiderel{=} A^{b} A^{c} \Gamma^a{}_{d(b,c)} = \cdb{dGamma01.101} \end{dmath*}
   \begin{dmath*} P_1 \hiderel{=} A^{b} A^{c} A^{e} \Gamma^a{}_{d(b,ce)} = \cdb{dGamma02.105} \end{dmath*}
   \begin{dmath*} P_2 \hiderel{=} A^{b} A^{c} A^{e} A^{f} \Gamma^a{}_{d(b,cef)} = \cdb{dGamma03.105} \end{dmath*}
\end{dgroup*}

% =================================================================================================
\section*{Stage 2: Impose Riemann normal coordinates}

Here we impose the RNC condition by setting the $\Gamma^{a}{}_{bc}$ to zero (but not their derivatives).

\begin{dgroup*}
   \begin{dmath*} A^{b} A^{c} \Gamma^a{}_{d(b,c)} = \cdb{dGamma01.201} \end{dmath*}
   \begin{dmath*} A^{b} A^{c} A^{e} \Gamma^a{}_{d(b,ce)} = \cdb{dGamma02.202} \end{dmath*}
   \begin{dmath*} A^{b} A^{c} A^{e} A^{f} \Gamma^a{}_{d(b,cef)} = \cdb{dGamma03.203} \end{dmath*}
\end{dgroup*}

% =================================================================================================
\section*{Stage 3: Replace partial derivatives of $\Gamma$ with partial derivatives of $R$}

The key point to note is that the partial derivatives of $\Gamma$ on the right hand side are all
symmetrized in exactly the same manner as the partial derivatives on the left hand side. Thus
results from the lower order equations can be fed into the later equations to completely eliminate
the partial derivatives of $\Gamma$.

\begin{dgroup*}
   \begin{dmath*} A^{b} A^{c} \Gamma^a{}_{d(b,c)} = \cdb{dGamma01.201} \end{dmath*}
   \begin{dmath*} A^{b} A^{c} A^{e} \Gamma^a{}_{d(b,ce)} = \cdb{dGamma02.202} \end{dmath*}
   \begin{dmath*} A^{b} A^{c} A^{e} A^{f} \Gamma^a{}_{d(b,cef)} = \cdb{dGamma03.402} \end{dmath*}
\end{dgroup*}

% =================================================================================================
\section*{Stage 4: Reformatting}

This is just simple reformatting.

\begin{dgroup*}
   \begin{dmath*}   3 A^b A^c\Gamma^a{}_{d(b,c)} = \cdb{scaled1.002} \end{dmath*}
   \begin{dmath*}   6 A^b A^c A^e \Gamma^a{}_{d(b,ce)} = \cdb{scaled2.002} \end{dmath*}
   \begin{dmath*}  15 A^b A^c A^e A^f \Gamma^a{}_{d(b,cef)} = \cdb{scaled3.002} \end{dmath*}
\end{dgroup*}

\clearpage

% =================================================================================================
\section*{Stage 1: Compute symmetrised derivatives}

\begin{cadabra}
   {a,b,c,d,e,f,g,h,i,j,k,l,m,n,o,p,q,r,s,t,u,v,w#}::Indices(position=independent).

   \nabla{#}::Derivative.
   \partial{#}::PartialDerivative.

   g_{a b}::Metric.
   g^{a b}::InverseMetric.
   g_{a}^{b}::KroneckerDelta.
   g^{a}_{b}::KroneckerDelta.

   R_{a b c d}::RiemannTensor.
   R^{a}_{b c d}::RiemannTensor.
   R_{a b c}^{d}::RiemannTensor.

   \Gamma^{a}_{b c}::TableauSymmetry(shape={2}, indices={1,2}).

   g_{a b}::Depends(\partial{#}).
   R_{a b c d}::Depends(\partial{#}).
   R^{a}_{b c d}::Depends(\partial{#}).
   \Gamma^{a}_{b c}::Depends(\partial{#}).

   # symmetrized partial derivatives of \Gamma

   dGamma01:= (1/3) A^{b} A^{c} ( R^{a}_{b c d} - \Gamma^{a}_{c e}\Gamma^{e}_{b d} ).
                                                        # cdb (dGamma01.101,dGamma01)

   dGamma02:= (6/4) A^{a}\partial_{a}{ @(dGamma01) }.   # cdb (dGamma02.101,dGamma02)
   distribute   (dGamma02)                              # cdb (dGamma02.102,dGamma02)
   product_rule (dGamma02)                              # cdb (dGamma02.103,dGamma02)
   unwrap       (dGamma02)                              # cdb (dGamma02.104,dGamma02)
   distribute   (dGamma02)                              # cdb (dGamma02.105,dGamma02)

   dGamma03:= (12/10) A^{a}\partial_{a}{ @(dGamma02) }. # cdb (dGamma03.101,dGamma03)
   distribute   (dGamma03)                              # cdb (dGamma03.102,dGamma03)
   product_rule (dGamma03)                              # cdb (dGamma03.103,dGamma03)
   unwrap       (dGamma03)                              # cdb (dGamma03.104,dGamma03)
   distribute   (dGamma03)                              # cdb (dGamma03.105,dGamma03)

   dGamma04:= (20/18) A^{a}\partial_{a}{ @(dGamma03) }. # cdb (dGamma04.101,dGamma04)
   distribute   (dGamma04)                              # cdb (dGamma04.102,dGamma04)
   product_rule (dGamma04)                              # cdb (dGamma04.103,dGamma04)
   unwrap       (dGamma04)                              # cdb (dGamma04.104,dGamma04)
   distribute   (dGamma04)                              # cdb (dGamma04.105,dGamma04)

   dGamma05:= (30/28) A^{a}\partial_{a}{ @(dGamma04) }. # cdb (dGamma05.101,dGamma05)
   distribute   (dGamma05)                              # cdb (dGamma05.102,dGamma05)
   product_rule (dGamma05)                              # cdb (dGamma05.103,dGamma05)
   unwrap       (dGamma05)                              # cdb (dGamma05.104,dGamma05)
   distribute   (dGamma05)                              # cdb (dGamma05.105,dGamma05)

\end{cadabra}

\clearpage

\begin{dgroup*}
   \begin{dmath*} \cdb*{dGamma01.101} \end{dmath*}
\end{dgroup*}

\begin{dgroup*}
   \begin{dmath*} \cdb*{dGamma02.101} \end{dmath*}
   \begin{dmath*} \cdb*{dGamma02.102} \end{dmath*}
   \begin{dmath*} \cdb*{dGamma02.103} \end{dmath*}
   \begin{dmath*} \cdb*{dGamma02.104} \end{dmath*}
   \begin{dmath*} \cdb*{dGamma02.105} \end{dmath*}
\end{dgroup*}

\begin{dgroup*}
   \begin{dmath*} \cdb*{dGamma03.101} \end{dmath*}
   \begin{dmath*} \cdb*{dGamma03.102} \end{dmath*}
   \begin{dmath*} \cdb*{dGamma03.103} \end{dmath*}
   \begin{dmath*} \cdb*{dGamma03.104} \end{dmath*}
   \begin{dmath*} \cdb*{dGamma03.105} \end{dmath*}
\end{dgroup*}

\begin{dgroup*}
   \begin{dmath*} \cdb*{dGamma04.101} \end{dmath*}
   \begin{dmath*} \cdb*{dGamma04.102} \end{dmath*}
   \begin{dmath*} \cdb*{dGamma04.103} \end{dmath*}
   \begin{dmath*} \cdb*{dGamma04.104} \end{dmath*}
   \begin{dmath*} \cdb*{dGamma04.105} \end{dmath*}
\end{dgroup*}

% too long for pdflatex

% \begin{dgroup*}
%    \begin{dmath*} \cdb*{dGamma05.101} \end{dmath*}
%    \begin{dmath*} \cdb*{dGamma05.102} \end{dmath*}
%    \begin{dmath*} \cdb*{dGamma05.103} \end{dmath*}
%    \begin{dmath*} \cdb*{dGamma05.104} \end{dmath*}
%    \begin{dmath*} \cdb*{dGamma05.105} \end{dmath*}
% \end{dgroup*}

\clearpage

% =================================================================================================
\section*{Stage 2: Impose Riemann normal coordinates}

\begin{cadabra}
   def impose_rnc (obj):
       # hide the derivatives of Gamma
       substitute (obj,$\partial_{d}{\Gamma^{a}_{b c}} -> zzz_{d}^{a}_{b c}$,repeat=True)
       substitute (obj,$\partial_{d e}{\Gamma^{a}_{b c}} -> zzz_{d e}^{a}_{b c}$,repeat=True)
       substitute (obj,$\partial_{d e f}{\Gamma^{a}_{b c}} -> zzz_{d e f}^{a}_{b c}$,repeat=True)
       substitute (obj,$\partial_{d e f g}{\Gamma^{a}_{b c}} -> zzz_{d e f g}^{a}_{b c}$,repeat=True)
       substitute (obj,$\partial_{d e f g h}{\Gamma^{a}_{b c}} -> zzz_{d e f g h}^{a}_{b c}$,repeat=True)
       # set Gamma to zero
       substitute (obj,$\Gamma^{a}_{b c} -> 0$,repeat=True)
       # recover the derivatives Gamma
       substitute (obj,$zzz_{d}^{a}_{b c} -> \partial_{d}{\Gamma^{a}_{b c}}$,repeat=True)
       substitute (obj,$zzz_{d e}^{a}_{b c} -> \partial_{d e}{\Gamma^{a}_{b c}}$,repeat=True)
       substitute (obj,$zzz_{d e f}^{a}_{b c} -> \partial_{d e f}{\Gamma^{a}_{b c}}$,repeat=True)
       substitute (obj,$zzz_{d e f g}^{a}_{b c} -> \partial_{d e f g}{\Gamma^{a}_{b c}}$,repeat=True)
       substitute (obj,$zzz_{d e f g h}^{a}_{b c} -> \partial_{d e f g h}{\Gamma^{a}_{b c}}$,repeat=True)
       return obj

   # switch to RNC

   dGamma01 = impose_rnc (dGamma01)   # cdb (dGamma01.201,dGamma01)
   dGamma02 = impose_rnc (dGamma02)   # cdb (dGamma02.202,dGamma02)
   dGamma03 = impose_rnc (dGamma03)   # cdb (dGamma03.203,dGamma03)
   dGamma04 = impose_rnc (dGamma04)   # cdb (dGamma04.204,dGamma04)
   dGamma05 = impose_rnc (dGamma05)   # cdb (dGamma05.205,dGamma05)

\end{cadabra}

\begin{dgroup*}
   \begin{dmath*} \cdb*{dGamma01.201} \end{dmath*}
   \begin{dmath*} \cdb*{dGamma02.202} \end{dmath*}
   \begin{dmath*} \cdb*{dGamma03.203} \end{dmath*}
   \begin{dmath*} \cdb*{dGamma04.204} \end{dmath*}
   \begin{dmath*} \cdb*{dGamma05.205} \end{dmath*}
\end{dgroup*}

\clearpage

% =================================================================================================
\section*{Stage 3: Replace partial derivatives of $\Gamma$ with partial derivatives of $R$}

\begin{cadabra}
   # use lower equations to eliminate partial derivs of Gamma from rhs

   # this produces experssions for the partial derivs of the Gamma's in terms of the Rabcd and its partial derivs

   substitute (dGamma03,$A^{c}A^{b}\partial_{c}{\Gamma^{a}_{b d}} -> @(dGamma01)$,repeat=True)                 # cdb(dGamma03.301,dGamma03)
   substitute (dGamma03,$A^{c}A^{b}\partial_{c}{\Gamma^{a}_{d b}} -> @(dGamma01)$,repeat=True)                 # cdb(dGamma03.302,dGamma03)
   distribute (dGamma03)                                                                                       # cdb(dGamma03.303,dGamma03)

   substitute (dGamma04,$A^{c}A^{b}A^{e}\partial_{c e}{\Gamma^{a}_{d b}} -> @(dGamma02)$,repeat=True)          # cdb(dGamma04.301,dGamma04)
   substitute (dGamma04,$A^{c}A^{b}A^{e}\partial_{c e}{\Gamma^{a}_{b d}} -> @(dGamma02)$,repeat=True)          # cdb(dGamma04.302,dGamma04)
   substitute (dGamma04,$A^{c}A^{b}\partial_{c}{\Gamma^{a}_{b d}} -> @(dGamma01)$,repeat=True)                 # cdb(dGamma04.303,dGamma04)
   substitute (dGamma04,$A^{c}A^{b}\partial_{c}{\Gamma^{a}_{d b}} -> @(dGamma01)$,repeat=True)                 # cdb(dGamma04.304,dGamma04)
   distribute (dGamma04)                                                                                       # cdb(dGamma04.305,dGamma04)

   substitute (dGamma05,$A^{c}A^{b}A^{e}A^{f}\partial_{c e f}{\Gamma^{a}_{d b}} -> @(dGamma03)$,repeat=True)   # cdb(dGamma05.301,dGamma05)
   substitute (dGamma05,$A^{c}A^{b}A^{e}A^{f}\partial_{c e f}{\Gamma^{a}_{b d}} -> @(dGamma03)$,repeat=True)   # cdb(dGamma05.302,dGamma05)
   substitute (dGamma05,$A^{c}A^{b}A^{e}\partial_{c e}{\Gamma^{a}_{d b}} -> @(dGamma02)$,repeat=True)          # cdb(dGamma05.303,dGamma05)
   substitute (dGamma05,$A^{c}A^{b}A^{e}\partial_{c e}{\Gamma^{a}_{b d}} -> @(dGamma02)$,repeat=True)          # cdb(dGamma05.304,dGamma05)
   substitute (dGamma05,$A^{c}A^{b}\partial_{c}{\Gamma^{a}_{b d}} -> @(dGamma01)$,repeat=True)                 # cdb(dGamma05.305,dGamma05)
   substitute (dGamma05,$A^{c}A^{b}\partial_{c}{\Gamma^{a}_{d b}} -> @(dGamma01)$,repeat=True)                 # cdb(dGamma05.306,dGamma05)
   distribute (dGamma05)                                                                                       # cdb(dGamma05.307,dGamma05)

\end{cadabra}

\clearpage

\begin{dgroup*}
   \begin{dmath*} \cdb*{dGamma03.301} \end{dmath*}
   \begin{dmath*} \cdb*{dGamma03.302} \end{dmath*}
   \begin{dmath*} \cdb*{dGamma03.303} \end{dmath*}
\end{dgroup*}

\begin{dgroup*}
   \begin{dmath*} \cdb*{dGamma04.301} \end{dmath*}
   \begin{dmath*} \cdb*{dGamma04.302} \end{dmath*}
   \begin{dmath*} \cdb*{dGamma04.303} \end{dmath*}
   \begin{dmath*} \cdb*{dGamma04.304} \end{dmath*}
   \begin{dmath*} \cdb*{dGamma04.305} \end{dmath*}
\end{dgroup*}

\begin{dgroup*}
   \begin{dmath*} \cdb*{dGamma05.301} \end{dmath*}
   \begin{dmath*} \cdb*{dGamma05.302} \end{dmath*}
   \begin{dmath*} \cdb*{dGamma05.303} \end{dmath*}
   \begin{dmath*} \cdb*{dGamma05.304} \end{dmath*}
   \begin{dmath*} \cdb*{dGamma05.305} \end{dmath*}
   \begin{dmath*} \cdb*{dGamma05.306} \end{dmath*}
   \begin{dmath*} \cdb*{dGamma05.307} \end{dmath*}
\end{dgroup*}

\clearpage

\begin{cadabra}
   # note:
   # canonicalise must not be used here because it may make changes like
   #    R^{a}_{b c d} -> - R_{b}^{a}_{c d}
   # these changes can not be applied inside a \partial, must defer use
   # of canocialise until we have \nabla acting on curvatures

   sort_product   (dGamma03) # cdb(dGamma03.401,dGamma03)
   rename_dummies (dGamma03) # cdb(dGamma03.402,dGamma03)
   # canonicalise   (dGamma03) # cdb(dGamma03.403,dGamma03)

   sort_product   (dGamma04) # cdb(dGamma04.401,dGamma04)
   rename_dummies (dGamma04) # cdb(dGamma04.402,dGamma04)
   # canonicalise   (dGamma04) # cdb(dGamma04.403,dGamma04)

   sort_product   (dGamma05) # cdb(dGamma05.401,dGamma05)
   rename_dummies (dGamma05) # cdb(dGamma05.402,dGamma05)
   # canonicalise   (dGamma05) # cdb(dGamma05.403,dGamma05)

\end{cadabra}

\clearpage

\begin{dgroup*}
   \begin{dmath*} \cdb*{dGamma03.401} \end{dmath*}
   \begin{dmath*} \cdb*{dGamma03.402} \end{dmath*}
   % \begin{dmath*} \cdb*{dGamma03.403} \end{dmath*}
\end{dgroup*}

\begin{dgroup*}
   \begin{dmath*} \cdb*{dGamma04.401} \end{dmath*}
   \begin{dmath*} \cdb*{dGamma04.402} \end{dmath*}
   % \begin{dmath*} \cdb*{dGamma04.403} \end{dmath*}
\end{dgroup*}

\begin{dgroup*}
   \begin{dmath*} \cdb*{dGamma05.401} \end{dmath*}
   \begin{dmath*} \cdb*{dGamma05.402} \end{dmath*}
   % \begin{dmath*} \cdb*{dGamma05.403} \end{dmath*}
\end{dgroup*}

\clearpage

\begin{cadabra}
   import cdblib

   cdblib.create ('dGamma.json')

   cdblib.put ('dGamma01',dGamma01,'dGamma.json')
   cdblib.put ('dGamma02',dGamma02,'dGamma.json')
   cdblib.put ('dGamma03',dGamma03,'dGamma.json')
   cdblib.put ('dGamma04',dGamma04,'dGamma.json')
   cdblib.put ('dGamma05',dGamma05,'dGamma.json')

\end{cadabra}

\clearpage

% =================================================================================================
\section*{Stage 4: Reformatting}

\begin{cadabra}
   # note: keeping numbering as is (out of order) to ensure R appears before \nabla R etc.
   def product_sort (obj):
       substitute (obj,$ A^{a}                              -> A001^{a}                 $)
       substitute (obj,$ x^{a}                              -> A002^{a}                 $)
       substitute (obj,$ g^{a b}                            -> A003^{a b}               $)
       substitute (obj,$ \partial_{e f g h}{R^{a}_{b c d}}  -> A008^{a}_{b c d e f g h} $)
       substitute (obj,$ \partial_{e f g}{R^{a}_{b c d}}    -> A007^{a}_{b c d e f g}   $)
       substitute (obj,$ \partial_{e f}{R^{a}_{b c d}}      -> A006^{a}_{b c d e f}     $)
       substitute (obj,$ \partial_{e}{R^{a}_{b c d}}        -> A005^{a}_{b c d e}       $)
       substitute (obj,$ R^{a}_{b c d}                      -> A004^{a}_{b c d}         $)
       sort_product   (obj)
       rename_dummies (obj)
       substitute (obj,$ A001^{a}                  -> A^{a}                             $)
       substitute (obj,$ A002^{a}                  -> x^{a}                             $)
       substitute (obj,$ A003^{a b}                -> g^{a b}                           $)
       substitute (obj,$ A004^{a}_{b c d}          -> R^{a}_{b c d}                     $)
       substitute (obj,$ A005^{a}_{b c d e}        -> \partial_{e}{R^{a}_{b c d}}       $)
       substitute (obj,$ A006^{a}_{b c d e f}      -> \partial_{e f}{R^{a}_{b c d}}     $)
       substitute (obj,$ A007^{a}_{b c d e f g}    -> \partial_{e f g}{R^{a}_{b c d}}   $)
       substitute (obj,$ A008^{a}_{b c d e f g h}  -> \partial_{e f g h}{R^{a}_{b c d}} $)

       return obj

   def reformat (obj,scale):
       bah  = Ex(str(scale))
       tmp := @(bah) @(obj).
       distribute     (tmp)
       tmp = product_sort (tmp)
       rename_dummies (tmp)
       factor_out     (tmp,$A^{a?}$)
       return tmp

   def get_term (obj,n):

       A^{a}::Weight(label=numA).

       foo := @(obj).
       bah  = Ex("numA = " + str(n))
       distribute  (foo)
       keep_weight (foo, bah)

       return foo

   Gterm01 := @(dGamma01).
   Gterm02 := @(dGamma02).
   Gterm03 := @(dGamma03).
   Gterm04 := @(dGamma04).
   Gterm05 := @(dGamma05).

   scaled1 = reformat (Gterm01,   3)   # cdb (scaled1.002,scaled1)
   scaled2 = reformat (Gterm02,   6)   # cdb (scaled2.002,scaled2)
   scaled3 = reformat (Gterm03,  15)   # cdb (scaled3.002,scaled3)
   scaled4 = reformat (Gterm04,   9)   # cdb (scaled4.002,scaled4)
   scaled5 = reformat (Gterm05, 252)   # cdb (scaled5.002,scaled5)

\end{cadabra}

\clearpage

% =================================================================================================
\section*{Symmetrised partial derivatives of the connection}

\begin{dgroup*}
   \begin{dmath*}   3 A^b A^c\Gamma^a{}_{d(b,c)} = \cdb{scaled1.002} \end{dmath*}
   \begin{dmath*}   6 A^b A^c A^e \Gamma^a{}_{d(b,ce)} = \cdb{scaled2.002} \end{dmath*}
   \begin{dmath*}  15 A^b A^c A^e A^f \Gamma^a{}_{d(b,cef)} = \cdb{scaled3.002} \end{dmath*}
   \begin{dmath*}   9 A^b A^c A^e A^f A^g \Gamma^a{}_{d(b,cefg)} = \cdb{scaled4.002} \end{dmath*}
   \begin{dmath*} 252 A^b A^c A^e A^f A^g A^h\Gamma^a{}_{d(b,cefgh)} = \cdb{scaled5.002} \end{dmath*}
\end{dgroup*}

\clearpage

% =================================================================================================
% export selected objects, these will later be imported into a library
% these are the objects that will appear in the paper

\begin{cadabra}
   substitute (scaled1,$A^{a}->1$)
   substitute (scaled2,$A^{a}->1$)
   substitute (scaled3,$A^{a}->1$)
   substitute (scaled4,$A^{a}->1$)
   substitute (scaled5,$A^{a}->1$)

   cdblib.create ('dGamma.export')

   # 6th order dGamma, scaled
   cdblib.put ('dGamma61scaled',scaled1,'dGamma.export')
   cdblib.put ('dGamma62scaled',scaled2,'dGamma.export')
   cdblib.put ('dGamma63scaled',scaled3,'dGamma.export')
   cdblib.put ('dGamma64scaled',scaled4,'dGamma.export')
   cdblib.put ('dGamma65scaled',scaled5,'dGamma.export')

   checkpoint.append (scaled1)
   checkpoint.append (scaled2)
   checkpoint.append (scaled3)
   checkpoint.append (scaled4)
   checkpoint.append (scaled5)

\end{cadabra}

% =================================================================================================
% export checkpoints in json format

\bgroup
\CdbSetup{action=hide}
\begin{cadabra}
   for i in range( len(checkpoint) ):
      cdblib.put ('check{:03d}'.format(i),checkpoint[i],checkpoint_file)
\end{cadabra}
\egroup

\end{document}


\begin{dgroup*}
   \begin{dmath*}   3 A^b A^c\Gamma^a{}_{d(b,c)} = \cdb{scaled1.002} \end{dmath*}
   \begin{dmath*}   6 A^b A^c A^e \Gamma^a{}_{d(b,ce)} = \cdb{scaled2.002} \end{dmath*}
   \begin{dmath*}  15 A^b A^c A^e A^f \Gamma^a{}_{d(b,cef)} = \cdb{scaled3.002} \end{dmath*}
   \begin{dmath*}   9 A^b A^c A^e A^f A^g \Gamma^a{}_{d(b,cefg)} = \cdb{scaled4.002} \end{dmath*}
   \begin{dmath*} 252 A^b A^c A^e A^f A^g A^h\Gamma^a{}_{d(b,cefgh)} = \cdb{scaled5.002} \end{dmath*}
\end{dgroup*}

\clearpage

% =================================================================================================
\section*{Symmetrised partial derivatives of $R^a{}_{bcd}$}
\documentclass[12pt]{cdblatex}
\usepackage{fancyhdr}
\usepackage{footer}

\begin{document}

% =================================================================================================
% create checkpoint file

\bgroup
\CdbSetup{action=hide}
\begin{cadabra}
   import cdblib
   checkpoint_file = 'tests/semantic/output/dRabcd.json'
   cdblib.create (checkpoint_file)
   checkpoint = []
\end{cadabra}
\egroup

% =================================================================================================
\section*{Symmetrised partial derivatives of the Riemann tensor}

Here we compute the symmetrised partial derivatoves $R^{a}{}_{(b{\Dot c}d,\ue)}$ in terms of
the symmetrised covariant derivatives $R^{a}{}_{(b{\Dot c}d;\ue)}$. Note that the dot over an
index indicates that that index does not take part in the symmetrisation.

We will use the algorithm described in section (10.3) of my lcb09-03 paper. Here we will make
one small change of notation -- the symbol $D^a$ will replaced with $A^a$.

We have lots of space (and no annoying editors to appease with brevity) so I will take the
liberty to expand slightly on what I wrote in the lcb0-03 paper.

Our starting point is the simple identity
\begin{align}
   \label{eqn:submain}
   \left( R^{a}{}_{cdb} B^{b}{}_{a} A^{c} A^{d} \right)_{;e} A^{e}
   =
   \left( R^{a}{}_{cdb} B^{b}{}_{a} A^{c} A^{d} \right)_{,e} A^{e}
\end{align}
This is true in all frames since the quantity inside the brackets is a scalar. We are free to
make any choice we like for $A^{a}$ and $B^{a}{}_{b}$ so let's choose $A^{a}$ to be the tangent
vector to any geodesic through the origin and choose the $B^{a}{}_{b}$ to be constants (i.e,
all partial derivatives are zero). We will also use local Riemann normal coordinates and as a
consequence, the $A^{a}$ will also be constant along the integral curves of $A$ (the geodesics
in an RNC are always of the form $x^{a}(s) = s A^{a}$ for some affine parameter $s$ on the
geodesic). Let $df/ds$ be the directional derivative of the function $f$ along the geodesics
defined by $A^{a}$ and assume that $s$ is the proper length along the geodesic (although any
affine parameter would be sufficient).

Thus at the origin we have, by choice,
\begin{gather*}
   0 = B^{a}{}_{b,c} = B^{a}{}_{b,cd} = B^{a}{}_{b,cde} = \dots\\[5pt]
   0 = dA^{a}/ds = d^2A^{a}/ds^2 = d^3A^{a}/ds^3 = \dots\\[5pt]
   0 = A^{a}{}_{,b} A^{b} = \left(A^{a}{}_{,b} A^{b}\right)_{,c} A^{c} = \left(\left(A^{a}{}_{,b} A^{b}\right)_{,c} A^{c}\right)_{,d} A^{d}\\[5pt]
   0 = A^{a}{}_{;b} A^{b} = \left(A^{a}{}_{;b} A^{b}\right)_{;c} A^{c} = \left(\left(A^{a}{}_{;b} A^{b}\right)_{;c} A^{c}\right)_{;d} A^{d}\\[5pt]
   df/ds = f_{,a} A^{a} = f_{;a} A^{a}\\[5pt]
   d^2f/ds^2 = \left ( f_{,a} A^{a} \right)_{,b} A^{b} = \left ( f_{;a} A^{a} \right)_{;b} A^{b}\\[5pt]
   d^3f/ds^3 = \left(\left ( f_{,a} A^{a} \right)_{,b} A^{b}\right)_{,c} A^{c} = \left(\left ( f_{;a} A^{a} \right)_{;b} A^{b}\right)_{;c} A^{c}
\end{gather*}
I admit I've gone overboard here in writing out more than I need to but it's handy to have all
of these equations laid bare in one convenient place.

Now put $f = R^{p}{}_{abq} B^{q}{}_{p} A^{a} A^{b}$. Then upon taking successive derivatives,
while taking full advantage of the asummptions just noted, we can eaily see that
\begin{align}
   \label{eqn:main}
   \left( R^{a}{}_{cdb} B^{b}{}_{a} \right)_{;\ue} A^{c} A^{d} A^{\ue}
   =
   \left( R^{a}{}_{cdb} \right)_{,\ue} B^{b}{}_{a} A^{c} A^{d} A^{\ue}
\end{align}
This is the equation that will be computed by the following Cadabra code. All of the
compuations will be carried out on the left hand side (in the first version of the paper I
swapped the left and righ hand sides).

We will need the successive covariant derivatives of $B$. The first covariant derivative is
just \begin{align*} B^{a}{}_{b;c} A^c &= \Gamma^{a}{}_{dc}B^{d}{}_{b} A^c -
\Gamma^{d}{}_{bc}B^{a}{}_{d} A^c \end{align*} The quantities on the left hand side are the
components of a tensor so further covariant derivatives of the right hand side can be computed
(despite the presence of the $\Gamma$'s) by application of the usual rule for a covariant
derivative of a mixed tensor.

% =================================================================================================
\section*{Stage 1: Symmetrised partial derivatives of $R$}

The first stage involves the expansion of the left side of (\ref{eqn:main}). This leads to
expressions for the symmetrized partial derivatives of $R_{abcd}$ in terms of the symmetrized
covariant derivatives of $R_{abcd}$ and $B^{a}{}_{b}$.

\begin{dgroup*}
   \begin{dmath*} \left( R^{a}{}_{cdb} \right)_{,e} B^{b}{}_{a} A^{c} A^{d} A^{e}
                  = \cdb{dRabcd01.108} \end{dmath*}
   \begin{dmath*} \left( R^{a}{}_{cdb} \right)_{,ef} B^{b}{}_{a} A^{c} A^{d} A^{e} A^{f}
                  = \cdb{dRabcd02.108} \end{dmath*}
   \begin{dmath*} \left( R^{a}{}_{cdb} \right)_{,efg} B^{b}{}_{a} A^{c} A^{d} A^{e} A^{f} A^{g}
                  = \cdb{dRabcd03.108} \end{dmath*}
\end{dgroup*}

% =================================================================================================
\section*{Stage 2: Symmetrised covariant derivatives of $B$}

In this stage the symmetrized covariant derivatives of $B^{a}{}_{b}$ are computed in terms of
its partial derivatives (which by choice are all zero) and the connection and its partial
derivatives (which in general are not zero).

\begin{dgroup*}
   \begin{dmath*} A^{c}\nabla_{c}\left(B^{a}{}_{b}\right)
                  = \cdb{dBab01.209} \end{dmath*}
   \begin{dmath*} A^{d}A^{c}\nabla_{d}\left(\nabla_{c}\left(B^{a}{}_{b}\right)\right)
                  = \cdb{dBab02.209} \end{dmath*}
   \begin{dmath*} A^{e}A^{d}A^{c}\nabla_{e}\left(\nabla_{d}\left(\nabla_{c}\left(B^{a}{}_{b}\right)\right)\right)
                  = \cdb{dBab03.209} \end{dmath*}
\end{dgroup*}

% =================================================================================================
\section*{Stage 3: Impose the Riemann normal coordinate condition on covariant derivs of $B$}

Here we impose the RNC condition (that $\Gamma = 0$ while $\partial\Gamma\not=0$).

\begin{dgroup*}
   \begin{dmath*} A^{c}\nabla_{c}\left(B^{a}{}_{b}\right)
                  = \cdb{dBab01.301} \end{dmath*}
   \begin{dmath*} A^{d}A^{c}\nabla_{d}\left(\nabla_{c}\left(B^{a}{}_{b}\right)\right)
                  = \cdb{dBab02.301} \end{dmath*}
   \begin{dmath*} A^{e}A^{d}A^{c}\nabla_{e}\left(\nabla_{d}\left(\nabla_{c}\left(B^{a}{}_{b}\right)\right)\right)
                  = \cdb{dBab03.301} \end{dmath*}
\end{dgroup*}

% =================================================================================================
\section*{Stage 4: Replace covariant derivs of $B$ with partial derivs of $\Gamma$}

This stage uses the results from the second stage to eliminate the $\nabla B$ terms from the
results of the first stage. This produces expressions for the symmetrized partial derivatives
of $R_{abcd}$ in terms of the symmetrized covariant derivatives of $R_{abcd}$ and the partial
derivatives of the connection. In this stage we also set the $B^{a}{}_{b}$ to equal 1.

\begin{dgroup*}
   \begin{dmath*} \left( R^{a}{}_{cdb} \right)_{,e} A^{c} A^{d} A^{e}
                  = \cdb{dRabcd01.401} \end{dmath*}
   \begin{dmath*} \left( R^{a}{}_{cdb} \right)_{,ef} A^{c} A^{d} A^{e} A^{e}
                  = \cdb{dRabcd02.401} \end{dmath*}
   \begin{dmath*} \left( R^{a}{}_{cdb} \right)_{,efg} A^{c} A^{d} A^{e} A^{f} A^{g}
                  = \cdb{dRabcd03.401} \end{dmath*}
\end{dgroup*}

% =================================================================================================
\section*{Stage 5: Replace partial derivs of $\Gamma$ with partial derivs of $R$}

The fifth stage draws in results from {\tt dGamma.tex} to replace the partial derivatives of
$\Gamma$ with partial derivatives of $R_{abcd}$.

\begin{dgroup*}
   \begin{dmath*} \left( R^{a}{}_{cdb} \right)_{,e} A^{c} A^{d} A^{e}
                  = \cdb{dRabcd01.500} \end{dmath*}
   \begin{dmath*} \left( R^{a}{}_{cdb} \right)_{,ef} A^{c} A^{d} A^{e} A^{f}
                  = \cdb{dRabcd02.505} \end{dmath*}
   \begin{dmath*} \left( R^{a}{}_{cdb} \right)_{,efg} A^{c} A^{d} A^{e} A^{f} A^{g}
                  = \cdb{dRabcd03.507} \end{dmath*}
\end{dgroup*}

% =================================================================================================
\section*{Stage 6: Replace partial derivs of $R$ with covariant derivs of $R$}

The final stage is to eliminate the $\partial R$ by using earlier results. For example, in the
equation for $\partial^3 R$ we see terms involving $\partial R$. These first order partial
derivatives can be replaced with the expression previously computed for $\partial R$ in terms
of $\nabla R$.

\begin{dgroup*}
   \begin{dmath*} \left( R^{a}{}_{cdb} \right)_{,e} A^{c} A^{d} A^{e}
                  = \cdb{dRabcd01.702} \end{dmath*}
   \begin{dmath*} \left( R^{a}{}_{cdb} \right)_{,ef} A^{c} A^{d} A^{e} A^{e}
                  = \cdb{dRabcd02.702} \end{dmath*}
   \begin{dmath*} \left( R^{a}{}_{cdb} \right)_{,efg} A^{c} A^{d} A^{e} A^{f} A^{g}
                  = \cdb{dRabcd03.702} \end{dmath*}
\end{dgroup*}

The end result are expressions for the symmetrized partial derivatives of $R_{abcd}$ solely in
terms of the symmetrized covariant derivatives of $R_{abcd}$.

\clearpage

% =================================================================================================
\section*{Shared properties}

\begin{cadabra}
   import time

   {a,b,c,d,e,f,g,h,i,j,k,l,m,n,o,p,q,r,s,t,u,v,w#}::Indices(position=independent).

   \nabla{#}::Derivative.
   \partial{#}::PartialDerivative.

   g_{a b}::Metric.
   g^{a b}::InverseMetric.
   g_{a}^{b}::KroneckerDelta.
   g^{a}_{b}::KroneckerDelta.

   R_{a b c d}::RiemannTensor.
   R^{a}_{b c d}::RiemannTensor.

   \Gamma^{a}_{b c}::TableauSymmetry(shape={2}, indices={1,2}).

   g_{a b}::Depends(\partial{#}).
   R_{a b c d}::Depends(\partial{#}).
   R^{a}_{b c d}::Depends(\partial{#}).
   \Gamma^{a}_{b c}::Depends(\partial{#}).

   B^{a}_{b::Depends(\nabla{#}).
   R_{a b c d}::Depends(\nabla{#}).
   R^{a}_{b c d}::Depends(\nabla{#}).

\end{cadabra}

\clearpage

% =================================================================================================
\section*{Stage 1: Symmetrised partial derivatives of $R$}

\begin{cadabra}
   def flatten_Rabcd (obj):
       substitute (obj,$R^{a}_{b c d}   -> g^{a e} R_{e b c d}$)
       substitute (obj,$R_{a}^{b}_{c d} -> g^{b e} R_{a e c d}$)
       substitute (obj,$R_{a b}^{c}_{b} -> g^{c e} R_{a b e d}$)
       substitute (obj,$R_{a b c}^{d}   -> g^{d e} R_{a b c e}$)
       unwrap     (obj)
       sort_product   (obj)
       rename_dummies (obj)
       return obj

   # compute the symmetric covariant derivatives of R^{a}_{bcd} B^{d}_{a}

   beg_stage_1 = time.time()

   dRabcd00:=R^{a}_{b c d} B^{d}_{a} A^{b} A^{c}.        # cdb(dRabcd00.101,dRabcd00)

   dRabcd01:=A^{a}\nabla_{a}{ @(dRabcd00) }.             # cdb(dRabcd01.101,dRabcd01)
   distribute     (dRabcd01)                             # cdb(dRabcd01.102,dRabcd01)
   product_rule   (dRabcd01)                             # cdb(dRabcd01.103,dRabcd01)
   distribute     (dRabcd01)                             # cdb(dRabcd01.104,dRabcd01)
   substitute     (dRabcd01,$\nabla_{a}{A^{b}} -> 0$)    # cdb(dRabcd01.105,dRabcd01)
   substitute     (dRabcd01,$\nabla_{a}{g^{b c}} -> 0$)  # cdb(dRabcd01.106,dRabcd01)

   sort_product   (dRabcd01)
   rename_dummies (dRabcd01)
   canonicalise   (dRabcd01)                             # cdb(dRabcd01.107,dRabcd01)
   dRabcd01 = flatten_Rabcd (dRabcd01)                   # cdb(dRabcd01.108,dRabcd01)

   dRabcd02:=A^{a}\nabla_{a}{ @(dRabcd01) }.             # cdb(dRabcd02.101,dRabcd02)
   distribute     (dRabcd02)                             # cdb(dRabcd02.102,dRabcd02)
   product_rule   (dRabcd02)                             # cdb(dRabcd02.103,dRabcd02)
   distribute     (dRabcd02)                             # cdb(dRabcd02.104,dRabcd02)
   substitute     (dRabcd02,$\nabla_{a}{A^{b}} -> 0$)    # cdb(dRabcd02.105,dRabcd02)
   substitute     (dRabcd02,$\nabla_{a}{g^{b c}} -> 0$)  # cdb(dRabcd02.106,dRabcd02)

   sort_product   (dRabcd02)
   rename_dummies (dRabcd02)
   canonicalise   (dRabcd02)                             # cdb(dRabcd02.107,dRabcd02)
   dRabcd02 = flatten_Rabcd (dRabcd02)                   # cdb(dRabcd02.108,dRabcd02)

   dRabcd03:=A^{a}\nabla_{a}{ @(dRabcd02) }.             # cdb(dRabcd03.101,dRabcd03)
   distribute     (dRabcd03)                             # cdb(dRabcd03.102,dRabcd03)
   product_rule   (dRabcd03)                             # cdb(dRabcd03.103,dRabcd03)
   distribute     (dRabcd03)                             # cdb(dRabcd03.104,dRabcd03)
   substitute     (dRabcd03,$\nabla_{a}{A^{b}} -> 0$)    # cdb(dRabcd03.105,dRabcd03)
   substitute     (dRabcd03,$\nabla_{a}{g^{b c}} -> 0$)  # cdb(dRabcd03.106,dRabcd03)

   sort_product   (dRabcd03)
   rename_dummies (dRabcd03)
   canonicalise   (dRabcd03)                             # cdb(dRabcd03.107,dRabcd03)
   dRabcd03 = flatten_Rabcd (dRabcd03)                   # cdb(dRabcd03.108,dRabcd03)

   dRabcd04:=A^{a}\nabla_{a}{ @(dRabcd03) }.
   distribute     (dRabcd04)
   product_rule   (dRabcd04)
   distribute     (dRabcd04)
   substitute     (dRabcd04,$\nabla_{a}{A^{b}} -> 0$)
   substitute     (dRabcd04,$\nabla_{a}{g^{b c}} -> 0$)

   sort_product   (dRabcd04)
   rename_dummies (dRabcd04)
   canonicalise   (dRabcd04)
   dRabcd04 = flatten_Rabcd (dRabcd04)

   dRabcd05:=A^{a}\nabla_{a}{ @(dRabcd04) }.
   distribute     (dRabcd05)
   product_rule   (dRabcd05)
   distribute     (dRabcd05)
   substitute     (dRabcd05,$\nabla_{a}{A^{b}} -> 0$)
   substitute     (dRabcd05,$\nabla_{a}{g^{b c}} -> 0$)

   sort_product   (dRabcd05)
   rename_dummies (dRabcd05)
   canonicalise   (dRabcd05)
   dRabcd05 = flatten_Rabcd (dRabcd05)

   def combine_nabla (obj):
       substitute (obj,$\nabla_{p}{\nabla_{q}{\nabla_{r}{\nabla_{s}{\nabla_{t}{A??}}}}}->\nabla_{p q r s t}{A??}$,repeat=True)
       substitute (obj,$\nabla_{p}{\nabla_{q}{\nabla_{r}{\nabla_{s}{A??}}}}->\nabla_{p q r s}{A??}$,repeat=True)
       substitute (obj,$\nabla_{p}{\nabla_{q}{\nabla_{r}{A??}}}->\nabla_{p q r}{A??}$,repeat=True)
       substitute (obj,$\nabla_{p}{\nabla_{q}{A??}}->\nabla_{p q}{A??}$,repeat=True)
       return obj

   dRabcd01 = combine_nabla (dRabcd01)
   dRabcd02 = combine_nabla (dRabcd02)
   dRabcd03 = combine_nabla (dRabcd03)
   dRabcd04 = combine_nabla (dRabcd04)
   dRabcd05 = combine_nabla (dRabcd05)

   end_stage_1 = time.time()
\end{cadabra}

\clearpage

\begin{dgroup*}
   \begin{dmath*} \cdb*{dRabcd00.101} \end{dmath*}
\end{dgroup*}

\begin{dgroup*}
   \begin{dmath*} \cdb*{dRabcd01.101} \end{dmath*}
   \begin{dmath*} \cdb*{dRabcd01.102} \end{dmath*}
   \begin{dmath*} \cdb*{dRabcd01.103} \end{dmath*}
   \begin{dmath*} \cdb*{dRabcd01.104} \end{dmath*}
   \begin{dmath*} \cdb*{dRabcd01.105} \end{dmath*}
   \begin{dmath*} \cdb*{dRabcd01.106} \end{dmath*}
   \begin{dmath*} \cdb*{dRabcd01.107} \end{dmath*}
   \begin{dmath*} \cdb*{dRabcd01.108} \end{dmath*}
\end{dgroup*}

\begin{dgroup*}
   \begin{dmath*} \cdb*{dRabcd02.101} \end{dmath*}
   \begin{dmath*} \cdb*{dRabcd02.102} \end{dmath*}
   \begin{dmath*} \cdb*{dRabcd02.103} \end{dmath*}
   \begin{dmath*} \cdb*{dRabcd02.104} \end{dmath*}
   \begin{dmath*} \cdb*{dRabcd02.105} \end{dmath*}
   \begin{dmath*} \cdb*{dRabcd02.106} \end{dmath*}
   \begin{dmath*} \cdb*{dRabcd02.107} \end{dmath*}
   \begin{dmath*} \cdb*{dRabcd02.108} \end{dmath*}
\end{dgroup*}

\begin{dgroup*}
   \begin{dmath*} \cdb*{dRabcd03.101} \end{dmath*}
   \begin{dmath*} \cdb*{dRabcd03.102} \end{dmath*}
   \begin{dmath*} \cdb*{dRabcd03.103} \end{dmath*}
   \begin{dmath*} \cdb*{dRabcd03.104} \end{dmath*}
   \begin{dmath*} \cdb*{dRabcd03.105} \end{dmath*}
   \begin{dmath*} \cdb*{dRabcd03.106} \end{dmath*}
   \begin{dmath*} \cdb*{dRabcd03.107} \end{dmath*}
   \begin{dmath*} \cdb*{dRabcd03.108} \end{dmath*}
\end{dgroup*}

\clearpage

% =================================================================================================
\section*{Stage 2: Symmetrised covariant derivatives of $B$}

\begin{cadabra}
   # compute the covariant derivatives of B^{a}_{b}, note B^{a}_{b,c} is zero, by choice
   # this method of computing covariant derivatives does not use auxillary fields

   beg_stage_2 = time.time()

   dBab00:=B^{a}_{b}.      # cdb(dBab00.201,dBab00)

   dBab01:=A^{c}\partial_{c}{ @(dBab00) } + \Gamma^{a}_{p q} W^{p}_{b} A^{q}
                                          - \Gamma^{p}_{b q} W^{a}_{p} A^{q}.
                                                         # cdb(dBab01.201,dBab01)
   distribute   (dBab01)                                 # cdb(dBab01.202,dBab01)
   product_rule (dBab01)                                 # cdb(dBab01.203,dBab01)
   distribute   (dBab01)                                 # cdb(dBab01.204,dBab01)
   substitute   (dBab01,$\partial_{a}{A^{b}} -> 0$)      # cdb(dBab01.205,dBab01)
   substitute   (dBab01,$\partial_{a}{B^{b}_{c}} -> 0$)  # cdb(dBab01.206,dBab01)
   substitute   (dBab01,$W^{a}_{b} -> @(dBab00)$)        # cdb(dBab01.207,dBab01)
   distribute   (dBab01)                                 # cdb(dBab01.208,dBab01)
   canonicalise (dBab01)                                 # cdb(dBab01.209,dBab01)

   dBab02:=A^{c}\partial_{c}{ @(dBab01) } + \Gamma^{a}_{p q} W^{p}_{b} A^{q}
                                          - \Gamma^{p}_{b q} W^{a}_{p} A^{q}.
                                                         # cdb(dBab02.201,dBab02)
   distribute   (dBab02)                                 # cdb(dBab02.202,dBab02)
   product_rule (dBab02)                                 # cdb(dBab02.203,dBab02)
   distribute   (dBab02)                                 # cdb(dBab02.204,dBab02)
   substitute   (dBab02,$\partial_{a}{A^{b}} -> 0$)      # cdb(dBab02.205,dBab02)
   substitute   (dBab02,$\partial_{a}{B^{b}_{c}} -> 0$)  # cdb(dBab02.206,dBab02)
   substitute   (dBab02,$W^{a}_{b} -> @(dBab01)$)        # cdb(dBab02.207,dBab02)
   distribute   (dBab02)                                 # cdb(dBab02.208,dBab02)
   canonicalise (dBab02)                                 # cdb(dBab02.209,dBab02)

   dBab03:=A^{c}\partial_{c}{ @(dBab02) } + \Gamma^{a}_{p q} W^{p}_{b} A^{q}
                                          - \Gamma^{p}_{b q} W^{a}_{p} A^{q}.
                                                         # cdb(dBab03.201,dBab03)
   distribute   (dBab03)                                 # cdb(dBab03.202,dBab03)
   product_rule (dBab03)                                 # cdb(dBab03.203,dBab03)
   distribute   (dBab03)                                 # cdb(dBab03.204,dBab03)
   substitute   (dBab03,$\partial_{a}{A^{b}} -> 0$)      # cdb(dBab03.205,dBab03)
   substitute   (dBab03,$\partial_{a}{B^{b}_{c}} -> 0$)  # cdb(dBab03.206,dBab03)
   substitute   (dBab03,$W^{a}_{b} -> @(dBab02)$)        # cdb(dBab03.207,dBab03)
   distribute   (dBab03)                                 # cdb(dBab03.208,dBab03)
   canonicalise (dBab03)                                 # cdb(dBab03.209,dBab03)

   dBab04:=A^{c}\partial_{c}{ @(dBab03) } + \Gamma^{a}_{p q} W^{p}_{b} A^{q}
                                          - \Gamma^{p}_{b q} W^{a}_{p} A^{q}.
   distribute   (dBab04)
   product_rule (dBab04)
   distribute   (dBab04)
   substitute   (dBab04,$\partial_{a}{A^{b}} -> 0$)
   substitute   (dBab04,$\partial_{a}{B^{b}_{c}} -> 0$)
   substitute   (dBab04,$W^{a}_{b} -> @(dBab03)$)
   distribute   (dBab04)
   canonicalise (dBab04)

   dBab05:=A^{c}\partial_{c}{ @(dBab04) } + \Gamma^{a}_{p q} W^{p}_{b} A^{q}
                                          - \Gamma^{p}_{b q} W^{a}_{p} A^{q}.
   distribute   (dBab05)
   product_rule (dBab05)
   distribute   (dBab05)
   substitute   (dBab05,$\partial_{a}{A^{b}} -> 0$)
   substitute   (dBab05,$\partial_{a}{B^{b}_{c}} -> 0$)
   substitute   (dBab05,$W^{a}_{b} -> @(dBab04)$)
   distribute   (dBab05)
   canonicalise (dBab05)

   end_stage_2 = time.time()
\end{cadabra}

\clearpage

\begin{dgroup*}
   \begin{dmath*} \cdb*{dBab00.201} \end{dmath*}
\end{dgroup*}

\begin{dgroup*}
   \begin{dmath*} \cdb*{dBab01.201} \end{dmath*}
   \begin{dmath*} \cdb*{dBab01.202} \end{dmath*}
   \begin{dmath*} \cdb*{dBab01.203} \end{dmath*}
   \begin{dmath*} \cdb*{dBab01.204} \end{dmath*}
   \begin{dmath*} \cdb*{dBab01.205} \end{dmath*}
   \begin{dmath*} \cdb*{dBab01.206} \end{dmath*}
   \begin{dmath*} \cdb*{dBab01.207} \end{dmath*}
   \begin{dmath*} \cdb*{dBab01.208} \end{dmath*}
   \begin{dmath*} \cdb*{dBab01.209} \end{dmath*}
\end{dgroup*}

\begin{dgroup*}
   \begin{dmath*} \cdb*{dBab02.201} \end{dmath*}
   \begin{dmath*} \cdb*{dBab02.202} \end{dmath*}
   \begin{dmath*} \cdb*{dBab02.203} \end{dmath*}
   \begin{dmath*} \cdb*{dBab02.204} \end{dmath*}
   \begin{dmath*} \cdb*{dBab02.205} \end{dmath*}
   \begin{dmath*} \cdb*{dBab02.206} \end{dmath*}
   \begin{dmath*} \cdb*{dBab02.207} \end{dmath*}
   \begin{dmath*} \cdb*{dBab02.208} \end{dmath*}
   \begin{dmath*} \cdb*{dBab02.209} \end{dmath*}
\end{dgroup*}

\begin{dgroup*}
   \begin{dmath*} \cdb*{dBab03.201} \end{dmath*}
   \begin{dmath*} \cdb*{dBab03.202} \end{dmath*}
   \begin{dmath*} \cdb*{dBab03.203} \end{dmath*}
   \begin{dmath*} \cdb*{dBab03.204} \end{dmath*}
   \begin{dmath*} \cdb*{dBab03.205} \end{dmath*}
   \begin{dmath*} \cdb*{dBab03.206} \end{dmath*}
   \begin{dmath*} \cdb*{dBab03.207} \end{dmath*}
   \begin{dmath*} \cdb*{dBab03.208} \end{dmath*}
   \begin{dmath*} \cdb*{dBab03.209} \end{dmath*}
\end{dgroup*}

\clearpage

% =================================================================================================
\section*{Stage 3: Impose the Riemann normal coordinate condition on covariant derivs of $B$}

\begin{cadabra}
   def impose_rnc (obj):
       # hide the derivatives of Gamma
       substitute (obj,$\partial_{d}{\Gamma^{a}_{b c}} -> zzz_{d}^{a}_{b c}$,repeat=True)
       substitute (obj,$\partial_{d e}{\Gamma^{a}_{b c}} -> zzz_{d e}^{a}_{b c}$,repeat=True)
       substitute (obj,$\partial_{d e f}{\Gamma^{a}_{b c}} -> zzz_{d e f}^{a}_{b c}$,repeat=True)
       substitute (obj,$\partial_{d e f g}{\Gamma^{a}_{b c}} -> zzz_{d e f g}^{a}_{b c}$,repeat=True)
       substitute (obj,$\partial_{d e f g h}{\Gamma^{a}_{b c}} -> zzz_{d e f g h}^{a}_{b c}$,repeat=True)
       # set Gamma to zero
       substitute (obj,$\Gamma^{a}_{b c} -> 0$,repeat=True)
       # recover the derivatives Gamma
       substitute (obj,$zzz_{d}^{a}_{b c} -> \partial_{d}{\Gamma^{a}_{b c}}$,repeat=True)
       substitute (obj,$zzz_{d e}^{a}_{b c} -> \partial_{d e}{\Gamma^{a}_{b c}}$,repeat=True)
       substitute (obj,$zzz_{d e f}^{a}_{b c} -> \partial_{d e f}{\Gamma^{a}_{b c}}$,repeat=True)
       substitute (obj,$zzz_{d e f g}^{a}_{b c} -> \partial_{d e f g}{\Gamma^{a}_{b c}}$,repeat=True)
       substitute (obj,$zzz_{d e f g h}^{a}_{b c} -> \partial_{d e f g h}{\Gamma^{a}_{b c}}$,repeat=True)
       return obj

   # switch to RNC

   beg_stage_3 = time.time()

   dBab01 = impose_rnc (dBab01)   # cdb (dBab01.301,dBab01)
   dBab02 = impose_rnc (dBab02)   # cdb (dBab02.301,dBab02)
   dBab03 = impose_rnc (dBab03)   # cdb (dBab03.301,dBab03)
   dBab04 = impose_rnc (dBab04)   # cdb (dBab04.301,dBab04)
   dBab05 = impose_rnc (dBab05)   # cdb (dBab05.301,dBab05)

   end_stage_3 = time.time()
\end{cadabra}

\clearpage

\begin{dgroup*}
   \begin{dmath*} \cdb*{dBab01.301} \end{dmath*}
   \begin{dmath*} \cdb*{dBab02.301} \end{dmath*}
   \begin{dmath*} \cdb*{dBab03.301} \end{dmath*}
   \begin{dmath*} \cdb*{dBab04.301} \end{dmath*}
   \begin{dmath*} \cdb*{dBab05.301} \end{dmath*}
\end{dgroup*}

\clearpage

% =================================================================================================
\section*{Stage 4: Replace covariant derivs of $B$ with partial derivs of $\Gamma$}

\begin{cadabra}
   # substitute covariant derivs of B^{a}_{b} into covariant derivs of R^{a}_{bcd}B^{d}_{a}
   # this produces expressions for the partial derivs of Rabcd its covariant derivs and partial derivs of Gamma
   # the partial derivs of Gamma will be eliminted later by using results imported from dGamma.json

   beg_stage_4 = time.time()

   substitute (dRabcd01,$A^{c}\nabla_{c}{B^{a}_{b}} -> @(dBab01)$,repeat=True);   distribute (dRabcd01)
   substitute (dRabcd02,$A^{c}\nabla_{c}{B^{a}_{b}} -> @(dBab01)$,repeat=True);   distribute (dRabcd02)
   substitute (dRabcd03,$A^{c}\nabla_{c}{B^{a}_{b}} -> @(dBab01)$,repeat=True);   distribute (dRabcd03)
   substitute (dRabcd04,$A^{c}\nabla_{c}{B^{a}_{b}} -> @(dBab01)$,repeat=True);   distribute (dRabcd04)
   substitute (dRabcd05,$A^{c}\nabla_{c}{B^{a}_{b}} -> @(dBab01)$,repeat=True);   distribute (dRabcd05)

   substitute (dRabcd02,$A^{c}A^{d}\nabla_{c d}{B^{a}_{b}} -> @(dBab02)$,repeat=True);   distribute (dRabcd02)
   substitute (dRabcd03,$A^{c}A^{d}\nabla_{c d}{B^{a}_{b}} -> @(dBab02)$,repeat=True);   distribute (dRabcd03)
   substitute (dRabcd04,$A^{c}A^{d}\nabla_{c d}{B^{a}_{b}} -> @(dBab02)$,repeat=True);   distribute (dRabcd04)
   substitute (dRabcd05,$A^{c}A^{d}\nabla_{c d}{B^{a}_{b}} -> @(dBab02)$,repeat=True);   distribute (dRabcd05)

   substitute (dRabcd03,$A^{c}A^{d}A^{e}\nabla_{c d e}{B^{a}_{b}} -> @(dBab03)$,repeat=True);   distribute (dRabcd03)
   substitute (dRabcd04,$A^{c}A^{d}A^{e}\nabla_{c d e}{B^{a}_{b}} -> @(dBab03)$,repeat=True);   distribute (dRabcd04)
   substitute (dRabcd05,$A^{c}A^{d}A^{e}\nabla_{c d e}{B^{a}_{b}} -> @(dBab03)$,repeat=True);   distribute (dRabcd05)

   substitute (dRabcd04,$A^{c}A^{d}A^{e}A^{f}\nabla_{c d e f}{B^{a}_{b}} -> @(dBab04)$,repeat=True); distribute (dRabcd04)
   substitute (dRabcd05,$A^{c}A^{d}A^{e}A^{f}\nabla_{c d e f}{B^{a}_{b}} -> @(dBab04)$,repeat=True); distribute (dRabcd05)

   substitute (dRabcd05,$A^{c}A^{d}A^{e}A^{f}A^{g}\nabla_{c d e f g}{B^{a}_{b}} -> @(dBab05)$,repeat=True); distribute (dRabcd05)

   # no longer need B, so let's get rid of it

   # two subtle tricks are used here
   # 1) rename A and B as A002 and A001 before sort_product,
   #    this ensures B will be to left of A after the sort
   # 2) indices on B changed from B^{a}_{b} to B_{b}^{a},
   #    this ensures that after factor_out B will have dummy indices B_{a}^{b}

   def remove_Bab (obj):
       foo := @(obj).
       substitute     (foo,$A^{a}->A002^{a},B^{a}_{b}->A001_{b}^{a}$)  # need this to sort B to the left of A
       sort_product   (foo)
       rename_dummies (foo)
       factor_out     (foo,$A001^{a?}_{b?},A002^{c?}$)
       substitute     (foo,$A001_{a}^{b}->1,A002^{a}->A^{a}$)  # recover A and set B = 1, free indices now ^{a}_{b}
       return foo

   dRabcd01 = remove_Bab (dRabcd01)   # cdb(dRabcd01.401,dRabcd01)
   dRabcd02 = remove_Bab (dRabcd02)   # cdb(dRabcd02.401,dRabcd02)
   dRabcd03 = remove_Bab (dRabcd03)   # cdb(dRabcd03.401,dRabcd03)
   dRabcd04 = remove_Bab (dRabcd04)   # cdb(dRabcd04.401,dRabcd04)
   dRabcd05 = remove_Bab (dRabcd05)   # cdb(dRabcd05.401,dRabcd05)

   end_stage_4 = time.time()
\end{cadabra}

\clearpage

\begin{dgroup*}
   \begin{dmath*} \cdb*{dRabcd01.401} \end{dmath*}
   \begin{dmath*} \cdb*{dRabcd02.401} \end{dmath*}
   \begin{dmath*} \cdb*{dRabcd03.401} \end{dmath*}
   \begin{dmath*} \cdb*{dRabcd04.401} \end{dmath*}
   \begin{dmath*} \cdb*{dRabcd05.401} \end{dmath*}
\end{dgroup*}

\clearpage

% =================================================================================================
\section*{Stage 5: Replace partial derivs of $\Gamma$ with partial derivs of $R$}

\begin{cadabra}
   import cdblib

   beg_stage_5 = time.time()

   dGamma01 = cdblib.get ('dGamma01','dGamma.json')  # cdb(dGamma01.500,dGamma01)
   dGamma02 = cdblib.get ('dGamma02','dGamma.json')  # cdb(dGamma02.500,dGamma02)
   dGamma03 = cdblib.get ('dGamma03','dGamma.json')  # cdb(dGamma03.500,dGamma03)
   dGamma04 = cdblib.get ('dGamma04','dGamma.json')  # cdb(dGamma04.500,dGamma04)
   dGamma05 = cdblib.get ('dGamma05','dGamma.json')  # cdb(dGamma05.500,dGamma05)

   distribute (dRabcd01)   # cdb(dRabcd01.500,dRabcd01)
   distribute (dRabcd02)   # cdb(dRabcd02.500,dRabcd02)
   distribute (dRabcd03)   # cdb(dRabcd03.500,dRabcd03)
   distribute (dRabcd04)   # cdb(dRabcd04.500,dRabcd04)
   distribute (dRabcd05)   # cdb(dRabcd05.500,dRabcd05)

   # use dGamma to eliminate the partial derivs of Gamma
   # this will introduces some lower order partial dervis of Rabcd on the rhs
   # these extra partial derivs of Rabcd will be eliminated (later) by substiting lower order dRabcd into the higher order dRabcd

   substitute (dRabcd02,$A^{c}A^{b}\partial_{c}{\Gamma^{a}_{d b}} -> @(dGamma01)$,repeat=True)                # cdb(dRabcd02.501,dRabcd02)
   substitute (dRabcd02,$A^{c}A^{b}\partial_{c}{\Gamma^{a}_{b d}} -> @(dGamma01)$,repeat=True)                # cdb(dRabcd02.502,dRabcd02)
   distribute (dRabcd02)                                                                                      # cdb(dRabcd02.503,dRabcd02)
   sort_product   (dRabcd02)                                                                                  # cdb(dRabcd02.504,dRabcd02)
   rename_dummies (dRabcd02)                                                                                  # cdb(dRabcd02.505,dRabcd02)


   substitute (dRabcd03,$A^{c}A^{b}A^{e}\partial_{c e}{\Gamma^{a}_{d b}} -> @(dGamma02)$,repeat=True)         # cdb(dRabcd03.501,dRabcd03)
   substitute (dRabcd03,$A^{c}A^{b}A^{e}\partial_{c e}{\Gamma^{a}_{b d}} -> @(dGamma02)$,repeat=True)         # cdb(dRabcd03.502,dRabcd03)
   substitute (dRabcd03,$A^{c}A^{b}\partial_{c}{\Gamma^{a}_{d b}} -> @(dGamma01)$,repeat=True)                # cdb(dRabcd03.503,dRabcd03)
   substitute (dRabcd03,$A^{c}A^{b}\partial_{c}{\Gamma^{a}_{b d}} -> @(dGamma01)$,repeat=True)                # cdb(dRabcd03.504,dRabcd03)
   distribute (dRabcd03)                                                                                      # cdb(dRabcd03.505,dRabcd03)
   sort_product   (dRabcd03)                                                                                  # cdb(dRabcd03.506,dRabcd03)
   rename_dummies (dRabcd03)                                                                                  # cdb(dRabcd03.507,dRabcd03)

   substitute (dRabcd04,$A^{c}A^{b}A^{e}A^{f}\partial_{c e f}{\Gamma^{a}_{d b}} -> @(dGamma03)$,repeat=True)  # cdb(dRabcd04.501,dRabcd04)
   substitute (dRabcd04,$A^{c}A^{b}A^{e}A^{f}\partial_{c e f}{\Gamma^{a}_{b d}} -> @(dGamma03)$,repeat=True)  # cdb(dRabcd04.502,dRabcd04)
   substitute (dRabcd04,$A^{c}A^{b}A^{e}\partial_{c e}{\Gamma^{a}_{d b}} -> @(dGamma02)$,repeat=True)         # cdb(dRabcd04.503,dRabcd04)
   substitute (dRabcd04,$A^{c}A^{b}A^{e}\partial_{c e}{\Gamma^{a}_{b d}} -> @(dGamma02)$,repeat=True)         # cdb(dRabcd04.504,dRabcd04)
   substitute (dRabcd04,$A^{c}A^{b}\partial_{c}{\Gamma^{a}_{d b}} -> @(dGamma01)$,repeat=True)                # cdb(dRabcd04.505,dRabcd04)
   substitute (dRabcd04,$A^{c}A^{b}\partial_{c}{\Gamma^{a}_{b d}} -> @(dGamma01)$,repeat=True)                # cdb(dRabcd04.506,dRabcd04)
   distribute (dRabcd04)                                                                                      # cdb(dRabcd04.507,dRabcd04)
   sort_product   (dRabcd04)                                                                                  # cdb(dRabcd04.508,dRabcd04)
   rename_dummies (dRabcd04)                                                                                  # cdb(dRabcd04.509,dRabcd04)

   substitute (dRabcd05,$A^{c}A^{b}A^{e}A^{f}A^{g}\partial_{c e f g}{\Gamma^{a}_{d b}} -> @(dGamma04)$,repeat=True)
   substitute (dRabcd05,$A^{c}A^{b}A^{e}A^{f}A^{g}\partial_{c e f g}{\Gamma^{a}_{b d}} -> @(dGamma04)$,repeat=True)
   substitute (dRabcd05,$A^{c}A^{b}A^{e}A^{f}\partial_{c e f}{\Gamma^{a}_{d b}} -> @(dGamma03)$,repeat=True)
   substitute (dRabcd05,$A^{c}A^{b}A^{e}A^{f}\partial_{c e f}{\Gamma^{a}_{b d}} -> @(dGamma03)$,repeat=True)
   substitute (dRabcd05,$A^{c}A^{b}A^{e}\partial_{c e}{\Gamma^{a}_{d b}} -> @(dGamma02)$,repeat=True)
   substitute (dRabcd05,$A^{c}A^{b}A^{e}\partial_{c e}{\Gamma^{a}_{b d}} -> @(dGamma02)$,repeat=True)
   substitute (dRabcd05,$A^{c}A^{b}\partial_{c}{\Gamma^{a}_{d b}} -> @(dGamma01)$,repeat=True)
   substitute (dRabcd05,$A^{c}A^{b}\partial_{c}{\Gamma^{a}_{b d}} -> @(dGamma01)$,repeat=True)
   distribute (dRabcd05)
   sort_product   (dRabcd05)
   rename_dummies (dRabcd05)

   end_stage_5 = time.time()
\end{cadabra}

\clearpage

\begin{dgroup*}
   \begin{dmath*} \cdb*{dRabcd01.500} \end{dmath*}
\end{dgroup*}

\begin{dgroup*}
   \begin{dmath*} \cdb*{dRabcd02.500} \end{dmath*}
   \begin{dmath*} \cdb*{dRabcd02.501} \end{dmath*}
   \begin{dmath*} \cdb*{dRabcd02.502} \end{dmath*}
   \begin{dmath*} \cdb*{dRabcd02.503} \end{dmath*}
   \begin{dmath*} \cdb*{dRabcd02.504} \end{dmath*}
   \begin{dmath*} \cdb*{dRabcd02.505} \end{dmath*}
\end{dgroup*}

\begin{dgroup*}
   \begin{dmath*} \cdb*{dRabcd03.500} \end{dmath*}
   \begin{dmath*} \cdb*{dRabcd03.501} \end{dmath*}
   \begin{dmath*} \cdb*{dRabcd03.502} \end{dmath*}
   \begin{dmath*} \cdb*{dRabcd03.503} \end{dmath*}
   \begin{dmath*} \cdb*{dRabcd03.504} \end{dmath*}
   \begin{dmath*} \cdb*{dRabcd03.505} \end{dmath*}
   \begin{dmath*} \cdb*{dRabcd03.506} \end{dmath*}
   \begin{dmath*} \cdb*{dRabcd03.507} \end{dmath*}
\end{dgroup*}

\begin{dgroup*}
   \begin{dmath*} \cdb*{dRabcd04.500} \end{dmath*}
   \begin{dmath*} \cdb*{dRabcd04.501} \end{dmath*}
   \begin{dmath*} \cdb*{dRabcd04.502} \end{dmath*}
   \begin{dmath*} \cdb*{dRabcd04.503} \end{dmath*}
   \begin{dmath*} \cdb*{dRabcd04.504} \end{dmath*}
   \begin{dmath*} \cdb*{dRabcd04.505} \end{dmath*}
   \begin{dmath*} \cdb*{dRabcd04.506} \end{dmath*}
   \begin{dmath*} \cdb*{dRabcd04.507} \end{dmath*}
   \begin{dmath*} \cdb*{dRabcd04.508} \end{dmath*}
   \begin{dmath*} \cdb*{dRabcd04.509} \end{dmath*}
\end{dgroup*}

\clearpage

% =================================================================================================
\section*{Stage 6: Replace partial derivs of $R$ with covariant derivs of $R$}

\begin{cadabra}
   # now eliminate remaining partial derivs of Rabcd by substitution from the lower order dRabcd

   # note that
   #   dRabcd01 = R^a_{cdb,e} A^c A^d A^e
   #   dRabcd02 = R^a_{cdb,ef} A^c A^d A^e A^f
   #   dRabcd03 = R^a_{cdb,efg} A^c A^d A^e A^f A^g

   # thus we can use
   #   dRabcd01 to eliminate 1st partial derivs of R in dRabcd03, dRabcd04, etc.
   #   dRabcd02 to eliminate 2nd partial derivs of R in dRabcd04, dRabcd05, etc.
   #   dRabcd03 to eliminate 3rd partial derivs of R in dRabcd05, dRabcd06, etc.

   beg_stage_6 = time.time()

   substitute (dRabcd03,$A^{c}A^{d}A^{e}\partial_{e}{R^{a}_{c d b}} -> @(dRabcd01)$,repeat=True)         # cdb(dRabcd03.601,dRabcd03)
   distribute (dRabcd03)                                                                                 # cdb(dRabcd03.602,dRabcd03)

   # note: dRabcd04 and dRabcd05 unused in this code (or any other code)

   substitute (dRabcd04,$A^{c}A^{d}A^{e}A^{f}\partial_{e f}{R^{a}_{c d b}} -> @(dRabcd02)$,repeat=True)  # cdb(dRabcd04.601,dRabcd04)
   substitute (dRabcd04,$A^{c}A^{d}A^{e}\partial_{e}{R^{a}_{c d b}} -> @(dRabcd01)$,repeat=True)         # cdb(dRabcd04.602,dRabcd04)
   distribute (dRabcd04)                                                                                 # cdb(dRabcd04.603,dRabcd04)

   substitute (dRabcd05,$A^{c}A^{d}A^{e}A^{f}A^{g}\partial_{e f g}{R^{a}_{c d b}} -> @(dRabcd03)$,repeat=True)
   substitute (dRabcd05,$A^{c}A^{d}A^{e}A^{f}\partial_{e f}{R^{a}_{c d b}} -> @(dRabcd02)$,repeat=True)
   substitute (dRabcd05,$A^{c}A^{d}A^{e}\partial_{e}{R^{a}_{c d b}} -> @(dRabcd01)$,repeat=True)
   distribute (dRabcd05)

   end_stage_6 = time.time()
\end{cadabra}

\clearpage

\begin{dgroup*}
   \begin{dmath*} \cdb*{dRabcd03.601} \end{dmath*}
   \begin{dmath*} \cdb*{dRabcd03.602} \end{dmath*}
\end{dgroup*}

\begin{dgroup*}
   \begin{dmath*} \cdb*{dRabcd04.601} \end{dmath*}
   \begin{dmath*} \cdb*{dRabcd04.602} \end{dmath*}
   \begin{dmath*} \cdb*{dRabcd04.603} \end{dmath*}
\end{dgroup*}

\clearpage

% =================================================================================================
\section*{Stage 7: Reformatting}

\begin{cadabra}
   beg_stage_7 = time.time()

   dRabcd01 = flatten_Rabcd (dRabcd01)  # cdb(dRabcd01.701,dRabcd01)
   dRabcd02 = flatten_Rabcd (dRabcd02)  # cdb(dRabcd02.701,dRabcd02)
   dRabcd03 = flatten_Rabcd (dRabcd03)  # cdb(dRabcd03.701,dRabcd03)
   dRabcd04 = flatten_Rabcd (dRabcd04)  # cdb(dRabcd04.701,dRabcd04)
   dRabcd05 = flatten_Rabcd (dRabcd05)  # cdb(dRabcd05.701,dRabcd05)

   canonicalise (dRabcd01)   # cdb(dRabcd01.702,dRabcd01)
   canonicalise (dRabcd02)   # cdb(dRabcd02.702,dRabcd02)
   canonicalise (dRabcd03)   # cdb(dRabcd03.702,dRabcd03)
   canonicalise (dRabcd04)   # cdb(dRabcd04.702,dRabcd04)
   canonicalise (dRabcd05)   # cdb(dRabcd05.702,dRabcd05)

   end_stage_7 = time.time()

   # cdbBeg (timing)
   print ("Stage 1: {:7.1f} secs\\hfill\\break".format(end_stage_1-beg_stage_1))
   print ("Stage 2: {:7.1f} secs\\hfill\\break".format(end_stage_2-beg_stage_2))
   print ("Stage 3: {:7.1f} secs\\hfill\\break".format(end_stage_3-beg_stage_3))
   print ("Stage 4: {:7.1f} secs\\hfill\\break".format(end_stage_4-beg_stage_4))
   print ("Stage 5: {:7.1f} secs\\hfill\\break".format(end_stage_5-beg_stage_5))
   print ("Stage 6: {:7.1f} secs\\hfill\\break".format(end_stage_6-beg_stage_6))
   print ("Stage 7: {:7.1f} secs".format(end_stage_7-beg_stage_7))
   # cdbEnd (timing)

\end{cadabra}

\clearpage

\begin{dgroup*}
   \begin{dmath*} \cdb*{dRabcd01.701} \end{dmath*}
   \begin{dmath*} \cdb*{dRabcd02.701} \end{dmath*}
   \begin{dmath*} \cdb*{dRabcd03.701} \end{dmath*}
   \begin{dmath*} \cdb*{dRabcd04.701} \end{dmath*}
   % \begin{dmath*} \cdb*{dRabcd05.701} \end{dmath*}
\end{dgroup*}

\clearpage

\begin{dgroup*}
   \begin{dmath*} \cdb*{dRabcd01.702} \end{dmath*}
   \begin{dmath*} \cdb*{dRabcd02.702} \end{dmath*}
   \begin{dmath*} \cdb*{dRabcd03.702} \end{dmath*}
   \begin{dmath*} \cdb*{dRabcd04.702} \end{dmath*}
   \begin{dmath*} \cdb*{dRabcd05.702} \end{dmath*}
\end{dgroup*}

\clearpage

\begin{cadabra}
   cdblib.create ('dRabcd.json')

   cdblib.put ('dRabcd01',dRabcd01,'dRabcd.json')
   cdblib.put ('dRabcd02',dRabcd02,'dRabcd.json')
   cdblib.put ('dRabcd03',dRabcd03,'dRabcd.json')
   cdblib.put ('dRabcd04',dRabcd04,'dRabcd.json')
   cdblib.put ('dRabcd05',dRabcd05,'dRabcd.json')

\end{cadabra}

% =================================================================================================
% the remaining code is just for pretty printing

\clearpage

\begin{cadabra}
   # note: keeping numbering as is (out of order) to ensure R appears before \nabla R etc.
   def product_sort (obj):
       substitute (obj,$ A^{a}                            -> A001^{a}               $)
       substitute (obj,$ x^{a}                            -> A002^{a}               $)
       substitute (obj,$ g^{a b}                          -> A003^{a b}             $)
       substitute (obj,$ \nabla_{e f g h}{R_{a b c d}}    -> A008_{a b c d e f g h} $)
       substitute (obj,$ \nabla_{e f g}{R_{a b c d}}      -> A007_{a b c d e f g}   $)
       substitute (obj,$ \nabla_{e f}{R_{a b c d}}        -> A006_{a b c d e f}     $)
       substitute (obj,$ \nabla_{e}{R_{a b c d}}          -> A005_{a b c d e}       $)
       substitute (obj,$ R_{a b c d}                      -> A004_{a b c d}         $)
       sort_product   (obj)
       rename_dummies (obj)
       substitute (obj,$ A001^{a}                  -> A^{a}                         $)
       substitute (obj,$ A002^{a}                  -> x^{a}                         $)
       substitute (obj,$ A003^{a b}                -> g^{a b}                       $)
       substitute (obj,$ A004_{a b c d}            -> R_{a b c d}                   $)
       substitute (obj,$ A005_{a b c d e}          -> \nabla_{e}{R_{a b c d}}       $)
       substitute (obj,$ A006_{a b c d e f}        -> \nabla_{e f}{R_{a b c d}}     $)
       substitute (obj,$ A007_{a b c d e f g}      -> \nabla_{e f g}{R_{a b c d}}   $)
       substitute (obj,$ A008_{a b c d e f g h}    -> \nabla_{e f g h}{R_{a b c d}} $)

       return obj

   def reformat (obj,scale):
       foo  = Ex(str(scale))
       bah := @(foo) @(obj).
       distribute     (bah)
       bah = product_sort (bah)
       rename_dummies (bah)
       canonicalise   (bah)
       factor_out     (bah,$A^{a?}$)
       ans := @(bah).
       return ans

   scaled1 = reformat (dRabcd01, 1)    # cdb(scaled1.601,scaled1)
   scaled2 = reformat (dRabcd02, 1)    # cdb(scaled2.601,scaled2)
   scaled3 = reformat (dRabcd03,-2)    # cdb(scaled3.601,scaled3)
   scaled4 = reformat (dRabcd04,-5)    # cdb(scaled4.601,scaled4)
   scaled5 = reformat (dRabcd05,-3)    # cdb(scaled5.601,scaled5)

\end{cadabra}

\clearpage

% =================================================================================================
\section*{Symmetrised partial derivatives of $R^a{}_{bcd}$}
\begin{dgroup*}
   \begin{dmath*}    A^c A^d A^e R^a{}_{cdb,e} = \cdb{scaled1.601} \end{dmath*}
   \begin{dmath*}    A^c A^d A^e A^{f} R^a{}_{cdb,ef} = \cdb{scaled2.601} \end{dmath*}
   \begin{dmath*} -2 A^c A^d A^e A^{f} A^{g} R^a{}_{cdb,efg} = \cdb{scaled3.601} \end{dmath*}
   \begin{dmath*} -5 A^c A^d A^e A^{f} A^{g} A^{h} R^a{}_{cdb,efgh} = \cdb{scaled4.601} \end{dmath*}
   \begin{dmath*} -3 A^c A^d A^e A^{f} A^{g} A^{h} A^{i}R^a{}_{cdb,efghi} = \cdb{scaled5.601} \end{dmath*}
\end{dgroup*}

% LCB: only need dRabcd01,02,03. So I could save time by not
%      computing dRabcd04,05

% computing just dRabcd0n to n=4 takes about 1 min 40 sec.
% but going to n=5 takes about 7 min

\clearpage

% =================================================================================================
% export selected objects, these will later be imported into a library
% these are the objects that will appear in the paper

\begin{cadabra}
   substitute (scaled1,$A^{a}->1$)
   substitute (scaled2,$A^{a}->1$)
   substitute (scaled3,$A^{a}->1$)
   substitute (scaled4,$A^{a}->1$)
   substitute (scaled5,$A^{a}->1$)

   cdblib.create ('dRabcd.export')

   # 6th order dRabcd, scaled
   cdblib.put ('dRabcd61scaled',scaled1,'dRabcd.export')
   cdblib.put ('dRabcd62scaled',scaled2,'dRabcd.export')
   cdblib.put ('dRabcd63scaled',scaled3,'dRabcd.export')
   cdblib.put ('dRabcd64scaled',scaled4,'dRabcd.export')
   cdblib.put ('dRabcd65scaled',scaled5,'dRabcd.export')

   checkpoint.append (scaled1)
   checkpoint.append (scaled2)
   checkpoint.append (scaled3)
   checkpoint.append (scaled4)
   checkpoint.append (scaled5)

\end{cadabra}

\clearpage

% =================================================================================================
\section*{Timing}

\cdb{timing}

% =================================================================================================
% export checkpoints in json format

\bgroup
\CdbSetup{action=hide}
\begin{cadabra}
   for i in range( len(checkpoint) ):
      cdblib.put ('check{:03d}'.format(i),checkpoint[i],checkpoint_file)
\end{cadabra}
\egroup

\end{document}


\begin{dgroup*}
   \begin{dmath*}    A^c A^d A^e R^a{}_{cdb,e} = \cdb{scaled1.601} \end{dmath*}
   \begin{dmath*}    A^c A^d A^e A^{f} R^a{}_{cdb,ef} = \cdb{scaled2.601} \end{dmath*}
   \begin{dmath*} -2 A^c A^d A^e A^{f} A^{g} R^a{}_{cdb,efg} = \cdb{scaled3.601} \end{dmath*}
   \begin{dmath*} -5 A^c A^d A^e A^{f} A^{g} A^{h} R^a{}_{cdb,efgh} = \cdb{scaled4.601} \end{dmath*}
   \begin{dmath*} -3 A^c A^d A^e A^{f} A^{g} A^{h} A^{i}R^a{}_{cdb,efghi} = \cdb{scaled5.601} \end{dmath*}
\end{dgroup*}

\clearpage

% =================================================================================================
\section*{The generalised connection in RNC}
\def\Date{19 Jan 2024}
% \def\FileID{file:}

\documentclass[12pt]{cdblatex}

\begin{document}

% =================================================================================================
% create checkpoint file

\bgroup
\CdbSetup{action=hide}
\begin{cadabra}
   import cdblib
   checkpoint_file = 'tests/semantic/output/genGamma.json'
   cdblib.create (checkpoint_file)
   checkpoint = []
\end{cadabra}
\egroup

% =================================================================================================
\section*{The generalised connections}

The generalised connections may be computed recursively using
\begin{align}
   \label{eq:GenGamma}
\Gamma^{a}{}_{b\uc d} = \Gamma^{a}{}_{(b\uc,d)}
                - (n+1) \Gamma^{a}{}_{p(\uc}
                        \Gamma^{p}{}_{bd)}
\end{align}
where $\uc$ contains $n>0$ indices. The sequence begins with the standard metric compatible connection
\begin{align}
   \Gamma^{d}_{ab} = \frac{1}{2} g^{dc}\left( g_{cb,a} + g_{ac,b} - g_{ab,c} \right)
\end{align}

Here we will use the results of {\tt metric.tex} and {\tt metric-inv.tex} to compute the metric connection
$\Gamma^{d}_{ab}$. But since the $g_{ab}$ and $g^{ab}$ provided by those codes are truncated at a
particular order in the curvatures (and thus are only approximations to the $g_{ab}$ and $g^{ab}$) similar
truncations will arise in the $\Gamma^{a}{}_{b\uc d}$.

Approximations will be denoted by the addition of an overbar to an object. In this notation the metric
$g$ can be written as
\begin{align}
   g = {\bar g} + \BigO{\eps^n}
\end{align}
in which ${\bar g}$ is the truncated polynomial approximation to $g$ and $\BigO{\eps^n}$ is the error term
(containing terms no smaller than $\eps^n$). The polynomial structure of ${\bar g}$ can be expressed as
\begin{align}
   \gabBar = \ngabBar{0}
           + \ngabBar{1}
           + \ngabBar{2}
           + \dots
           + \ngabBar{p}
\end{align}
in which each terms like $\overset{m}{\bar g}$ contains only terms of order $m$. This notation will be applied
to other quantities in particular the generalised connections.

The notation $\BigO{\eps^n}$ denotes terms in the curvatures that are of order $\eps^n$. What does this actually mean?
Each term in $R$ is of order $\eps^2$ while each derivative of $R$ carries an extra power of $\eps$.
Thus $R\cdot R = \BigO{\eps^4}$, $R\cdot R\cdot\nabla R = \BigO{\eps^7}$ and $R\cdot R\cdot\nabla^2R = \BigO{\eps^8}$.

We will also adopt the convention that an object is said to be an $\BigO{\eps^{m}}$ approximation when the corresponding error term is $\BigO{\eps^{m+1}}$.

Consider the $\BigO{\eps^{m}}$ approximation of the generalised connection, namely,
\begin{align}
   \GammaBar^{a}{}_{b\ucn d}
      = \nGammaBar{0}^{a}{}_{b\ucn d}
      + \nGammaBar{1}^{a}{}_{b\ucn d}
      + \nGammaBar{2}^{a}{}_{b\ucn d}
      + \dots
      + \nGammaBar{m}^{a}{}_{b\ucn d}
\end{align}
where $\ucn$ denotes a set of indices such as $c_1c_2c_3\dots c_n$.

The first thing to note is that
\begin{align}
   0 = \nGammaBar{1+n}^{a}{}_{(b\ucn,d)}
\end{align}

There are two proofs of this claim. For the first proof, note (by inspection) that the order $\BigO{\eps^p}$
approximation for $\GammaBar^{a}{}_{b\ucn d}$ is a polynomial in $x$ of degree $p-n-1$. Thus
$\nGammaBar{1+n}^{a}{}_{(b\ucn,d)}$ is a polynomial in $x$ of degree
zero, i.e., a constant. However, we know that all generalised connections vanish at the origin of the RNC frame.
Thus this constant must be zero. The second proof makes explicit use of the first (and second?) Bianchi identity,
that is $0=R_{a(bcd)}$. The term $\nGammaBar{1+n}^{a}{}_{(b\ucn,d)}$ will
itself consist of a sum of terms built from combinations of $x$, $R$, $\nabla R$ etc. The $x^{a}$ will always
appear in a contraction with one of the indices on $R_{abcd}$ or one of its derivatives. Consider any one of
these terms, denoted by $A$, and assume for the moment that $1+n$ is an even number, say $1+n=2p$. The indices
$(b\ucn,d)$ must somehow be assigned to the factors that comprise $A$. Our aim is to show that at least one $R$
factor in $A$ will receive 3 of these indices and thus by the Bianchi identities will be zero. If there are too
many $R$ factors then the Bianchi identities will not come into play. So how many $R$ factors can we expect?
Since $A$ is a term in an $\BigO{\eps^{(n+1)}}$ approximation there can be no more than $(n+1)/2=p$ Riemann
factors. There will be at least one $x$ term contracted with one of the $p$ Riemann factors. However, we have
$n+2=2p+1$ indices to distribute amongst the $x$ term and $p$ Riemann factors. One of the indices is a derivative
index and will have nett effect of transferring that index from $x$ to one of the Riemann factors. The remaining
$2p$ indices must be distributed amongst the $p$ Riemann factors. It is not possible to avoid assigning three
indices to at least one of the Riemann factors. Thus, by the Bianchi identity, this $A$ term must vanish. Similar
arguments can be applied to the other cases where the $A$ terms consists of products of $R$ and its derivatives
and in the case where $n+1$ is an odd number. The analysis always comes down to the distribution of the indices
$(b\ucn,d)$ amongst the factors of a typical $A$ term. In all cases the Bianchi identity will enter the play and
force $A$ to be zero.

A corollary of the second proof is that for all $m<n+2$
\begin{align}
   0 = \nGammaBar{m}^{a}{}_{b\ucn d}
\end{align}
The proof follows exactly that of the second proof given above.

We can use the above results to streamline the computation of the generalised connections.
We begin with the formal expression for the $\BigO{\eps^m}$ approximations
\begin{align}
   \Gamma^{a}{}_{bc}
      &= \nGammaBar{2}^{a}{}_{bc}
       + \nGammaBar{3}^{a}{}_{bc}
       + \nGammaBar{4}^{a}{}_{bc}
       + \dots
       + \nGammaBar{m}^{a}{}_{bc}\\
   \Gamma^{a}{}_{b\uc}
      &= \nGammaBar{n+1}^{a}{}_{b\uc}
       + \nGammaBar{n+2}^{a}{}_{b\uc}
       + \nGammaBar{n+3}^{a}{}_{b\uc}
       + \dots
       + \nGammaBar{m}^{a}{}_{b\uc}\\
   \Gamma^{a}{}_{b\uc d}
      &= \nGammaBar{n+2}^{a}{}_{b\uc d}
       + \nGammaBar{n+3}^{a}{}_{b\uc d}
       + \nGammaBar{n+4}^{a}{}_{b\uc d}
       + \dots
       + \nGammaBar{m}^{a}{}_{b\uc d}\label{eq:GenGammaA}
\end{align}
These can be substituted into equation (\ref{eq:GenGamma}) with the result
\def\m{\hskip 4pt}
\begin{align}
   \Gamma^{a}{}_{b\uc d}
      &= \nGammaBar{n+1}^{a}{}_{(b\uc,d)}
       + \nGammaBar{n+2}^{a}{}_{(b\uc,d)}
       + \nGammaBar{n+3}^{a}{}_{(b\uc,d)}
       + \dots
       + \nGammaBar{m}^{a}{}_{(b\uc,d)}
       -(n+1)\left(\m \nGammaBar{n+1}^{a}{}_{p\uc}
                    + \nGammaBar{n+2}^{a}{}_{p\uc}
                    + \nGammaBar{n+3}^{a}{}_{p\uc}
                    + \dots
                    + \nGammaBar{m}^{a}{}_{p\uc}\right)
             \left(   \nGammaBar{2}^{p}{}_{bd}
                    + \nGammaBar{3}^{p}{}_{bd}
                    + \nGammaBar{4}^{p}{}_{bd}
                    + \dots
                    + \nGammaBar{m}^{p}{}_{bd}\right)\label{eq:GenGammaB}
\end{align}
where it is understood that in expanding the pair of bracketed terms in the last result the terms should be
symmetrised over $b\uc d$ and also truncated to terms of order $\BigO{\eps^m}$. Note that the first term
on the right hand side of this equation vanishes by way of the results described above.

Comparing the order $m$ terms in equation (\ref{eq:GenGammaA}) and (\ref{eq:GenGammaB}) leads to the
following equation
\begin{align}
   \nGammaBar{m}^{a}{}_{b\uc d}
   = \nGammaBar{m}^{a}{}_{(b\uc,d)}
   - (n+1)\left(\m \nGammaBar{m-2}^{a}{}_{p(\uc}
                   \nGammaBar{  2}^{p}{}_{bd)}
                  +\nGammaBar{m-3}^{a}{}_{p(\uc}
                   \nGammaBar{  3}^{p}{}_{bd)}
                  +\nGammaBar{m-4}^{a}{}_{p(\uc}
                   \nGammaBar{  4}^{p}{}_{bd)}
                  + \dots
                  + \nGammaBar{  n+1}^{a}{}_{p(\uc}
                    \nGammaBar{m-n-1}^{p}{}_{bd)}
   \right)
   \label{eq:GenGammaC}
\end{align}
This one equation is all that is needed to compute all of the
$\nGammaBar{p}^{a}{}_{b\uc d}$ for $p=3,4,5,\dots m$ given just
the $\nGammaBar{p}^{a}{}_{bd}$ for $p=2,3,4,\dots m$. For example,
suppose $m=5$ and suppose that we are given
$\nGammaBar{p}^{a}{}_{bd}$ for $p=2,3,4,5$. Then with $n=1$ we can
use equation (\ref{eq:GenGammaC}) to compute in turn,
$\nGammaBar{p}^{a}{}_{bc_1d}$ for $p=3,4,5$. Then with $n=2$ we
compute
$\nGammaBar{p}^{a}{}_{bc_1c_2d}$ for $p=4,5$ and finally with $n=3$
we compute $\nGammaBar{p}^{a}{}_{bc_1c_2c_3d}$ for $p=5$. There
are no terms like
$\nGammaBar{p}^{a}{}_{bc_1c_2c_3c_4d}$ for $p\le5$ due to the
corollary given earlier.

\clearpage

The explicit computations for $m=5$ are as follows.

For $n=1$,
\begin{align}
   \nGammaBar{3}^{a}{}_{bc_1d}
   &=
   \nGammaBar{3}^{a}{}_{(bc_1,d)}\\
   %-----------------------------------------------------------------------
   \nGammaBar{4}^{a}{}_{bc_1d}
   &=
   \nGammaBar{4}^{a}{}_{(bc_1,d)}
   - 2 \nGammaBar{2}^{a}{}_{p(c_1}
       \nGammaBar{2}^{p}{}_{bd)}\\
   %-----------------------------------------------------------------------
   \nGammaBar{5}^{a}{}_{bc_1d}
   &=
   \nGammaBar{5}^{a}{}_{(bc_1,d)}
   - 2 \nGammaBar{3}^{a}{}_{p(c_1}
       \nGammaBar{2}^{p}{}_{bd)}
   - 2 \nGammaBar{2}^{a}{}_{p(c_1}
       \nGammaBar{3}^{p}{}_{bd)}
\end{align}

For $n=2$,
\begin{align}
   \nGammaBar{4}^{a}{}_{bc_1c_2d}
   &=
   \nGammaBar{4}^{a}{}_{(bc_1c_2,d)}\\
   %-----------------------------------------------------------------------
   \nGammaBar{5}^{a}{}_{bc_1c_2d}
   &=
   \nGammaBar{5}^{a}{}_{(bc_1c_2,d)}
   - 3 \nGammaBar{2}^{a}{}_{p(c_1c_2}
       \nGammaBar{2}^{p}{}_{bd)}
\end{align}

For $n=3$,
\begin{align}
   \nGammaBar{5}^{a}{}_{bc_1c_2c_3d}
   &=
   \nGammaBar{5}^{a}{}_{(bc_1c_2c_3,d)}
\end{align}


\clearpage

\begin{cadabra}
   {a,b,c,d,e,f,g,h,i,j,k,l,m,n,o,p,q,r,s,t,u,v,c1,c2,c3,c4,c5,w#}::Indices(position=independent).

   D{#}::Derivative.
   \nabla{#}::Derivative.
   \partial{#}::PartialDerivative.

   g_{a b}::Metric.
   g^{a b}::InverseMetric.
   g_{a}^{b}::KroneckerDelta.
   g^{a}_{b}::KroneckerDelta.
   \delta^{a}_{b}::KroneckerDelta.
   \delta_{a}^{b}::KroneckerDelta.

   R_{a b c d}::RiemannTensor.
   R^{a}_{b c d}::RiemannTensor.
   R_{a b c}^{d}::RiemannTensor.

   \Gamma^{a}_{b c}::TableauSymmetry(shape={2}, indices={1,2}).

   x^{a}::Depends(D{#}).

   g_{a b}::Depends(\partial{#}).
   R_{a b c d}::Depends(\partial{#}).
   R^{a}_{b c d}::Depends(\partial{#}).
   \Gamma^{a}_{b c}::Depends(\partial{#}).

   R_{a b c d}::Depends(\nabla{#}).
   R^{a}_{b c d}::Depends(\nabla{#}).

   import cdblib

   term0 = cdblib.get ('GammaRterm0','connection.json')
   term1 = cdblib.get ('GammaRterm2','connection.json')
   term2 = cdblib.get ('GammaRterm3','connection.json')
   term3 = cdblib.get ('GammaRterm4','connection.json')
   term4 = cdblib.get ('GammaRterm5','connection.json')

   # LCB: these terms were not computed in connection.tex so set them to zero
   #      maybe in the future I will compute down to term6.

   term5 := 0.
   term6 := 0.

   # genGmn : m = eps order of Rabcd terms
   #          n = number of c indices

   # --------------------------------------------------------------------------
   # rules for building the genGmn

   # note: after applying each rule, must symmetrise over (b c1 c2 ... cn d)

   # n = 0

   genG20 := genG2^{a}_{b d}.
   genG30 := genG3^{a}_{b d}.
   genG40 := genG4^{a}_{b d}.
   genG50 := genG5^{a}_{b d}.

   defG20 := genG2^{d}_{a b} -> @(term1).
   defG30 := genG3^{d}_{a b} -> @(term2).
   defG40 := genG4^{d}_{a b} -> @(term3).
   defG50 := genG5^{d}_{a b} -> @(term4).

   # LCB: rncGamma in connection.json limited to "term4" (ie. to 4th order in x)
   #      so can only compute genG3*, genG4* and genG5* (at this stage)
   #      but it doesn't hurt to provide the definitions for genG6*, genG7* etc. we just won't use them (at this atage)

   defG60 := genG6^{d}_{a b} -> @(term5).
   defG70 := genG7^{d}_{a b} -> @(term6).

   # n = 1

   defG31 := genG3^{a}_{b c1 d} -> D_{d}{genG3^{a}_{b c1}}.

   defG41 := genG4^{a}_{b c1 d} -> D_{d}{genG4^{a}_{b c1}}
                                   - 2 genG2^{a}_{p c1} genG2^{p}_{b d}.

   defG51 := genG5^{a}_{b c1 d} -> D_{d}{genG5^{a}_{b c1}}
                                   - 2 genG3^{a}_{p c1} genG2^{p}_{b d}
                                   - 2 genG2^{a}_{p c1} genG3^{p}_{b d}.

   defG61 := genG6^{a}_{b c1 d} -> D_{d}{genG6^{a}_{b c1}}
                                   - 2 genG4^{a}_{p c1} genG2^{p}_{b d}
                                   - 2 genG3^{a}_{p c1} genG3^{p}_{b d}
                                   - 2 genG3^{a}_{p c1} genG4^{p}_{b d}.

   defG71 := genG7^{a}_{b c1 d} -> D_{d}{genG7^{a}_{b c1}}
                                   - 2 genG5^{a}_{p c1} genG2^{p}_{b d}
                                   - 2 genG4^{a}_{p c1} genG3^{p}_{b d}
                                   - 2 genG3^{a}_{p c1} genG4^{p}_{b d}
                                   - 2 genG2^{a}_{p c1} genG5^{p}_{b d}.

   # n = 2

   defG42 := genG4^{a}_{b c1 c2 d} -> D_{d}{genG4^{a}_{b c1 c2}}.

   defG52 := genG5^{a}_{b c1 c2 d} -> D_{d}{genG5^{a}_{b c1 c2}}
                                      - 3 genG3^{a}_{p c1 c2} genG2^{p}_{b d}.

   defG62 := genG6^{a}_{b c1 c2 d} -> D_{d}{genG6^{a}_{b c1 c2}}
                                      - 3 genG4^{a}_{p c1 c2} genG2^{p}_{b d}
                                      - 3 genG3^{a}_{p c1 c2} genG3^{p}_{b d}.

   defG72 := genG7^{a}_{b c1 c2 d} -> D_{d}{genG7^{a}_{b c1 c2}}
                                      - 3 genG5^{a}_{p c1 c2} genG2^{p}_{b d}
                                      - 3 genG4^{a}_{p c1 c2} genG3^{p}_{b d}
                                      - 3 genG3^{a}_{p c1 c2} genG4^{p}_{b d}.

   # n = 3

   defG53 := genG5^{a}_{b c1 c2 c3 d} -> D_{d}{genG5^{a}_{b c1 c2 c3}}.

   defG63 := genG6^{a}_{b c1 c2 c3 d} -> D_{d}{genG6^{a}_{b c1 c2 c3}}
                                         - 4 genG3^{a}_{p c1 c2 c3} genG3^{p}_{b d}.

   defG73 := genG7^{a}_{b c1 c2 c3 d} -> D_{d}{genG7^{a}_{b c1 c2 c3}}
                                         - 4 genG4^{a}_{p c1 c2 c3} genG3^{p}_{b d}
                                         - 4 genG3^{a}_{p c1 c2 c3} genG4^{p}_{b d}.

   # n = 4

   defG64 := genG6^{a}_{b c1 c2 c3 c4 d} -> D_{d}{genG6^{a}_{b c1 c2 c3 c4}}.

   defG74 := genG7^{a}_{b c1 c2 c3 c4 d} -> D_{d}{genG7^{a}_{b c1 c2 c3 c4}}
                                            - 5 genG5^{a}_{p c1 c2 c3 c4} genG2^{p}_{b d}.

   # n = 5

   defG75 := genG7^{a}_{b c1 c2 c3 c4 c5 d} -> D_{d}{genG7^{a}_{b c1 c2 c3 c4 c5}}.

   # --------------------------------------------------------------------------
   # build the genGmn

   # ==========================================================================
   # n = 1

   genG31 := genG3^{a}_{b c1 d}.                              # cdb (genG31.000,genG31)
   genG41 := genG4^{a}_{b c1 d}.                              # cdb (genG41.000,genG41)
   genG51 := genG5^{a}_{b c1 d}.
   # genG61 := genG6^{a}_{b c1 d}.
   # genG71 := genG7^{a}_{b c1 d}.

   # --------------------------------------------------------------------------
   substitute     (genG20,defG20)                             # cdb (genG20.001,genG20)
   substitute     (genG30,defG30)                             # cdb (genG30.001,genG30)
   substitute     (genG40,defG40)                             # cdb (genG40.001,genG40)
   substitute     (genG50,defG50)                             # cdb (genG50.001,genG50)

   # --------------------------------------------------------------------------
   substitute     (genG31,defG31)                             # cdb (genG31.001,genG31)
   substitute     (genG31,defG30)                             # cdb (genG31.002,genG31)

   distribute     (genG31)                                    # cdb (genG31.002,genG31)
   unwrap         (genG31)                                    # cdb (genG31.003,genG31)
   product_rule   (genG31)                                    # cdb (genG31.004,genG31)
   distribute     (genG31)                                    # cdb (genG31.005,genG31)
   substitute     (genG31,$D_{a}{x^b}->\delta_{a}^{b}$)       # cdb (genG31.006,genG31)
   eliminate_kronecker (genG31)                               # cdb (genG31.007,genG31)
   sym            (genG31,$_{b}, _{c1}, _{d}$)
   sort_product   (genG31)                                    # cdb (genG31.008,genG31)
   rename_dummies (genG31)                                    # cdb (genG31.009,genG31)
   canonicalise   (genG31)                                    # cdb (genG31.010,genG31)

   # --------------------------------------------------------------------------
   substitute     (genG41,defG41)                             # cdb (genG41.001,genG41)
   substitute     (genG41,defG40)                             # cdb (genG41.002,genG41)
   substitute     (genG41,defG20,repeat=True)                 # cdb (genG41.003,genG41)

   distribute     (genG41)                                    # cdb (genG41.004,genG41)
   unwrap         (genG41)                                    # cdb (genG41.005,genG41)
   product_rule   (genG41)                                    # cdb (genG41.006,genG41)
   distribute     (genG41)                                    # cdb (genG41.007,genG41)
   substitute     (genG41,$D_{a}{x^b}->\delta_{a}^{b}$)       # cdb (genG41.008,genG41)
   eliminate_kronecker (genG41)                               # cdb (genG41.009,genG41)
   sym            (genG41,$_{b}, _{c1}, _{d}$)
   sort_product   (genG41)                                    # cdb (genG41.010,genG41)
   rename_dummies (genG41)                                    # cdb (genG41.011,genG41)
   canonicalise   (genG41)                                    # cdb (genG41.012,genG41)

   # --------------------------------------------------------------------------
   substitute     (genG51,defG51)
   substitute     (genG51,defG50)
   substitute     (genG51,defG30,repeat=True)
   substitute     (genG51,defG20,repeat=True)

   distribute     (genG51)
   unwrap         (genG51)
   product_rule   (genG51)
   distribute     (genG51)
   substitute     (genG51,$D_{a}{x^b}->\delta_{a}^{b}$)
   eliminate_kronecker (genG51)
   sym            (genG51,$_{b}, _{c1}, _{d}$)
   sort_product   (genG51)
   rename_dummies (genG51)
   canonicalise   (genG51)

   # update the rules

   defG31 := genG3^{a}_{b c1 d} -> @(genG31).
   defG41 := genG4^{a}_{b c1 d} -> @(genG41).
   defG51 := genG5^{a}_{b c1 d} -> @(genG51).

   # ==========================================================================
   # n = 2

   genG42 := genG4^{a}_{b c1 c2 d}.                           # cdb (genG42.000,genG42)
   genG52 := genG5^{a}_{b c1 c2 d}.
   # genG62 := genG6^{a}_{b c1 c2 d}.
   # genG72 := genG7^{a}_{b c1 c2 d}.

   # --------------------------------------------------------------------------
   substitute     (genG42,defG42)                             # cdb (genG42.001,genG42)
   substitute     (genG42,defG41)                             # cdb (genG42.002,genG42)

   distribute     (genG42)                                    # cdb (genG42.003,genG42)
   unwrap         (genG42)                                    # cdb (genG42.004,genG42)
   product_rule   (genG42)                                    # cdb (genG42.005,genG42)
   distribute     (genG42)                                    # cdb (genG42.006,genG42)
   substitute     (genG42,$D_{a}{x^b}->\delta_{a}^{b}$)       # cdb (genG42.007,genG42)
   eliminate_kronecker (genG42)                               # cdb (genG42.008,genG42)
   sym            (genG42,$_{b}, _{c1}, _{c2}, _{d}$)
   sort_product   (genG42)                                    # cdb (genG42.009,genG42)
   rename_dummies (genG42)                                    # cdb (genG42.010,genG42)
   canonicalise   (genG42)                                    # cdb (genG42.011,genG42)

   # --------------------------------------------------------------------------
   substitute     (genG52,defG52)
   substitute     (genG52,defG51)
   substitute     (genG52,defG31,repeat=True)
   substitute     (genG52,defG20,repeat=True)

   distribute     (genG52)
   unwrap         (genG52)
   product_rule   (genG52)
   distribute     (genG52)
   substitute     (genG52,$D_{a}{x^b}->\delta_{a}^{b}$)
   eliminate_kronecker (genG52)
   sym            (genG52,$_{b}, _{c1}, _{c2}, _{d}$)
   sort_product   (genG52)
   rename_dummies (genG52)
   canonicalise   (genG52)                                    # cdb (genG52.001,genG52)

   # update the rules

   defG42 := genG4^{a}_{b c1 c2 d} -> @(genG42).
   defG52 := genG5^{a}_{b c1 c2 d} -> @(genG52).

   # ==========================================================================
   # n = 3

   genG53 := genG5^{a}_{b c1 c2 c3 d}.
   # genG63 := genG6^{a}_{b c1 c2 c3 d}.
   # genG73 := genG7^{a}_{b c1 c2 c3 d}.

   # --------------------------------------------------------------------------
   substitute     (genG53,defG53)
   substitute     (genG53,defG52)

   distribute     (genG53)
   unwrap         (genG53)
   product_rule   (genG53)
   distribute     (genG53)
   substitute     (genG53,$D_{a}{x^b}->\delta_{a}^{b}$)
   eliminate_kronecker (genG53)
   sym            (genG53,$_{b}, _{c1}, _{c2}, _{c3}, _{d}$)
   sort_product   (genG53)
   rename_dummies (genG53)
   canonicalise   (genG53)                                    # cdb (genG53.001,genG53)

   # update the rules

   defG53 := genG5^{a}_{b c1 c2 c3 d} -> @(genG53).

\end{cadabra}

\clearpage

\clearpage

\begin{dgroup*}
   \begin{dmath*} \cdb*{genG31.000} \end{dmath*}
   \begin{dmath*} \cdb*{genG31.001} \end{dmath*}
   \begin{dmath*} \cdb*{genG31.002} \end{dmath*}
   \begin{dmath*} \cdb*{genG31.003} \end{dmath*}
   \begin{dmath*} \cdb*{genG31.004} \end{dmath*}
   \begin{dmath*} \cdb*{genG31.005} \end{dmath*}
   \begin{dmath*} \cdb*{genG31.006} \end{dmath*}
   \begin{dmath*} \cdb*{genG31.007} \end{dmath*}
   \begin{dmath*} \cdb*{genG31.008} \end{dmath*}
   \begin{dmath*} \cdb*{genG31.009} \end{dmath*}
   \begin{dmath*} \cdb*{genG31.010} \end{dmath*}
\end{dgroup*}

\clearpage

\begin{dgroup*}
   \begin{dmath*} \cdb*{genG41.000} \end{dmath*}
   \begin{dmath*} \cdb*{genG41.001} \end{dmath*}
   \begin{dmath*} \cdb*{genG41.002} \end{dmath*}
   \begin{dmath*} \cdb*{genG41.003} \end{dmath*}
   \begin{dmath*} \cdb*{genG41.004} \end{dmath*}
   \begin{dmath*} \cdb*{genG41.005} \end{dmath*}
   \begin{dmath*} \cdb*{genG41.006} \end{dmath*}
   \begin{dmath*} \cdb*{genG41.007} \end{dmath*}
   \begin{dmath*} \cdb*{genG41.008} \end{dmath*}
   \begin{dmath*} \cdb*{genG41.009} \end{dmath*}
   % \begin{dmath*} \cdb*{genG41.010} \end{dmath*}
   % \begin{dmath*} \cdb*{genG41.011} \end{dmath*}
   \begin{dmath*} \cdb*{genG41.012} \end{dmath*}
\end{dgroup*}

\clearpage

\begin{dgroup*}
   \begin{dmath*} \cdb*{genG42.000} \end{dmath*}
   \begin{dmath*} \cdb*{genG42.001} \end{dmath*}
   \begin{dmath*} \cdb*{genG42.002} \end{dmath*}
   \begin{dmath*} \cdb*{genG42.003} \end{dmath*}
   \begin{dmath*} \cdb*{genG42.004} \end{dmath*}
   \begin{dmath*} \cdb*{genG42.005} \end{dmath*}
   % \begin{dmath*} \cdb*{genG42.006} \end{dmath*}
   % \begin{dmath*} \cdb*{genG42.007} \end{dmath*}
   % \begin{dmath*} \cdb*{genG42.008} \end{dmath*}
   % \begin{dmath*} \cdb*{genG42.009} \end{dmath*}
   % \begin{dmath*} \cdb*{genG42.010} \end{dmath*}
   \begin{dmath*} \cdb*{genG42.011} \end{dmath*}
\end{dgroup*}

\clearpage

\begin{cadabra}
   # note: keeping numbering as is (out of order) to ensure R appears before \nabla R etc.
   def product_sort (obj):
       substitute (obj,$ A^{a}                            -> A001^{a}               $)
       substitute (obj,$ x^{a}                            -> A002^{a}               $)
       substitute (obj,$ g^{a b}                          -> A003^{a b}             $)
       substitute (obj,$ \nabla_{e f g h}{R_{a b c d}}    -> A008_{a b c d e f g h} $)
       substitute (obj,$ \nabla_{e f g}{R_{a b c d}}      -> A007_{a b c d e f g}   $)
       substitute (obj,$ \nabla_{e f}{R_{a b c d}}        -> A006_{a b c d e f}     $)
       substitute (obj,$ \nabla_{e}{R_{a b c d}}          -> A005_{a b c d e}       $)
       substitute (obj,$ R_{a b c d}                      -> A004_{a b c d}         $)
       sort_product   (obj)
       rename_dummies (obj)
       substitute (obj,$ A001^{a}                  -> A^{a}                         $)
       substitute (obj,$ A002^{a}                  -> x^{a}                         $)
       substitute (obj,$ A003^{a b}                -> g^{a b}                       $)
       substitute (obj,$ A004_{a b c d}            -> R_{a b c d}                   $)
       substitute (obj,$ A005_{a b c d e}          -> \nabla_{e}{R_{a b c d}}       $)
       substitute (obj,$ A006_{a b c d e f}        -> \nabla_{e f}{R_{a b c d}}     $)
       substitute (obj,$ A007_{a b c d e f g}      -> \nabla_{e f g}{R_{a b c d}}   $)
       substitute (obj,$ A008_{a b c d e f g h}    -> \nabla_{e f g h}{R_{a b c d}} $)

       return obj

   # --------------------------------------------------------------------------
   symG20 := @(genG20) A^{b} A^{d}.                           # cdb (symG20.100,symG20)

   distribute            (symG20)                             # cdb (symG20.101,symG20)
   symG20 = product_sort (symG20)                             # cdb (symG20.102,symG20)
   rename_dummies        (symG20)                             # cdb (symG20.103,symG20)
   canonicalise          (symG20)                             # cdb (symG20.104,symG20)

   # --------------------------------------------------------------------------
   symG30 := @(genG30) A^{b} A^{d}.                           # cdb (symG30.100,symG30)

   distribute            (symG30)                             # cdb (symG30.101,symG30)
   symG30 = product_sort (symG30)                             # cdb (symG30.102,symG30)
   rename_dummies        (symG30)                             # cdb (symG30.103,symG30)
   canonicalise          (symG30)                             # cdb (symG30.104,symG30)

   # --------------------------------------------------------------------------
   symG40 := @(genG40) A^{b} A^{d}.                           # cdb (symG40.100,symG40)

   distribute            (symG40)                             # cdb (symG40.101,symG40)
   symG40 = product_sort (symG40)                             # cdb (symG40.102,symG40)
   rename_dummies        (symG40)                             # cdb (symG40.103,symG40)
   canonicalise          (symG40)                             # cdb (symG40.104,symG40)

   # --------------------------------------------------------------------------
   symG50 := @(genG50) A^{b} A^{d}.                           # cdb (symG50.100,symG50)

   distribute            (symG50)                             # cdb (symG50.101,symG50)
   symG50 = product_sort (symG50)                             # cdb (symG50.102,symG50)
   rename_dummies        (symG50)                             # cdb (symG50.103,symG50)
   canonicalise          (symG50)                             # cdb (symG50.104,symG50)

   # --------------------------------------------------------------------------
   symG31 := @(genG31) A^{b} A^{c1} A^{d}.                    # cdb (symG31.100,symG31)

   distribute            (symG31)                             # cdb (symG31.101,symG31)
   symG31 = product_sort (symG31)                             # cdb (symG31.102,symG31)
   rename_dummies        (symG31)                             # cdb (symG31.103,symG31)
   canonicalise          (symG31)                             # cdb (symG31.104,symG31)

   # --------------------------------------------------------------------------
   symG41 := @(genG41) A^{b} A^{c1} A^{d}.                    # cdb (symG41.100,symG41)

   distribute            (symG41)                             # cdb (symG41.101,symG41)
   symG41 = product_sort (symG41)                             # cdb (symG41.102,symG41)
   rename_dummies        (symG41)                             # cdb (symG41.103,symG41)
   canonicalise          (symG41)                             # cdb (symG41.104,symG41)

   # --------------------------------------------------------------------------
   symG51 := @(genG51) A^{b} A^{c1} A^{d}.                    # cdb (symG51.100,symG51)

   distribute            (symG51)                             # cdb (symG51.101,symG51)
   symG51 = product_sort (symG51)                             # cdb (symG51.102,symG51)
   rename_dummies        (symG51)                             # cdb (symG51.103,symG51)
   canonicalise          (symG51)                             # cdb (symG51.104,symG51)

   # --------------------------------------------------------------------------
   symG42 := @(genG42) A^{b} A^{c1} A^{c2} A^{d}.             # cdb (symG42.100,symG42)

   distribute            (symG42)                             # cdb (symG42.101,symG42)
   symG42 = product_sort (symG42)                             # cdb (symG42.102,symG42)
   rename_dummies        (symG42)                             # cdb (symG42.103,symG42)
   canonicalise          (symG42)                             # cdb (symG42.104,symG42)

   # --------------------------------------------------------------------------
   symG52 := @(genG52) A^{b} A^{c1} A^{c2} A^{d}.             # cdb (symG52.100,symG52)

   distribute            (symG52)                             # cdb (symG52.101,symG52)
   symG52 = product_sort (symG52)                             # cdb (symG52.102,symG52)
   rename_dummies        (symG52)                             # cdb (symG52.103,symG52)
   canonicalise          (symG52)                             # cdb (symG52.104,symG52)

   # --------------------------------------------------------------------------
   symG53 := @(genG53) A^{b} A^{c1} A^{c2} A^{c3} A^{d}.      # cdb (symG53.100,symG53)

   distribute            (symG53)                             # cdb (symG53.101,symG53)
   symG53 = product_sort (symG53)                             # cdb (symG53.102,symG53)
   rename_dummies        (symG53)                             # cdb (symG53.103,symG53)
   canonicalise          (symG53)                             # cdb (symG53.104,symG53)

\end{cadabra}

\clearpage

\begin{dgroup*}
   \begin{dmath*} \cdb*{symG31.100} \end{dmath*}
   \begin{dmath*} \cdb*{symG31.101} \end{dmath*}
   \begin{dmath*} \cdb*{symG31.102} \end{dmath*}
   \begin{dmath*} \cdb*{symG31.103} \end{dmath*}
   \begin{dmath*} \cdb*{symG31.104} \end{dmath*}
\end{dgroup*}

\clearpage

\begin{dgroup*}
   \begin{dmath*} \cdb*{symG41.100} \end{dmath*}
   \begin{dmath*} \cdb*{symG41.101} \end{dmath*}
   \begin{dmath*} \cdb*{symG41.102} \end{dmath*}
   \begin{dmath*} \cdb*{symG41.103} \end{dmath*}
   \begin{dmath*} \cdb*{symG41.104} \end{dmath*}
\end{dgroup*}

\clearpage

\begin{dgroup*}
   \begin{dmath*} \cdb*{symG51.104} \end{dmath*}
\end{dgroup*}

\clearpage

\begin{dgroup*}
   \begin{dmath*} \cdb*{symG42.104} \end{dmath*}
   \begin{dmath*} \cdb*{symG52.104} \end{dmath*}
\end{dgroup*}

\clearpage

\begin{dgroup*}
   \begin{dmath*} \cdb*{symG53.104} \end{dmath*}
\end{dgroup*}

\clearpage

\begin{cadabra}
   def reformat (obj,scale):
       foo  = Ex(str(scale))
       bah := @(foo) @(obj).
       distribute (bah)
       factor_out (bah,$A^{a?},x^{b?}$)
       ans := @(bah) / @(foo).
       return ans

   fooG20 = reformat (symG20,3)
   fooG30 = reformat (symG30,12)
   fooG40 = reformat (symG40,360)
   fooG50 = reformat (symG50,180)

   fooG31 = reformat (symG31,2)
   fooG41 = reformat (symG41,120)
   fooG51 = reformat (symG51,180)

   fooG42 = reformat (symG42,15)
   fooG52 = reformat (symG52,90)

   fooG53 = reformat (symG53,3)

   genGamma0 := @(fooG20) + @(fooG30) + @(fooG40) + @(fooG50).  # cdb (genGamma0.000,genGamma0)
   genGamma1 := @(fooG31) + @(fooG41) + @(fooG51).              # cdb (genGamma1.000,genGamma1)
   genGamma2 := @(fooG42) + @(fooG52).                          # cdb (genGamma2.000,genGamma2)
   genGamma3 := @(fooG53).                                      # cdb (genGamma3.000,genGamma3)

   cdblib.create ('genGamma.json')

   cdblib.put ('genGamma0',genGamma0,'genGamma.json')
   cdblib.put ('genGamma1',genGamma1,'genGamma.json')
   cdblib.put ('genGamma2',genGamma2,'genGamma.json')
   cdblib.put ('genGamma3',genGamma3,'genGamma.json')

   cdblib.put ('genGamma01',fooG20,'genGamma.json')
   cdblib.put ('genGamma02',fooG30,'genGamma.json')
   cdblib.put ('genGamma03',fooG40,'genGamma.json')
   cdblib.put ('genGamma04',fooG50,'genGamma.json')

   cdblib.put ('genGamma11',fooG31,'genGamma.json')
   cdblib.put ('genGamma12',fooG41,'genGamma.json')
   cdblib.put ('genGamma13',fooG51,'genGamma.json')

   cdblib.put ('genGamma21',fooG42,'genGamma.json')
   cdblib.put ('genGamma22',fooG52,'genGamma.json')

   cdblib.put ('genGamma31',fooG53,'genGamma.json')

\end{cadabra}

\clearpage

% =================================================================================================
\section*{The generalised connection in Riemann normal coordinates}

\begin{dgroup*}
   \begin{dmath*} A^b A^c \Gamma^{a}_{b c}(x) = \cdb{genGamma0.000} \end{dmath*}
   \begin{dmath*} A^b A^c A^d \Gamma^{a}_{b c d}(x) = \cdb{genGamma1.000} \end{dmath*}
   \begin{dmath*} A^b A^c A^d A^e \Gamma^{a}_{b c d e}(x) = \cdb{genGamma2.000} \end{dmath*}
   \begin{dmath*} A^b A^c A^d A^e A^f \Gamma^{a}_{b c d e f}(x) = \cdb{genGamma3.000} \end{dmath*}
\end{dgroup*}

\clearpage

\begin{cadabra}
   scaledGamma0 := 360 @(genGamma0).  # cdb (scaledGamma0.001,scaledGamma0)
   scaledGamma1 := 360 @(genGamma1).  # cdb (scaledGamma1.001,scaledGamma1)
   scaledGamma2 :=  90 @(genGamma2).  # cdb (scaledGamma2.001,scaledGamma2)
   scaledGamma3 :=   3 @(genGamma3).  # cdb (scaledGamma3.001,scaledGamma3)

\end{cadabra}

\clearpage

% =================================================================================================
\section*{The generalised connection in Riemann normal coordinates}

This is the same as the previous page but with a small change in the format to avoid fractions.

\begin{dgroup*}
   \begin{dmath*} 360 A^b A^c \Gamma^{a}_{b c}(x) = \cdb{scaledGamma0.001} \end{dmath*}
   \begin{dmath*} 360 A^b A^c A^d \Gamma^{a}_{b c d}(x) = \cdb{scaledGamma1.001} \end{dmath*}
   \begin{dmath*}  90 A^b A^c A^d A^e \Gamma^{a}_{b c d e}(x) = \cdb{scaledGamma2.001} \end{dmath*}
   \begin{dmath*}   3 A^b A^c A^d A^e A^f \Gamma^{a}_{b c d e f}(x) = \cdb{scaledGamma3.001} \end{dmath*}
\end{dgroup*}

\clearpage

% =================================================================================================
% compute the first three generalised Gamms (including \Gamma^{a}_{bc})
% this is just provide the LaTeX code for the first three \Gamma's in section 7 of the paper
% these results are never used anywhere else

\begin{cadabra}
   deriv01:=B^{a}:

   deriv02:=-\Gamma^{a}_{b c} B^{b} B^{c}:                # cdb (deriv02.100,deriv02)

   deriv03:=\nabla{@(deriv02)}.                           # cdb (deriv03.100,deriv03)
   distribute     (deriv03)
   product_rule   (deriv03)                               # cdb (deriv03.101,deriv03)
   substitute     (deriv03,$\nabla{B^{a}}->@(deriv02)$)   # cdb (deriv03.102,deriv03)
   substitute     (deriv03,$\nabla{\Gamma^{m}_{s t}}->B^{d}\partial_{d}{\Gamma^{m}_{s t}}$)   # cdb (deriv03.103,deriv03)
   sort_product   (deriv03)                               # cdb (deriv03.104,deriv03)
   rename_dummies (deriv03)                               # cdb (deriv03.105,deriv03)
   canonicalise   (deriv03)                               # cdb (deriv03.106,deriv03)

   deriv04:=\nabla{@(deriv03)}.                           # cdb (deriv04.100,deriv04)
   distribute     (deriv04)
   product_rule   (deriv04)                               # cdb (deriv04.101,deriv04)
   substitute     (deriv04,$\nabla{B^{a}}->@(deriv02)$)   # cdb (deriv04.102,deriv04)
   substitute     (deriv04,$\nabla{\Gamma^{m}_{s t}}->B^{d}\partial_{d}{\Gamma^{m}_{s t}}$)   # cdb (deriv04.103,deriv04)
   substitute     (deriv04,$\nabla{\partial_{e}{\Gamma^{m}_{s t}}}->B^{d}\partial_{d e}{\Gamma^{m}_{s t}}$)   # cdb (deriv04.104,deriv04)
   sort_product   (deriv04)                               # cdb (deriv04.105,deriv04)
   rename_dummies (deriv04)                               # cdb (deriv04.106,deriv04)
   canonicalise   (deriv04)                               # cdb (deriv04.107,deriv04)

   pderiv02 := -@(deriv02).                # cdb (pderiv02.100,pderiv02)
   factor_out (pderiv02, $B^{a?}$)         # cdb (pderiv02.101,pderiv02)
   substitute (pderiv02, $B^{a} -> 1$)     # cdb (pderiv02.102,pderiv02)

   pderiv03 := -@(deriv03).                # cdb (pderiv03.100,pderiv03)
   factor_out (pderiv03, $B^{a?}$)         # cdb (pderiv03.101,pderiv03)
   substitute (pderiv03, $B^{a} -> 1$)     # cdb (pderiv03.102,pderiv03)

   pderiv04 := -@(deriv04).                # cdb (pderiv04.100,pderiv04)
   factor_out (pderiv04, $B^{a?}$)         # cdb (pderiv04.101,pderiv04)
   substitute (pderiv04, $B^{a} -> 1$)     # cdb (pderiv04.102,pderiv04)
\end{cadabra}

\clearpage

% =================================================================================================
\section*{The generalised connection in generic coordinates (for the paper section 7)}

\begin{dgroup*}[spread=5pt]
   \begin{dmath*} {\Gamma}^{a}_{(bc)}(x) = \Cdb*{pderiv02.102} \end{dmath*}
   \begin{dmath*} {\Gamma}^{a}_{(bcd)}(x) = \Cdb*{pderiv03.102} \end{dmath*}
   \begin{dmath*} {\Gamma}^{a}_{(bcde)}(x) = \Cdb*{pderiv04.102} \end{dmath*}
\end{dgroup*}

\clearpage

% =================================================================================================
% export selected objects, these will later be imported into a library
% these are the objects that will appear in the paper

\begin{cadabra}
   tmp0 := @(fooG20) + @(fooG30).
   tmp1 := @(fooG31).

   alt0 := @(genGamma0).
   alt1 := @(genGamma1).
   alt2 := @(genGamma2).
   alt3 := @(genGamma3).

   alt0scaled := @(scaledGamma0).
   alt1scaled := @(scaledGamma1).
   alt2scaled := @(scaledGamma2).
   alt3scaled := @(scaledGamma3).

   substitute (tmp0, $A^{a}->1$)
   substitute (tmp1, $A^{a}->1$)

   substitute (alt0, $A^{a}->1$)
   substitute (alt1, $A^{a}->1$)
   substitute (alt2, $A^{a}->1$)
   substitute (alt3, $A^{a}->1$)

   substitute (alt0scaled, $A^{a}->1$)
   substitute (alt1scaled, $A^{a}->1$)
   substitute (alt2scaled, $A^{a}->1$)
   substitute (alt3scaled, $A^{a}->1$)

   cdblib.create ('genGamma.export')

   # 4th order gen gamma
   cdblib.put ('gen_gamma_0_4th',tmp0,'genGamma.export')
   cdblib.put ('gen_gamma_1_4th',tmp1,'genGamma.export')

   # 6th order gen gamma
   cdblib.put ('gen_gamma_0',alt0,'genGamma.export')
   cdblib.put ('gen_gamma_1',alt1,'genGamma.export')
   cdblib.put ('gen_gamma_2',alt2,'genGamma.export')
   cdblib.put ('gen_gamma_3',alt3,'genGamma.export')

   # 6th order gen gamma scaled
   cdblib.put ('gen_gamma_0_scaled',alt0scaled,'genGamma.export')
   cdblib.put ('gen_gamma_1_scaled',alt1scaled,'genGamma.export')
   cdblib.put ('gen_gamma_2_scaled',alt2scaled,'genGamma.export')
   cdblib.put ('gen_gamma_3_scaled',alt3scaled,'genGamma.export')

   # gen gamma in terms of partial derivs of Gamma^{a}_{bc}
   cdblib.put ('gen_gamma_pderiv0',pderiv02,'genGamma.export')
   cdblib.put ('gen_gamma_pderiv1',pderiv03,'genGamma.export')
   cdblib.put ('gen_gamma_pderiv2',pderiv04,'genGamma.export')

   checkpoint.append (tmp0)
   checkpoint.append (tmp1)

   checkpoint.append (alt0)
   checkpoint.append (alt1)
   checkpoint.append (alt2)
   checkpoint.append (alt3)

   checkpoint.append (alt0scaled)
   checkpoint.append (alt1scaled)
   checkpoint.append (alt2scaled)
   checkpoint.append (alt3scaled)

   checkpoint.append (pderiv02)
   checkpoint.append (pderiv03)
   checkpoint.append (pderiv04)
\end{cadabra}

% =================================================================================================
% export checkpoints in json format

\bgroup
\CdbSetup{action=hide}
\begin{cadabra}
   for i in range( len(checkpoint) ):
      cdblib.put ('check{:03d}'.format(i),checkpoint[i],checkpoint_file)
\end{cadabra}
\egroup

\end{document}


\begin{dgroup*}
   \begin{dmath*} A^b A^c \Gamma^{a}_{b c} = \cdb{genGamma0.000} \end{dmath*}
   \begin{dmath*} A^b A^c A^d \Gamma^{a}_{b c d} = \cdb{genGamma1.000} \end{dmath*}
   \begin{dmath*} A^b A^c A^d A^e \Gamma^{a}_{b c d e} = \cdb{genGamma2.000} \end{dmath*}
   \begin{dmath*} A^b A^c A^d A^e A^f \Gamma^{a}_{b c d e f} = \cdb{genGamma3.000} \end{dmath*}
\end{dgroup*}

\clearpage

% =================================================================================================
\section*{The generalised connection in RNC}

This is the same as the previous page but with a small change in the format to avoid fractions.

\begin{dgroup*}
   \begin{dmath*} 360 A^b A^c \Gamma^{a}_{b c} = \cdb{scaledGamma0.001} \end{dmath*}
   \begin{dmath*} 360 A^b A^c A^d \Gamma^{a}_{b c d} = \cdb{scaledGamma1.001} \end{dmath*}
   \begin{dmath*}  90 A^b A^c A^d A^e \Gamma^{a}_{b c d e} = \cdb{scaledGamma2.001} \end{dmath*}
   \begin{dmath*}   3 A^b A^c A^d A^e A^f \Gamma^{a}_{b c d e f} = \cdb{scaledGamma3.001} \end{dmath*}
\end{dgroup*}

\clearpage

% =================================================================================================
\section*{Convert from generic (x) to local RNC coords (y)}
\def\Date{19 Jan 2024}
% \def\FileID{file:}

\documentclass[12pt]{cdblatex}

\begin{document}

\section*{\jobname}

\CdbSetup{action=hide}

\begin{cadabra}
   import shared

   import cdblib

   term00A = cdblib.get ('check000','expected/gen2rnc.json')
   term01A = cdblib.get ('check001','expected/gen2rnc.json')
   term02A = cdblib.get ('check002','expected/gen2rnc.json')
   term03A = cdblib.get ('check003','expected/gen2rnc.json')
   term04A = cdblib.get ('check004','expected/gen2rnc.json')

   term00B = cdblib.get ('check000','output/gen2rnc.json')
   term01B = cdblib.get ('check001','output/gen2rnc.json')
   term02B = cdblib.get ('check002','output/gen2rnc.json')
   term03B = cdblib.get ('check003','output/gen2rnc.json')
   term04B = cdblib.get ('check004','output/gen2rnc.json')

   diff000 = shared.check (term00A,term00B)   # cdb (diff000,diff000)
   diff001 = shared.check (term01A,term01B)   # cdb (diff001,diff001)
   diff002 = shared.check (term02A,term02B)   # cdb (diff002,diff002)
   diff003 = shared.check (term03A,term03B)   # cdb (diff003,diff003)
   diff004 = shared.check (term04A,term04B)   # cdb (diff004,diff004)

\end{cadabra}

\begin{dgroup*}
   \Dmath*{ \cdb*{diff000} }
   \Dmath*{ \cdb*{diff001} }
   \Dmath*{ \cdb*{diff002} }
   \Dmath*{ \cdb*{diff003} }
   \Dmath*{ \cdb*{diff004} }
\end{dgroup*}

\end{document}


\begin{align*}
   y^a = \ny{0}^{a} + \ny{1}^{a} + \ny{2}^{a} + \ny{3}^{a} + \ny{4}^{a}
\end{align*}

\begin{dgroup*}
   \begin{dmath*}     \ny{0}^{a} = \cdb{scaled1.002} \end{dmath*}
   \begin{dmath*}   2 \ny{1}^{a} = \cdb{scaled2.002} \end{dmath*}
   \begin{dmath*}   6 \ny{2}^{a} = \cdb{scaled3.002} \end{dmath*}
   \begin{dmath*}  24 \ny{3}^{a} = \cdb{scaled4.002} \end{dmath*}
   \begin{dmath*} 360 \ny{4}^{a} = \cdb{scaled5.002} \end{dmath*}
\end{dgroup*}

\clearpage

% =================================================================================================
\section*{The geodesic ivp}
\def\Date{30 Jul 2024}
% \def\FileID{file:}

\documentclass[12pt]{cdblatex}

\lstset{gobble=2}

\begin{document}

% =================================================================================================
% create checkpoint file

\bgroup
\CdbSetup{action=hide}
\begin{cadabra}
   import cdblib
   checkpoint_file = 'tests/semantic/output/geodesic-ivp.json'
   cdblib.create (checkpoint_file)
   checkpoint = []
\end{cadabra}
\egroup

% =================================================================================================
\section*{Geodesic IVP}

Our game here is to find the solution of
\begin{align*}
   0 = \frac{d^2 x^{a}}{ds^2} + \Gamma^{a}_{bc}(x) \frac{dx^b}{ds} \frac{dx^c}{ds}
\end{align*}
subject to the initial conditions $x^{a}(s) = x^a$ and $dx^a(s)/ds={\Dot x}^{a}$ at $s=0$.

% =================================================================================================
\section*{Algorithm}

By successive differentiation of the above equation we can compute
\begin{align*}
   \frac{d^n x^{a}}{ds^n} = -\Gamma^{a}_{\udn}\frac{dx^{\udn}}{ds}
\end{align*}
at $s=0$ for $n=2,3,4,\dotsc$. The $\Gamma^{a}_{\udn}$ are the \emph{generalised connections}.

We can then construct the Taylor series solution for $x^{a}(s)$
\begin{align*}
   x^a(s) = x^a + s {\Dot x}^a - \sum_{k=2}^\infty\>\frac{s^{k}}{k!} \Gamma^{a}_{\udk}{\Dot x}^{\udk}
\end{align*}

\clearpage

\begin{cadabra}
   {a,b,c,d,e,f,g,h,i,j,k,l,m,n,o,p,q,r,s,t,u,v,w#}::Indices(position=independent).

   \nabla{#}::Derivative.

   import cdblib

   # change signs to account for - sign in front of the sum for x^a(s), see above preamble

   def flip_sign (obj):
       return Ex(0) - obj

   sterm21 = flip_sign (cdblib.get ('genGamma01','genGamma.json'))
   sterm22 = flip_sign (cdblib.get ('genGamma02','genGamma.json'))
   sterm23 = flip_sign (cdblib.get ('genGamma03','genGamma.json'))
   sterm24 = flip_sign (cdblib.get ('genGamma04','genGamma.json'))

   sterm31 = flip_sign (cdblib.get ('genGamma11','genGamma.json'))
   sterm32 = flip_sign (cdblib.get ('genGamma12','genGamma.json'))
   sterm33 = flip_sign (cdblib.get ('genGamma13','genGamma.json'))

   sterm41 = flip_sign (cdblib.get ('genGamma21','genGamma.json'))
   sterm42 = flip_sign (cdblib.get ('genGamma22','genGamma.json'))

   sterm51 = flip_sign (cdblib.get ('genGamma31','genGamma.json'))

   # note: the various ivp21, ivp31  etc. are the pieces of the Taylor series
   #       for the ivp but *without* the leading 1/n! of the Taylor series

   ivp21 := @(sterm21).                                          # cdb (ivp21.000,ivp21)

   ivp31 := @(sterm21) + @(sterm22).                             # cdb (ivp31.000,ivp31)
   ivp32 := @(sterm31).                                          # cdb (ivp32.000,ivp32)

   ivp41 := @(sterm21) + @(sterm22) + @(sterm23).                # cdb (ivp41.000,ivp41)
   ivp42 := @(sterm31) + @(sterm32).                             # cdb (ivp42.000,ivp42)
   ivp43 := @(sterm41).                                          # cdb (ivp43.000,ivp43)

   ivp51 := @(sterm21) + @(sterm22) + @(sterm23) + @(sterm24).   # cdb (ivp51.000,ivp51)
   ivp52 := @(sterm31) + @(sterm32) + @(sterm33).                # cdb (ivp52.000,ivp52)
   ivp53 := @(sterm41) + @(sterm42).                             # cdb (ivp53.000,ivp53)
   ivp54 := @(sterm51).                                          # cdb (ivp54.000,ivp54)

   factor_out (ivp21,$A^{a?}$)                                   # cdb (ivp21.001,ivp21)

   factor_out (ivp31,$A^{a?}$)                                   # cdb (ivp31.001,ivp31)
   factor_out (ivp32,$A^{a?}$)                                   # cdb (ivp32.001,ivp32)

   factor_out (ivp41,$A^{a?}$)                                   # cdb (ivp41.001,ivp41)
   factor_out (ivp42,$A^{a?}$)                                   # cdb (ivp42.001,ivp42)
   factor_out (ivp43,$A^{a?}$)                                   # cdb (ivp43.001,ivp43)

   factor_out (ivp51,$A^{a?}$)                                   # cdb (ivp51.001,ivp51)
   factor_out (ivp52,$A^{a?}$)                                   # cdb (ivp52.001,ivp52)
   factor_out (ivp53,$A^{a?}$)                                   # cdb (ivp53.001,ivp53)
   factor_out (ivp54,$A^{a?}$)                                   # cdb (ivp54.001,ivp54)

   v{#}::LaTeXForm("\dot{x}").

   substitute (ivp21, $A^{a} -> v^{a}$)                          # cdb (ivp21.002,ivp21)

   substitute (ivp31, $A^{a} -> v^{a}$)                          # cdb (ivp31.002,ivp31)
   substitute (ivp32, $A^{a} -> v^{a}$)                          # cdb (ivp32.002,ivp32)

   substitute (ivp41, $A^{a} -> v^{a}$)                          # cdb (ivp41.002,ivp41)
   substitute (ivp42, $A^{a} -> v^{a}$)                          # cdb (ivp42.002,ivp42)
   substitute (ivp43, $A^{a} -> v^{a}$)                          # cdb (ivp43.002,ivp43)

   substitute (ivp51, $A^{a} -> v^{a}$)                          # cdb (ivp51.002,ivp51)
   substitute (ivp52, $A^{a} -> v^{a}$)                          # cdb (ivp52.002,ivp52)
   substitute (ivp53, $A^{a} -> v^{a}$)                          # cdb (ivp53.002,ivp53)
   substitute (ivp54, $A^{a} -> v^{a}$)                          # cdb (ivp54.002,ivp54)

   # build the Taylor series
   # note the inclusion of the 1/n! factors

   ivp2 := x^{a} + s v^{a} + (1/2) (s**2) @(ivp21).
   ivp3 := x^{a} + s v^{a} + (1/2) (s**2) @(ivp31) + (1/6) (s**3) @(ivp32).
   ivp4 := x^{a} + s v^{a} + (1/2) (s**2) @(ivp41) + (1/6) (s**3) @(ivp42) + (1/24) (s**4) @(ivp43).
   ivp5 := x^{a} + s v^{a} + (1/2) (s**2) @(ivp51) + (1/6) (s**3) @(ivp52) + (1/24) (s**4) @(ivp53) + (1/120) (s**5) @(ivp54).

   # cdb (ivp2.000,ivp2)
   # cdb (ivp3.000,ivp3)
   # cdb (ivp4.000,ivp4)
   # cdb (ivp5.000,ivp5)

   # now construct the scaled terms for ivp5

   sterm2 := @(sterm21) + @(sterm22) + @(sterm23) + @(sterm24).  # cdb (sterm2.000,sterm2)
   sterm3 := @(sterm31) + @(sterm32) + @(sterm33).               # cdb (sterm3.000,sterm3)
   sterm4 := @(sterm41) + @(sterm42).                            # cdb (sterm4.000,sterm4)
   sterm5 := @(sterm51).                                         # cdb (sterm5.000,sterm5)

   factor_out (sterm2,$A^{a?}$)                                  # cdb (sterm2.001,sterm2)
   factor_out (sterm3,$A^{a?}$)                                  # cdb (sterm3.001,sterm3)
   factor_out (sterm4,$A^{a?}$)                                  # cdb (sterm4.001,sterm4)
   factor_out (sterm5,$A^{a?}$)                                  # cdb (sterm5.001,sterm5)

   sterm2 := 360 @(sterm2).
   sterm3 := 360 @(sterm3).
   sterm4 :=  90 @(sterm4).
   sterm5 :=   3 @(sterm5).

   substitute (sterm2,$A^{a}->1$)                                # cdb (sterm2.002,sterm2)
   substitute (sterm3,$A^{a}->1$)                                # cdb (sterm3.002,sterm3)
   substitute (sterm4,$A^{a}->1$)                                # cdb (sterm4.002,sterm4)
   substitute (sterm5,$A^{a}->1$)                                # cdb (sterm5.002,sterm5)

\end{cadabra}

% =================================================================================================
% the remaining code is just for pretty printing

\clearpage

% =================================================================================================
\section*{The geodesic ivp}

\begin{align*}
   x^{a}(s) = x^{a}
            + s {\dot{x}^a}
            + \frac{s^2}{2!} {\dot{x}^b} {\dot{x}^c} A^{a}_{bc}
            + \frac{s^3}{3!} {\dot{x}^b} {\dot{x}^c} {\dot{x}^d} A^{a}_{bcd}
            + \frac{s^4}{4!} {\dot{x}^b} {\dot{x}^c} {\dot{x}^d} {\dot{x}^e} A^{a}_{bcde}
            + \frac{s^5}{5!} {\dot{x}^b} {\dot{x}^c} {\dot{x}^d} {\dot{x}^e} {\dot{x}^f} A^{a}_{bcdef}
            + \dotsb
\end{align*}
\begin{dgroup*}
   \begin{dmath*} 360 A^{a}_{bc} = \cdb{sterm2.002} \end{dmath*}
   \begin{dmath*} 360 A^{a}_{bcd} = \cdb{sterm3.002} \end{dmath*}
   \begin{dmath*}  90 A^{a}_{bcde} = \cdb{sterm4.002} \end{dmath*}
   \begin{dmath*}   3 A^{a}_{bcdef} = \cdb{sterm5.002} \end{dmath*}
\end{dgroup*}

\clearpage

% =================================================================================================
% export selected objects, these will later be imported into a library
% these are the objects that will appear in the paper

\begin{cadabra}
   sterm2short := @(sterm21) + @(sterm22).             # cdb (sterm2.short.001,sterm2short)
   sterm3short := @(sterm31).                          # cdb (sterm3.short.001,sterm3short)
   sterm2shortscaled := 12 @(sterm2short).             # cdb (sterm2.short.scaled.002,sterm2shortscaled)
   sterm3shortscaled :=  2 @(sterm3short).             # cdb (sterm3.short.scaled.002,sterm3shortscaled)

   substitute (sterm2shortscaled,$A^{a}->1$)           # cdb (sterm2.short.scaled.003,sterm2shortscaled)
   substitute (sterm3shortscaled,$A^{a}->1$)           # cdb (sterm3.short.scaled.003,sterm3shortscaled)

   cdblib.create ('geodesic-ivp.export')

   # 4th order ivp terms scaled
   cdblib.put ('ivp42',sterm2shortscaled,'geodesic-ivp.export')
   cdblib.put ('ivp43',sterm3shortscaled,'geodesic-ivp.export')

   # 6th order ivp terms scaled
   cdblib.put ('ivp62',sterm2,'geodesic-ivp.export')
   cdblib.put ('ivp63',sterm3,'geodesic-ivp.export')
   cdblib.put ('ivp64',sterm4,'geodesic-ivp.export')
   cdblib.put ('ivp65',sterm5,'geodesic-ivp.export')

   checkpoint.append (sterm2shortscaled)
   checkpoint.append (sterm3shortscaled)

   checkpoint.append (sterm2)
   checkpoint.append (sterm3)
   checkpoint.append (sterm4)
   checkpoint.append (sterm5)

   cdblib.create ('geodesic-ivp.json')

   cdblib.put ('ivp21',ivp21,'geodesic-ivp.json')

   cdblib.put ('ivp31',ivp31,'geodesic-ivp.json')
   cdblib.put ('ivp32',ivp32,'geodesic-ivp.json')

   cdblib.put ('ivp41',ivp41,'geodesic-ivp.json')
   cdblib.put ('ivp42',ivp42,'geodesic-ivp.json')
   cdblib.put ('ivp43',ivp43,'geodesic-ivp.json')

   cdblib.put ('ivp51',ivp51,'geodesic-ivp.json')
   cdblib.put ('ivp52',ivp52,'geodesic-ivp.json')
   cdblib.put ('ivp53',ivp53,'geodesic-ivp.json')
   cdblib.put ('ivp54',ivp54,'geodesic-ivp.json')

   cdblib.put ('ivp2',ivp2,'geodesic-ivp.json')
   cdblib.put ('ivp3',ivp3,'geodesic-ivp.json')
   cdblib.put ('ivp4',ivp4,'geodesic-ivp.json')
   cdblib.put ('ivp5',ivp5,'geodesic-ivp.json')
\end{cadabra}

% just to check that we are exporting the correct 4th order terms

\begin{dgroup*}
   \begin{dmath*} \cdb*{sterm2.short.001} \end{dmath*}
   \begin{dmath*} \cdb*{sterm3.short.001} \end{dmath*}
   \begin{dmath*} \cdb*{sterm2.short.scaled.002} \end{dmath*}
   \begin{dmath*} \cdb*{sterm3.short.scaled.002} \end{dmath*}
   \begin{dmath*} \cdb*{sterm2.short.scaled.003} \end{dmath*}
   \begin{dmath*} \cdb*{sterm3.short.scaled.003} \end{dmath*}
\end{dgroup*}

\begin{dgroup*}
   \begin{dmath*} \cdb*{ivp21.002} \end{dmath*}
   \begin{dmath*} \cdb*{ivp31.002} \end{dmath*}
   \begin{dmath*} \cdb*{ivp32.002} \end{dmath*}
   \begin{dmath*} \cdb*{ivp41.002} \end{dmath*}
   \begin{dmath*} \cdb*{ivp42.002} \end{dmath*}
   \begin{dmath*} \cdb*{ivp43.002} \end{dmath*}
   \begin{dmath*} \cdb*{ivp51.002} \end{dmath*}
   \begin{dmath*} \cdb*{ivp52.002} \end{dmath*}
   \begin{dmath*} \cdb*{ivp53.002} \end{dmath*}
   \begin{dmath*} \cdb*{ivp54.002} \end{dmath*}
\end{dgroup*}

\begin{dgroup*}
   \begin{dmath*} \cdb*{ivp2.000} \end{dmath*}
   \begin{dmath*} \cdb*{ivp3.000} \end{dmath*}
   \begin{dmath*} \cdb*{ivp4.000} \end{dmath*}
   % \begin{dmath*} \cdb*{ivp5.000} \end{dmath*}
\end{dgroup*}

% =================================================================================================
% export checkpoints in json format

\bgroup
\CdbSetup{action=hide}
\begin{cadabra}
   for i in range( len(checkpoint) ):
      cdblib.put ('check{:03d}'.format(i),checkpoint[i],checkpoint_file)
\end{cadabra}
\egroup

\end{document}


\begin{align*}
   x^{d}(s) = x^{d}
            + s {\dot{x}^d}
            + \frac{s^2}{2!} {\dot{x}^a} {\dot{x}^b} A^{d}_{ab}
            + \frac{s^3}{3!} {\dot{x}^a} {\dot{x}^b} {\dot{x}^c} A^{d}_{abc}
            + \frac{s^4}{4!} {\dot{x}^a} {\dot{x}^b} {\dot{x}^c} {\dot{x}^d} A^{d}_{abce}
            + \frac{s^5}{5!} {\dot{x}^a} {\dot{x}^b} {\dot{x}^c} {\dot{x}^d} {\dot{x}^e} A^{d}_{abcef}
            + \dotsb
\end{align*}
\begin{dgroup*}
   \begin{dmath*} 360 A^{d}_{ab} = \cdb{sterm2.002} \end{dmath*}
   \begin{dmath*} 360 A^{d}_{abc} = \cdb{sterm3.002} \end{dmath*}
   \begin{dmath*}  90 A^{d}_{abce} = \cdb{sterm4.002} \end{dmath*}
   \begin{dmath*}   3 A^{d}_{abcef} = \cdb{sterm5.002} \end{dmath*}
\end{dgroup*}

\clearpage

% =================================================================================================
\section*{Geodesic boundary value problem to terms linear in $R$}
\documentclass[12pt]{cdblatex}
\usepackage{fancyhdr}
\usepackage{footer}

\begin{document}

% =================================================================================================
% create checkpoint file

\bgroup
\CdbSetup{action=hide}
\begin{cadabra}
   import cdblib
   checkpoint_file = 'tests/semantic/output/geodesic-bvp.json'
   cdblib.create (checkpoint_file)
   checkpoint = []
\end{cadabra}
\egroup

% =================================================================================================
\section*{Geodesic BVP}

Consider a geodesic that connects two points $P_i$ and $P_j$ with RNC coordinates
$x^a_i$ and $x^a_j$. Our aim is to construct a solution $x^a(s)$ of the geodesic
equation such that $x^a(0)= x^a_i$ and $x^a(1)=x^a_j$.

We will do this in two stages. First we will solve
\begin{align}
   \label{eq:twoptBVP}
   x^a_j = x^a_i + y^a - \sum_{k=2}^\infty\>\frac{1}{k!}\>\Gamma^{a}_{\ubk}y^{.\ubk}
\end{align}
for $y^a$ as an explicit polynomial in $x^a_i$ and $x^a_j$. The functions $\Gamma^{a}_{\ubk}$ are
the generalised connections for the RNC frame evaluated at $x^a=x^a_i$.

In the second stage, we will substitute our expression for $y^a$ into
\begin{align}
   \label{eq:solBVP}
   x^a(s) = x^a_i + s y^a - \sum_{k=2}^\infty\>\frac{1}{k!}\>\Gamma^{a}_{\ubk}y^{.\ubk} s^k
\end{align}
to obtain the desired solution to the two point boundary value problem.

% =================================================================================================
\section*{Stage 1: The fixed point iteration scheme}

First we rewrite the main equation \eqref{eq:twoptBVP} in the suggestive form
\begin{align*}
   y^a = \Dx^a + \sum_{k=2}^\infty\>\frac{1}{k!}\>\Gamma^{a}_{\ubk}y^{\ubk}
\end{align*}
where $\Dx^a = x^a_j-x^a_i$. Our approximate solution for $y^a$ will be taken to be the partial
sums for the infinite series. Thus we will solve
\begin{align*}
   \ny{n}^a = \Dx^a + \sum_{k=2}^n\>\frac{1}{k!}\>\Gamma^{a}_{\ubk}\ny{n}^{\ubk}
\end{align*}
for $\ny{n}^a$. Note that in the last term of the sum, the $\Gamma^{a}_{\ubn}$ will contain
curvature terms of order $\BigO{\eps^n}$. Thus in truncating the series at this point we will
loose contributions to the curvature terms of order $\BigO{\eps^{n+1}}$ and higher. So to be
consistent we must truncate all terms of the partial sum to order $\BigO{\eps^n}$ (i.e., exclude
any contributions from terms $\BigO{\eps^{n+1}}$ and higher, these are the terms that would
couple with the terms that we excluded when truncating the original infinite series). Let
$\nT{k}$ be the operator that truncates its argument to contain terms no higher than
$\BigO{\eps^n}$. Then we have the following modified version of the equation for $\ny{n}^a$
\begin{align*}
   \ny{n}^a = \Dx^a
            + \sum_{k=2}^n\>\frac{1}{k!}\>\nT{k}\left(\Gamma^{a}_{\ubk}\ny{n}^{\ubk}\right)
\end{align*}
Finally we note that since $\Gamma^{a}_{\ubk} = \BigO{\eps^k}$, we can use lower order estimates
for the $\ny{k}^a$ in the right hand side of the sum. This allows us to compute $\ny{n}^a$ by
successive approximations such as
\begin{align*}
   \ny{0}^a &= \Dx^a\\
   \ny{2}^a &= \ny{0}^a + \frac{1}{2!}\nT{2}\left(\Gamma^a_{bc}\>\ny{0}^b \ny{0}^c\right)\\[5pt]
   \ny{3}^a &= \ny{0}^a + \frac{1}{2!}\nT{3}\left(\Gamma^a_{bc}\>\ny{2}^b \ny{2}^c\right)
                        + \frac{1}{3!}\nT{3}\left(\Gamma^a_{bcd}\>\ny{0}^b \ny{0}^c \ny{0}^d\right)\\[5pt]
   \ny{4}^a &= \ny{0}^a + \frac{1}{2!}\nT{4}\left(\Gamma^a_{bc}\>\ny{3}^b \ny{3}^c\right)
                        + \frac{1}{3!}\nT{4}\left(\Gamma^a_{bcd}\>\ny{2}^b \ny{2}^c \ny{2}^d\right)
                        + \frac{1}{4!}\nT{4}\left(\Gamma^a_{bcde}\>\ny{0}^b \ny{0}^c \ny{0}^d \ny{0}^e\right)\\[5pt]
   \ny{5}^a &= \ny{0}^a + \frac{1}{2!}\nT{5}\left(\Gamma^a_{bc}\>\ny{4}^b \ny{4}^c\right)
                        + \frac{1}{3!}\nT{5}\left(\Gamma^a_{bcd}\>\ny{3}^b \ny{3}^c \ny{3}^d\right)
                        + \frac{1}{4!}\nT{5}\left(\Gamma^a_{bcde}\>\ny{2}^b \ny{2}^c \ny{2}^d \ny{2}^e\right)
                        + \frac{1}{5!}\nT{5}\left(\Gamma^a_{bcdef}\>\ny{0}^b \ny{0}^c \ny{0}^d \ny{0}^e \ny{0}^f\right)
\end{align*}
and so on. Note that there are no $\ny{1}^a$ terms.

% =================================================================================================
\section*{Stage 2: Introduce the generalised connections}

This is the final stage -- it introduces the generalised connecstion after the
completion of the fixed point scheme.

All that needs be done is to substitute our expression for $y^a$ into \eqref{eq:solBVP}
% \begin{align}
%    \label{eq:solBVP}
%    x^a(s) = x^a_i + s y^a - \sum_{k=2}^\infty\>\frac{1}{k!}\>\Gamma^{a}{}_{\ubk} y^{.\ubk} s^k
% \end{align}
to obtain the desired solution to the two point boundary value problem.

The generalised connections $\Gamma^{a}{}_{\ubk}$ are taken from the results of the
{\tts genGamma} code.

\clearpage

% =================================================================================================
\section*{Stage 1: The fixed point iteration scheme}

\begin{cadabra}
   import time

   {a,b,c,d,e,f,g,h,i,j,k,l,m,n,o,p,q,r,s,t,u,v,w#}::Indices(position=independent).

   \nabla{#}::Derivative.

   g_{a b}::Metric.
   g^{a b}::InverseMetric.

   R_{a b c d}::RiemannTensor.
   R_{a b c d}::Depends(\nabla{#}).

   {Gam22^{a}_{b c},Gam23^{a}_{b c},Gam24^{a}_{b c},Gam25^{a}_{b c}}::TableauSymmetry(shape={2}, indices={1,2}).
   {Gam33^{a}_{b c d},Gam34^{a}_{b c d},Gam35^{a}_{b c d}}::TableauSymmetry(shape={3}, indices={1,2,3}).
   {Gam44^{a}_{b c d e},Gam45^{a}_{b c d e}}::TableauSymmetry(shape={4}, indices={1,2,3,4}).
   {Gam55^{a}_{b c d e f}}::TableauSymmetry(shape={5}, indices={1,2,3,4,5}).

   {Gam22^{a}_{b c}}::Weight(label=eps,value=2).
   {Gam23^{a}_{b c},Gam33^{a}_{b c d}}::Weight(label=eps,value=3).
   {Gam24^{a}_{b c},Gam34^{a}_{b c d},Gam44^{a}_{b c d e}}::Weight(label=eps,value=4).
   {Gam25^{a}_{b c},Gam35^{a}_{b c d},Gam45^{a}_{b c d e},Gam55^{a}_{b c d e f}}::Weight(label=eps,value=5).

   {Dx^{a}}::Weight(label=eps,value=0).

   {y00^{a},y20^{a},y30^{a},y40^{a},y50^{a}}::Weight(label=eps,value=0).
   {y22^{a},y32^{a},y42^{a},y52^{a}}::Weight(label=eps,value=2).
   {y33^{a},y43^{a},y53^{a}}::Weight(label=eps,value=3).
   {y44^{a},y54^{a}}::Weight(label=eps,value=4).
   {y55^{a}}::Weight(label=eps,value=5).

   # Dx{#}::LaTeXForm{"{\Dx}"}.  # LCB: currently causes a bug, it kills ::KeepWeight for Dx

   # note: keeping numbering as is (out of order) to ensure R appears before \nabla R etc.
   def product_sort (obj):
       substitute (obj,$ x^{a}                            -> A001^{a}               $)
       substitute (obj,$ Dx^{a}                           -> A002^{a}               $)
       substitute (obj,$ g^{a b}                          -> A003^{a b}             $)
       substitute (obj,$ \nabla_{e f g h}{R_{a b c d}}    -> A008_{a b c d e f g h} $)
       substitute (obj,$ \nabla_{e f g}{R_{a b c d}}      -> A007_{a b c d e f g}   $)
       substitute (obj,$ \nabla_{e f}{R_{a b c d}}        -> A006_{a b c d e f}     $)
       substitute (obj,$ \nabla_{e}{R_{a b c d}}          -> A005_{a b c d e}       $)
       substitute (obj,$ R_{a b c d}                      -> A004_{a b c d}         $)
       sort_product   (obj)
       rename_dummies (obj)
       substitute (obj,$ A001^{a}                  -> x^{a}                         $)
       substitute (obj,$ A002^{a}                  -> Dx^{a}                        $)
       substitute (obj,$ A003^{a b}                -> g^{a b}                       $)
       substitute (obj,$ A004_{a b c d}            -> R_{a b c d}                   $)
       substitute (obj,$ A005_{a b c d e}          -> \nabla_{e}{R_{a b c d}}       $)
       substitute (obj,$ A006_{a b c d e f}        -> \nabla_{e f}{R_{a b c d}}     $)
       substitute (obj,$ A007_{a b c d e f g}      -> \nabla_{e f g}{R_{a b c d}}   $)
       substitute (obj,$ A008_{a b c d e f g h}    -> \nabla_{e f g h}{R_{a b c d}} $)

       return obj

   def get_term (obj,n):

       tmp := @(obj).
       foo = Ex("eps = " + str(n))
       distribute  (tmp)
       keep_weight (tmp, foo)

       return tmp

   def truncate (obj,n):

       ans = Ex(0)

       for i in range (0,n+1):
          foo := @(obj).
          bah = Ex("eps = " + str(i))
          distribute  (foo)
          keep_weight (foo, bah)
          ans = ans + foo

       return ans

   def substitute_eps (obj):
       substitute     (obj,epsy0)
       substitute     (obj,epsy2)
       substitute     (obj,epsy3)
       substitute     (obj,epsy4)
       substitute     (obj,epsy5)
       substitute     (obj,epsGam2)
       substitute     (obj,epsGam3)
       substitute     (obj,epsGam4)
       substitute     (obj,epsGam5)
       distribute     (obj)
       obj = truncate     (obj,5)
       obj = product_sort (obj)
       rename_dummies (obj)
       canonicalise   (obj)

       return obj

   beg_stage_1 = time.time()

   # yn = y expanded to terms upto and including O(eps^n)

   y0 := Dx^{a}.
   y2 := Dx^{a} +   (1/2) Gam^{a}_{b c} y0^{b} y0^{c}.
   y3 := Dx^{a} +   (1/2) Gam^{a}_{b c} y2^{b} y2^{c}
                +   (1/6) Gam^{a}_{b c d} y0^{b} y0^{c} y0^{d}.
   y4 := Dx^{a} +   (1/2) Gam^{a}_{b c} y3^{b} y3^{c}
                +   (1/6) Gam^{a}_{b c d} y2^{b} y2^{c} y2^{d}
                +  (1/24) Gam^{a}_{b c d e} y0^{b} y0^{c} y0^{d} y0^{e}.
   y5 := Dx^{a} +   (1/2) Gam^{a}_{b c} y4^{b} y4^{c}
                +   (1/6) Gam^{a}_{b c d} y3^{b} y3^{c} y3^{d}
                +  (1/24) Gam^{a}_{b c d e} y2^{b} y2^{c} y2^{d} y2^{e}
                + (1/120) Gam^{a}_{b c d e f} y0^{b} y0^{c} y0^{d} y0^{e} y0^{f}.

   # epsyN = y expanded to terms upto and including O(eps^N)
   # yPQ = O(eps^Q) term of epsyP

   # expand to O(eps^5)

   epsy0 := y0^{a} -> y00^{a}.
   epsy2 := y2^{a} -> y20^{a}+y22^{a}.
   epsy3 := y3^{a} -> y30^{a}+y32^{a}+y33^{a}.
   epsy4 := y4^{a} -> y40^{a}+y42^{a}+y43^{a}+y44^{a}.
   epsy5 := y5^{a} -> y50^{a}+y52^{a}+y53^{a}+y54^{a}+y55^{a}.

   # epsGamN = gen. gamma with N lower indices (epsGam2 = the connection)
   # epsGamPQ = O(eps^Q) term of epsGamP

   epsGam2 := Gam^{a}_{b c} -> Gam22^{a}_{b c}+Gam23^{a}_{b c}+Gam24^{a}_{b c}+Gam25^{a}_{b c}.
   epsGam3 := Gam^{a}_{b c d} -> Gam33^{a}_{b c d}+Gam34^{a}_{b c d}+Gam35^{a}_{b c d}.
   epsGam4 := Gam^{a}_{b c d e} -> Gam44^{a}_{b c d e}+Gam45^{a}_{b c d e}.
   epsGam5 := Gam^{a}_{b c d e f} -> Gam55^{a}_{b c d e f}.

   y0 = substitute_eps (y0)   # cdb (y0.001,y0)
   y2 = substitute_eps (y2)   # cdb (y2.001,y2)
   y3 = substitute_eps (y3)   # cdb (y3.001,y3)
   y4 = substitute_eps (y4)   # cdb (y4.001,y4)
   y5 = substitute_eps (y5)   # cdb (y5.001,y5)

   y0 = truncate (y0,1)       # cdb (y0.002,y0)
   y2 = truncate (y2,2)       # cdb (y2.002,y2)
   y3 = truncate (y3,3)       # cdb (y3.002,y3)
   y4 = truncate (y4,4)       # cdb (y4.002,y4)
   y5 = truncate (y5,5)       # cdb (y5.002,y5)

   defy0 := y0^{a} -> @(y0).
   defy2 := y2^{a} -> @(y2).
   defy3 := y3^{a} -> @(y3).
   defy4 := y4^{a} -> @(y4).
   defy5 := y5^{a} -> @(y5).

   # -----------------------------------
   def tidy (obj):
       obj = product_sort (obj)
       rename_dummies     (obj)
       canonicalise       (obj)
       return obj

   # -----------------------------------
   # y0

   y00 := @(y0).           # cdb (y00.101,y00)

   defy00 := y00^{a} -> @(y00).

   # -----------------------------------
   # y2

   substitute (y2,defy00)

   distribute (y2)

   y20 = get_term (y2,0)   # cdb (y20.101,y20)
   y22 = get_term (y2,2)   # cdb (y22.101,y22)

   y20 = tidy (y20)        # cdb (y20.201,y20)
   y22 = tidy (y22)        # cdb (y22.201,y22)

   defy20 := y20^{a} -> @(y20).
   defy22 := y22^{a} -> @(y22).

   # -----------------------------------
   # y3

   substitute (y3,defy00)

   substitute (y3,defy20)
   substitute (y3,defy22)

   distribute (y3)

   y30 = get_term (y3,0)   # cdb (y30.101,y30)
   y32 = get_term (y3,2)   # cdb (y32.101,y32)
   y33 = get_term (y3,3)   # cdb (y33.101,y33)

   y30 = tidy (y30)        # cdb (y30.201,y30)
   y32 = tidy (y32)        # cdb (y32.201,y32)
   y33 = tidy (y33)        # cdb (y33.201,y33)

   defy30 := y30^{a} -> @(y30).
   defy32 := y32^{a} -> @(y32).
   defy33 := y33^{a} -> @(y33).

   # -----------------------------------
   # y4

   substitute (y4,defy00)

   substitute (y4,defy20)
   substitute (y4,defy22)

   substitute (y4,defy30)
   substitute (y4,defy32)
   substitute (y4,defy33)

   distribute (y4)

   y40 = get_term (y4,0)   # cdb (y40.101,y40)
   y42 = get_term (y4,2)   # cdb (y42.101,y42)
   y43 = get_term (y4,3)   # cdb (y43.101,y43)
   y44 = get_term (y4,4)   # cdb (y44.101,y44)

   y40 = tidy (y40)        # cdb (y40.201,y40)
   y42 = tidy (y42)        # cdb (y42.201,y42)
   y43 = tidy (y43)        # cdb (y43.201,y43)
   y44 = tidy (y44)        # cdb (y44.201,y44)

   defy40 := y40^{a} -> @(y40).
   defy42 := y42^{a} -> @(y42).
   defy43 := y43^{a} -> @(y43).
   defy44 := y44^{a} -> @(y44).

   # -----------------------------------
   # y5

   substitute (y5,defy00)

   substitute (y5,defy20)
   substitute (y5,defy22)

   substitute (y5,defy30)
   substitute (y5,defy32)
   substitute (y5,defy33)

   substitute (y5,defy40)
   substitute (y5,defy42)
   substitute (y5,defy43)
   substitute (y5,defy44)

   distribute (y5)

   y50 = get_term (y5,0)   # cdb (y50.101,y50)
   y52 = get_term (y5,2)   # cdb (y52.101,y52)
   y53 = get_term (y5,3)   # cdb (y53.101,y53)
   y54 = get_term (y5,4)   # cdb (y54.101,y54)
   y55 = get_term (y5,5)   # cdb (y55.101,y55)

   y50 = tidy (y50)        # cdb (y50.201,y50)
   y52 = tidy (y52)        # cdb (y52.201,y52)
   y53 = tidy (y53)        # cdb (y53.201,y53)
   y54 = tidy (y54)        # cdb (y54.201,y54)
   y55 = tidy (y55)        # cdb (y55.201,y55)

   defy50 := y50^{a} -> @(y50).
   defy52 := y52^{a} -> @(y52).
   defy53 := y53^{a} -> @(y53).
   defy54 := y54^{a} -> @(y54).
   defy55 := y55^{a} -> @(y55).

   end_stage_1 = time.time()

\end{cadabra}

\clearpage
\begin{dgroup*}
   \begin{dmath*} \cdb*{y0.001} \end{dmath*}
   \begin{dmath*} \cdb*{y2.001} \end{dmath*}
   \begin{dmath*} \cdb*{y3.001} \end{dmath*}
   \begin{dmath*} \cdb*{y4.001} \end{dmath*}
   \begin{dmath*} \cdb*{y5.001} \end{dmath*}
\end{dgroup*}

\clearpage
\begin{dgroup*}
   \begin{dmath*} \cdb*{y0.002} \end{dmath*}
   \begin{dmath*} \cdb*{y2.002} \end{dmath*}
   \begin{dmath*} \cdb*{y3.002} \end{dmath*}
   \begin{dmath*} \cdb*{y4.002} \end{dmath*}
   \begin{dmath*} \cdb*{y5.002} \end{dmath*}
\end{dgroup*}

\clearpage
\begin{dgroup*}
   \begin{dmath*} \cdb*{y00.101} \end{dmath*}
\end{dgroup*}

\begin{dgroup*}
   \begin{dmath*} \cdb*{y20.201} \end{dmath*}
   \begin{dmath*} \cdb*{y22.201} \end{dmath*}
\end{dgroup*}

\begin{dgroup*}
   \begin{dmath*} \cdb*{y30.201} \end{dmath*}
   \begin{dmath*} \cdb*{y32.201} \end{dmath*}
   \begin{dmath*} \cdb*{y33.201} \end{dmath*}
\end{dgroup*}

\begin{dgroup*}
   \begin{dmath*} \cdb*{y40.201} \end{dmath*}
   \begin{dmath*} \cdb*{y42.201} \end{dmath*}
   \begin{dmath*} \cdb*{y43.201} \end{dmath*}
   \begin{dmath*} \cdb*{y44.201} \end{dmath*}
\end{dgroup*}

\clearpage
\begin{dgroup*}
   \begin{dmath*} \cdb*{y50.201} \end{dmath*}
   \begin{dmath*} \cdb*{y52.201} \end{dmath*}
   \begin{dmath*} \cdb*{y53.201} \end{dmath*}
   \begin{dmath*} \cdb*{y54.201} \end{dmath*}
   \begin{dmath*} \cdb*{y55.201} \end{dmath*}
\end{dgroup*}

\clearpage

% =================================================================================================
\section*{Stage 2a: Introduce the generalised connections, build terms of $y^{a}$}

\begin{cadabra}
   def substitute_gam (obj):

       substitute (obj,defGam22)
       substitute (obj,defGam23)
       substitute (obj,defGam24)
       substitute (obj,defGam25)

       substitute (obj,defGam33)
       substitute (obj,defGam34)
       substitute (obj,defGam35)

       substitute (obj,defGam44)
       substitute (obj,defGam45)

       substitute (obj,defGam55)

       distribute (obj)
       return obj

   import cdblib

   beg_stage_2a = time.time()

   Gam22 = cdblib.get ('genGamma01','genGamma.json')
   Gam23 = cdblib.get ('genGamma02','genGamma.json')
   Gam24 = cdblib.get ('genGamma03','genGamma.json')
   Gam25 = cdblib.get ('genGamma04','genGamma.json')

   Gam33 = cdblib.get ('genGamma11','genGamma.json')
   Gam34 = cdblib.get ('genGamma12','genGamma.json')
   Gam35 = cdblib.get ('genGamma13','genGamma.json')

   Gam44 = cdblib.get ('genGamma21','genGamma.json')
   Gam45 = cdblib.get ('genGamma22','genGamma.json')

   Gam55 = cdblib.get ('genGamma31','genGamma.json')

   # peel off the A^{a}, must then symmetrise over revealed indices

   substitute (Gam22,$A^{a}->1$)
   substitute (Gam23,$A^{a}->1$)
   substitute (Gam24,$A^{a}->1$)
   substitute (Gam25,$A^{a}->1$)

   substitute (Gam33,$A^{a}->1$)
   substitute (Gam34,$A^{a}->1$)
   substitute (Gam35,$A^{a}->1$)

   substitute (Gam44,$A^{a}->1$)
   substitute (Gam45,$A^{a}->1$)

   substitute (Gam55,$A^{a}->1$)

   # now symmetrise

   sym (Gam22,$_{b},_{c}$)
   sym (Gam23,$_{b},_{c}$)
   sym (Gam24,$_{b},_{c}$)
   sym (Gam25,$_{b},_{c}$)

   sym (Gam33,$_{b},_{c},_{d}$)
   sym (Gam34,$_{b},_{c},_{d}$)
   sym (Gam35,$_{b},_{c},_{d}$)

   sym (Gam44,$_{b},_{c},_{d},_{e}$)
   sym (Gam45,$_{b},_{c},_{d},_{e}$)

   sym (Gam55,$_{b},_{c},_{d},_{e},_{f}$)

   defGam22 := Gam22^{a}_{b c} -> @(Gam22).
   defGam23 := Gam23^{a}_{b c} -> @(Gam23).
   defGam24 := Gam24^{a}_{b c} -> @(Gam24).
   defGam25 := Gam25^{a}_{b c} -> @(Gam25).

   defGam33 := Gam33^{a}_{b c d} -> @(Gam33).
   defGam34 := Gam34^{a}_{b c d} -> @(Gam34).
   defGam35 := Gam35^{a}_{b c d} -> @(Gam35).

   defGam44 := Gam44^{a}_{b c d e} -> @(Gam44).
   defGam45 := Gam45^{a}_{b c d e} -> @(Gam45).

   defGam55 := Gam55^{a}_{b c d e f} -> @(Gam55).

   # ---------------------------------------------------
   # y2

   y22 = substitute_gam (y22)

   y22 = tidy (y22)                                        # cdb (y22.301,y22)

   y2 := @(y20) + @(y22).                                  # cdb (y2.301,y2)

   # ---------------------------------------------------
   # y3

   y32 = substitute_gam (y32)
   y33 = substitute_gam (y33)

   y32 = tidy (y32)                                        # cdb (y32.301,y32)
   y33 = tidy (y33)                                        # cdb (y33.301,y33)

   y3 := @(y30) + @(y32) + @(y33).                         # cdb (y3.301,y3)

   # ---------------------------------------------------
   # y4

   y42 = substitute_gam (y42)
   y43 = substitute_gam (y43)
   y44 = substitute_gam (y44)

   y42 = tidy (y42)                                        # cdb (y42.301,y42)
   y43 = tidy (y43)                                        # cdb (y43.301,y43)
   y44 = tidy (y44)                                        # cdb (y44.301,y44)

   y4 := @(y40) + @(y42) + @(y43) + @(y44).                # cdb (y4.301,y4)

   # ---------------------------------------------------
   # y5

   y52 = substitute_gam (y52)
   y53 = substitute_gam (y53)
   y54 = substitute_gam (y54)
   y55 = substitute_gam (y55)

   y52 = tidy (y52)                                        # cdb (y52.301,y52)
   y53 = tidy (y53)                                        # cdb (y53.301,y53)
   y54 = tidy (y54)                                        # cdb (y54.301,y54)
   y55 = tidy (y55)                                        # cdb (y55.301,y55)

   y5 := @(y50) + @(y52) + @(y53) + @(y54) + @(y55).       # cdb (y5.301,y5)

   # ---------------------------------------------------
   cdblib.create ('geodesic-bvp.json')

   cdblib.put ('y2',y2,'geodesic-bvp.json')
   cdblib.put ('y3',y3,'geodesic-bvp.json')
   cdblib.put ('y4',y4,'geodesic-bvp.json')
   cdblib.put ('y5',y5,'geodesic-bvp.json')

   cdblib.put ('y20',y20,'geodesic-bvp.json')
   cdblib.put ('y22',y22,'geodesic-bvp.json')

   cdblib.put ('y30',y30,'geodesic-bvp.json')
   cdblib.put ('y32',y32,'geodesic-bvp.json')
   cdblib.put ('y33',y33,'geodesic-bvp.json')

   cdblib.put ('y40',y40,'geodesic-bvp.json')
   cdblib.put ('y42',y42,'geodesic-bvp.json')
   cdblib.put ('y43',y43,'geodesic-bvp.json')
   cdblib.put ('y44',y44,'geodesic-bvp.json')

   cdblib.put ('y50',y50,'geodesic-bvp.json')
   cdblib.put ('y52',y52,'geodesic-bvp.json')
   cdblib.put ('y53',y53,'geodesic-bvp.json')
   cdblib.put ('y54',y54,'geodesic-bvp.json')
   cdblib.put ('y55',y55,'geodesic-bvp.json')

   end_stage_2a = time.time()

\end{cadabra}

% note that:
%   y00 = y20 = y30 = y40 = y50
%   y22 = y32 = y42 = y52
%   y33 = y43 = y53
%   y44 = y54
%   y55

\clearpage

\begin{dgroup*}
   \begin{dmath*} \cdb*{y50.201} \end{dmath*}  % unchanged
   \begin{dmath*} \cdb*{y52.301} \end{dmath*}
   \begin{dmath*} \cdb*{y53.301} \end{dmath*}
   \begin{dmath*} \cdb*{y54.301} \end{dmath*}
   \begin{dmath*} \cdb*{y55.301} \end{dmath*}
\end{dgroup*}

\clearpage

% =================================================================================================
\section*{Stage 2b: Building the terms of $x^a(s)$}

\begin{cadabra}
   def substitute_y (obj):
       substitute (obj,defy00)
       substitute (obj,defy20)
       substitute (obj,defy30)
       substitute (obj,defy32)
       substitute (obj,defy40)
       substitute (obj,defy42)
       substitute (obj,defy43)
       distribute (obj)
       return obj

   beg_stage_2b = time.time()

   term2 := Gam^{a}_{b c} y4^{b} y4^{c}.
   term3 := Gam^{a}_{b c d} y3^{b} y3^{c} y3^{d}.
   term4 := Gam^{a}_{b c d e} y2^{b} y2^{c} y2^{d} y2^{e}.
   term5 := Gam^{a}_{b c d e f} y0^{b} y0^{c} y0^{d} y0^{e} y0^{f}.

   term2 = substitute_eps (term2)   # cdb (term2.401,term2)
   term3 = substitute_eps (term3)   # cdb (term3.401,term3)
   term4 = substitute_eps (term4)   # cdb (term4.401,term4)
   term5 = substitute_eps (term5)   # cdb (term5.401,term5)

   term2 = substitute_y (term2)
   term3 = substitute_y (term3)
   term4 = substitute_y (term4)
   term5 = substitute_y (term5)

   term2 = substitute_gam (term2)
   term3 = substitute_gam (term3)
   term4 = substitute_gam (term4)
   term5 = substitute_gam (term5)

   term2 = tidy (term2)   # cdb (term2.501,term2)
   term3 = tidy (term3)   # cdb (term3.501,term3)
   term4 = tidy (term4)   # cdb (term4.501,term4)
   term5 = tidy (term5)   # cdb (term5.501,term5)

\end{cadabra}

\clearpage
\begin{dgroup*}
   \begin{dmath*} \cdb*{term2.401} \end{dmath*}
   \begin{dmath*} \cdb*{term3.401} \end{dmath*}
   \begin{dmath*} \cdb*{term4.401} \end{dmath*}
   \begin{dmath*} \cdb*{term5.401} \end{dmath*}
\end{dgroup*}

\clearpage
\begin{dgroup*}
   \begin{dmath*} \cdb*{term2.501} \end{dmath*}
   \begin{dmath*} \cdb*{term3.501} \end{dmath*}
   \begin{dmath*} \cdb*{term4.501} \end{dmath*}
   \begin{dmath*} \cdb*{term5.501} \end{dmath*}
\end{dgroup*}

\clearpage

\begin{cadabra}
   # Check:
   #    x^{a} at s=1 should equal x^{a} + Dx^{a}
   #    but x^{a}(s) = x^{a} + s y^{a} - \sum (1/n!) @(termn) s^n
   #    thus foo should equal Dx^{a} and it does (yeah)

   foo := @(y5)
        -   (1/2) @(term2)
        -   (1/6) @(term3)
        -  (1/24) @(term4)
        - (1/120) @(term5).

   distribute         (foo)
   obj = product_sort (foo)
   rename_dummies     (foo)
   canonicalise       (foo)     # cdb (foo.001,foo)

   term2 :=   (1/2) @(term2).   # cdb(term2.502,term2)
   term3 :=   (1/6) @(term3).   # cdb(term3.502,term3)
   term4 :=  (1/24) @(term4).   # cdb(term4.502,term4)
   term5 := (1/120) @(term5).   # cdb(term5.502,term5)

   end_stage_2b = time.time()

\end{cadabra}

\begin{dgroup*}
   \begin{dmath*} \cdb*{foo.001} \end{dmath*}
\end{dgroup*}

\begin{dgroup*}
   \begin{dmath*} \cdb*{y2.301} \end{dmath*}
   \begin{dmath*} \cdb*{y3.301} \end{dmath*}
   \begin{dmath*} \cdb*{y4.301} \end{dmath*}
   % \begin{dmath*} \cdb*{y5.301} \end{dmath*}
\end{dgroup*}

\clearpage

% =================================================================================================
\section*{Stage 3: Reformatting and output}

\begin{cadabra}
   def get_Rterm (obj,n):

   # I would like to assign different weights to \nabla_{a}, \nabla_{a b}, \nabla_{a b c} etc. but no matter
   # what I do it appears that Cadabra assigns the same weight to all of these regardless of the number of subscripts.
   # It seems that the weight is assigned to the symbol \nabla alone. So I'm forced to use the following substitution trick.

       Q_{a b c d}::Weight(label=numR,value=2).
       Q_{a b c d e}::Weight(label=numR,value=3).
       Q_{a b c d e f}::Weight(label=numR,value=4).
       Q_{a b c d e f g}::Weight(label=numR,value=5).

       tmp := @(obj).

       distribute (tmp)

       substitute (tmp, $\nabla_{e f g}{R_{a b c d}} -> Q_{a b c d e f g}$)
       substitute (tmp, $\nabla_{e f}{R_{a b c d}} -> Q_{a b c d e f}$)
       substitute (tmp, $\nabla_{e}{R_{a b c d}} -> Q_{a b c d e}$)
       substitute (tmp, $R_{a b c d} -> Q_{a b c d}$)

       foo := @(tmp).
       bah = Ex("numR = " + str(n))
       keep_weight (foo, bah)

       substitute (foo, $Q_{a b c d e f g} -> \nabla_{e f g}{R_{a b c d}}$)
       substitute (foo, $Q_{a b c d e f} -> \nabla_{e f}{R_{a b c d}}$)
       substitute (foo, $Q_{a b c d e} -> \nabla_{e}{R_{a b c d}}$)
       substitute (foo, $Q_{a b c d} -> R_{a b c d}$)

       return foo

   def reformat (obj,scale):
       foo  = Ex(str(scale))
       bah := @(foo) @(obj).
       distribute     (bah)
       bah = product_sort (bah)
       rename_dummies (bah)
       canonicalise   (bah)
       substitute     (bah,$Dx^{b}->zzz^{b}$)
       factor_out     (bah,$x^{a?},zzz^{b?}$)
       substitute     (bah,$zzz^{b}->Dx^{b}$)
       ans := @(bah) / @(foo).
       return ans

   def rescale (obj,scale):
       foo  = Ex(str(scale))
       bah := @(foo) @(obj).
       distribute  (bah)
       substitute  (bah,$Dx^{b}->zzz^{b}$)
       factor_out  (bah,$x^{a?},zzz^{b?}$)
       substitute  (bah,$zzz^{b}->Dx^{b}$)
       return bah

   beg_stage_3 = time.time()

   Rterm22 = get_Rterm (term2,2)                           # cdb(Rterm22.101,Rterm22)
   Rterm23 = get_Rterm (term2,3)                           # cdb(Rterm23.101,Rterm23)
   Rterm24 = get_Rterm (term2,4)                           # cdb(Rterm24.101,Rterm24)
   Rterm25 = get_Rterm (term2,5)                           # cdb(Rterm25.101,Rterm25)

   Rterm32 = get_Rterm (term3,2)                           # cdb(Rterm32.101,Rterm32)  # zero
   Rterm33 = get_Rterm (term3,3)                           # cdb(Rterm33.101,Rterm33)
   Rterm34 = get_Rterm (term3,4)                           # cdb(Rterm34.101,Rterm34)
   Rterm35 = get_Rterm (term3,5)                           # cdb(Rterm35.101,Rterm35)

   Rterm42 = get_Rterm (term4,2)                           # cdb(Rterm42.101,Rterm42)  # zero
   Rterm43 = get_Rterm (term4,3)                           # cdb(Rterm43.101,Rterm43)  # zero
   Rterm44 = get_Rterm (term4,4)                           # cdb(Rterm44.101,Rterm44)
   Rterm45 = get_Rterm (term4,5)                           # cdb(Rterm45.101,Rterm45)

   Rterm52 = get_Rterm (term5,2)                           # cdb(Rterm52.101,Rterm52)  # zero
   Rterm53 = get_Rterm (term5,3)                           # cdb(Rterm53.101,Rterm53)  # zero
   Rterm54 = get_Rterm (term5,4)                           # cdb(Rterm54.101,Rterm54)  # zero
   Rterm55 = get_Rterm (term5,5)                           # cdb(Rterm55.101,Rterm55)

   Rterm22 = rescale ( reformat (Rterm22,   -3),    -3 )   # cdb(Rterm22.102,Rterm22)
   Rterm23 = rescale ( reformat (Rterm23,  -24),   -24 )   # cdb(Rterm23.102,Rterm23)
   Rterm24 = rescale ( reformat (Rterm24, -720),  -720 )   # cdb(Rterm24.102,Rterm24)
   Rterm25 = rescale ( reformat (Rterm25, -360),  -360 )   # cdb(Rterm25.102,Rterm25)

   Rterm33 = rescale ( reformat (Rterm33,  -12),   -12 )   # cdb(Rterm33.102,Rterm33)
   Rterm34 = rescale ( reformat (Rterm34, -720),  -720 )   # cdb(Rterm34.102,Rterm34)
   Rterm35 = rescale ( reformat (Rterm35,-1080), -1080 )   # cdb(Rterm35.102,Rterm35)

   Rterm44 = rescale ( reformat (Rterm44, -180),  -180 )   # cdb(Rterm44.102,Rterm44)
   Rterm45 = rescale ( reformat (Rterm45,-2160), -2160 )   # cdb(Rterm45.102,Rterm45)

   Rterm55 = rescale ( reformat (Rterm55, -360),  -360 )   # cdb(Rterm55.102,Rterm55)
\end{cadabra}

\clearpage

\begin{cadabra}
   # ----------------------------------------------------------------
   # bvp to terms linear in R

   tmp2 := -(1/3) @(Rterm22).

   bvp2 := x^{a}
        + s Dx^{a}
        + (s-s**2) @(tmp2).                                  # cdb(bvp.601,bvp2)

   cdblib.put ('bvp2',bvp2,'geodesic-bvp.json')
   cdblib.put ('bvp22',tmp2,'geodesic-bvp.json')

   y2 := Dx^{a} + @(tmp2).                                   # cdb(y2.600,y2)

   # ----------------------------------------------------------------
   # bvp to terms linear in dR

   tmp2 :=  -(1/3) @(Rterm22) - (1/24) @(Rterm23).
   tmp3 := -(1/12) @(Rterm33).

   bvp3 := x^{a}
        + s Dx^{a}
        + (s-s**2) @(tmp2)
        + (s-s**3) @(tmp3).                                  # cdb(bvp.602,bvp3)

   cdblib.put ('bvp3',bvp3,'geodesic-bvp.json')
   cdblib.put ('bvp32',tmp2,'geodesic-bvp.json')
   cdblib.put ('bvp33',tmp3,'geodesic-bvp.json')

   y3 := Dx^{a} + @(tmp2) + @(tmp3).                         # cdb(y3.600,y3)

   # ----------------------------------------------------------------
   # bvp to terms linear in d^2 R

   tmp2 :=   -(1/3) @(Rterm22) -  (1/24) @(Rterm23) - (1/720) @(Rterm24).
   tmp3 :=  -(1/12) @(Rterm33) - (1/720) @(Rterm34).
   tmp4 := -(1/180) @(Rterm44).

   bvp4 := x^{a}
        + s Dx^{a}
        + (s-s**2) @(tmp2)
        + (s-s**3) @(tmp3)
        + (s-s**4) @(tmp4).                                  # cdb(bvp.603,bvp4)

   cdblib.put ('bvp4',bvp4,'geodesic-bvp.json')
   cdblib.put ('bvp42',tmp2,'geodesic-bvp.json')
   cdblib.put ('bvp43',tmp3,'geodesic-bvp.json')
   cdblib.put ('bvp44',tmp4,'geodesic-bvp.json')

   y4 := Dx^{a} + @(tmp2) + @(tmp3) + @(tmp4).               # cdb(y4.600,y4)

   # ----------------------------------------------------------------
   # bvp to terms linear in d^3 R

   tmp2 := @(term2).
   tmp3 := @(term3).
   tmp4 := @(term4).
   tmp5 := @(term5).

   bvp5 := x^{a}
        + s Dx^{a}
        + (s-s**2) @(tmp2)
        + (s-s**3) @(tmp3)
        + (s-s**4) @(tmp4)
        + (s-s**5) @(tmp5).                                  # cdb(bvp.604,bvp5)

   cdblib.put ('bvp5',bvp5,'geodesic-bvp.json')
   cdblib.put ('bvp52',term2,'geodesic-bvp.json')
   cdblib.put ('bvp53',term3,'geodesic-bvp.json')
   cdblib.put ('bvp54',term4,'geodesic-bvp.json')
   cdblib.put ('bvp55',term5,'geodesic-bvp.json')

   y5 := Dx^{a} + @(tmp2) + @(tmp3) + @(tmp4) + @(tmp5).     # cdb(y5.600,y5)

   end_stage_3 = time.time()

   # cdbBeg (timing)
   print ("Stage 1:  {:7.1f} secs\\hfill\\break".format(end_stage_1-beg_stage_1))
   print ("Stage 2a: {:7.1f} secs\\hfill\\break".format(end_stage_2a-beg_stage_2a))
   print ("Stage 2b: {:7.1f} secs\\hfill\\break".format(end_stage_2b-beg_stage_2b))
   print ("Stage 3:  {:7.1f} secs\\hfill\\break".format(end_stage_3-beg_stage_3))
   # cdbEnd (timing)

\end{cadabra}

\clearpage

% -------------------------------------------------------------------------------------------------
\subsection*{Non-unit tangent vectors at $P$}

These are not unit vectors, their length is the geodesic distance from $P$ to $Q$

\begin{dgroup*}
   \begin{dmath*} \cdb*{y2.600} \end{dmath*}
   \begin{dmath*} \cdb*{y3.600} \end{dmath*}
   \begin{dmath*} \cdb*{y4.600} \end{dmath*}
   % \begin{dmath*} \cdb*{y5.600} \end{dmath*}
\end{dgroup*}

\clearpage

% =================================================================================================
\section*{Geodesic boundary value problem to terms linear in $R$}

\begin{dgroup*}
   \begin{dmath*} x^{a}(s) = \cdb{bvp.601} + \BigO{s^3,\eps^3} \end{dmath*}
\end{dgroup*}

\begin{align*}
   x^{a}(s) &= x^{a} + s Dx^{a}
                     + (s-s^2) x^{a}_2
                     + \BigO{s^3,\eps^3}
\end{align*}

\begin{dgroup*}
   \begin{dmath*} x^{a}_2 = \nx{2}^{a}_2 + \BigO{\eps^3} \end{dmath*}
   \begin{dmath*} -3 \nx{2}^{a}_2 = \cdb{Rterm22.102} \end{dmath*}
\end{dgroup*}

\clearpage

% =================================================================================================
\section*{Geodesic boundary value problem to terms linear in $\nabla R$}

\begin{dgroup*}
   \begin{dmath*} x^{a}(s) = \cdb{bvp.602} + \BigO{s^4,\eps^4} \end{dmath*}
\end{dgroup*}

\begin{align*}
   x^{a}(s) &= x^{a} + s Dx^{a}
                     + (s-s^2) x^{a}_2
                     + (s-s^3) x^{a}_3
                     + \BigO{s^4,\eps^4}
\end{align*}

\begin{dgroup*}
   \begin{dmath*} x^{a}_2 = \nx{2}^{a}_2 + \nx{3}^{a}_2 + \BigO{\eps^4} \end{dmath*}
   \begin{dmath*}   -3 \nx{2}^{a}_2 = \cdb{Rterm22.102} \end{dmath*}
   \begin{dmath*}  -24 \nx{3}^{a}_2 = \cdb{Rterm23.102} \end{dmath*}
\end{dgroup*}

\begin{dgroup*}
   \begin{dmath*} x^{a}_3 = \nx{3}^{a}_3 + \BigO{\eps^4} \end{dmath*}
   \begin{dmath*}   -12 \nx{3}^{a}_3 = \cdb{Rterm33.102} \end{dmath*}
\end{dgroup*}

% =================================================================================================
\section*{Geodesic boundary value problem to terms linear in $\nabla^2 R$}

\begin{dgroup*}
   \begin{dmath*} x^{a}(s) = \cdb{bvp.603} + \BigO{s^5,\eps^5} \end{dmath*}
\end{dgroup*}

\begin{align*}
   x^{a}(s) &= x^{a} + s Dx^{a}
                     + (s-s^2) x^{a}_2
                     + (s-s^3) x^{a}_3
                     + (s-s^4) x^{a}_4
                     + \BigO{s^5,\eps^5}
\end{align*}

\begin{dgroup*}
   \begin{dmath*} x^{a}_2 = \nx{2}^{a}_2 + \nx{3}^{a}_2 + \nx{4}^{a}_2 + \BigO{\eps^5} \end{dmath*}
   \begin{dmath*}   -3 \nx{2}^{a}_2 = \cdb{Rterm22.102} \end{dmath*}
   \begin{dmath*}  -24 \nx{3}^{a}_2 = \cdb{Rterm23.102} \end{dmath*}
   \begin{dmath*} -720 \nx{4}^{a}_2 = \cdb{Rterm24.102} \end{dmath*}
\end{dgroup*}

\begin{dgroup*}
   \begin{dmath*} x^{a}_3 = \nx{3}^{a}_3 + \nx{4}^{a}_3 + \BigO{\eps^5} \end{dmath*}
   \begin{dmath*}   -12 \nx{3}^{a}_3 = \cdb{Rterm33.102} \end{dmath*}
   \begin{dmath*}  -720 \nx{4}^{a}_3 = \cdb{Rterm34.102} \end{dmath*}
\end{dgroup*}

\begin{dgroup*}
   \begin{dmath*} x^{a}_4 = \nx{4}^{a}_4 + \BigO{\eps^5} \end{dmath*}
   \begin{dmath*}  -180 \nx{4}^{a}_4 = \cdb{Rterm44.102} \end{dmath*}
\end{dgroup*}

\clearpage

% =================================================================================================
\section*{Geodesic boundary value problem to terms linear in $\nabla^3 R$}

The geodesic that connects the points with RNC coordinates $x^a$ and $x^a+Dx^a$ is described, for $0\le s\le 1$, by
%
% \begin{dgroup*}
%    \begin{dmath*} x^{a}(s) = \cdb{bvp.604} + \BigO{s^6,\eps^6} \end{dmath*} % too big for pdfLaTeX
% \end{dgroup*}
%
\begin{align*}
   x^{a}(s) &= x^{a} + s Dx^{a}
                     + (s-s^2) x^{a}_2
                     + (s-s^3) x^{a}_3
                     + (s-s^4) x^{a}_4
                     + (s-s^5) x^{a}_5
                     + \BigO{s^6,\eps^6}
\end{align*}

\begin{dgroup*}
   \begin{dmath*} x^{a}_2 = \nx{2}^{a}_2 + \nx{3}^{a}_2 + \nx{4}^{a}_2 + \nx{5}^{a}_2 + \BigO{\eps^6} \end{dmath*}
   \begin{dmath*}   -3 \nx{2}^{a}_2 = \cdb{Rterm22.102} \end{dmath*}
   \begin{dmath*}  -24 \nx{3}^{a}_2 = \cdb{Rterm23.102} \end{dmath*}
   \begin{dmath*} -720 \nx{4}^{a}_2 = \cdb{Rterm24.102} \end{dmath*}
   \begin{dmath*} -360 \nx{5}^{a}_2 = \cdb{Rterm25.102} \end{dmath*}
\end{dgroup*}

\clearpage

\begin{dgroup*}
   \begin{dmath*} x^{a}_3 = \nx{3}^{a}_3 + \nx{4}^{a}_3 + \nx{5}^{a}_3 + \BigO{\eps^6} \end{dmath*}
   \begin{dmath*}   -12 \nx{3}^{a}_3 = \cdb{Rterm33.102} \end{dmath*}
   \begin{dmath*}  -720 \nx{4}^{a}_3 = \cdb{Rterm34.102} \end{dmath*}
   \begin{dmath*} -1080 \nx{5}^{a}_3 = \cdb{Rterm35.102} \end{dmath*}
\end{dgroup*}

\begin{dgroup*}
   \begin{dmath*} x^{a}_4 = \nx{4}^{a}_4 + \nx{5}^{a}_4 + \BigO{\eps^6} \end{dmath*}
   \begin{dmath*}  -180 \nx{4}^{a}_4 = \cdb{Rterm44.102} \end{dmath*}
   \begin{dmath*} -2160 \nx{5}^{a}_4 = \cdb{Rterm45.102} \end{dmath*}
\end{dgroup*}

\begin{dgroup*}
   \begin{dmath*} x^{a}_5 = \nx{5}^{a}_5 + \BigO{\eps^6} \end{dmath*}
   \begin{dmath*} -360 \nx{5}^{a}_5 = \cdb{Rterm55.102} \end{dmath*}
\end{dgroup*}

\clearpage

% =================================================================================================
% export selected objects, these will later be imported into a library
% these are the objects that will appear in the paper

\begin{cadabra}
   tmp2 := 8 @(Rterm22) + @(Rterm23).
   tmp3 := @(Rterm33).

   factor_out     (tmp2,$Dx^{a?}$) # cdb(tmp2.001,tmp2)
   rename_dummies (tmp2)
   factor_out     (tmp2,$Dx^{a?}$) # cdb(tmp2.002,tmp2)

   bvp4 := x^{a}
        + \lam Dx^{a}
        - (1/24) (\lam-\lam**2) @(tmp2)
        - (1/12) (\lam-\lam**3) @(tmp3).     # cdb(bvp4,bvp4)

   cdblib.create ('geodesic-bvp.export')

   # 4th order bvp
   cdblib.put ('bvp4',bvp4,'geodesic-bvp.export')

   # 6th order bvp terms, scaled
   cdblib.put ('bvp622',Rterm22,'geodesic-bvp.export')
   cdblib.put ('bvp623',Rterm23,'geodesic-bvp.export')
   cdblib.put ('bvp624',Rterm24,'geodesic-bvp.export')
   cdblib.put ('bvp625',Rterm25,'geodesic-bvp.export')

   cdblib.put ('bvp633',Rterm33,'geodesic-bvp.export')
   cdblib.put ('bvp634',Rterm34,'geodesic-bvp.export')
   cdblib.put ('bvp635',Rterm35,'geodesic-bvp.export')

   cdblib.put ('bvp644',Rterm44,'geodesic-bvp.export')
   cdblib.put ('bvp645',Rterm45,'geodesic-bvp.export')

   cdblib.put ('bvp655',Rterm55,'geodesic-bvp.export')

   checkpoint.append (bvp4)

   checkpoint.append (Rterm22)
   checkpoint.append (Rterm23)
   checkpoint.append (Rterm24)
   checkpoint.append (Rterm25)

   checkpoint.append (Rterm33)
   checkpoint.append (Rterm34)
   checkpoint.append (Rterm35)

   checkpoint.append (Rterm44)
   checkpoint.append (Rterm45)

   checkpoint.append (Rterm55)
\end{cadabra}

\clearpage

% =================================================================================================
\section*{Timing}

\cdb{timing}

% =================================================================================================
% export checkpoints in json format

\bgroup
\CdbSetup{action=hide}
\begin{cadabra}
   for i in range( len(checkpoint) ):
      cdblib.put ('check{:03d}'.format(i),checkpoint[i],checkpoint_file)
\end{cadabra}
\egroup

\end{document}


\begin{dgroup*}
   \begin{dmath*} x^{a}(s) = \cdb{bvp.601} + \BigO{s^3,\eps^3} \end{dmath*}
\end{dgroup*}

\begin{align*}
   x^{a}(s) &= x^{a} + s Dx^{a}
                     + (s-s^2) x^{a}_2
                     + \BigO{s^3,\eps^3}
\end{align*}

\begin{dgroup*}
   \begin{dmath*} x^{a}_2 = \nx{2}^{a}_2 + \BigO{\eps^3} \end{dmath*}
   \begin{dmath*} -3 \nx{2}^{a}_2 = \cdb{Rterm22.102} \end{dmath*}
\end{dgroup*}

\clearpage

% =================================================================================================
\section*{Geodesic boundary value problem to terms linear in $\nabla R$}

\begin{dgroup*}
   \begin{dmath*} x^{a}(s) = \cdb{bvp.602} + \BigO{s^4,\eps^4} \end{dmath*}
\end{dgroup*}

\begin{align*}
   x^{a}(s) &= x^{a} + s Dx^{a}
                     + (s-s^2) x^{a}_2
                     + (s-s^3) x^{a}_3
                     + \BigO{s^4,\eps^4}
\end{align*}

\begin{dgroup*}
   \begin{dmath*} x^{a}_2 = \nx{2}^{a}_2 + \nx{3}^{a}_2 + \BigO{\eps^4} \end{dmath*}
   \begin{dmath*}   -3 \nx{2}^{a}_2 = \cdb{Rterm22.102} \end{dmath*}
   \begin{dmath*}  -24 \nx{3}^{a}_2 = \cdb{Rterm23.102} \end{dmath*}
\end{dgroup*}

\begin{dgroup*}
   \begin{dmath*} x^{a}_3 = \nx{3}^{a}_3 + \BigO{\eps^4} \end{dmath*}
   \begin{dmath*}   -12 \nx{3}^{a}_3 = \cdb{Rterm33.102} \end{dmath*}
\end{dgroup*}

% =================================================================================================
\section*{Geodesic boundary value problem to terms linear in $\nabla^2 R$}

\begin{dgroup*}
   \begin{dmath*} x^{a}(s) = \cdb{bvp.603} + \BigO{s^5,\eps^5} \end{dmath*}
\end{dgroup*}

\begin{align*}
   x^{a}(s) &= x^{a} + s Dx^{a}
                     + (s-s^2) x^{a}_2
                     + (s-s^3) x^{a}_3
                     + (s-s^4) x^{a}_4
                     + \BigO{s^5,\eps^5}
\end{align*}

\begin{dgroup*}
   \begin{dmath*} x^{a}_2 = \nx{2}^{a}_2 + \nx{3}^{a}_2 + \nx{4}^{a}_2 + \BigO{\eps^5} \end{dmath*}
   \begin{dmath*}   -3 \nx{2}^{a}_2 = \cdb{Rterm22.102} \end{dmath*}
   \begin{dmath*}  -24 \nx{3}^{a}_2 = \cdb{Rterm23.102} \end{dmath*}
   \begin{dmath*} -720 \nx{4}^{a}_2 = \cdb{Rterm24.102} \end{dmath*}
\end{dgroup*}

\begin{dgroup*}
   \begin{dmath*} x^{a}_3 = \nx{3}^{a}_3 + \nx{4}^{a}_3 + \BigO{\eps^5} \end{dmath*}
   \begin{dmath*}   -12 \nx{3}^{a}_3 = \cdb{Rterm33.102} \end{dmath*}
   \begin{dmath*}  -720 \nx{4}^{a}_3 = \cdb{Rterm34.102} \end{dmath*}
\end{dgroup*}

\begin{dgroup*}
   \begin{dmath*} x^{a}_4 = \nx{4}^{a}_4 + \BigO{\eps^5} \end{dmath*}
   \begin{dmath*}  -180 \nx{4}^{a}_4 = \cdb{Rterm44.102} \end{dmath*}
\end{dgroup*}

\clearpage

% =================================================================================================
\section*{Geodesic boundary value problem to terms linear in $\nabla^3 R$}

The geodesic that connects the points with RNC coordinates $x^a$ and $x^a+Dx^a$ is described, for $0\le s\le 1$, by
%
% \begin{dgroup*}
%    \begin{dmath*} x^{a}(s) = \cdb{bvp.604} + \BigO{s^6,\eps^6} \end{dmath*} % too big for pdfLaTeX
% \end{dgroup*}
%
\begin{align*}
   x^{a}(s) &= x^{a} + s Dx^{a}
                     + (s-s^2) x^{a}_2
                     + (s-s^3) x^{a}_3
                     + (s-s^4) x^{a}_4
                     + (s-s^5) x^{a}_5
                     + \BigO{s^6,\eps^6}
\end{align*}

\begin{dgroup*}
   \begin{dmath*} x^{a}_2 = \nx{2}^{a}_2 + \nx{3}^{a}_2 + \nx{4}^{a}_2 + \nx{5}^{a}_2 + \BigO{\eps^6} \end{dmath*}
   \begin{dmath*}   -3 \nx{2}^{a}_2 = \cdb{Rterm22.102} \end{dmath*}
   \begin{dmath*}  -24 \nx{3}^{a}_2 = \cdb{Rterm23.102} \end{dmath*}
   \begin{dmath*} -720 \nx{4}^{a}_2 = \cdb{Rterm24.102} \end{dmath*}
   \begin{dmath*} -360 \nx{5}^{a}_2 = \cdb{Rterm25.102} \end{dmath*}
\end{dgroup*}

\clearpage

\begin{dgroup*}
   \begin{dmath*} x^{a}_3 = \nx{3}^{a}_3 + \nx{4}^{a}_3 + \nx{5}^{a}_3 + \BigO{\eps^6} \end{dmath*}
   \begin{dmath*}   -12 \nx{3}^{a}_3 = \cdb{Rterm33.102} \end{dmath*}
   \begin{dmath*}  -720 \nx{4}^{a}_3 = \cdb{Rterm34.102} \end{dmath*}
   \begin{dmath*} -1080 \nx{5}^{a}_3 = \cdb{Rterm35.102} \end{dmath*}
\end{dgroup*}

\begin{dgroup*}
   \begin{dmath*} x^{a}_4 = \nx{4}^{a}_4 + \nx{5}^{a}_4 + \BigO{\eps^6} \end{dmath*}
   \begin{dmath*}  -180 \nx{4}^{a}_4 = \cdb{Rterm44.102} \end{dmath*}
   \begin{dmath*} -2160 \nx{5}^{a}_4 = \cdb{Rterm45.102} \end{dmath*}
\end{dgroup*}

\begin{dgroup*}
   \begin{dmath*} x^{a}_5 = \nx{5}^{a}_5 + \BigO{\eps^6} \end{dmath*}
   \begin{dmath*} -360 \nx{5}^{a}_5 = \cdb{Rterm55.102} \end{dmath*}
\end{dgroup*}

\clearpage

% =================================================================================================
\section*{Geodesic arc-length}
\def\Date{19 Jan 2024}
% \def\FileID{file:}

\documentclass[12pt]{cdblatex}

\begin{document}

% =================================================================================================
% create checkpoint file

\bgroup
\CdbSetup{action=hide}
\begin{cadabra}
   import cdblib
   checkpoint_file = 'tests/semantic/output/geodesic-lsq.json'
   cdblib.create (checkpoint_file)
   checkpoint = []
\end{cadabra}
\egroup

% =================================================================================================
\section*{Geodesic arc-length}

Give a pair of points $P$ and $Q$ the geodesic arc-length can be computed using
\begin{align}
   L_{PQ} = \int_P^Q\>\left(g_{ab}(x)\frac{dx^a}{ds}\frac{dx^b}{ds}\right)^{1/2}\>ds
\end{align}
Since the path is a geodesic the integrand is constant and thus
\begin{align}
   L^2_{PQ} = \left.g_{ab}(x)\frac{dx^a}{ds}\frac{dx^b}{ds}\right\vert_{P}
\end{align}
where $s$ is a re-scaled parameter (0 at $P$ and 1 at $Q$). The point $P$ has
RNC coordinates $x^{a}$ while the point $Q$ has coordinates $x^{a} + Dx^{a}$.

The vector $dx^a/ds$ at $P$ is given by the solution of the geodesic boundary value
problem. This was found in the previous code ({\tts geodesic-bvp}). That is
\begin{align}
   \left.\frac{dx^b}{ds}\right\vert_{P} = y^{a}
\end{align}
and thus
\begin{align}
   \label{eq:lsq}
   L^2_{PQ} = g_{ab}(x) y^{a} y^{b}
\end{align}

It is possible to directly evaluate the right hand side of (\ref{eq:lsq}) using the results from
the {\tts geodesic-bvp} and {\tts metric} codes. The result would need to be truncated (to an
order consistent with the results form those codes). But doing so would be computationaly
expensive as at least half of the terms will be thrown away. A better approach is compute just
the terms that will survive the truncation. This is done by expanding $g_{ab}(x)$ and $y^{a}$ as
a truncated series in the curvatures and its derivatives.

The $g_{ab}(x)$ and $y^{a}$ are written in a (truncated) formal power series in the curvature and
its derivatives
\begin{align}
   y^{a} &= \ny{0}^a + \ny{2}^a + \ny{3}^a + \ny{4}^a + \ny{5}^a + \BigO{\eps^6}\\
   g_{a b}(x) &=   \ngab{0}_{a b}
                 + \ngab{2}_{a b}
                 + \ngab{3}_{a b}
                 + \ngab{4}_{a b}
                 + \ngab{5}_{a b}
                 + \BigO{\eps^6}
\end{align}
Note that this use of $\ny{i}$ differs from that used in {\tts geodesic-bvp}. Here the
index above $y^{a}$ denotes a particular term in the curvature expansion while in
{\tts geodesic-bvp} the index denoted the iteration number (in the fixed point scheme
used to solve the BVP for $y^{a}$).

% \clarepage

% =================================================================================================
\section*{Stage 1}
The formal curvature expansions are substituted into equation (\ref{eq:lsq}), expanded and
truncated to retain terms of order $\BigO{\eps^5}$ or less. The expansion to 4th order terms is
as follows.

\begin{dgroup*}
   \begin{dmath*} L^2_{PQ} = \cdb{lsq4.002} \end{dmath*}
\end{dgroup*}

\documentclass[12pt]{cdblatex}
\usepackage{fancyhdr}
\usepackage{footer}

\begin{document}

% =================================================================================================
% create checkpoint file

\bgroup
\CdbSetup{action=hide}
\begin{cadabra}
   import cdblib
   checkpoint_file = 'tests/semantic/output/rnc2rnc.json'
   cdblib.create (checkpoint_file)
   checkpoint = []
\end{cadabra}
\egroup

% =================================================================================================
\section*{From one RNC to another}

Consider an RNC frame with RNC cooridnates $x^{a}$.

In the {\tts geodesic-bvp} code the two point boundary value problem (for the geodesic connecting
two points) was solved. There is a bonus in that calculation -- it can be trivaly adapted to the
case of transforming form one RNC into another.

The starting point is the basic equation for the geodesic connecting $P$ (with coordinaties
$x^{a}$) to Q (with coordinates $x^{a} + Dx^{a}$)
\begin{equation*}
   x^a(s) = x^a_i + s y^a - \sum_{k=2}^\infty\>\frac{1}{k!}\>\Gamma^{a}{}_{\ubk}y^{.\ubk} s^k
\end{equation*}
The affine parameter $s$ varies form 0 (at $P$) to 1 (at $Q$).

A new RNC frame, with origin at $P$, can be defined via the $y^{a}$ with the coordinates of $Q$ in
the new RNC frame defined by $y^{a}$ (since $s=1$ at $Q$). Recall that in an RNC all geodesics
through the origin are described by $y^{a}(s) = s y^{a}$. Thus the transformation from $x^a$ to
$y^a$ satisfies
\begin{equation*}
   x^a = x^a_i + y^a - \sum_{k=2}^\infty\>\frac{1}{k!}\>\Gamma^{a}{}_{\ubk}y^{.\ubk}
\end{equation*}
where the $\Gamma^{a}{}_{\ubk}$ are the generalised connections of the $x^a$ frame evaluated at
$x^a=0$. This equation can be inverted to express $y^a$ in terms of $x^a$. This computation is
done in the {\tts geodesic-bvp} code -- we only quote the results here (at the end).

The new $y^a$ frame has origin at $P$. Its coordinate axes are aligned with those (at $P$) of the
origianl RNC frame. To see this just note that $\partial x^a/\partial y^b = \delta^a_b$ at $P$.
Thus the metric at $P$ in the new frame has values $g_{ab}(x)$ (i.e., exactly those of the
original RNC frame). Note that this means that the coordinate axes of the new frame are not
ncessarily orthogonal.

The calculations in this code are trivial. It uses the $y^{a}$ found in {\tts geodesic-bvp} as
the basis of the transformation from $x^{a}$ to $y^{a}$. Most of the code involves reformatting
the $y^{a}$.

\clearpage

\begin{cadabra}
   {a,b,c,d,e,f,g,h,i,j,k,l,m,n,o,p,q,r,s,t,u,v,w#}::Indices(position=independent).

   \nabla{#}::Derivative.

   g_{a b}::Metric.
   g^{a b}::InverseMetric.

   R_{a b c d}::RiemannTensor.
   R^{a}_{b c d}::RiemannTensor.

   # Dx{#}::LaTeXForm{"{\Dx}"}.  # LCB: currently causes a bug, it kills ::KeepWeight for Dx

   import cdblib

   Y5 = cdblib.get ('y5','geodesic-bvp.json')

   Y50 = cdblib.get ('y50','geodesic-bvp.json')
   Y52 = cdblib.get ('y52','geodesic-bvp.json')
   Y53 = cdblib.get ('y53','geodesic-bvp.json')
   Y54 = cdblib.get ('y54','geodesic-bvp.json')
   Y55 = cdblib.get ('y55','geodesic-bvp.json')

   # this copies y5* from geodesic-bvp.json to rnc2rnc.json

   cdblib.create ('rnc2rnc.json')

   cdblib.put ('rnc2rnc',Y5,'rnc2rnc.json')

   cdblib.put ('rnc2rnc0',Y50,'rnc2rnc.json')
   cdblib.put ('rnc2rnc2',Y52,'rnc2rnc.json')
   cdblib.put ('rnc2rnc3',Y53,'rnc2rnc.json')
   cdblib.put ('rnc2rnc4',Y54,'rnc2rnc.json')
   cdblib.put ('rnc2rnc5',Y55,'rnc2rnc.json')

\end{cadabra}

% =================================================================================================
% the remaining code is just for pretty printing

\clearpage

\begin{cadabra}
   # note: keeping numbering as is (out of order) to ensure R appears before \nabla R etc.
   def product_sort (obj):
       substitute (obj,$ x^{a}                            -> A001^{a}               $)
       substitute (obj,$ Dx^{a}                           -> A002^{a}               $)
       substitute (obj,$ g^{a b}                          -> A003^{a b}             $)
       substitute (obj,$ \nabla_{e f g h}{R_{a b c d}}    -> A008_{a b c d e f g h} $)
       substitute (obj,$ \nabla_{e f g}{R_{a b c d}}      -> A007_{a b c d e f g}   $)
       substitute (obj,$ \nabla_{e f}{R_{a b c d}}        -> A006_{a b c d e f}     $)
       substitute (obj,$ \nabla_{e}{R_{a b c d}}          -> A005_{a b c d e}       $)
       substitute (obj,$ R_{a b c d}                      -> A004_{a b c d}         $)
       sort_product   (obj)
       rename_dummies (obj)
       substitute (obj,$ A001^{a}                  -> x^{a}                         $)
       substitute (obj,$ A002^{a}                  -> Dx^{a}                        $)
       substitute (obj,$ A003^{a b}                -> g^{a b}                       $)
       substitute (obj,$ A004_{a b c d}            -> R_{a b c d}                   $)
       substitute (obj,$ A005_{a b c d e}          -> \nabla_{e}{R_{a b c d}}       $)
       substitute (obj,$ A006_{a b c d e f}        -> \nabla_{e f}{R_{a b c d}}     $)
       substitute (obj,$ A007_{a b c d e f g}      -> \nabla_{e f g}{R_{a b c d}}   $)
       substitute (obj,$ A008_{a b c d e f g h}    -> \nabla_{e f g h}{R_{a b c d}} $)

       return obj

   def get_xDxterm (obj,n,m):

       x^{a}::Weight(label=numx,value=1).
       Dx^{a}::Weight(label=numDx,value=1).

       tmp := @(obj).
       distribute  (tmp)

       foo = Ex("numx = " + str(n))
       bah = Ex("numDx = " + str(m))
       keep_weight (tmp, foo)
       keep_weight (tmp, bah)

       return tmp

   def reformat (obj,scale):
       foo  = Ex(str(scale))
       bah := @(foo) @(obj).
       distribute     (bah)
       bah = product_sort (bah)
       rename_dummies (bah)
       canonicalise   (bah)
       substitute     (bah,$Dx^{b}->zzz^{b}$)
       factor_out     (bah,$x^{a?},zzz^{b?}$)
       substitute     (bah,$zzz^{b}->Dx^{b}$)
       ans := @(bah) / @(foo).
       return ans

   def rescale (obj,scale):
       foo  = Ex(str(scale))
       bah := @(foo) @(obj).
       distribute  (bah)
       substitute  (bah,$Dx^{b}->zzz^{b}$)
       factor_out  (bah,$x^{a?},zzz^{b?}$)
       substitute  (bah,$zzz^{b}->Dx^{b}$)
       return bah

   term0 := @(Y50).  # cdb (term0.101,term0)
   term2 := @(Y52).  # cdb (term2.101,term2)
   term3 := @(Y53).  # cdb (term3.101,term3)
   term4 := @(Y54).  # cdb (term4.101,term4)
   term5 := @(Y55).  # cdb (term5.101,term5)

   term0 = reformat (term0,1)  # cdb (term0.102,term0)
   term2 = reformat (term2,1)  # cdb (term2.102,term2)
   term3 = reformat (term3,1)  # cdb (term3.102,term3)
   term4 = reformat (term4,1)  # cdb (term4.102,term4)
   term5 = reformat (term5,1)  # cdb (term5.102,term5)

   xDxterm12 = get_xDxterm (term2,1,2)   # cdb(xDxterm12.101,xDxterm12)

   xDxterm13 = get_xDxterm (term3,1,3)   # cdb(xDxterm13.101,xDxterm13)
   xDxterm22 = get_xDxterm (term3,2,2)   # cdb(xDxterm22.101,xDxterm22)

   xDxterm14 = get_xDxterm (term4,1,4)   # cdb(xDxterm14.101,xDxterm14)
   xDxterm23 = get_xDxterm (term4,2,3)   # cdb(xDxterm23.101,xDxterm23)
   xDxterm32 = get_xDxterm (term4,3,2)   # cdb(xDxterm32.101,xDxterm32)

   xDxterm15 = get_xDxterm (term5,1,5)   # cdb(xDxterm15.101,xDxterm15)
   xDxterm24 = get_xDxterm (term5,2,4)   # cdb(xDxterm24.101,xDxterm24)
   xDxterm33 = get_xDxterm (term5,3,3)   # cdb(xDxterm33.101,xDxterm33)
   xDxterm42 = get_xDxterm (term5,4,2)   # cdb(xDxterm42.101,xDxterm42)


   xDxterm12 = rescale ( reformat (xDxterm12,    3),     3 )   # cdb(xDxterm12.102,xDxterm12)

   xDxterm13 = rescale ( reformat (xDxterm13,   12),   -12 )   # cdb(xDxterm13.102,xDxterm13)
   xDxterm22 = rescale ( reformat (xDxterm22,   24),   -24 )   # cdb(xDxterm22.102,xDxterm22)

   xDxterm14 = rescale ( reformat (xDxterm14,  180),  -180 )   # cdb(xDxterm14.102,xDxterm14)
   xDxterm23 = rescale ( reformat (xDxterm23,  720),  -720 )   # cdb(xDxterm23.102,xDxterm23)
   xDxterm32 = rescale ( reformat (xDxterm32,  720),  -720 )   # cdb(xDxterm32.102,xDxterm32)

   xDxterm15 = rescale ( reformat (xDxterm15,  360),  -360 )   # cdb(xDxterm15.102,xDxterm15)
   xDxterm24 = rescale ( reformat (xDxterm24, 2160), -2160 )   # cdb(xDxterm24.102,xDxterm24)
   xDxterm33 = rescale ( reformat (xDxterm33, 1080), -1080 )   # cdb(xDxterm33.102,xDxterm33)
   xDxterm42 = rescale ( reformat (xDxterm42,  360),  -360 )   # cdb(xDxterm42.102,xDxterm42)

   checkpoint.append (term0)
   checkpoint.append (term2)
   checkpoint.append (term3)
   checkpoint.append (term4)
   checkpoint.append (term5)

\end{cadabra}

\clearpage

% =================================================================================================
\section*{Tranformation between two RNC frames}

\begin{align*}
     y^{a} = \ny{0}^{a} + \ny{2}^{a} + \ny{3}^{a} + \ny{4}^{a} + \ny{5}^{a} + \BigO{\eps^6}
\end{align*}

\begin{dgroup*}
   \begin{dmath*} \ny{0}^{a} = \cdb{term0.102} \end{dmath*}
   \begin{dmath*} \ny{2}^{a} = \cdb{term2.102} \end{dmath*}
   \begin{dmath*} \ny{3}^{a} = \cdb{term3.102} \end{dmath*}
   \begin{dmath*} \ny{4}^{a} = \cdb{term4.102} \end{dmath*}
   \begin{dmath*} \ny{5}^{a} = \cdb{term5.102} \end{dmath*}
\end{dgroup*}

\clearpage

% =================================================================================================
\section*{Tranformation between two RNC frames}

Same as before but with an improved format (maybe) for the expressions.

\begin{align}
   y^{a} = \ny{0}^{a} + \ny{2}^{a} + \ny{3}^{a} + \ny{4}^{a} + \ny{5}^{a} + \BigO{\eps^6}
\end{align}

\begin{dgroup}
   \begin{dmath} \ny{0}^{a} = Dx^{a} \end{dmath}
\end{dgroup}

\begin{dgroup}
   \begin{dmath} \ny{2}^{a} = \ny{2}^{a}_1 \end{dmath}
   \begin{dmath}   3 \ny{2}^{a}_1 = \cdb{xDxterm12.102} \end{dmath}
\end{dgroup}

\begin{dgroup}
   \begin{dmath} \ny{3}^{a} = \ny{3}^{a}_1 + \ny{3}^{a}_2 \end{dmath}
   \begin{dmath} -12 \ny{3}^{a}_1 = \cdb{xDxterm13.102} \end{dmath}
   \begin{dmath} -24 \ny{3}^{a}_2 = \cdb{xDxterm22.102} \end{dmath}
\end{dgroup}

\begin{dgroup}
   \begin{dmath} \ny{4}^{a} = \ny{4}^{a}_1 + \ny{4}^{a}_2 + \ny{4}^{a}_3 \end{dmath}
   \begin{dmath} -180 \ny{4}^{a}_1 = \cdb{xDxterm14.102} \end{dmath}
   \begin{dmath} -720 \ny{4}^{a}_2 = \cdb{xDxterm23.102} \end{dmath}
   \begin{dmath} -720 \ny{4}^{a}_3 = \cdb{xDxterm32.102} \end{dmath}
\end{dgroup}

\begin{dgroup}
   \begin{dmath} \ny{5}^{a} = \ny{5}^{a}_1 + \ny{5}^{a}_2 + \ny{5}^{a}_3 + \ny{5}^{a}_4 \end{dmath}
   \begin{dmath}  -360 \ny{5}^{a}_1 = \cdb{xDxterm15.102} \end{dmath}
   \begin{dmath} -2160 \ny{5}^{a}_2 = \cdb{xDxterm24.102} \end{dmath}
   \begin{dmath} -1080 \ny{5}^{a}_3 = \cdb{xDxterm33.102} \end{dmath}
   \begin{dmath}  -360 \ny{5}^{a}_4 = \cdb{xDxterm42.102} \end{dmath}
\end{dgroup}

% =================================================================================================
% export checkpoints in json format

\bgroup
\CdbSetup{action=hide}
\begin{cadabra}
   for i in range( len(checkpoint) ):
      cdblib.put ('check{:03d}'.format(i),checkpoint[i],checkpoint_file)
\end{cadabra}
\egroup

\end{document}

\documentclass[12pt]{cdblatex}

\begin{document}

\section*{\jobname}

\CdbSetup{action=hide}

\begin{cadabra}
   import shared

   import cdblib

   term00A = cdblib.get ('check000','expected/metric.json')
   term01A = cdblib.get ('check001','expected/metric.json')
   term02A = cdblib.get ('check002','expected/metric.json')
   term03A = cdblib.get ('check003','expected/metric.json')
   term04A = cdblib.get ('check004','expected/metric.json')
   term05A = cdblib.get ('check005','expected/metric.json')
   term06A = cdblib.get ('check005','expected/metric.json')
   term07A = cdblib.get ('check005','expected/metric.json')

   term00B = cdblib.get ('check000','output/metric.json')
   term01B = cdblib.get ('check001','output/metric.json')
   term02B = cdblib.get ('check002','output/metric.json')
   term03B = cdblib.get ('check003','output/metric.json')
   term04B = cdblib.get ('check004','output/metric.json')
   term05B = cdblib.get ('check005','output/metric.json')
   term06B = cdblib.get ('check005','output/metric.json')
   term07B = cdblib.get ('check005','output/metric.json')

   # bug: can't push this function into shared.py
   #      no synatx error, but cadabra doesn't cancel equal terms
   # see ~/cadabra/bugs/bug02

   def check (objA,objB):
       tmp := @(objA) - @(objB).
       distribute         (tmp)
       tmp = shared.standard_indices (tmp)
       tmp = shared.product_sort (tmp)
       rename_dummies     (tmp)
       canonicalise       (tmp)

       return tmp

   diff000 = shared.check (term00A,term00B)   # cdb (diff000,diff000)
   diff001 = shared.check (term01A,term01B)   # cdb (diff001,diff001)
   diff002 = shared.check (term02A,term02B)   # cdb (diff002,diff002)
   diff003 = shared.check (term03A,term03B)   # cdb (diff003,diff003)
   diff004 = shared.check (term04A,term04B)   # cdb (diff004,diff004)
   diff005 = shared.check (term05A,term05B)   # cdb (diff005,diff005)
   diff006 = shared.check (term06A,term06B)   # cdb (diff006,diff006)
   diff007 = shared.check (term07A,term07B)   # cdb (diff007,diff007)

\end{cadabra}

\begin{dgroup*}
   \Dmath*{ \cdb*{diff000} }
   \Dmath*{ \cdb*{diff001} }
   \Dmath*{ \cdb*{diff002} }
   \Dmath*{ \cdb*{diff003} }
   \Dmath*{ \cdb*{diff004} }
   \Dmath*{ \cdb*{diff005} }
   \Dmath*{ \cdb*{diff006} }
   \Dmath*{ \cdb*{diff007} }
\end{dgroup*}

\end{document}


From {\tts geodesic-bvp} (actually from {\tts rnc2rnc which reformatted the results nicely}) we have
% \begin{align*}
%      y^{a} = \ny{0}^{a} + \ny{2}^{a} + \ny{3}^{a} + \ny{4}^{a} + \BigO{\eps^5}
% \end{align*}

\begin{dgroup*}
   \begin{dmath*} \ny{0}^{a} = \cdb{term0.102} \end{dmath*}
   \begin{dmath*} \ny{2}^{a} = \cdb{term2.102} \end{dmath*}
   \begin{dmath*} \ny{3}^{a} = \cdb{term3.102} \end{dmath*}
   \begin{dmath*} \ny{4}^{a} = \cdb{term4.102} \end{dmath*}
\end{dgroup*}

and from {\tts metric} we have

% \begin{align*}
%      g_{a b}(x) =
%      \ngab{0}_{a b}
%    + \ngab{2}_{a b}
%    + \ngab{3}_{a b}
%    + \ngab{4}_{a b}
%    + \BigO{\eps^6}
% \end{align*}
\begin{dgroup*}
   \begin{dmath*}     \ngab{0}_{a b} = \cdb{scaled0.601} \end{dmath*}
   \begin{dmath*}   3 \ngab{2}_{a b} = \cdb{scaled2.601} \end{dmath*}
   \begin{dmath*}   6 \ngab{3}_{a b} = \cdb{scaled3.601} \end{dmath*}
   \begin{dmath*} 180 \ngab{4}_{a b} = \cdb{scaled4.601} \end{dmath*}
\end{dgroup*}

% =================================================================================================
\section*{Stage 2}
The results from the {\tts geodesic-bvp} and {\tts metric} codes are read to provide
values for the $\ny{n}^{a}$ and $\ngab{m}_{ab}$. These are substituted into the result from
Stage 1, et volia, the final answer. To 4th-order terms the result is given by

\Dmath*{ L^2_{PQ} = \cdb{lsq5.301} + \BigO{\eps^5}}

\clearpage

% =================================================================================================
\section*{Shared properties}

\begin{cadabra}
   {a,b,c,d,e,f,g,h,i,j,k,l,m,n,o,p,q,r,s,t,u,v,w#}::Indices(position=independent).

   D{#}::Derivative.
   \nabla{#}::Derivative.
   \partial{#}::PartialDerivative.

   g_{a b}::Metric.
   g^{a b}::InverseMetric.
   g_{a}^{b}::KroneckerDelta.
   g^{a}_{b}::KroneckerDelta.
   \delta^{a}_{b}::KroneckerDelta.
   \delta_{a}^{b}::KroneckerDelta.

   R_{a b c d}::RiemannTensor.
   R^{a}_{b c d}::RiemannTensor.
   R_{a b c}^{d}::RiemannTensor.

   \Gamma^{a}_{b c}::TableauSymmetry(shape={2}, indices={1,2}).
   \Gamma^{a}_{b c d}::TableauSymmetry(shape={3}, indices={1,2,3}).
   \Gamma^{a}_{b c d e}::TableauSymmetry(shape={4}, indices={1,2,3,4}).
   \Gamma^{a}_{b c d e f}::TableauSymmetry(shape={5}, indices={1,2,3,4,5}).

   x^{a}::Depends(D{#}).

   g_{a b}::Depends(\partial{#}).
   R_{a b c d}::Depends(\partial{#}).
   R^{a}_{b c d}::Depends(\partial{#}).
   \Gamma^{a}_{b c}::Depends(\partial{#}).

   R_{a b c d}::Depends(\nabla{#}).
   R^{a}_{b c d}::Depends(\nabla{#}).

   g0{#}::LaTeXForm("\ngab{0}").
   g2{#}::LaTeXForm("\ngab{2}").
   g3{#}::LaTeXForm("\ngab{3}").
   g4{#}::LaTeXForm("\ngab{4}").
   g5{#}::LaTeXForm("\ngab{5}").

   y0{#}::LaTeXForm("\ny{0}").
   y2{#}::LaTeXForm("\ny{2}").
   y3{#}::LaTeXForm("\ny{3}").
   y4{#}::LaTeXForm("\ny{4}").
   y5{#}::LaTeXForm("\ny{5}").

\end{cadabra}

\clearpage

% =================================================================================================
\section*{Stage 1: The formal expansion}

\begin{cadabra}
   g0_{a b}::Symmetric.
   g2_{a b}::Symmetric.
   g3_{a b}::Symmetric.
   g4_{a b}::Symmetric.
   g5_{a b}::Symmetric.

   g0_{a b}::Weight(label=num,value=0).
   g2_{a b}::Weight(label=num,value=2).
   g3_{a b}::Weight(label=num,value=3).
   g4_{a b}::Weight(label=num,value=4).
   g5_{a b}::Weight(label=num,value=5).

   y0^{a}::Weight(label=num,value=0).
   y2^{a}::Weight(label=num,value=2).
   y3^{a}::Weight(label=num,value=3).
   y4^{a}::Weight(label=num,value=4).
   y5^{a}::Weight(label=num,value=5).

   # note: keeping numbering as is (out of order) to ensure R appears before \nabla R etc.
   def product_sort (obj):
       substitute (obj,$ A^{a}                            -> A001^{a}               $)
       substitute (obj,$ x^{a}                            -> A002^{a}               $)
       substitute (obj,$ Dx^{a}                           -> A003^{a}               $)
       substitute (obj,$ g_{a b}                          -> A004_{a b}             $)
       substitute (obj,$ g^{a b}                          -> A005^{a b}             $)
       substitute (obj,$ \nabla_{e f g h}{R_{a b c d}}    -> A010_{a b c d e f g h} $)
       substitute (obj,$ \nabla_{e f g}{R_{a b c d}}      -> A009_{a b c d e f g}   $)
       substitute (obj,$ \nabla_{e f}{R_{a b c d}}        -> A008_{a b c d e f}     $)
       substitute (obj,$ \nabla_{e}{R_{a b c d}}          -> A007_{a b c d e}       $)
       substitute (obj,$ R_{a b c d}                      -> A006_{a b c d}         $)
       sort_product   (obj)
       rename_dummies (obj)
       substitute (obj,$ A001^{a}                  -> A^{a}                         $)
       substitute (obj,$ A002^{a}                  -> x^{a}                         $)
       substitute (obj,$ A003^{a}                  -> Dx^{a}                        $)
       substitute (obj,$ A004_{a b}                -> g_{a b}                       $)
       substitute (obj,$ A005^{a b}                -> g^{a b}                       $)
       substitute (obj,$ A006_{a b c d}            -> R_{a b c d}                   $)
       substitute (obj,$ A007_{a b c d e}          -> \nabla_{e}{R_{a b c d}}       $)
       substitute (obj,$ A008_{a b c d e f}        -> \nabla_{e f}{R_{a b c d}}     $)
       substitute (obj,$ A009_{a b c d e f g}      -> \nabla_{e f g}{R_{a b c d}}   $)
       substitute (obj,$ A010_{a b c d e f g h}    -> \nabla_{e f g h}{R_{a b c d}} $)

       return obj

   def truncate (obj,n):
       ans = Ex(0)

       for i in range (0,n+1):
          foo := @(obj).
          bah = Ex("num = " + str(i))
          keep_weight (foo, bah)
          ans = ans + foo

       return ans

   # expansions wrt the curvature

   defgab := g_{a b} -> g0_{a b} + g2_{a b} + g3_{a b} + g4_{a b} + g5_{a b}.
   defy   := y^{a}   -> y0^{a} + y2^{a} + y3^{a} + y4^{a} + y5^{a}.

   lsq    := g_{a b} y^{a} y^{b}.

   substitute (lsq,defgab)
   substitute (lsq,defy)
   distribute (lsq)

   def tidy (obj):
       foo := @(obj).
       sort_product    (foo)
       rename_dummies  (foo)
       canonicalise    (foo)
       return foo

   lsq0 = tidy ( truncate (lsq,0) )  # cdb (lsq0.002,lsq0)
   lsq2 = tidy ( truncate (lsq,2) )  # cdb (lsq2.002,lsq2)
   lsq3 = tidy ( truncate (lsq,3) )  # cdb (lsq3.002,lsq3)
   lsq4 = tidy ( truncate (lsq,4) )  # cdb (lsq4.002,lsq4)
   lsq5 = tidy ( truncate (lsq,5) )  # cdb (lsq5.002,lsq5)

   d20 := @(lsq2) - @(lsq0).         # cdb (d20.001,d20)   # check, should contain only O(2) terms
   d32 := @(lsq3) - @(lsq2).         # cdb (d32.001,d32)   # check, should contain only O(3) terms
   d43 := @(lsq4) - @(lsq3).         # cdb (d43.001,d43)   # check, should contain only O(4) terms
   d54 := @(lsq5) - @(lsq4).         # cdb (d54.001,d54)   # check, should contain only O(5) terms

   d5 := @(lsq5) - @(lsq).           # cdb (d5.001,d5)
   d5  = tidy (d5)                   # cdb (d5.002,d5)  # all higher order terms, should see no O(5) terms

\end{cadabra}

\clearpage

\begin{dgroup*}
   \begin{dmath*} \cdb*{lsq0.002} \end{dmath*}
   \begin{dmath*} \cdb*{lsq2.002} \end{dmath*}
   \begin{dmath*} \cdb*{lsq3.002} \end{dmath*}
   \begin{dmath*} \cdb*{lsq4.002} \end{dmath*}
   \begin{dmath*} \cdb*{lsq5.002} \end{dmath*}
\end{dgroup*}

\begin{dgroup*}
   \begin{dmath*} \cdb*{d20.001} \end{dmath*}
   \begin{dmath*} \cdb*{d32.001} \end{dmath*}
   \begin{dmath*} \cdb*{d43.001} \end{dmath*}
   \begin{dmath*} \cdb*{d54.001} \end{dmath*}
   \begin{dmath*} \cdb*{d5.002} \end{dmath*}
\end{dgroup*}

\clearpage

% =================================================================================================
\section*{Stage 2: Substution of $\ny{n}^{a}$ and $\ngab{m}_{ab}$}

\begin{cadabra}
   import cdblib

   g0ab = cdblib.get('g_ab_0','metric.json')
   g2ab = cdblib.get('g_ab_2','metric.json')
   g3ab = cdblib.get('g_ab_3','metric.json')
   g4ab = cdblib.get('g_ab_4','metric.json')
   g5ab = cdblib.get('g_ab_5','metric.json')

   defg0ab := g0_{a b} -> @(g0ab).
   defg2ab := g2_{a b} -> @(g2ab).
   defg3ab := g3_{a b} -> @(g3ab).
   defg4ab := g4_{a b} -> @(g4ab).
   defg5ab := g5_{a b} -> @(g5ab).

   y0a = cdblib.get('y50','geodesic-bvp.json')
   y2a = cdblib.get('y52','geodesic-bvp.json')
   y3a = cdblib.get('y53','geodesic-bvp.json')
   y4a = cdblib.get('y54','geodesic-bvp.json')
   y5a = cdblib.get('y55','geodesic-bvp.json')

   defy0a := y0^{a} -> @(y0a).
   defy2a := y2^{a} -> @(y2a).
   defy3a := y3^{a} -> @(y3a).
   defy4a := y4^{a} -> @(y4a).
   defy5a := y5^{a} -> @(y5a).

   def substitute_gab_ya (obj):

      foo := @(obj).

      substitute (foo,defg0ab)
      substitute (foo,defg2ab)
      substitute (foo,defg3ab)
      substitute (foo,defg4ab)
      substitute (foo,defg5ab)

      substitute (foo,defy0a)
      substitute (foo,defy2a)
      substitute (foo,defy3a)
      substitute (foo,defy4a)
      substitute (foo,defy5a)

      distribute     (foo)
      sort_product   (foo)
      rename_dummies (foo)
      canonicalise   (foo)

      substitute     (foo,$g_{a b} g^{c b} -> \delta^{c}_{a}$)
      eliminate_kronecker (foo)
      foo = product_sort  (foo)
      rename_dummies      (foo)
      canonicalise        (foo)

      return foo

   def get_Rterm (obj,n):

   # I would like to assign different weights to \nabla_{a}, \nabla_{a b}, \nabla_{a b c} etc. but no matter
   # what I do it appears that Cadabra assigns the same weight to all of these regardless of the number of subscripts.
   # It seems that the weight is assigned to the symbol \nabla alone. So I'm forced to use the following substitution trick.

       Q_{a b c d}::Weight(label=numR,value=2).
       Q_{a b c d e}::Weight(label=numR,value=3).
       Q_{a b c d e f}::Weight(label=numR,value=4).
       Q_{a b c d e f g}::Weight(label=numR,value=5).

       tmp := @(obj).

       distribute (tmp)

       substitute (tmp, $\nabla_{e f g}{R_{a b c d}} -> Q_{a b c d e f g}$)
       substitute (tmp, $\nabla_{e f}{R_{a b c d}} -> Q_{a b c d e f}$)
       substitute (tmp, $\nabla_{e}{R_{a b c d}} -> Q_{a b c d e}$)
       substitute (tmp, $R_{a b c d} -> Q_{a b c d}$)

       foo := @(tmp).
       bah = Ex("numR = " + str(n))
       keep_weight (foo, bah)

       substitute (foo, $Q_{a b c d e f g} -> \nabla_{e f g}{R_{a b c d}}$)
       substitute (foo, $Q_{a b c d e f} -> \nabla_{e f}{R_{a b c d}}$)
       substitute (foo, $Q_{a b c d e} -> \nabla_{e}{R_{a b c d}}$)
       substitute (foo, $Q_{a b c d} -> R_{a b c d}$)

       return foo

   lsq2 = substitute_gab_ya (lsq2)  # cdb (lsq2.101,lsq2)
   lsq3 = substitute_gab_ya (lsq3)  # cdb (lsq3.101,lsq3)
   lsq4 = substitute_gab_ya (lsq4)  # cdb (lsq4.101,lsq4)
   lsq5 = substitute_gab_ya (lsq5)  # cdb (lsq5.101,lsq5)

   lsq50 = get_Rterm (lsq5,0)
   lsq52 = get_Rterm (lsq5,2)
   lsq53 = get_Rterm (lsq5,3)
   lsq54 = get_Rterm (lsq5,4)
   lsq55 = get_Rterm (lsq5,5)

   cdblib.create ('geodesic-lsq.json')

   cdblib.put ('lsq2',lsq2,'geodesic-lsq.json')
   cdblib.put ('lsq3',lsq3,'geodesic-lsq.json')
   cdblib.put ('lsq4',lsq4,'geodesic-lsq.json')
   cdblib.put ('lsq5',lsq5,'geodesic-lsq.json')

   cdblib.put ('lsq50',lsq50,'geodesic-lsq.json')
   cdblib.put ('lsq52',lsq52,'geodesic-lsq.json')
   cdblib.put ('lsq53',lsq53,'geodesic-lsq.json')
   cdblib.put ('lsq54',lsq54,'geodesic-lsq.json')
   cdblib.put ('lsq55',lsq55,'geodesic-lsq.json')

\end{cadabra}

\clearpage
\begin{dgroup*}
   \begin{dmath*} \cdb*{lsq2.101} \end{dmath*}
   \begin{dmath*} \cdb*{lsq3.101} \end{dmath*}
   \begin{dmath*} \cdb*{lsq4.101} \end{dmath*}
   \begin{dmath*} \cdb*{lsq5.101} \end{dmath*}
\end{dgroup*}

% =================================================================================================
% the remaining code is just for pretty printing

\clearpage

% =================================================================================================
\section*{Stage 3: Reformatting}

\begin{cadabra}
   def reformat (obj,scale):
      foo  = Ex(str(scale))
      bah := @(foo) @(obj).
      distribute     (bah)
      bah = product_sort (bah)
      rename_dummies (bah)
      canonicalise   (bah)
      substitute     (bah,$Dx^{b}->zzz^{b}$)
      factor_out     (bah,$x^{a?},zzz^{b?}$)
      substitute     (bah,$zzz^{b}->Dx^{b}$)
      ans := @(bah) / @(foo).
      return ans

   def rescale (obj,scale):
      foo  = Ex(str(scale))
      bah := @(foo) @(obj).
      distribute  (bah)
      substitute  (bah,$Dx^{b}->zzz^{b}$)
      factor_out  (bah,$x^{a?},zzz^{b?}$)
      substitute  (bah,$zzz^{b}->Dx^{b}$)
      return bah

   Rterm0 := @(lsq50).
   Rterm2 := @(lsq52).
   Rterm3 := @(lsq53).
   Rterm4 := @(lsq54).
   Rterm5 := @(lsq55).

   Rterm0 = reformat (Rterm0,   1)    # cdb(Rterm0.301,Rterm0) # LCB: returns Dx before g, not what I want
   Rterm2 = reformat (Rterm2,   3)    # cdb(Rterm2.301,Rterm2)
   Rterm3 = reformat (Rterm3,  12)    # cdb(Rterm3.301,Rterm3)
   Rterm4 = reformat (Rterm4, 360)    # cdb(Rterm4.301,Rterm4)
   Rterm5 = reformat (Rterm5,1080)    # cdb(Rterm5.301,Rterm5)

   Rterm0 := g_{a b} Dx^{a} Dx^{b}.   # LCB: fixes the order of terms, g before Dx,

   lsq3 := @(Rterm0) + @(Rterm2).                                      # cdb (lsq4.301,lsq3)
   lsq4 := @(Rterm0) + @(Rterm2) + @(Rterm3).                          # cdb (lsq4.301,lsq4)
   lsq5 := @(Rterm0) + @(Rterm2) + @(Rterm3) + @(Rterm4).              # cdb (lsq5.301,lsq5)
   lsq6 := @(Rterm0) + @(Rterm2) + @(Rterm3) + @(Rterm4) + @(Rterm5).  # cdb (lsq5.301,lsq6)

   lsq  := @(lsq6).                   # cdb (lsq.301,lsq)

   scaled0 = rescale (Rterm0,    1)   # cdb (scaled0.301,scaled0) # LCB: returns Dx before g, not what I want
   scaled2 = rescale (Rterm2,    3)   # cdb (scaled2.301,scaled2)
   scaled3 = rescale (Rterm3,   12)   # cdb (scaled3.301,scaled3)
   scaled4 = rescale (Rterm4,  360)   # cdb (scaled4.301,scaled4)
   scaled5 = rescale (Rterm5, 1080)   # cdb (scaled5.301,scaled5)

   scaled0 := g_{a b} Dx^{a} Dx^{b}.  # cdb (scaled0.301,scaled0) # LCB: fixes the order of terms, g before Dx, good

\end{cadabra}

\clearpage

% =================================================================================================
\section*{Geodesic arc-length}

\begin{dgroup*}[spread=5pt]
   % LCB: which of these is correct?
   % \begin{dmath*} \left(\Delta s\right)^2 = \cdb{lsq.301} + \BigO{\eps^6,Dx^6} \end{dmath*}
   % \begin{dmath*} \left(\Delta s\right)^2 = \cdb{lsq.301} + \BigO{\eps^6,Dx^7} \end{dmath*}
   \begin{dmath*} \left(\Delta s\right)^2 = \cdb{lsq.301} + \BigO{\eps^6} \end{dmath*}
\end{dgroup*}

\clearpage

% =================================================================================================
\section*{Geodesic arc-length curvature expansion}

\begin{align*}
   % LCB: which of these is correct?
   % \left(\Delta s\right)^2 = \nD{0} + \nD{2} + \nD{3} + \nD{4} + \nD{5} + \BigO{\eps^6,Dx^6}
   % \left(\Delta s\right)^2 = \nD{0} + \nD{2} + \nD{3} + \nD{4} + \nD{5} + \BigO{\eps^6,Dx^7}
   \left(\Delta s\right)^2 = \nD{0} + \nD{2} + \nD{3} + \nD{4} + \nD{5} + \BigO{\eps^6}
\end{align*}

\begin{dgroup*}[spread=5pt]
   \begin{dmath*}      \nD{0} = \cdb{scaled0.301} \end{dmath*}
   \begin{dmath*}    3 \nD{2} = \cdb{scaled2.301} \end{dmath*}
   \begin{dmath*}   12 \nD{3} = \cdb{scaled3.301} \end{dmath*}
   \begin{dmath*}  360 \nD{4} = \cdb{scaled4.301} \end{dmath*}
   \begin{dmath*} 1080 \nD{5} = \cdb{scaled5.301} \end{dmath*}
\end{dgroup*}

\clearpage

% =================================================================================================
% export selected objects, these will later be imported into a library
% these are the objects that will appear in the paper

\begin{cadabra}
   cdblib.create ('geodesic-lsq.export')

   # 3rd to 6th order lsq
   cdblib.put ('lsq3',lsq3,'geodesic-lsq.export')
   cdblib.put ('lsq4',lsq4,'geodesic-lsq.export')
   cdblib.put ('lsq5',lsq5,'geodesic-lsq.export')
   cdblib.put ('lsq6',lsq6,'geodesic-lsq.export')

   # 6th order lsq terms, scaled
   cdblib.put ('lsq60',scaled0,'geodesic-lsq.export')
   cdblib.put ('lsq62',scaled2,'geodesic-lsq.export')
   cdblib.put ('lsq63',scaled3,'geodesic-lsq.export')
   cdblib.put ('lsq64',scaled4,'geodesic-lsq.export')
   cdblib.put ('lsq65',scaled5,'geodesic-lsq.export')

   checkpoint.append (lsq4)

   checkpoint.append (scaled0)
   checkpoint.append (scaled2)
   checkpoint.append (scaled3)
   checkpoint.append (scaled4)
   checkpoint.append (scaled5)
\end{cadabra}

% =================================================================================================
% export checkpoints in json format

\bgroup
\CdbSetup{action=hide}
\begin{cadabra}
   for i in range( len(checkpoint) ):
      cdblib.put ('check{:03d}'.format(i),checkpoint[i],checkpoint_file)
\end{cadabra}
\egroup

\end{document}


\begin{dgroup*}[spread=5pt]
   % LCB: which of these is correct?
   % \begin{dmath*} \left(\Delta s\right)^2 = \cdb{lsq.301} + \BigO{\eps^6,Dx^6} \end{dmath*}
   % \begin{dmath*} \left(\Delta s\right)^2 = \cdb{lsq.301} + \BigO{\eps^6,Dx^7} \end{dmath*}
   \begin{dmath*} \left(\Delta s\right)^2 = \cdb{lsq.301} + \BigO{\eps^6} \end{dmath*}
\end{dgroup*}

\clearpage

% =================================================================================================
\section*{Geodesic arc-length curvature expansion}

\begin{align*}
   % LCB: which of these is correct?
   % \left(\Delta s\right)^2 = \nD{0} + \nD{2} + \nD{3} + \nD{4} + \nD{5} + \BigO{\eps^6,Dx^6}
   % \left(\Delta s\right)^2 = \nD{0} + \nD{2} + \nD{3} + \nD{4} + \nD{5} + \BigO{\eps^6,Dx^7}
   \left(\Delta s\right)^2 = \nD{0} + \nD{2} + \nD{3} + \nD{4} + \nD{5} + \BigO{\eps^6}
\end{align*}

\begin{dgroup*}[spread=5pt]
   \begin{dmath*}      \nD{0} = \cdb{scaled0.301} \end{dmath*}
   \begin{dmath*}    3 \nD{2} = \cdb{scaled2.301} \end{dmath*}
   \begin{dmath*}   12 \nD{3} = \cdb{scaled3.301} \end{dmath*}
   \begin{dmath*}  360 \nD{4} = \cdb{scaled4.301} \end{dmath*}
   \begin{dmath*} 1080 \nD{5} = \cdb{scaled5.301} \end{dmath*}
\end{dgroup*}

\clearpage

% =================================================================================================
\section*{Tranformation between two RNC frames}
\documentclass[12pt]{cdblatex}
\usepackage{fancyhdr}
\usepackage{footer}

\begin{document}

% =================================================================================================
% create checkpoint file

\bgroup
\CdbSetup{action=hide}
\begin{cadabra}
   import cdblib
   checkpoint_file = 'tests/semantic/output/rnc2rnc.json'
   cdblib.create (checkpoint_file)
   checkpoint = []
\end{cadabra}
\egroup

% =================================================================================================
\section*{From one RNC to another}

Consider an RNC frame with RNC cooridnates $x^{a}$.

In the {\tts geodesic-bvp} code the two point boundary value problem (for the geodesic connecting
two points) was solved. There is a bonus in that calculation -- it can be trivaly adapted to the
case of transforming form one RNC into another.

The starting point is the basic equation for the geodesic connecting $P$ (with coordinaties
$x^{a}$) to Q (with coordinates $x^{a} + Dx^{a}$)
\begin{equation*}
   x^a(s) = x^a_i + s y^a - \sum_{k=2}^\infty\>\frac{1}{k!}\>\Gamma^{a}{}_{\ubk}y^{.\ubk} s^k
\end{equation*}
The affine parameter $s$ varies form 0 (at $P$) to 1 (at $Q$).

A new RNC frame, with origin at $P$, can be defined via the $y^{a}$ with the coordinates of $Q$ in
the new RNC frame defined by $y^{a}$ (since $s=1$ at $Q$). Recall that in an RNC all geodesics
through the origin are described by $y^{a}(s) = s y^{a}$. Thus the transformation from $x^a$ to
$y^a$ satisfies
\begin{equation*}
   x^a = x^a_i + y^a - \sum_{k=2}^\infty\>\frac{1}{k!}\>\Gamma^{a}{}_{\ubk}y^{.\ubk}
\end{equation*}
where the $\Gamma^{a}{}_{\ubk}$ are the generalised connections of the $x^a$ frame evaluated at
$x^a=0$. This equation can be inverted to express $y^a$ in terms of $x^a$. This computation is
done in the {\tts geodesic-bvp} code -- we only quote the results here (at the end).

The new $y^a$ frame has origin at $P$. Its coordinate axes are aligned with those (at $P$) of the
origianl RNC frame. To see this just note that $\partial x^a/\partial y^b = \delta^a_b$ at $P$.
Thus the metric at $P$ in the new frame has values $g_{ab}(x)$ (i.e., exactly those of the
original RNC frame). Note that this means that the coordinate axes of the new frame are not
ncessarily orthogonal.

The calculations in this code are trivial. It uses the $y^{a}$ found in {\tts geodesic-bvp} as
the basis of the transformation from $x^{a}$ to $y^{a}$. Most of the code involves reformatting
the $y^{a}$.

\clearpage

\begin{cadabra}
   {a,b,c,d,e,f,g,h,i,j,k,l,m,n,o,p,q,r,s,t,u,v,w#}::Indices(position=independent).

   \nabla{#}::Derivative.

   g_{a b}::Metric.
   g^{a b}::InverseMetric.

   R_{a b c d}::RiemannTensor.
   R^{a}_{b c d}::RiemannTensor.

   # Dx{#}::LaTeXForm{"{\Dx}"}.  # LCB: currently causes a bug, it kills ::KeepWeight for Dx

   import cdblib

   Y5 = cdblib.get ('y5','geodesic-bvp.json')

   Y50 = cdblib.get ('y50','geodesic-bvp.json')
   Y52 = cdblib.get ('y52','geodesic-bvp.json')
   Y53 = cdblib.get ('y53','geodesic-bvp.json')
   Y54 = cdblib.get ('y54','geodesic-bvp.json')
   Y55 = cdblib.get ('y55','geodesic-bvp.json')

   # this copies y5* from geodesic-bvp.json to rnc2rnc.json

   cdblib.create ('rnc2rnc.json')

   cdblib.put ('rnc2rnc',Y5,'rnc2rnc.json')

   cdblib.put ('rnc2rnc0',Y50,'rnc2rnc.json')
   cdblib.put ('rnc2rnc2',Y52,'rnc2rnc.json')
   cdblib.put ('rnc2rnc3',Y53,'rnc2rnc.json')
   cdblib.put ('rnc2rnc4',Y54,'rnc2rnc.json')
   cdblib.put ('rnc2rnc5',Y55,'rnc2rnc.json')

\end{cadabra}

% =================================================================================================
% the remaining code is just for pretty printing

\clearpage

\begin{cadabra}
   # note: keeping numbering as is (out of order) to ensure R appears before \nabla R etc.
   def product_sort (obj):
       substitute (obj,$ x^{a}                            -> A001^{a}               $)
       substitute (obj,$ Dx^{a}                           -> A002^{a}               $)
       substitute (obj,$ g^{a b}                          -> A003^{a b}             $)
       substitute (obj,$ \nabla_{e f g h}{R_{a b c d}}    -> A008_{a b c d e f g h} $)
       substitute (obj,$ \nabla_{e f g}{R_{a b c d}}      -> A007_{a b c d e f g}   $)
       substitute (obj,$ \nabla_{e f}{R_{a b c d}}        -> A006_{a b c d e f}     $)
       substitute (obj,$ \nabla_{e}{R_{a b c d}}          -> A005_{a b c d e}       $)
       substitute (obj,$ R_{a b c d}                      -> A004_{a b c d}         $)
       sort_product   (obj)
       rename_dummies (obj)
       substitute (obj,$ A001^{a}                  -> x^{a}                         $)
       substitute (obj,$ A002^{a}                  -> Dx^{a}                        $)
       substitute (obj,$ A003^{a b}                -> g^{a b}                       $)
       substitute (obj,$ A004_{a b c d}            -> R_{a b c d}                   $)
       substitute (obj,$ A005_{a b c d e}          -> \nabla_{e}{R_{a b c d}}       $)
       substitute (obj,$ A006_{a b c d e f}        -> \nabla_{e f}{R_{a b c d}}     $)
       substitute (obj,$ A007_{a b c d e f g}      -> \nabla_{e f g}{R_{a b c d}}   $)
       substitute (obj,$ A008_{a b c d e f g h}    -> \nabla_{e f g h}{R_{a b c d}} $)

       return obj

   def get_xDxterm (obj,n,m):

       x^{a}::Weight(label=numx,value=1).
       Dx^{a}::Weight(label=numDx,value=1).

       tmp := @(obj).
       distribute  (tmp)

       foo = Ex("numx = " + str(n))
       bah = Ex("numDx = " + str(m))
       keep_weight (tmp, foo)
       keep_weight (tmp, bah)

       return tmp

   def reformat (obj,scale):
       foo  = Ex(str(scale))
       bah := @(foo) @(obj).
       distribute     (bah)
       bah = product_sort (bah)
       rename_dummies (bah)
       canonicalise   (bah)
       substitute     (bah,$Dx^{b}->zzz^{b}$)
       factor_out     (bah,$x^{a?},zzz^{b?}$)
       substitute     (bah,$zzz^{b}->Dx^{b}$)
       ans := @(bah) / @(foo).
       return ans

   def rescale (obj,scale):
       foo  = Ex(str(scale))
       bah := @(foo) @(obj).
       distribute  (bah)
       substitute  (bah,$Dx^{b}->zzz^{b}$)
       factor_out  (bah,$x^{a?},zzz^{b?}$)
       substitute  (bah,$zzz^{b}->Dx^{b}$)
       return bah

   term0 := @(Y50).  # cdb (term0.101,term0)
   term2 := @(Y52).  # cdb (term2.101,term2)
   term3 := @(Y53).  # cdb (term3.101,term3)
   term4 := @(Y54).  # cdb (term4.101,term4)
   term5 := @(Y55).  # cdb (term5.101,term5)

   term0 = reformat (term0,1)  # cdb (term0.102,term0)
   term2 = reformat (term2,1)  # cdb (term2.102,term2)
   term3 = reformat (term3,1)  # cdb (term3.102,term3)
   term4 = reformat (term4,1)  # cdb (term4.102,term4)
   term5 = reformat (term5,1)  # cdb (term5.102,term5)

   xDxterm12 = get_xDxterm (term2,1,2)   # cdb(xDxterm12.101,xDxterm12)

   xDxterm13 = get_xDxterm (term3,1,3)   # cdb(xDxterm13.101,xDxterm13)
   xDxterm22 = get_xDxterm (term3,2,2)   # cdb(xDxterm22.101,xDxterm22)

   xDxterm14 = get_xDxterm (term4,1,4)   # cdb(xDxterm14.101,xDxterm14)
   xDxterm23 = get_xDxterm (term4,2,3)   # cdb(xDxterm23.101,xDxterm23)
   xDxterm32 = get_xDxterm (term4,3,2)   # cdb(xDxterm32.101,xDxterm32)

   xDxterm15 = get_xDxterm (term5,1,5)   # cdb(xDxterm15.101,xDxterm15)
   xDxterm24 = get_xDxterm (term5,2,4)   # cdb(xDxterm24.101,xDxterm24)
   xDxterm33 = get_xDxterm (term5,3,3)   # cdb(xDxterm33.101,xDxterm33)
   xDxterm42 = get_xDxterm (term5,4,2)   # cdb(xDxterm42.101,xDxterm42)


   xDxterm12 = rescale ( reformat (xDxterm12,    3),     3 )   # cdb(xDxterm12.102,xDxterm12)

   xDxterm13 = rescale ( reformat (xDxterm13,   12),   -12 )   # cdb(xDxterm13.102,xDxterm13)
   xDxterm22 = rescale ( reformat (xDxterm22,   24),   -24 )   # cdb(xDxterm22.102,xDxterm22)

   xDxterm14 = rescale ( reformat (xDxterm14,  180),  -180 )   # cdb(xDxterm14.102,xDxterm14)
   xDxterm23 = rescale ( reformat (xDxterm23,  720),  -720 )   # cdb(xDxterm23.102,xDxterm23)
   xDxterm32 = rescale ( reformat (xDxterm32,  720),  -720 )   # cdb(xDxterm32.102,xDxterm32)

   xDxterm15 = rescale ( reformat (xDxterm15,  360),  -360 )   # cdb(xDxterm15.102,xDxterm15)
   xDxterm24 = rescale ( reformat (xDxterm24, 2160), -2160 )   # cdb(xDxterm24.102,xDxterm24)
   xDxterm33 = rescale ( reformat (xDxterm33, 1080), -1080 )   # cdb(xDxterm33.102,xDxterm33)
   xDxterm42 = rescale ( reformat (xDxterm42,  360),  -360 )   # cdb(xDxterm42.102,xDxterm42)

   checkpoint.append (term0)
   checkpoint.append (term2)
   checkpoint.append (term3)
   checkpoint.append (term4)
   checkpoint.append (term5)

\end{cadabra}

\clearpage

% =================================================================================================
\section*{Tranformation between two RNC frames}

\begin{align*}
     y^{a} = \ny{0}^{a} + \ny{2}^{a} + \ny{3}^{a} + \ny{4}^{a} + \ny{5}^{a} + \BigO{\eps^6}
\end{align*}

\begin{dgroup*}
   \begin{dmath*} \ny{0}^{a} = \cdb{term0.102} \end{dmath*}
   \begin{dmath*} \ny{2}^{a} = \cdb{term2.102} \end{dmath*}
   \begin{dmath*} \ny{3}^{a} = \cdb{term3.102} \end{dmath*}
   \begin{dmath*} \ny{4}^{a} = \cdb{term4.102} \end{dmath*}
   \begin{dmath*} \ny{5}^{a} = \cdb{term5.102} \end{dmath*}
\end{dgroup*}

\clearpage

% =================================================================================================
\section*{Tranformation between two RNC frames}

Same as before but with an improved format (maybe) for the expressions.

\begin{align}
   y^{a} = \ny{0}^{a} + \ny{2}^{a} + \ny{3}^{a} + \ny{4}^{a} + \ny{5}^{a} + \BigO{\eps^6}
\end{align}

\begin{dgroup}
   \begin{dmath} \ny{0}^{a} = Dx^{a} \end{dmath}
\end{dgroup}

\begin{dgroup}
   \begin{dmath} \ny{2}^{a} = \ny{2}^{a}_1 \end{dmath}
   \begin{dmath}   3 \ny{2}^{a}_1 = \cdb{xDxterm12.102} \end{dmath}
\end{dgroup}

\begin{dgroup}
   \begin{dmath} \ny{3}^{a} = \ny{3}^{a}_1 + \ny{3}^{a}_2 \end{dmath}
   \begin{dmath} -12 \ny{3}^{a}_1 = \cdb{xDxterm13.102} \end{dmath}
   \begin{dmath} -24 \ny{3}^{a}_2 = \cdb{xDxterm22.102} \end{dmath}
\end{dgroup}

\begin{dgroup}
   \begin{dmath} \ny{4}^{a} = \ny{4}^{a}_1 + \ny{4}^{a}_2 + \ny{4}^{a}_3 \end{dmath}
   \begin{dmath} -180 \ny{4}^{a}_1 = \cdb{xDxterm14.102} \end{dmath}
   \begin{dmath} -720 \ny{4}^{a}_2 = \cdb{xDxterm23.102} \end{dmath}
   \begin{dmath} -720 \ny{4}^{a}_3 = \cdb{xDxterm32.102} \end{dmath}
\end{dgroup}

\begin{dgroup}
   \begin{dmath} \ny{5}^{a} = \ny{5}^{a}_1 + \ny{5}^{a}_2 + \ny{5}^{a}_3 + \ny{5}^{a}_4 \end{dmath}
   \begin{dmath}  -360 \ny{5}^{a}_1 = \cdb{xDxterm15.102} \end{dmath}
   \begin{dmath} -2160 \ny{5}^{a}_2 = \cdb{xDxterm24.102} \end{dmath}
   \begin{dmath} -1080 \ny{5}^{a}_3 = \cdb{xDxterm33.102} \end{dmath}
   \begin{dmath}  -360 \ny{5}^{a}_4 = \cdb{xDxterm42.102} \end{dmath}
\end{dgroup}

% =================================================================================================
% export checkpoints in json format

\bgroup
\CdbSetup{action=hide}
\begin{cadabra}
   for i in range( len(checkpoint) ):
      cdblib.put ('check{:03d}'.format(i),checkpoint[i],checkpoint_file)
\end{cadabra}
\egroup

\end{document}


\begin{align*}
   y^{a} = \ny{0}^{a} + \ny{2}^{a} + \ny{3}^{a} + \ny{4}^{a} + \ny{5}^{a} + \BigO{\eps^6}
\end{align*}

\begin{dgroup*}
   \begin{dmath*} \ny{0}^{a} = Dx^{a} \end{dmath*}
\end{dgroup*}

\begin{dgroup*}
   \begin{dmath*} \ny{2}^{a} = \ny{2}^{a}_1 \end{dmath*}
   \begin{dmath*}   3 \ny{2}^{a}_1 = \cdb{xDxterm12.102} \end{dmath*}
\end{dgroup*}

\begin{dgroup*}
   \begin{dmath*} \ny{3}^{a} = \ny{3}^{a}_1 + \ny{3}^{a}_2 \end{dmath*}
   \begin{dmath*} -12 \ny{3}^{a}_1 = \cdb{xDxterm13.102} \end{dmath*}
   \begin{dmath*} -24 \ny{3}^{a}_2 = \cdb{xDxterm22.102} \end{dmath*}
\end{dgroup*}

\begin{dgroup*}
   \begin{dmath*} \ny{4}^{a} = \ny{4}^{a}_1 + \ny{4}^{a}_2 + \ny{4}^{a}_3 \end{dmath*}
   \begin{dmath*} -180 \ny{4}^{a}_1 = \cdb{xDxterm14.102} \end{dmath*}
   \begin{dmath*} -720 \ny{4}^{a}_2 = \cdb{xDxterm23.102} \end{dmath*}
   \begin{dmath*} -720 \ny{4}^{a}_3 = \cdb{xDxterm32.102} \end{dmath*}
\end{dgroup*}

\begin{dgroup*}
   \begin{dmath*} \ny{5}^{a} = \ny{5}^{a}_1 + \ny{5}^{a}_2 + \ny{5}^{a}_3 + \ny{5}^{a}_4 \end{dmath*}
   \begin{dmath*}  -360 \ny{5}^{a}_1 = \cdb{xDxterm15.102} \end{dmath*}
   \begin{dmath*} -2160 \ny{5}^{a}_2 = \cdb{xDxterm24.102} \end{dmath*}
   \begin{dmath*} -1080 \ny{5}^{a}_3 = \cdb{xDxterm33.102} \end{dmath*}
   \begin{dmath*}  -360 \ny{5}^{a}_4 = \cdb{xDxterm42.102} \end{dmath*}
\end{dgroup*}

\end{document}
