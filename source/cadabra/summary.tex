\documentclass[a4paper,12pt]{article}
\usepackage{cdblatex}
% \usepackage{amsmath}% part of cdblatex
% \usepackage{amssymb}% ditto
% \usepackage{breqn}%   ditto
\usepackage{hyperref}
\usepackage{geometry}
\usepackage{summary}
\usepackage{config}

\geometry{a4paper,landscape,margin=2cm}
% \geometry{a4paper,portrait,margin=2cm}
\hypersetup{colorlinks=true}% use false for journals and paper
\numberwithin{equation}{section}% requires amsmath

\begin{document}

% =================================================================================================
\section*{Notes}

The convention for the curvature used in these notes conforms to that of Misner-Thorne-Wheeler
(MTW, eq. 11.12) , namely
\begin{align*}
   V^{a}{}_{;bc} - V^{a}{}_{;cb} = - R^{a}{}_{dbc} V^{d}
\end{align*}

Also, note the following shorthand for mixed covariant derivatives
\begin{align*}
   \nabla_a\left(\nabla_b\right) &= \nabla_{ab}\\
   \nabla_a\left(\nabla_b\left(\nabla_c\right)\right) &= \nabla_{abc}\\
   \nabla_a\left(\nabla_b\left(\nabla_c\left(\nabla_d\right)\right)\right) &= \nabla_{abcd}
\end{align*}
and so on.

In terms of $\nabla$ the above MTW definition of $R^{a}{}_{bcd}$ can written as
\begin{align*}
   \left(\nabla_{cb}-\nabla_{bc}\right) V^{a} = - R^{a}{}_{dbc} V^{d}
\end{align*}

% -------------------------------------------------------------------------------------------------
\subsection*{Symmetrisation}
\documentclass[12pt]{cdblatex}
\usepackage{fancyhdr}
\usepackage{footer}

\begin{document}

\section*{\jobname}

\CdbSetup{action=hide}

\begin{cadabra}
   import shared

   import cdblib

   term00A = cdblib.get ('check000','expected/dGamma.json')
   term01A = cdblib.get ('check001','expected/dGamma.json')
   term02A = cdblib.get ('check002','expected/dGamma.json')
   term03A = cdblib.get ('check003','expected/dGamma.json')
   term04A = cdblib.get ('check004','expected/dGamma.json')

   term00B = cdblib.get ('check000','output/dGamma.json')
   term01B = cdblib.get ('check001','output/dGamma.json')
   term02B = cdblib.get ('check002','output/dGamma.json')
   term03B = cdblib.get ('check003','output/dGamma.json')
   term04B = cdblib.get ('check004','output/dGamma.json')

   diff000 = shared.check (term00A,term00B)   # cdb (diff000,diff000)
   diff001 = shared.check (term01A,term01B)   # cdb (diff001,diff001)
   diff002 = shared.check (term02A,term02B)   # cdb (diff002,diff002)
   diff003 = shared.check (term03A,term03B)   # cdb (diff003,diff003)
   diff004 = shared.check (term04A,term04B)   # cdb (diff004,diff004)

\end{cadabra}

\begin{dgroup*}
   \Dmath*{ \cdb*{diff000} }
   \Dmath*{ \cdb*{diff001} }
   \Dmath*{ \cdb*{diff002} }
   \Dmath*{ \cdb*{diff003} }
   \Dmath*{ \cdb*{diff004} }
\end{dgroup*}

\end{document}


In the following pages there will be frequent constructions of the form
\begin{dgroup*}
   \begin{dmath*}  3 A^b A^c\Gamma^a{}_{d(b,c)} = \cdb{scaled1.002} \end{dmath*}
   \begin{dmath*}  6 A^b A^c A^e \Gamma^a{}_{d(b,ce)} = \cdb{scaled2.002} \end{dmath*}
   \begin{dmath*} 15 A^b A^c A^e A^f \Gamma^a{}_{d(b,cef)} = \cdb{scaled3.002} \end{dmath*}
\end{dgroup*}
The vector $A^{a}$ has no special meaning. Its purpose is to indicate that the
associciated tensor is symmetric over a selection of its indices. If the $A^{a}$ were not included
then the right hand side would either need to be spelt out in full or some other device would
be needed to denote the symmetries. The symmetrisation brackets are included on the left hand
side though they are redundant (in the presence of the $A{a}$).

\clearpage

% =================================================================================================
\section*{The metric in RNC}
\def\Date{19 Jan 2024}
% \def\FileID{file:}

\documentclass[12pt]{cdblatex}

\begin{document}

% =================================================================================================
% create checkpoint file

\bgroup
\CdbSetup{action=hide}
\begin{cadabra}
   import cdblib
   checkpoint_file = 'tests/semantic/output/metric.json'
   cdblib.create (checkpoint_file)
   checkpoint = []
\end{cadabra}
\egroup

% =================================================================================================
\section*{The metric tensor in Riemann normal coordinates}

In this notebook we compute the recursive sequences
\begin{align}
\label{eq:pdgab}
g_{ab,d\ue} &=  \left(g_{cb}\Gamma^{c}{}_{a(d}\right){}_{,\ue)}
               + \left(g_{ac}\Gamma^{c}{}_{b(d}\right){}_{,\ue)}\\[10pt]
\label{eq:pdGamma}
(n+3)\Gamma^a{}_{d(b,c\ue)} &= (n+1)\left(R^a{}_{(bc\Dot d,\ue)}
                               - \left(\Gamma^a{}_{f(c}\Gamma^f{}_{b{\Dot d}}\right){}_{,\ue)}\right)
\end{align}
for $n=1,2,3,\cdots$. Note in these equations that the (extended) index $\ue$ contains $n$
normal indices.

We then construct a Taylor series for the metric using
\begin{dmath*}[spread=5pt]
g_{ab}(x) = g_{ab} + g_{ab,c}x^c + \frac{1}{2!} g_{ab,cd}x^cx^d + \frac{1}{3!} g_{ab,cde}x^cx^dx^e + \cdots
          = g_{ab} + \sum_{n=1}^\infty\> \frac{1}{n!}\>g_{ab,\uc}\>x^{.\uc}
\end{dmath*}

% =================================================================================================
\section*{Stage 1: Symmetrised partial derivatives of $g_{ab}$}

In this stage, equation (\ref{eq:pdgab}) is used to express the symmetrised partial derivatives
of the metric in terms of the symmetrised partial derivatives of the connection.

\begin{dgroup*}
   \begin{dmath*} g_{ab,c} A^{c} = \cdb{term1.200} \end{dmath*}
   \begin{dmath*} g_{ab,cd} A^{c} A^{d} = \cdb{term2.200} \end{dmath*}
   \begin{dmath*} g_{ab,cde} A^{c} A^{d} A^{e} = \cdb{term3.200} \end{dmath*}
\end{dgroup*}

% =================================================================================================
\section*{Stage 2: Replace derivatives of $\Gamma$ with partial derivs of $R$}

Now we use the results from {\verb|dGamma|} to replace derivatives of $\Gamma$ with
partial derivatives of $R$. These were computed in {\verb|dGamma|} using equation
(\ref{eq:pdGamma}) above.

\begin{dgroup*}
   \begin{dmath*} g_{ab,c} A^{c} = \cdb{term1.200} \end{dmath*}
   \begin{dmath*} g_{ab,cd} A^{c} A^{d} = \cdb{term2.303} \end{dmath*}
   \begin{dmath*} g_{ab,cde} A^{c} A^{d} A^{e} = \cdb{term3.305} \end{dmath*}
\end{dgroup*}

% =================================================================================================
\section*{Stage 3: Replace partial derivs of $R$ with covariant derivs of $R$}

Next we use the results from {\verb|dRabcd|} to replace the partial derivatives of $R$ with
covariant deriavtives.

\begin{dgroup*}
   \begin{dmath*} g_{ab,c} A^{c} = \cdb{term1.404} \end{dmath*}
   \begin{dmath*} g_{ab,cd} A^{c} A^{d} = \cdb{term2.404} \end{dmath*}
   \begin{dmath*} g_{ab,cde} A^{c} A^{d} A^{e} = \cdb{term3.403} \end{dmath*}
\end{dgroup*}

% =================================================================================================
\section*{Stage 4: Build the Taylor series for $g_{ab}$, reformatting and output}

Each of the above expressions constitutues one term in the Taylor series for the metric.
We also make the trivial change $A\rightarrow x$. Then we do some trivial reformatting.

\begin{align*}
   g_{ab}(x) &=   g_{ab}
                + g_{ab,c} x^c
                + \frac{1}{2!} g_{ab,cd} x^c x^d
                + \frac{1}{3!} g_{ab,cde} x^c x^d x^e +  \BigO{\eps^4}\\
             &= \cdb{metric4.501} + \BigO{\eps^4}
\end{align*}

\clearpage

% =================================================================================================
\section*{Shared properties}

\begin{cadabra}
   import time

   def flatten_Rabcd (obj):
       substitute (obj,$R^{a}_{b c d}   -> g^{a e} R_{e b c d}$)
       substitute (obj,$R_{a}^{b}_{c d} -> g^{b e} R_{a e c d}$)
       substitute (obj,$R_{a b}^{c}_{b} -> g^{c e} R_{a b e d}$)
       substitute (obj,$R_{a b c}^{d}   -> g^{d e} R_{a b c e}$)
       unwrap     (obj)
       return obj

   def impose_rnc (obj):
       # hide the derivatives of Gamma
       substitute (obj,$\partial_{d}{\Gamma^{a}_{b c}} -> zzz_{d}^{a}_{b c}$,repeat=True)
       substitute (obj,$\partial_{d e}{\Gamma^{a}_{b c}} -> zzz_{d e}^{a}_{b c}$,repeat=True)
       substitute (obj,$\partial_{d e f}{\Gamma^{a}_{b c}} -> zzz_{d e f}^{a}_{b c}$,repeat=True)
       substitute (obj,$\partial_{d e f g}{\Gamma^{a}_{b c}} -> zzz_{d e f g}^{a}_{b c}$,repeat=True)
       substitute (obj,$\partial_{d e f g h}{\Gamma^{a}_{b c}} -> zzz_{d e f g h}^{a}_{b c}$,repeat=True)
       # set Gamma to zero
       substitute (obj,$\Gamma^{a}_{b c} -> 0$,repeat=True)
       # recover the derivatives Gamma
       substitute (obj,$zzz_{d}^{a}_{b c} -> \partial_{d}{\Gamma^{a}_{b c}}$,repeat=True)
       substitute (obj,$zzz_{d e}^{a}_{b c} -> \partial_{d e}{\Gamma^{a}_{b c}}$,repeat=True)
       substitute (obj,$zzz_{d e f}^{a}_{b c} -> \partial_{d e f}{\Gamma^{a}_{b c}}$,repeat=True)
       substitute (obj,$zzz_{d e f g}^{a}_{b c} -> \partial_{d e f g}{\Gamma^{a}_{b c}}$,repeat=True)
       substitute (obj,$zzz_{d e f g h}^{a}_{b c} -> \partial_{d e f g h}{\Gamma^{a}_{b c}}$,repeat=True)
       return obj

   def get_xterm (obj,n):

       x^{a}::Weight(label=numx).

       foo := @(obj).
       bah  = Ex("numx = " + str(n))
       keep_weight (foo,bah)

       return foo

   # note: keeping numbering as is (out of order) to ensure R appears before \nabla R etc.
   def product_sort (obj):
       substitute (obj,$ A^{a}                             -> A001^{a}                  $)
       substitute (obj,$ x^{a}                             -> A002^{a}                  $)
       substitute (obj,$ g_{a b}                           -> A003_{a b}                $)
       substitute (obj,$ g^{a b}                           -> A004^{a b}                $)
       substitute (obj,$ \nabla_{e f g h}{R_{a b c d}}     -> A010_{a b c d e f g h}    $)
       substitute (obj,$ \nabla_{e f g}{R_{a b c d}}       -> A009_{a b c d e f g}      $)
       substitute (obj,$ \nabla_{e f}{R_{a b c d}}         -> A008_{a b c d e f}        $)
       substitute (obj,$ \nabla_{e}{R_{a b c d}}           -> A007_{a b c d e}          $)
       substitute (obj,$ \partial_{e f g h}{R_{a b c d}}   -> A014_{a b c d e f g h}    $)
       substitute (obj,$ \partial_{e f g}{R_{a b c d}}     -> A013_{a b c d e f g}      $)
       substitute (obj,$ \partial_{e f}{R_{a b c d}}       -> A012_{a b c d e f}        $)
       substitute (obj,$ \partial_{e}{R_{a b c d}}         -> A011_{a b c d e}          $)
       substitute (obj,$ \partial_{e f g h}{R^{a}_{b c d}} -> A018^{a}_{b c d e f g h}  $)
       substitute (obj,$ \partial_{e f g}{R^{a}_{b c d}}   -> A017^{a}_{b c d e f g}    $)
       substitute (obj,$ \partial_{e f}{R^{a}_{b c d}}     -> A016^{a}_{b c d e f}      $)
       substitute (obj,$ \partial_{e}{R^{a}_{b c d}}       -> A015^{a}_{b c d e}        $)
       substitute (obj,$ R_{a b c d}                       -> A005_{a b c d}            $)
       substitute (obj,$ R^{a}_{b c d}                     -> A006^{a}_{b c d}          $)
       sort_product   (obj)
       rename_dummies (obj)
       substitute (obj,$ A001^{a}                  -> A^{a}                             $)
       substitute (obj,$ A002^{a}                  -> x^{a}                             $)
       substitute (obj,$ A003_{a b}                -> g_{a b}                           $)
       substitute (obj,$ A004^{a b}                -> g^{a b}                           $)
       substitute (obj,$ A005_{a b c d}            -> R_{a b c d}                       $)
       substitute (obj,$ A006^{a}_{b c d}          -> R^{a}_{b c d}                     $)
       substitute (obj,$ A007_{a b c d e}          -> \nabla_{e}{R_{a b c d}}           $)
       substitute (obj,$ A008_{a b c d e f}        -> \nabla_{e f}{R_{a b c d}}         $)
       substitute (obj,$ A009_{a b c d e f g}      -> \nabla_{e f g}{R_{a b c d}}       $)
       substitute (obj,$ A010_{a b c d e f g h}    -> \nabla_{e f g h}{R_{a b c d}}     $)
       substitute (obj,$ A011_{a b c d e}          -> \partial_{e}{R_{a b c d}}         $)
       substitute (obj,$ A012_{a b c d e f}        -> \partial_{e f}{R_{a b c d}}       $)
       substitute (obj,$ A013_{a b c d e f g}      -> \partial_{e f g}{R_{a b c d}}     $)
       substitute (obj,$ A014_{a b c d e f g h}    -> \partial_{e f g h}{R_{a b c d}}   $)
       substitute (obj,$ A015^{a}_{b c d e}        -> \partial_{e}{R^{a}_{b c d}}       $)
       substitute (obj,$ A016^{a}_{b c d e f}      -> \partial_{e f}{R^{a}_{b c d}}     $)
       substitute (obj,$ A017^{a}_{b c d e f g}    -> \partial_{e f g}{R^{a}_{b c d}}   $)
       substitute (obj,$ A018^{a}_{b c d e f g h}  -> \partial_{e f g h}{R^{a}_{b c d}} $)

       return obj

   def reformat_xterm (obj,scale):
       foo  = Ex(str(scale))
       bah := @(foo) @(obj).
       distribute     (bah)
       bah = product_sort (bah)
       rename_dummies (bah)
       canonicalise   (bah)
       factor_out     (bah,$x^{a?}$)
       ans := @(bah) / @(foo).
       return ans

   def rescale_xterm (obj,scale):
       foo  = Ex(str(scale))
       bah := @(foo) @(obj).
       distribute  (bah)
       factor_out  (bah,$x^{a?}$)
       return bah

   {a,b,c,d,e,f,g,h,i,j,k,l,m,n,o,p,q,r,s,t,u,v,w#}::Indices(position=independent).

   \nabla{#}::Derivative.
   \partial{#}::PartialDerivative.

   g_{a b}::Metric.
   g^{a b}::InverseMetric.
   g_{a}^{b}::KroneckerDelta.
   g^{a}_{b}::KroneckerDelta.

   R_{a b c d}::RiemannTensor.
   R^{a}_{b c d}::RiemannTensor.
   R_{a b c}^{d}::RiemannTensor.

   \Gamma^{a}_{b c}::TableauSymmetry(shape={2}, indices={1,2}).

   g_{a b}::Depends(\partial{#}).
   R_{a b c d}::Depends(\partial{#}).
   R^{a}_{b c d}::Depends(\partial{#}).
   \Gamma^{a}_{b c}::Depends(\partial{#}).

   R_{a b c d}::Depends(\nabla{#}).
   R^{a}_{b c d}::Depends(\nabla{#}).

\end{cadabra}

\clearpage

% =================================================================================================
\section*{Stage 1: Symmetrised partial derivatives of $g_{ab}$}

\begin{cadabra}
   beg_stage_1 = time.time()

   # symmetrised partial derivatives of g_{ab}

   gab00:=g_{a b}.                                              # cdb (gab00.101,gab00)

   gab01:=g_{c b}\Gamma^{c}_{a d} + g_{a c}\Gamma^{c}_{b d}.    # cdb (gab01.101,gab01)

   gab02:=\partial_{e}{ @(gab01) }.                             # cdb (gab02.101,gab02)
   distribute   (gab02)                                         # cdb (gab02.102,gab02)
   product_rule (gab02)                                         # cdb (gab02.103,gab02)
   substitute   (gab02, $\partial_{d}{g_{a b}} -> @(gab01)$)    # cdb (gab02.104,gab02)
   distribute   (gab02)                                         # cdb (gab02.105,gab02)

   gab03:=\partial_{f}{ @(gab02) }.                             # cdb (gab03.101,gab03)
   distribute   (gab03)                                         # cdb (gab03.102,gab03)
   product_rule (gab03)                                         # cdb (gab03.103,gab03)
   substitute   (gab03, $\partial_{d}{g_{a b}} -> @(gab01)$)    # cdb (gab03.104,gab03)
   distribute   (gab03)                                         # cdb (gab03.105,gab03)

   gab04:=\partial_{g}{ @(gab03) }.                             # cdb (gab04.101,gab04)
   distribute   (gab04)                                         # cdb (gab04.102,gab04)
   product_rule (gab04)                                         # cdb (gab04.103,gab04)
   substitute   (gab04, $\partial_{d}{g_{a b}} -> @(gab01)$)    # cdb (gab04.104,gab04)
   distribute   (gab04)                                         # cdb (gab04.105,gab04)

   gab05:=\partial_{h}{ @(gab04) }.                             # cdb (gab05.101,gab05)
   distribute   (gab05)                                         # cdb (gab05.102,gab05)
   product_rule (gab05)                                         # cdb (gab05.103,gab05)
   substitute   (gab05, $\partial_{d}{g_{a b}} -> @(gab01)$)    # cdb (gab05.104,gab05)
   distribute   (gab05)                                         # cdb (gab05.105,gab05)

   gab00 = impose_rnc (gab00)   # cdb (gab00.102,gab00)
   gab01 = impose_rnc (gab01)   # cdb (gab01.102,gab01)
   gab02 = impose_rnc (gab02)   # cdb (gab02.106,gab02)
   gab03 = impose_rnc (gab03)   # cdb (gab03.106,gab03)
   gab04 = impose_rnc (gab04)   # cdb (gab04.106,gab04)
   gab05 = impose_rnc (gab05)   # cdb (gab05.106,gab05)

\end{cadabra}

\clearpage

\begin{dgroup*}
   \begin{dmath*} \cdb*{gab00.101} \end{dmath*}
   \begin{dmath*} \cdb*{gab00.102} \end{dmath*}
   \begin{dmath*} \cdb*{gab01.101} \end{dmath*}
   \begin{dmath*} \cdb*{gab01.102} \end{dmath*}
\end{dgroup*}

\begin{dgroup*}
   \begin{dmath*} \cdb*{gab02.101} \end{dmath*}
   \begin{dmath*} \cdb*{gab02.102} \end{dmath*}
   \begin{dmath*} \cdb*{gab02.103} \end{dmath*}
   \begin{dmath*} \cdb*{gab02.104} \end{dmath*}
   \begin{dmath*} \cdb*{gab02.105} \end{dmath*}
   \begin{dmath*} \cdb*{gab02.106} \end{dmath*}
\end{dgroup*}

\begin{dgroup*}
   \begin{dmath*} \cdb*{gab03.101} \end{dmath*}
   \begin{dmath*} \cdb*{gab03.102} \end{dmath*}
   \begin{dmath*} \cdb*{gab03.103} \end{dmath*}
   \begin{dmath*} \cdb*{gab03.104} \end{dmath*}
   \begin{dmath*} \cdb*{gab03.105} \end{dmath*}
   \begin{dmath*} \cdb*{gab03.106} \end{dmath*}
\end{dgroup*}

\begin{dgroup*}
   \begin{dmath*} \cdb*{gab04.101} \end{dmath*}
   \begin{dmath*} \cdb*{gab04.102} \end{dmath*}
   \begin{dmath*} \cdb*{gab04.103} \end{dmath*}
   \begin{dmath*} \cdb*{gab04.104} \end{dmath*}
   \begin{dmath*} \cdb*{gab04.105} \end{dmath*}
   \begin{dmath*} \cdb*{gab04.106} \end{dmath*}
\end{dgroup*}

\begin{cadabra}
   # prepare first six terms in the Taylor series expansion of g_{ab}(x)

   term0:= @(gab00).
   distribute (term0)                             # cdb(term0.200,term0)

   term1:= @(gab01) A^d.
   distribute (term1)                             # cdb(term1.200,term1)

   term2:= @(gab02) A^d A^e.
   distribute (term2)                             # cdb(term2.200,term2)

   term3:= @(gab03) A^d A^e A^f.
   distribute (term3)                             # cdb(term3.200,term3)

   term4:= @(gab04) A^d A^e A^f A^g.
   distribute (term4)                             # cdb(term4.200,term4)

   term5:= @(gab05) A^d A^e A^f A^g A^h.
   distribute (term5)                             # cdb(term5.200,term5)

   end_stage_1 = time.time()
\end{cadabra}

\begin{dgroup*}
   \begin{dmath*} \cdb*{term0.200} \end{dmath*}
   \begin{dmath*} \cdb*{term1.200} \end{dmath*}
   \begin{dmath*} \cdb*{term2.200} \end{dmath*}
   \begin{dmath*} \cdb*{term3.200} \end{dmath*}
   % \begin{dmath*} \cdb*{term4.200} \end{dmath*}
   % \begin{dmath*} \cdb*{term5.200} \end{dmath*}
\end{dgroup*}

\clearpage

% =================================================================================================
\section*{Stage 2: Replace derivatives of $\Gamma$ with partial derivs of $R$}

\begin{cadabra}
   import cdblib

   beg_stage_2 = time.time()

   dGamma01 = cdblib.get ('dGamma01','dGamma.json')  # cdb(dGamma01.300,dGamma01)
   dGamma02 = cdblib.get ('dGamma02','dGamma.json')  # cdb(dGamma02.300,dGamma02)
   dGamma03 = cdblib.get ('dGamma03','dGamma.json')  # cdb(dGamma03.300,dGamma03)
   dGamma04 = cdblib.get ('dGamma04','dGamma.json')  # cdb(dGamma04.300,dGamma04)
   dGamma05 = cdblib.get ('dGamma05','dGamma.json')  # cdb(dGamma05.300,dGamma05)

   # replace partial derivs of \Gamma with products and derivs of Riemann tensor

   substitute (term2,$\partial_{c}{\Gamma^{a}_{b d}}A^{c}A^{b} -> @(dGamma01)$,repeat=True)                       # cdb(term2.301,term2)
   substitute (term2,$\partial_{c}{\Gamma^{a}_{d b}}A^{c}A^{b} -> @(dGamma01)$,repeat=True)                       # cdb(term2.302,term2)
   distribute (term2)                                                                                             # cdb(term2.303,term2)

   substitute (term3,$\partial_{c e}{\Gamma^{a}_{d b}}A^{c}A^{b}A^{e} -> @(dGamma02)$,repeat=True)                # cdb(term3.301,term3)
   substitute (term3,$\partial_{c e}{\Gamma^{a}_{b d}}A^{c}A^{b}A^{e} -> @(dGamma02)$,repeat=True)                # cdb(term3.302,term3)
   substitute (term3,$\partial_{c}{\Gamma^{a}_{b d}}A^{c}A^{b} -> @(dGamma01)$,repeat=True)                       # cdb(term3.303,term3)
   substitute (term3,$\partial_{c}{\Gamma^{a}_{d b}}A^{c}A^{b} -> @(dGamma01)$,repeat=True)                       # cdb(term3.304,term3)
   distribute (term3)                                                                                             # cdb(term3.305,term3)

   substitute (term4,$\partial_{c e f}{\Gamma^{a}_{d b}}A^{c}A^{b}A^{e}A^{f} -> @(dGamma03)$,repeat=True)         # cdb(term4.301,term4)
   substitute (term4,$\partial_{c e f}{\Gamma^{a}_{b d}}A^{c}A^{b}A^{e}A^{f} -> @(dGamma03)$,repeat=True)         # cdb(term4.302,term4)
   substitute (term4,$\partial_{c e}{\Gamma^{a}_{d b}}A^{c}A^{b}A^{e} -> @(dGamma02)$,repeat=True)                # cdb(term4.303,term4)
   substitute (term4,$\partial_{c e}{\Gamma^{a}_{b d}}A^{c}A^{b}A^{e} -> @(dGamma02)$,repeat=True)                # cdb(term4.304,term4)
   substitute (term4,$\partial_{c}{\Gamma^{a}_{b d}}A^{c}A^{b} -> @(dGamma01)$,repeat=True)                       # cdb(term4.305,term4)
   substitute (term4,$\partial_{c}{\Gamma^{a}_{d b}}A^{c}A^{b} -> @(dGamma01)$,repeat=True)                       # cdb(term4.306,term4)
   distribute (term4)                                                                                             # cdb(term4.307,term4)

   substitute (term5,$\partial_{c e f g}{\Gamma^{a}_{d b}}A^{c}A^{b}A^{e}A^{f}A^{g} -> @(dGamma04)$,repeat=True)  # cdb(term5.301,term5)
   substitute (term5,$\partial_{c e f g}{\Gamma^{a}_{b d}}A^{c}A^{b}A^{e}A^{f}A^{g} -> @(dGamma04)$,repeat=True)  # cdb(term5.302,term5)
   substitute (term5,$\partial_{c e f}{\Gamma^{a}_{d b}}A^{c}A^{b}A^{e}A^{f} -> @(dGamma03)$,repeat=True)         # cdb(term5.303,term5)
   substitute (term5,$\partial_{c e f}{\Gamma^{a}_{b d}}A^{c}A^{b}A^{e}A^{f} -> @(dGamma03)$,repeat=True)         # cdb(term5.304,term5)
   substitute (term5,$\partial_{c e}{\Gamma^{a}_{d b}}A^{c}A^{b}A^{e} -> @(dGamma02)$,repeat=True)                # cdb(term5.305,term5)
   substitute (term5,$\partial_{c e}{\Gamma^{a}_{b d}}A^{c}A^{b}A^{e} -> @(dGamma02)$,repeat=True)                # cdb(term5.306,term5)
   substitute (term5,$\partial_{c}{\Gamma^{a}_{b d}}A^{c}A^{b} -> @(dGamma01)$,repeat=True)                       # cdb(term5.307,term5)
   substitute (term5,$\partial_{c}{\Gamma^{a}_{d b}}A^{c}A^{b} -> @(dGamma01)$,repeat=True)                       # cdb(term5.308,term5)
   distribute (term5)                                                                                             # cdb(term5.309,term5)

   end_stage_2 = time.time()

   # -------------------------------------------------------------------------------------------
   # this block of Xterms only produces formatted output, it's not part of the main computation
   # -------------------------------------------------------------------------------------------

   # the metric in terms of partial derivatives of Rabcd

   metric:=@(term0)
         + (1/1) @(term1)  # zero
         + (1/2) @(term2)
         + (1/6) @(term3)
         + (1/24) @(term4)
         + (1/120) @(term5).  # cdb(metric.301,metric)

   substitute (metric,$A^{a} -> x^{a}$)  # cdb (metric.302,metric)

   # reformat and tidy up

   Xterm0 := @(term0).
   Xterm1 := (1/1) @(term1).
   Xterm2 := (1/2) @(term2).
   Xterm3 := (1/6) @(term3).
   Xterm4 := (1/24) @(term4).
   Xterm5 := (1/120) @(term5).

   substitute (Xterm0,$A^{a} -> x^{a}$)
   substitute (Xterm1,$A^{a} -> x^{a}$)
   substitute (Xterm2,$A^{a} -> x^{a}$)
   substitute (Xterm3,$A^{a} -> x^{a}$)
   substitute (Xterm4,$A^{a} -> x^{a}$)
   substitute (Xterm5,$A^{a} -> x^{a}$)

   substitute (Xterm2,$g_{a b} \partial_{c}{R^{b}_{d e f}} -> \partial_{c}{R_{a d e f}}$)  # cdb(Xterm2.301,Xterm2)
   substitute (Xterm3,$g_{a b} \partial_{c}{R^{b}_{d e f}} -> \partial_{c}{R_{a d e f}}$)  # cdb(Xterm3.301,Xterm3)
   substitute (Xterm4,$g_{a b} \partial_{c}{R^{b}_{d e f}} -> \partial_{c}{R_{a d e f}}$)  # cdb(Xterm4.301,Xterm4)
   substitute (Xterm5,$g_{a b} \partial_{c}{R^{b}_{d e f}} -> \partial_{c}{R_{a d e f}}$)  # cdb(Xterm5.301,Xterm5)

   substitute (Xterm2,$g_{b a} \partial_{c}{R^{b}_{d e f}} -> \partial_{c}{R_{a d e f}}$)  # cdb(Xterm2.301,Xterm2)
   substitute (Xterm3,$g_{b a} \partial_{c}{R^{b}_{d e f}} -> \partial_{c}{R_{a d e f}}$)  # cdb(Xterm3.301,Xterm3)
   substitute (Xterm4,$g_{b a} \partial_{c}{R^{b}_{d e f}} -> \partial_{c}{R_{a d e f}}$)  # cdb(Xterm4.301,Xterm4)
   substitute (Xterm5,$g_{b a} \partial_{c}{R^{b}_{d e f}} -> \partial_{c}{R_{a d e f}}$)  # cdb(Xterm5.301,Xterm5)

   eliminate_metric (Xterm2)  # cdb(Xterm2.302,Xterm2)
   eliminate_metric (Xterm3)  # cdb(Xterm3.302,Xterm3)
   eliminate_metric (Xterm4)  # cdb(Xterm4.302,Xterm4)
   eliminate_metric (Xterm5)  # cdb(Xterm5.302,Xterm5)

   sort_product     (Xterm2)  # cdb(Xterm2.303,Xterm2)
   sort_product     (Xterm3)  # cdb(Xterm3.303,Xterm3)
   sort_product     (Xterm4)  # cdb(Xterm4.303,Xterm4)
   sort_product     (Xterm5)  # cdb(Xterm5.303,Xterm5)

   rename_dummies   (Xterm2)  # cdb(Xterm2.304,Xterm2)
   rename_dummies   (Xterm3)  # cdb(Xterm3.304,Xterm3)
   rename_dummies   (Xterm4)  # cdb(Xterm4.304,Xterm4)
   rename_dummies   (Xterm5)  # cdb(Xterm5.304,Xterm5)

   canonicalise     (Xterm2)  # cdb(Xterm2.305,Xterm2)
   canonicalise     (Xterm3)  # cdb(Xterm3.305,Xterm3)
   canonicalise     (Xterm4)  # cdb(Xterm4.305,Xterm4)
   canonicalise     (Xterm5)  # cdb(Xterm5.305,Xterm5)

   # push upper index to the left
   def tidy_Rabcd (obj):
       substitute (obj,$R_{a b c}^{d} -> - R^{d}_{c a b}$)
       substitute (obj,$R_{a b}^{c}_{d} -> R^{c}_{d a b}$)
       substitute (obj,$R_{a}^{b}_{c d} -> - R^{b}_{a c d}$)
       return obj

   Xterm0 = tidy_Rabcd (Xterm0)  # cdb(Xterm0.666,Xterm0)
   Xterm2 = tidy_Rabcd (Xterm2)  # cdb(Xterm2.666,Xterm2)
   Xterm3 = tidy_Rabcd (Xterm3)  # cdb(Xterm3.666,Xterm3)
   Xterm4 = tidy_Rabcd (Xterm4)  # cdb(Xterm4.666,Xterm4)
   Xterm5 = tidy_Rabcd (Xterm5)  # cdb(Xterm5.666,Xterm5)

   Xterm0 = reformat_xterm (Xterm0,  1)    # cdb(Xterm0.301,Xterm0)
   Xterm2 = reformat_xterm (Xterm2,  3)    # cdb(Xterm2.301,Xterm2)
   Xterm3 = reformat_xterm (Xterm3,  6)    # cdb(Xterm3.301,Xterm3)
   Xterm4 = reformat_xterm (Xterm4,360)    # cdb(Xterm4.301,Xterm4)
   Xterm5 = reformat_xterm (Xterm5,180)    # cdb(Xterm5.301,Xterm5)

   # canonicalise from reformat_xterm will slide upper index from left hand side
   # so now we slide the upper index back to the left

   Xterm0 = tidy_Rabcd (Xterm0)  # cdb(Xterm0.667,Xterm0)
   Xterm2 = tidy_Rabcd (Xterm2)  # cdb(Xterm2.667,Xterm2)
   Xterm3 = tidy_Rabcd (Xterm3)  # cdb(Xterm3.667,Xterm3)
   Xterm4 = tidy_Rabcd (Xterm4)  # cdb(Xterm4.667,Xterm4)
   Xterm5 = tidy_Rabcd (Xterm5)  # cdb(Xterm5.667,Xterm5)

   # metric to 3rd, 4th, 5th and 6th order terms in powers of x^a

   Metric3 := @(Xterm0) + @(Xterm2).                                      # cdb (Metric3.301,Metric3)
   Metric4 := @(Xterm0) + @(Xterm2) + @(Xterm3).                          # cdb (Metric4.301,Metric4)
   Metric5 := @(Xterm0) + @(Xterm2) + @(Xterm3) + @(Xterm4).              # cdb (Metric5.301,Metric5)
   Metric6 := @(Xterm0) + @(Xterm2) + @(Xterm3) + @(Xterm4) + @(Xterm5).  # cdb (Metric6.301,Metric6)

   # ------------------------------------------------------------------------------------
   # end of format block
   # ------------------------------------------------------------------------------------

\end{cadabra}

\clearpage

\begin{dgroup*}
   \begin{dmath*} \cdb*{term2.301} \end{dmath*}
   \begin{dmath*} \cdb*{term2.302} \end{dmath*}
   \begin{dmath*} \cdb*{term2.303} \end{dmath*}
\end{dgroup*}

\begin{dgroup*}
   \begin{dmath*} \cdb*{term3.301} \end{dmath*}
   \begin{dmath*} \cdb*{term3.302} \end{dmath*}
   \begin{dmath*} \cdb*{term3.303} \end{dmath*}
   \begin{dmath*} \cdb*{term3.304} \end{dmath*}
   \begin{dmath*} \cdb*{term3.305} \end{dmath*}
\end{dgroup*}

\begin{dgroup*}
   \begin{dmath*} \cdb*{term4.301} \end{dmath*}
   \begin{dmath*} \cdb*{term4.302} \end{dmath*}
   \begin{dmath*} \cdb*{term4.303} \end{dmath*}
   \begin{dmath*} \cdb*{term4.304} \end{dmath*}
   \begin{dmath*} \cdb*{term4.305} \end{dmath*}
   \begin{dmath*} \cdb*{term4.306} \end{dmath*}
   \begin{dmath*} \cdb*{term4.307} \end{dmath*}
\end{dgroup*}

\clearpage

\begin{dgroup*}
   \begin{dmath*} g_{ab}(x) = \cdb{Metric3.301} \end{dmath*}
   \begin{dmath*} g_{ab}(x) = \cdb{Metric4.301} \end{dmath*}
   \begin{dmath*} g_{ab}(x) = \cdb{Metric5.301} \end{dmath*}
   \begin{dmath*} g_{ab}(x) = \cdb{Metric6.301} \end{dmath*}
\end{dgroup*}

\clearpage

% =================================================================================================
\section*{Stage 3: Replace partial derivs of $R$ with covariant derivs of $R$}

\begin{cadabra}
   beg_stage_3 = time.time()

   # now convert partial derivs of Rabcd to covariant derivs

   dRabcd01 = cdblib.get ('dRabcd01','dRabcd.json')  # cdb(dRabcd01.400,dRabcd01)
   dRabcd02 = cdblib.get ('dRabcd02','dRabcd.json')  # cdb(dRabcd02.400,dRabcd02)
   dRabcd03 = cdblib.get ('dRabcd03','dRabcd.json')  # cdb(dRabcd03.400,dRabcd03)

   # term1 & term2 need no special care, just a bit of tidying

   eliminate_metric (term1)   # cdb(term1.401,term1)
   sort_product     (term1)   # cdb(term1.402,term1)
   rename_dummies   (term1)   # cdb(term1.403,term1)
   canonicalise     (term1)   # cdb(term1.404,term1)

   eliminate_metric (term2)   # cdb(term2.401,term2)
   sort_product     (term2)   # cdb(term2.402,term2)
   rename_dummies   (term2)   # cdb(term2.403,term2)
   canonicalise     (term2)   # cdb(term2.404,term2)

   # replace partial derivatives of Riemann tensor in term3, term4 etc. with covariant derivatives of Rabcd

   tmp01 := @(dRabcd01).      # cdb(tmp01.403,tmp01)
   tmp02 := @(dRabcd02).      # cdb(tmp02.403,tmp02)
   tmp03 := @(dRabcd03).      # cdb(tmp03.403,tmp03)

   substitute (term3,$A^{c}A^{d}A^{e}\partial_{e}{R^{a}_{c d b}} ->   @(tmp01)$,repeat=True)         # cdb(term3.401,term3)
   substitute (term3,$A^{c}A^{d}A^{e}\partial_{e}{R^{a}_{c b d}} -> - @(tmp01)$,repeat=True)         # cdb(term3.402,term3)
   distribute (term3)                                                                                # cdb(term3.403,term3)

   substitute (term4,$A^{c}A^{d}A^{e}A^{f}\partial_{e f}{R^{a}_{c d b}} ->   @(tmp02)$,repeat=True)  # cdb(term4.401,term4)
   substitute (term4,$A^{c}A^{d}A^{e}A^{f}\partial_{e f}{R^{a}_{c b d}} -> - @(tmp02)$,repeat=True)  # cdb(term4.402,term4)
   substitute (term4,$A^{c}A^{d}A^{e}\partial_{e}{R^{a}_{c d b}} ->   @(tmp01)$,repeat=True)         # cdb(term4.403,term4)
   substitute (term4,$A^{c}A^{d}A^{e}\partial_{e}{R^{a}_{c b d}} -> - @(tmp01)$,repeat=True)         # cdb(term4.404,term4)
   distribute (term4)                                                                                # cdb(term4.405,term4)

   substitute (term5,$A^{c}A^{d}A^{e}A^{f}A^{g}\partial_{e f g}{R^{a}_{c d b}} ->   @(tmp03)$,repeat=True)
   substitute (term5,$A^{c}A^{d}A^{e}A^{f}A^{g}\partial_{e f g}{R^{a}_{c b d}} -> - @(tmp03)$,repeat=True)
   substitute (term5,$A^{c}A^{d}A^{e}A^{f}\partial_{e f}{R^{a}_{c d b}} ->   @(tmp02)$,repeat=True)
   substitute (term5,$A^{c}A^{d}A^{e}A^{f}\partial_{e f}{R^{a}_{c b d}} -> - @(tmp02)$,repeat=True)
   substitute (term5,$A^{c}A^{d}A^{e}\partial_{e}{R^{a}_{c d b}} ->   @(tmp01)$,repeat=True)
   substitute (term5,$A^{c}A^{d}A^{e}\partial_{e}{R^{a}_{c b d}} -> - @(tmp01)$,repeat=True)
   distribute (term5)

   end_stage_3 = time.time()
\end{cadabra}

\begin{dgroup*}
   \begin{dmath*} \cdb*{tmp01.403} \end{dmath*}
   \begin{dmath*} \cdb*{tmp02.403} \end{dmath*}
   \begin{dmath*} \cdb*{tmp03.403} \end{dmath*}
\end{dgroup*}

\clearpage

\begin{dgroup*}
   \begin{dmath*} \cdb*{term1.401} \end{dmath*}
   \begin{dmath*} \cdb*{term1.402} \end{dmath*}
   \begin{dmath*} \cdb*{term1.403} \end{dmath*}
   \begin{dmath*} \cdb*{term1.404} \end{dmath*}
\end{dgroup*}

\begin{dgroup*}
   \begin{dmath*} \cdb*{term2.401} \end{dmath*}
   \begin{dmath*} \cdb*{term2.402} \end{dmath*}
   \begin{dmath*} \cdb*{term2.403} \end{dmath*}
   \begin{dmath*} \cdb*{term2.404} \end{dmath*}
\end{dgroup*}

\begin{dgroup*}
   \begin{dmath*} \cdb*{term3.401} \end{dmath*}
   \begin{dmath*} \cdb*{term3.402} \end{dmath*}
   \begin{dmath*} \cdb*{term3.403} \end{dmath*}
\end{dgroup*}

\begin{dgroup*}
   \begin{dmath*} \cdb*{term4.401} \end{dmath*}
   \begin{dmath*} \cdb*{term4.402} \end{dmath*}
   \begin{dmath*} \cdb*{term4.403} \end{dmath*}
   \begin{dmath*} \cdb*{term4.404} \end{dmath*}
   \begin{dmath*} \cdb*{term4.405} \end{dmath*}
\end{dgroup*}

\clearpage

% =================================================================================================
\section*{Stage 4: Build the Taylor series for $g_{ab}$, reformatting and output}

\begin{cadabra}
   beg_stage_4 = time.time()
   # final housekeeping

   term1 = flatten_Rabcd (term1)         # cdb(term1.501,term1)
   term2 = flatten_Rabcd (term2)         # cdb(term2.501,term2)
   term3 = flatten_Rabcd (term3)         # cdb(term3.501,term3)
   term4 = flatten_Rabcd (term4)         # cdb(term4.501,term4)
   term5 = flatten_Rabcd (term5)         # cdb(term5.501,term5)

   eliminate_metric (term1)
   eliminate_metric (term2)
   eliminate_metric (term3)
   eliminate_metric (term4)
   eliminate_metric (term5)

   eliminate_kronecker (term1)
   eliminate_kronecker (term2)
   eliminate_kronecker (term3)
   eliminate_kronecker (term4)
   eliminate_kronecker (term5)

   sort_product (term1)
   sort_product (term2)
   sort_product (term3)
   sort_product (term4)
   sort_product (term5)

   rename_dummies (term1)
   rename_dummies (term2)
   rename_dummies (term3)
   rename_dummies (term4)
   rename_dummies (term5)

   canonicalise (term1)                  # cdb(term1.502,term1)
   canonicalise (term2)                  # cdb(term2.502,term2)
   canonicalise (term3)                  # cdb(term3.502,term3)
   canonicalise (term4)                  # cdb(term4.502,term4)
   canonicalise (term5)                  # cdb(term5.502,term5)

   # this is out final answer

   metric:=@(term0)
         + (1/1) @(term1)
         + (1/2) @(term2)
         + (1/6) @(term3)
         + (1/24) @(term4)
         + (1/120) @(term5).             # cdb(metric.501,metric)

   substitute (metric,$A^{a} -> x^{a}$)  # cdb (metric.502,metric)

   cdblib.create ('metric.json')

   cdblib.put ('g_ab',metric,'metric.json')

   # extract the terms of the metric in powers of x

   term0 = get_xterm (metric,0)          # cdb(term0.503,term0)
   term1 = get_xterm (metric,1)          # cdb(term1.503,term1)
   term2 = get_xterm (metric,2)          # cdb(term2.503,term2)
   term3 = get_xterm (metric,3)          # cdb(term3.503,term3)
   term4 = get_xterm (metric,4)          # cdb(term4.503,term4)
   term5 = get_xterm (metric,5)          # cdb(term5.503,term5)

   cdblib.put ('g_ab_0',term0,'metric.json')
   cdblib.put ('g_ab_1',term1,'metric.json')
   cdblib.put ('g_ab_2',term2,'metric.json')
   cdblib.put ('g_ab_3',term3,'metric.json')
   cdblib.put ('g_ab_4',term4,'metric.json')
   cdblib.put ('g_ab_5',term5,'metric.json')

   # this version of "metric" is used only in the commentary at the start of this notebook

   metric4:=@(term0) + @(term1) + @(term2) + @(term3).  # cdb(metric4.501,metric4)

\end{cadabra}

\clearpage

\begin{dgroup*}
   \begin{dmath*} \cdb*{term2.501} \end{dmath*}
   \begin{dmath*} \cdb*{term2.502} \end{dmath*}
\end{dgroup*}

\begin{dgroup*}
   \begin{dmath*} \cdb*{term3.501} \end{dmath*}
   \begin{dmath*} \cdb*{term3.502} \end{dmath*}
\end{dgroup*}

\begin{dgroup*}
   \begin{dmath*} \cdb*{term4.501} \end{dmath*}
   \begin{dmath*} \cdb*{term4.502} \end{dmath*}
\end{dgroup*}

\begin{dgroup*}
   \begin{dmath*} \cdb*{term5.501} \end{dmath*}
   \begin{dmath*} \cdb*{term5.502} \end{dmath*}
\end{dgroup*}

\clearpage

\begin{dgroup*}
   \begin{dmath*} \cdb*{metric.501} \end{dmath*}
   \begin{dmath*} \cdb*{metric.502} \end{dmath*}
\end{dgroup*}

\clearpage

\begin{dgroup*}
   \begin{dmath*} \cdb*{term0.503} \end{dmath*}
   \begin{dmath*} \cdb*{term1.503} \end{dmath*}
   \begin{dmath*} \cdb*{term2.503} \end{dmath*}
   \begin{dmath*} \cdb*{term3.503} \end{dmath*}
   \begin{dmath*} \cdb*{term4.503} \end{dmath*}
   \begin{dmath*} \cdb*{term5.503} \end{dmath*}
\end{dgroup*}

% =================================================================================================
% the remaining code is just for pretty printing

\clearpage

\begin{cadabra}
   Xterm0 := @(term0).
   Xterm1 := @(term1).  # zero
   Xterm2 := @(term2).
   Xterm3 := @(term3).
   Xterm4 := @(term4).
   Xterm5 := @(term5).

   Xterm0 = reformat_xterm (Xterm0,  1)    # cdb(Xterm0.601,Xterm0)
   Xterm2 = reformat_xterm (Xterm2,  3)    # cdb(Xterm2.601,Xterm2)
   Xterm3 = reformat_xterm (Xterm3,  6)    # cdb(Xterm3.601,Xterm3)
   Xterm4 = reformat_xterm (Xterm4,180)    # cdb(Xterm4.601,Xterm4)
   Xterm5 = reformat_xterm (Xterm5, 90)    # cdb(Xterm5.601,Xterm5)

   gab3   := @(Xterm0) + @(Xterm2).                                      # cdb (gab3.601,gab3)
   gab4   := @(Xterm0) + @(Xterm2) + @(Xterm3).                          # cdb (gab4.601,gab4)
   gab5   := @(Xterm0) + @(Xterm2) + @(Xterm3) + @(Xterm4).              # cdb (gab5.601,gab5)
   gab6   := @(Xterm0) + @(Xterm2) + @(Xterm3) + @(Xterm4) + @(Xterm5).  # cdb (gab6.601,gab6)

   Metric := @(Xterm0) + @(Xterm2) + @(Xterm3) + @(Xterm4) + @(Xterm5).  # cdb (Metric.601,Metric)

   scaled0 = rescale_xterm (Xterm0,  1)    # cdb(scaled0.601,scaled0)
   scaled2 = rescale_xterm (Xterm2,  3)    # cdb(scaled2.601,scaled2)
   scaled3 = rescale_xterm (Xterm3,  6)    # cdb(scaled3.601,scaled3)
   scaled4 = rescale_xterm (Xterm4,180)    # cdb(scaled4.601,scaled4)
   scaled5 = rescale_xterm (Xterm5, 90)    # cdb(scaled5.601,scaled5)

   end_stage_4 = time.time()
\end{cadabra}

\clearpage

% =================================================================================================
\section*{The metric in Riemann normal coordinates}

\begin{dgroup*}
   \begin{dmath*} g_{a b}(x) = \cdb{Metric.601}+\BigO{\eps^6} \end{dmath*}
\end{dgroup*}

\clearpage

% =================================================================================================
\section*{Curvature expansion of the metric}
\begin{align*}
     g_{a b}(x) =
     \ngab{0}_{a b}
   + \ngab{2}_{a b}
   + \ngab{3}_{a b}
   + \ngab{4}_{a b}
   + \ngab{5}_{a b}+\BigO{\eps^6}
\end{align*}
\begin{dgroup*}
   \begin{dmath*}     \ngab{0}_{a b} = \cdb{scaled0.601} \end{dmath*}
   \begin{dmath*}   3 \ngab{2}_{a b} = \cdb{scaled2.601} \end{dmath*}
   \begin{dmath*}   6 \ngab{3}_{a b} = \cdb{scaled3.601} \end{dmath*}
   \begin{dmath*} 180 \ngab{4}_{a b} = \cdb{scaled4.601} \end{dmath*}
   \begin{dmath*}  90 \ngab{5}_{a b} = \cdb{scaled5.601} \end{dmath*}
\end{dgroup*}

\clearpage

% =================================================================================================
% export selected objects, these will later be imported into a library
% these are the objects that will appear in the paper

\begin{cadabra}
   cdblib.create ('metric.export')

   cdblib.put ('g_ab_3',Metric3,'metric.export')  # R and \partial R
   cdblib.put ('g_ab_4',Metric4,'metric.export')
   cdblib.put ('g_ab_5',Metric5,'metric.export')
   cdblib.put ('g_ab_6',Metric6,'metric.export')

   cdblib.put ('g_ab',  Metric, 'metric.export')  # R and \nabla R

   cdblib.put ('g_ab_scaled0',scaled0,'metric.export')
   cdblib.put ('g_ab_scaled2',scaled2,'metric.export')
   cdblib.put ('g_ab_scaled3',scaled3,'metric.export')
   cdblib.put ('g_ab_scaled4',scaled4,'metric.export')
   cdblib.put ('g_ab_scaled5',scaled5,'metric.export')

   checkpoint.append (Metric4)
   checkpoint.append (Metric6)

   checkpoint.append (Metric)

   checkpoint.append (scaled0)
   checkpoint.append (scaled2)
   checkpoint.append (scaled3)
   checkpoint.append (scaled4)
   checkpoint.append (scaled5)

   # cdbBeg (timing)
   print ("Stage 1: {:7.1f} secs\\hfill\\break".format(end_stage_1-beg_stage_1))
   print ("Stage 2: {:7.1f} secs\\hfill\\break".format(end_stage_2-beg_stage_2))
   print ("Stage 3: {:7.1f} secs\\hfill\\break".format(end_stage_3-beg_stage_3))
   print ("Stage 4: {:7.1f} secs".format(end_stage_4-beg_stage_4))
   # cdbEnd (timing)

\end{cadabra}

\clearpage

% =================================================================================================
\section*{Timing}

\cdb{timing}

% =================================================================================================
% export checkpoints in json format

\bgroup
\CdbSetup{action=hide}
\begin{cadabra}
   for i in range( len(checkpoint) ):
      cdblib.put ('check{:03d}'.format(i),checkpoint[i],checkpoint_file)
\end{cadabra}
\egroup

\end{document}


\begin{dgroup*}
   \begin{dmath*} g_{a b}(x) = \cdb{Metric.601}+\BigO{\eps^6} \end{dmath*}
\end{dgroup*}

% =================================================================================================
\section*{Curvature expansion of the metric}
\begin{align*}
     g_{a b}(x) =
     \ngab{0}_{a b}
   + \ngab{2}_{a b}
   + \ngab{3}_{a b}
   + \ngab{4}_{a b}
   + \ngab{5}_{a b}+\BigO{\eps^6}
\end{align*}
\begin{dgroup*}
   \begin{dmath*}     \ngab{0}_{a b} = \cdb{scaled0.601} \end{dmath*}
   \begin{dmath*}   3 \ngab{2}_{a b} = \cdb{scaled2.601} \end{dmath*}
   \begin{dmath*}   6 \ngab{3}_{a b} = \cdb{scaled3.601} \end{dmath*}
   \begin{dmath*} 180 \ngab{4}_{a b} = \cdb{scaled4.601} \end{dmath*}
   \begin{dmath*}  90 \ngab{5}_{a b} = \cdb{scaled5.601} \end{dmath*}
\end{dgroup*}

\clearpage

% =================================================================================================
\section*{The inverse metric in RNC}
\documentclass[12pt]{cdblatex}
\usepackage{fancyhdr}
\usepackage{footer}

\begin{document}

\section*{\jobname}

\CdbSetup{action=hide}

\begin{cadabra}
   import shared

   import cdblib

   term00A = cdblib.get ('check000','expected/metric-inv.json')
   term01A = cdblib.get ('check001','expected/metric-inv.json')
   term02A = cdblib.get ('check002','expected/metric-inv.json')
   term03A = cdblib.get ('check003','expected/metric-inv.json')
   term04A = cdblib.get ('check004','expected/metric-inv.json')
   term05A = cdblib.get ('check005','expected/metric-inv.json')
   term06A = cdblib.get ('check005','expected/metric-inv.json')
   term07A = cdblib.get ('check005','expected/metric-inv.json')

   term00B = cdblib.get ('check000','output/metric-inv.json')
   term01B = cdblib.get ('check001','output/metric-inv.json')
   term02B = cdblib.get ('check002','output/metric-inv.json')
   term03B = cdblib.get ('check003','output/metric-inv.json')
   term04B = cdblib.get ('check004','output/metric-inv.json')
   term05B = cdblib.get ('check005','output/metric-inv.json')
   term06B = cdblib.get ('check005','output/metric-inv.json')
   term07B = cdblib.get ('check005','output/metric-inv.json')

   diff000 = shared.check (term00A,term00B)   # cdb (diff000,diff000)
   diff001 = shared.check (term01A,term01B)   # cdb (diff001,diff001)
   diff002 = shared.check (term02A,term02B)   # cdb (diff002,diff002)
   diff003 = shared.check (term03A,term03B)   # cdb (diff003,diff003)
   diff004 = shared.check (term04A,term04B)   # cdb (diff004,diff004)
   diff005 = shared.check (term05A,term05B)   # cdb (diff005,diff005)
   diff006 = shared.check (term06A,term06B)   # cdb (diff006,diff006)
   diff007 = shared.check (term07A,term07B)   # cdb (diff007,diff007)

\end{cadabra}

\begin{dgroup*}
   \Dmath*{ \cdb*{diff000} }
   \Dmath*{ \cdb*{diff001} }
   \Dmath*{ \cdb*{diff002} }
   \Dmath*{ \cdb*{diff003} }
   \Dmath*{ \cdb*{diff004} }
   \Dmath*{ \cdb*{diff005} }
   \Dmath*{ \cdb*{diff006} }
   \Dmath*{ \cdb*{diff007} }
\end{dgroup*}

\end{document}


\begin{dgroup*}
   \begin{dmath*} g^{a b}(x) = \cdb{Metric.601}+\BigO{\eps^6} \end{dmath*}
\end{dgroup*}

% =================================================================================================
\section*{Curvature expansion of the inverse metric}
\begin{align*}
     g^{a b}(x) =
     \ngab{0}^{a b}
   + \ngab{2}^{a b}
   + \ngab{3}^{a b}
   + \ngab{4}^{a b}
   + \ngab{5}^{a b}+\BigO{\eps^6}
\end{align*}
\begin{dgroup*}
   \begin{dmath*}    \ngab{0}^{a b} = \cdb{scaled0.601} \end{dmath*}
   \begin{dmath*}  3 \ngab{2}^{a b} = \cdb{scaled2.601} \end{dmath*}
   \begin{dmath*}  6 \ngab{3}^{a b} = \cdb{scaled3.601} \end{dmath*}
   \begin{dmath*} 60 \ngab{4}^{a b} = \cdb{scaled4.601} \end{dmath*}
   \begin{dmath*} 90 \ngab{5}^{a b} = \cdb{scaled5.601} \end{dmath*}
\end{dgroup*}

\clearpage

% =================================================================================================
\section*{The metric determinant in RNC}
\def\Date{19 Jan 2024}
% \def\FileID{file:}

\documentclass[12pt]{cdblatex}

\begin{document}

% =================================================================================================
% create checkpoint file

\bgroup
\CdbSetup{action=hide}
\begin{cadabra}
   import cdblib
   checkpoint_file = 'tests/semantic/output/detg2.json'
   cdblib.create (checkpoint_file)
   checkpoint = []
\end{cadabra}
\egroup

% =================================================================================================
\section*{The determinant of the metric}

Our game here is to compute (the leading terms) in $\det g$ of the metric in RNC form
\begin{dgroup*}
   \begin{dmath*} g_{a b}(x) = \cdb{gab.001}+\BigO{\eps^5} \end{dmath*}
\end{dgroup*}
For the sake of simplicity let's assume that we are working in 3-dimensions. The following
analysis is easily generalsied to other dimensions (and the final answers for $\det g$ and
friends are unchanged).

Define $\eps^{abc}_{ijk}$ by
\begin{align}
   \eps^{abc}_{ijk} =
        \delta^a_i \delta^b_j \delta^c_k - \delta^b_i \delta^a_j \delta^c_k
      + \delta^c_i \delta^a_j \delta^b_k - \delta^c_i \delta^b_j \delta^a_k
      + \delta^b_i \delta^c_j \delta^a_k - \delta^a_i \delta^c_j \delta^b_k
\end{align}
It is easy to see that $\eps^{abc}_{ijk}$ is anti-symmetric in both its upper and lower
indices. A trivial computation shows that for any $3{}\times{}3$ square matrix $M_{ab}$,
\begin{align}
   \eps^{abc}_{123} M_{1a} M_{2b} M_{3c}
   = \left(
          \delta^a_1 \delta^b_2 \delta^c_3 - \delta^b_1 \delta^a_2 \delta^c_3
        + \delta^c_1 \delta^a_2 \delta^b_3 - \delta^c_1 \delta^b_2 \delta^a_3
        + \delta^b_1 \delta^c_2 \delta^a_3 - \delta^a_1 \delta^c_2 \delta^b_3
     \right)M_{1a} M_{2b} M_{3c}
   = \det M
\end{align}
This can be easily generalised to
\begin{align}
   \eps^{abc}_{ijk} M_{pa} M_{qb} M_{rc}
   =
   \begin{cases}
      \pm \det M &\text{when $(ijk)$ and $(pqr)$ are permutations of $(123)$}\\
      0 & \text{otherwise}
   \end{cases}
\end{align}
The $\pm$ sign in the above depends on the particular permutations of $(ijk)$ and $(pqr)$. If
both permutations are even or both odd then the sign is $+1$ otherwise the sign is $-1$.
The same arguments can also be applied to a matrix inverse $N^{-1}$ leading to
\begin{align}
   \eps^{ijk}_{uvw} N^{pu} N^{qv} M^{rw}
   =
   \begin{cases}
      \pm \det {N^{-1}} &\text{when $(ijk)$ and $(pqr)$ are permutations of $(123)$}\\
      0 & \text{otherwise}
   \end{cases}
\end{align}
Note that the $\pm$ in this case will match exactly that for the case of $\det M$. Thus,
multiplying both expressions and summing over all choices for $(ijk)$ and $(pqr)$ leads
to
\begin{align}
   \sum_{\substack{(ijk)\\(pqr)}}\left(\det N^{-1}\right) \det M
   = \eps^{ijk}_{uvw} N^{pu} N^{qv} M^{rw} \eps^{abc}_{ijk} M_{pa} M_{qb} M_{rc}
\end{align}
where the sum on the left hand side includes just those $(ijk)$ and $(prq)$ that are
permutations of $(123)$. There are $3!$ choices for $(ijk)$ and $3!$ choices for
$(pqr)$ and thus the left hand side is easily reduced to $(3!)^2 \det M/\det N$ where
$\det N = 1/\det N^{-1}$. For the right hand side notice that
\begin{align}
   \eps^{ijk}_{uvw} \eps^{abc}_{ijk} = 3! \eps^{abc}_{uvw}
\end{align}
which leads to
\begin{align}
   \det M = \frac{1}{3!} \det N \eps^{abc}_{uvw} M_{pa} M_{qb} M_{rc} N^{pu} N^{qv} N^{rw}
\end{align}

For our RNC metric we will set $N^{ab} = g^{ab}$ and $M_{ij} = g_{ij}(x)$. Since $g^{ab}$ is
of the form ${\rm diag}(-1,1,1,1)$ we have $\det g = -1$ and thus
\begin{align}
   \det g(x) = - \frac{1}{3!} \eps^{abc}_{ijk}\, g_{pa}(x)\, g_{qb}(x)\, g_{rc}(x)\, g^{ip} g^{jq} g^{kr}
\end{align}

The $\eps^{abc}_{ijk}$ can be constructed in Cadabra by applying the \verb|asym| algorithm
to the upper indices of $\delta^a_i \delta^b_j \delta^c_k$. Note that \verb|asym| will
include the $1/3!$ coeffcient as part of its output.

The following code computes $-\det g$ rather than $\det g$.

{\bf Note} that Calzetta etal. use an opposite sign for $R_{abcd}$ so when comparing the
following results against Calzetta do take note of this flipped sign in $R_{abcd}$.

\clearpage

\begin{cadabra}
   {a,b,c,d,e,f,g,h,i,j,k,l,m,n,o,p,q,r,s,t,u,v,w#}::Indices(position=independent).

   {a,b,c,d,e,f,g,h,i,j,k,l,m,n,o,p,q,r,s,t,u,v,w#}::Integer(1..2).

   \nabla{#}::Derivative.

   d{#}::KroneckerDelta.

   g^{a b}::Symmetric.
   g_{a b}::Symmetric.

   R_{a b c d}::RiemannTensor.

   x^{a}::Weight(label=numx,value=1).

   def truncate (obj,n):

       ans = Ex(0)

       for i in range (0,n+1):
          foo := @(obj).
          bah = Ex("numx = " + str(i))
          keep_weight (foo, bah)
          ans = ans + foo

       return ans

   import cdblib

   g0ab = cdblib.get('g_ab_0','metric.json')
   g1ab = cdblib.get('g_ab_1','metric.json')  # zero in RNC
   g2ab = cdblib.get('g_ab_2','metric.json')
   g3ab = cdblib.get('g_ab_3','metric.json')
   g4ab = cdblib.get('g_ab_4','metric.json')
   g5ab = cdblib.get('g_ab_5','metric.json')

   gab := @(g0ab) + @(g1ab) + @(g2ab) + @(g3ab) + @(g4ab) + @(g5ab).  # cdb (gab.001,gab)
   gxab := gx_{a b} -> @(gab).

   eps := d^{a}_{i} d^{b}_{j}.   # cdb(eps.001,eps)
   asym (eps,$^{a},^{b}$)        # cdb(eps.002,eps) # includes a factor of 1/2!

   # compute negative Ndetg rather than det g
   Ndetg := @(eps) gx_{p a} gx_{q b} g^{i p} g^{j q}.  # note 1/2! included in eps

   substitute       (Ndetg,gxab)
   distribute       (Ndetg)
   Ndetg = truncate (Ndetg,5)                                          # cdb (Ndetg.001,Ndetg)
   substitute       (Ndetg,$g^{a b} g_{b c} -> d^{a}_{c}$,repeat=True) # cdb (Ndetg.002,Ndetg)
   eliminate_kronecker (Ndetg)                                         # cdb (Ndetg.003,Ndetg)
   sort_product     (Ndetg)                                            # cdb (Ndetg.004,Ndetg)
   rename_dummies   (Ndetg)                                            # cdb (Ndetg.005,Ndetg)
   canonicalise     (Ndetg)                                            # cdb (Ndetg.006,Ndetg)

   # introduce the Ricci tensor

   substitute     (Ndetg,$R_{a b c d} g^{a c} -> R_{b d}$,repeat=True)                                  # cdb (Ndetg.101,Ndetg)
   substitute     (Ndetg,$\nabla_{a}{R_{b c d e}} g^{b d}  -> \nabla_{a}{R_{c e}}$,repeat=True)         # cdb (Ndetg.102,Ndetg)
   substitute     (Ndetg,$\nabla_{a b}{R_{c d e f}} g^{c e}  -> \nabla_{a b}{R_{d f}}$,repeat=True)     # cdb (Ndetg.103,Ndetg)
   substitute     (Ndetg,$\nabla_{a b c}{R_{d e f g}} g^{d f}  -> \nabla_{a b c}{R_{e g}}$,repeat=True) # cdb (Ndetg.104,Ndetg)

   # the following are based on sqrt-detg.tex

   sqrtNdetg := 1/2 + (1/2) @(Ndetg)
               - (1/8) (1/9) R_{a b} R_{c d} x^{a} x^{b} x^{c} x^{d}
               - (1/4) (1/18) R_{a b} \nabla_{c}{R_{d e}} x^{a} x^{b} x^{c} x^{d} x^{e}.
               # cdb (sqrtNdetg.001,sqrtNdetg)

   sort_product   (sqrtNdetg)                                          # cdb (sqrtNdetg.002,sqrtNdetg)
   rename_dummies (sqrtNdetg)                                          # cdb (sqrtNdetg.003,sqrtNdetg)
   canonicalise   (sqrtNdetg)                                          # cdb (sqrtNdetg.004,sqrtNdetg)

   logNdetg := -1 + @(Ndetg)
               - (1/2) (1/9) R_{a b} R_{c d} x^{a} x^{b} x^{c} x^{d}
               - (1/18) R_{a b} \nabla_{c}{R_{d e}} x^{a} x^{b} x^{c} x^{d} x^{e}.
               # cdb (logNdetg.001,logNdetg)

   sort_product   (logNdetg)                                           # cdb (logNdetg.002,logNdetg)
   rename_dummies (logNdetg)                                           # cdb (logNdetg.003,logNdetg)
   canonicalise   (logNdetg)                                           # cdb (logNdetg.004,logNdetg)

\end{cadabra}

\clearpage

\begin{dgroup*}
   \begin{dmath*} \cdb*{eps.001} \end{dmath*}
   \begin{dmath*} \cdb*{eps.002} \end{dmath*}
   \begin{dmath*} \cdb*{Ndetg.001} \end{dmath*}
   \begin{dmath*} \cdb*{Ndetg.002} \end{dmath*}
   \begin{dmath*} \cdb*{Ndetg.003} \end{dmath*}
   \begin{dmath*} \cdb*{Ndetg.004} \end{dmath*}
   \begin{dmath*} \cdb*{Ndetg.005} \end{dmath*}
   \begin{dmath*} \cdb*{Ndetg.006} \end{dmath*}
   \begin{dmath*} \cdb*{Ndetg.104} \end{dmath*}
   \begin{dmath*} \cdb*{sqrtNdetg.004} \end{dmath*}
   \begin{dmath*} \cdb*{logNdetg.004} \end{dmath*}
\end{dgroup*}

% =================================================================================================
% the remaining code is just for pretty printing

\clearpage

\begin{cadabra}
   # note: keeping numbering as is (out of order) to ensure R appears before \nabla R etc.
   def product_sort (obj):
       substitute (obj,$ x^{a}                            -> A000^{a}               $)
       substitute (obj,$ g^{a b}                          -> A001^{a b}             $)
       substitute (obj,$ \nabla_{c d e f}{R_{a b}}        -> A007_{a b c d e f}     $)
       substitute (obj,$ \nabla_{c d e}{R_{a b}}          -> A006_{a b c d e}       $)
       substitute (obj,$ \nabla_{c d}{R_{a b}}            -> A005_{a b c d}         $)
       substitute (obj,$ \nabla_{c}{R_{a b}}              -> A004_{a b c}           $)
       substitute (obj,$ \nabla_{e f g h}{R_{a b c d}}    -> A011_{a b c d e f g h} $)
       substitute (obj,$ \nabla_{e f g}{R_{a b c d}}      -> A010_{a b c d e f g}   $)
       substitute (obj,$ \nabla_{e f}{R_{a b c d}}        -> A009_{a b c d e f}     $)
       substitute (obj,$ \nabla_{e}{R_{a b c d}}          -> A008_{a b c d e}       $)
       substitute (obj,$ R_{a b}                          -> A002_{a b}             $)
       substitute (obj,$ R_{a b c d}                      -> A003_{a b c d}         $)
       sort_product   (obj)
       rename_dummies (obj)
       substitute (obj,$ A000^{a}                 -> x^{a}                          $)
       substitute (obj,$ A001^{a b}               -> g^{a b}                        $)
       substitute (obj,$ A002_{a b}               -> R_{a b}                        $)
       substitute (obj,$ A003_{a b c d}           -> R_{a b c d}                    $)
       substitute (obj,$ A004_{a b c}             -> \nabla_{c}{R_{a b}}            $)
       substitute (obj,$ A005_{a b c d}           -> \nabla_{c d}{R_{a b}}          $)
       substitute (obj,$ A006_{a b c d e}         -> \nabla_{c d e}{R_{a b}}        $)
       substitute (obj,$ A007_{a b c d e f}       -> \nabla_{c d e f}{R_{a b}}      $)
       substitute (obj,$ A008_{a b c d e}         -> \nabla_{e}{R_{a b c d}}        $)
       substitute (obj,$ A009_{a b c d e f}       -> \nabla_{e f}{R_{a b c d}}      $)
       substitute (obj,$ A010_{a b c d e f g}     -> \nabla_{e f g}{R_{a b c d}}    $)
       substitute (obj,$ A011_{a b c d e f g h}   -> \nabla_{e f g h}{R_{a b c d}}  $)

       return obj

   def get_term (obj,n):

       x^{a}::Weight(label=numx).

       foo := @(obj).
       bah  = Ex("numx = " + str(n))
       keep_weight (foo,bah)

       return foo

   def reformat (obj,scale):
       foo  = Ex(str(scale))
       bah := @(foo) @(obj).
       distribute     (bah)
       bah = product_sort (bah)
       rename_dummies (bah)
       canonicalise   (bah)
       sort_sum       (bah)
       factor_out     (bah,$x^{a?}$)
       ans := @(bah) / @(foo).
       return ans

   def rescale (obj,scale):
       foo  = Ex(str(scale))
       bah := @(foo) @(obj).
       distribute  (bah)
       factor_out  (bah,$x^{a?}$)
       return bah

   # ---------------------------------------------------------------
   # reformat Ndetg

   Rterm0 = get_term (Ndetg,0)       # cdb(Rterm0.701,Rterm0)
   Rterm1 = get_term (Ndetg,1)       # cdb(Rterm1.701,Rterm1)
   Rterm2 = get_term (Ndetg,2)       # cdb(Rterm2.701,Rterm2)
   Rterm3 = get_term (Ndetg,3)       # cdb(Rterm3.701,Rterm3)
   Rterm4 = get_term (Ndetg,4)       # cdb(Rterm4.701,Rterm4)
   Rterm5 = get_term (Ndetg,5)       # cdb(Rterm5.701,Rterm5)

   Rterm0 = reformat (Rterm0,  1)    # cdb(Rterm0.702,Rterm0)
   Rterm1 = reformat (Rterm1,  1)    # cdb(Rterm1.702,Rterm1)
   Rterm2 = reformat (Rterm2,  3)    # cdb(Rterm2.702,Rterm2)
   Rterm3 = reformat (Rterm3,  6)    # cdb(Rterm3.702,Rterm3)
   Rterm4 = reformat (Rterm4,180)    # cdb(Rterm4.702,Rterm4)
   Rterm5 = reformat (Rterm5, 90)    # cdb(Rterm5.702,Rterm5)

   Ndetg := @(Rterm0) + @(Rterm1) + @(Rterm2) + @(Rterm3) + @(Rterm4) + @(Rterm5).  # cdb (Ndetg.701,Ndetg)

   # ---------------------------------------------------------------
   # reformat sqrtNdetg

   Rterm0 = get_term (sqrtNdetg,0)   # cdb(Rterm0.801,Rterm0)
   Rterm1 = get_term (sqrtNdetg,1)   # cdb(Rterm1.801,Rterm1)
   Rterm2 = get_term (sqrtNdetg,2)   # cdb(Rterm2.801,Rterm2)
   Rterm3 = get_term (sqrtNdetg,3)   # cdb(Rterm3.801,Rterm3)
   Rterm4 = get_term (sqrtNdetg,4)   # cdb(Rterm4.801,Rterm4)
   Rterm5 = get_term (sqrtNdetg,5)   # cdb(Rterm5.801,Rterm5)

   Rterm0 = reformat (Rterm0,  1)    # cdb(Rterm0.802,Rterm0)
   Rterm1 = reformat (Rterm1,  1)    # cdb(Rterm1.802,Rterm1)
   Rterm2 = reformat (Rterm2,  6)    # cdb(Rterm2.802,Rterm2)
   Rterm3 = reformat (Rterm3, 12)    # cdb(Rterm3.802,Rterm3)
   Rterm4 = reformat (Rterm4,360)    # cdb(Rterm4.802,Rterm4)
   Rterm5 = reformat (Rterm5,360)    # cdb(Rterm5.802,Rterm5)

   sqrtNdetg := @(Rterm0) + @(Rterm1) + @(Rterm2) + @(Rterm3) + @(Rterm4) + @(Rterm5).  # cdb (sqrtNdetg.801,sqrtNdetg)

   # ---------------------------------------------------------------
   # reformat logNdetg

   Rterm0 = get_term (logNdetg,0)    # cdb(Rterm0.901,Rterm0)
   Rterm1 = get_term (logNdetg,1)    # cdb(Rterm1.901,Rterm1)
   Rterm2 = get_term (logNdetg,2)    # cdb(Rterm2.901,Rterm2)
   Rterm3 = get_term (logNdetg,3)    # cdb(Rterm3.901,Rterm3)
   Rterm4 = get_term (logNdetg,4)    # cdb(Rterm4.901,Rterm4)
   Rterm5 = get_term (logNdetg,5)    # cdb(Rterm5.901,Rterm5)

   Rterm0 = reformat (Rterm0,  1)    # cdb(Rterm0.902,Rterm0)
   Rterm1 = reformat (Rterm1,  1)    # cdb(Rterm1.902,Rterm1)
   Rterm2 = reformat (Rterm2,  3)    # cdb(Rterm2.902,Rterm2)
   Rterm3 = reformat (Rterm3,  6)    # cdb(Rterm3.902,Rterm3)
   Rterm4 = reformat (Rterm4,180)    # cdb(Rterm4.902,Rterm4)
   Rterm5 = reformat (Rterm5, 90)    # cdb(Rterm5.902,Rterm5)

   logNdetg := @(Rterm0) + @(Rterm1) + @(Rterm2) + @(Rterm3) + @(Rterm4) + @(Rterm5).  # cdb (logNdetg.901,logNdetg)

\end{cadabra}

\clearpage

% =================================================================================================
\section*{The metric determinant in Riemann normal coordinates}

\begin{dgroup*}
   \Dmath*{-\det g(x) = \cdb{Ndetg.701}+\BigO{\eps^6}}
\end{dgroup*}

% =================================================================================================
\section*{The volume element in Riemann normal coordinates}

If $-\det g(x)$ is non-negative then we also have
%
\begin{dgroup*}
   \Dmath*{\sqrt{-\det g(x)} = \cdb{sqrtNdetg.801}+\BigO{\eps^6}}
\end{dgroup*}

% =================================================================================================
\section*{The log of -detg in Riemann normal coordinates}

Apart from the signs, this matches exactly the expression given by Calzetta etal. (eq. A14)

\begin{dgroup*}
   \Dmath*{\log\left(-\det g(x)\right) = \cdb{logNdetg.901}+\BigO{\eps^6}}
\end{dgroup*}

\clearpage

% =================================================================================================
% export selected objects, these will later be imported into a library
% these are the objects that will appear in the paper

\begin{cadabra}
   cdblib.create ('detg2.export')

   cdblib.put ('Ndetg',    Ndetg,    'detg2.export')
   cdblib.put ('sqrtNdetg',sqrtNdetg,'detg2.export')
   cdblib.put ('logNdetg', logNdetg, 'detg2.export')

   checkpoint.append (Ndetg)
   checkpoint.append (sqrtNdetg)
   checkpoint.append (logNdetg)

\end{cadabra}

% =================================================================================================
% export checkpoints in json format

\bgroup
\CdbSetup{action=hide}
\begin{cadabra}
   for i in range( len(checkpoint) ):
      cdblib.put ('check{:03d}'.format(i),checkpoint[i],checkpoint_file)
\end{cadabra}
\egroup

\end{document}


\begin{dgroup*}
   \begin{dmath*} -\det g(x) = \cdb{Ndetg.701}+\BigO{\eps^5} \end{dmath*}
\end{dgroup*}

% =================================================================================================
\section*{The metric Jacobian in RNC}

\begin{dgroup*}
   \begin{dmath*} \sqrt{-\det g(x)} = \cdb{sqrtNdetg.801}+\BigO{\eps^5} \end{dmath*}
\end{dgroup*}

% =================================================================================================
\section*{The log of detg in RNC}

\begin{dgroup*}
   \begin{dmath*} \log{-\det g(x)\vert} = \cdb{logNdetg.901}+\BigO{\eps^5} \end{dmath*}
\end{dgroup*}

\clearpage

% =================================================================================================
\section*{The connection in RNC}
\def\Date{19 Jan 2024}
% \def\FileID{file:}

\documentclass[12pt]{cdblatex}

\begin{document}

% =================================================================================================
% create checkpoint file

\bgroup
\CdbSetup{action=hide}
\begin{cadabra}
   import cdblib
   checkpoint_file = 'tests/semantic/output/connection.json'
   cdblib.create (checkpoint_file)
   checkpoint = []
\end{cadabra}
\egroup

% =================================================================================================
\section*{The connection}

Here we use the output from {\tt metric.tex} and {\tt metric-inv.tex} to compute the metric connection
$\Gamma^{d}_{ab}$. We use the standard metric compatible connection
\begin{align}
   \label{eqn:Gamma}
   \Gamma^{d}_{ab} = \frac{1}{2} g^{dc}\left( g_{cb,a} + g_{ac,b} - g_{ab,c} \right)
\end{align}

Since {\tt metric.tex} and {\tt metric-inv.tex} generate truncated expressions for $g_{ab}$ and
$g^{ab}$ a similar truncation must be applied to this computation of $\Gamma^{d}_{ab}$. The naive
choice is to truncate $\Gamma^{d}_{ab}$ \emph{after} it has been fully evaluated on the truncated
expersions for $g_{ab}$ and $g^{ab}$. This will work but it wastes time and memory (big time).

A better approach is to truncate $\Gamma^{d}_{ab}$ during its construction. That is, we take
careful note of how the terms in the finite series for $g_{ab}$ and $g^{ab}$ combine to produce
the terms of a particular order in the expansion of $\Gamma^{d}_{ab}$.

Suppose $g_{ab}$ and $g^{ab}$ are known to say fourth order. We can write each of these as follows
\begin{align}
   g_{ab} &= \ngab{0}_{ab} + \ngab{1}_{ab} + \ngab{2}_{ab} + \ngab{3}_{ab} + \ngab{4}_{ab}\\
   g^{ab} &= \ngab{0}^{ab} + \ngab{1}^{ab} + \ngab{2}^{ab} + \ngab{3}^{ab} + \ngab{4}^{ab}
\end{align}
where $\ngab{n}$ denotes a term of order $\BigO{\eps^n}$. A similar expansion applies for $\Gamma^{d}_{ab}$, that is
\begin{align}
   \Gamma^{d}_{ab} = \nGamma{0}^{d}_{ab}
                   + \nGamma{1}^{d}_{ab}
                   + \nGamma{2}^{d}_{ab}
                   + \nGamma{3}^{d}_{ab}
                   + \nGamma{4}^{d}_{ab}
\end{align}
After substituting these formal expansions into the equation \eqref{eqn:Gamma} and then matching
corresponing terms we obtain
\begin{align}
   \nGamma{n}^{d}_{ab}
   =
   \frac{1}{2} \sum_{i=0}^{i=n} \ngab{i}^{dc}\left( \ngab{n-i}_{cb,a} + \ngab{n-i}_{ac,b} - \ngab{n-i}_{ab,c} \right)
\end{align}

We use this equation to compute the successive terms in $\Gamma^{d}_{ab}$.

\clearpage

\begin{cadabra}
   {a,b,c,d,e,f,g,h,i,j,k,l,m,n,o,p,q,r,s,t,u,v,w#}::Indices(position=independent).

   D{#}::Derivative.
   \nabla{#}::Derivative.
   \partial{#}::PartialDerivative.

   g_{a b}::Metric.
   g^{a b}::InverseMetric.
   g_{a}^{b}::KroneckerDelta.
   g^{a}_{b}::KroneckerDelta.
   \delta^{a}_{b}::KroneckerDelta.
   \delta_{a}^{b}::KroneckerDelta.

   R_{a b c d}::RiemannTensor.
   R^{a}_{b c d}::RiemannTensor.

   x^{a}::Depends(D{#}).

   R_{a b c d}::Depends(\nabla{#}).
   R^{a}_{b c d}::Depends(\nabla{#}).

   import cdblib

   gab = cdblib.get ('g_ab','metric.json')      # cdb(gab.000,gab)
   iab = cdblib.get ('g^ab','metric-inv.json')  # cdb(iab.000,iab)

   defgab := g_{a b} -> @(gab).
   defiab := g^{a b} -> @(iab).

   dgab := D_{a}{g_{c b}} + D_{b}{g_{a c}} - D_{c}{g_{a b}}.  # cdb(dgab.001,dgab)

   substitute   (dgab,defgab)

   distribute   (dgab)              # cdb(dgab.002,dgab)
   unwrap       (dgab)              # cdb(dgab.003,dgab)
   product_rule (dgab)              # cdb(dgab.004,dgab)
   distribute   (dgab)              # cdb(dgab.005,dgab)
   substitute   (dgab,$D_{a}{x^{b}}->\delta^{b}_{a}$,repeat=True)  # cdb(dgab.006,dgab)
   eliminate_kronecker (dgab)       # cdb(dgab.007,dgab)
   sort_product   (dgab)            # cdb(dgab.008,dgab)
   rename_dummies (dgab)            # cdb(dgab.009,dgab)
   canonicalise   (dgab)            # cdb(dgab.010,dgab)

\end{cadabra}

\clearpage

\begin{dgroup*}
   \begin{dmath*} \cdb*{gab.000} \end{dmath*}
\end{dgroup*}

\begin{dgroup*}
   \begin{dmath*} \cdb*{dgab.001} \end{dmath*}
   \begin{dmath*} \cdb*{dgab.002} \end{dmath*}
   % \begin{dmath*} \cdb*{dgab.003} \end{dmath*} % too big for pdfLaTeX
   % \begin{dmath*} \cdb*{dgab.004} \end{dmath*}
   % \begin{dmath*} \cdb*{dgab.005} \end{dmath*}
   % \begin{dmath*} \cdb*{dgab.006} \end{dmath*}
   % \begin{dmath*} \cdb*{dgab.007} \end{dmath*}
   % \begin{dmath*} \cdb*{dgab.008} \end{dmath*}
   % \begin{dmath*} \cdb*{dgab.009} \end{dmath*}
   \begin{dmath*} \cdb*{dgab.010} \end{dmath*}
\end{dgroup*}

\clearpage

\begin{cadabra}
   # Note:
   # Computing Gamma directly by (1/2) iab dgab and *then* truncating to lower order
   # is not optimal. We only want the leading oder terms (to 4th order in x). But the direct
   # calculation would compute *all* terms before the truncation. This does work but it
   # is slower than the following code.
   #
   # The better approach (as adopted in this code) is to extract all of the terms of iab
   # and dgab then construct the leading order terms of Gamma (to fifth order) term by term.

   def get_Rterm (obj,n):

   # I would like to assign different weights to \nabla_{a}, \nabla_{a b}, \nabla_{a b c} etc. but no matter
   # what I do it appears that Cadabra assigns the same weight to all of these regardless of the number of subscripts.
   # It seems that the weight is assigned to the symbol \nabla alone. So I'm forced to use the following substitution trick.

       Q_{a b c d}::Weight(label=numR,value=2).
       Q_{a b c d e}::Weight(label=numR,value=3).
       Q_{a b c d e f}::Weight(label=numR,value=4).
       Q_{a b c d e f g}::Weight(label=numR,value=5).

       tmp := @(obj).

       distribute (tmp)

       substitute (tmp, $\nabla_{e f g}{R_{a b c d}} -> Q_{a b c d e f g}$)
       substitute (tmp, $\nabla_{e f}{R_{a b c d}} -> Q_{a b c d e f}$)
       substitute (tmp, $\nabla_{e}{R_{a b c d}} -> Q_{a b c d e}$)
       substitute (tmp, $R_{a b c d} -> Q_{a b c d}$)

       foo := @(tmp).
       bah = Ex("numR = " + str(n))
       keep_weight (foo, bah)

       substitute (foo, $Q_{a b c d e f g} -> \nabla_{e f g}{R_{a b c d}}$)
       substitute (foo, $Q_{a b c d e f} -> \nabla_{e f}{R_{a b c d}}$)
       substitute (foo, $Q_{a b c d e} -> \nabla_{e}{R_{a b c d}}$)
       substitute (foo, $Q_{a b c d} -> R_{a b c d}$)

       return foo

   # terms of the curvature expansion of dg_{ab}

   dgab00 = get_Rterm (dgab,0)   # cdb(dgab00.105,dgab00)  # zero
   dgab01 = get_Rterm (dgab,1)   # cdb(dgab01.105,dgab01)  # zero
   dgab02 = get_Rterm (dgab,2)   # cdb(dgab02.105,dgab02)
   dgab03 = get_Rterm (dgab,3)   # cdb(dgab03.105,dgab03)
   dgab04 = get_Rterm (dgab,4)   # cdb(dgab04.105,dgab04)
   dgab05 = get_Rterm (dgab,5)   # cdb(dgab05.105,dgab05)

   # Convert free indices on iab from ^{a b} to ^{d c}
   # This ensures we can later build products like @(iab) @(dgab) knowing that the indices are correctly ordered.
   # Without this step we would be using free indices ^{a b} and _{a b c}. Thus the product @(iab) @(dgab) would
   # have just one free index _{c}. This is clearly wrong.

   tmp := @(iab) \delta_{a}^{d} \delta_{b}^{c}.

   distribute     (tmp)
   eliminate_kronecker (tmp)
   sort_product   (tmp)
   rename_dummies (tmp)
   canonicalise   (tmp)

   idc := @(tmp).

   # terms of the curvature expansion of g^{ab}

   idc00 = get_Rterm (idc,0)   # cdb(idc00.105,idc00)
   idc01 = get_Rterm (idc,1)   # cdb(idc01.105,idc01)  # zero
   idc02 = get_Rterm (idc,2)   # cdb(idc02.105,idc02)
   idc03 = get_Rterm (idc,3)   # cdb(idc03.105,idc03)
   idc04 = get_Rterm (idc,4)   # cdb(idc04.105,idc04)
   idc05 = get_Rterm (idc,5)   # cdb(idc05.105,idc05)

\end{cadabra}

\clearpage

\begin{dgroup*}
   \begin{dmath*} \cdb*{dgab00.105} \end{dmath*}
   \begin{dmath*} \cdb*{dgab01.105} \end{dmath*}
   \begin{dmath*} \cdb*{dgab02.105} \end{dmath*}
   \begin{dmath*} \cdb*{dgab03.105} \end{dmath*}
   \begin{dmath*} \cdb*{dgab04.105} \end{dmath*}
   \begin{dmath*} \cdb*{dgab05.105} \end{dmath*}
\end{dgroup*}

\clearpage

\begin{dgroup*}
   \begin{dmath*} \cdb*{idc00.105} \end{dmath*}
   \begin{dmath*} \cdb*{idc01.105} \end{dmath*}
   \begin{dmath*} \cdb*{idc02.105} \end{dmath*}
   \begin{dmath*} \cdb*{idc03.105} \end{dmath*}
   \begin{dmath*} \cdb*{idc04.105} \end{dmath*}
   \begin{dmath*} \cdb*{idc05.105} \end{dmath*}
\end{dgroup*}

\clearpage

\begin{cadabra}
   # idc  = g^{d c}
   # dgab = D_{a}{g_{c b}} + D_{b}{g_{a c}} - D_{c}{g_{a b}}

   # terms of the curvature expansion of \Gamma^{d}_{a b}

   # term0 := (1/2)  @(idc00) @(dgab00).
   # term1 := (1/2) (@(idc01) @(dgab00) + @(idc00) @(dgab01)).
   # term2 := (1/2) (@(idc02) @(dgab00) + @(idc01) @(dgab01) + @(idc00) @(dgab02)).
   # term3 := (1/2) (@(idc03) @(dgab00) + @(idc02) @(dgab01) + @(idc01) @(dgab02) + @(idc00) @(dgab03)).
   # term4 := (1/2) (@(idc04) @(dgab00) + @(idc03) @(dgab01) + @(idc02) @(dgab02) + @(idc01) @(dgab03) + @(idc00) @(dgab04)).
   # term5 := (1/2) (@(idc05) @(dgab00) + @(idc04) @(dgab01) + @(idc03) @(dgab02) + @(idc02) @(dgab03) + @(idc01) @(dgab04) + @(idc00) @(dgab05)).

   # simplidied version of the above after noting dgab00 = dgab01 = 0

   term0 := 0.
   term1 := 0.
   term2 := (1/2) (@(idc00) @(dgab02)).
   term3 := (1/2) (@(idc01) @(dgab02) + @(idc00) @(dgab03)).
   term4 := (1/2) (@(idc02) @(dgab02) + @(idc01) @(dgab03) + @(idc00) @(dgab04)).
   term5 := (1/2) (@(idc03) @(dgab02) + @(idc02) @(dgab03) + @(idc01) @(dgab04) + @(idc00) @(dgab05)).

   def tidy_terms (obj):
       substitute     (obj,$x^{a}->AA^{a}$,repeat=True)  # will force AA to the left of all terms
       distribute     (obj)
       sort_product   (obj)
       rename_dummies (obj)
       canonicalise   (obj)
       substitute     (obj,$AA^{a}->x^{a}$,repeat=True)  # replace AA with x
       factor_out     (obj,$x^{a?}$)

       return obj

   term0 = tidy_terms (term0)   # cdb(term0.201,term0)  # zero
   term1 = tidy_terms (term1)   # cdb(term1.201,term1)  # zero
   term2 = tidy_terms (term2)   # cdb(term2.201,term2)
   term3 = tidy_terms (term3)   # cdb(term3.201,term3)
   term4 = tidy_terms (term4)   # cdb(term4.201,term4)
   term5 = tidy_terms (term5)   # cdb(term5.201,term5)

   Gamma := @(term0) + @(term1) + @(term2) + @(term3) + @(term4) + @(term5). # cdb(Gamma.200,Gamma)

\end{cadabra}

\clearpage

\begin{dgroup*}
   \begin{dmath*} \cdb*{term0.201} \end{dmath*}
   \begin{dmath*} \cdb*{term1.201} \end{dmath*}
   \begin{dmath*} \cdb*{term2.201} \end{dmath*}
   \begin{dmath*} \cdb*{term3.201} \end{dmath*}
   \begin{dmath*} \cdb*{term4.201} \end{dmath*}
   \begin{dmath*} \cdb*{term5.201} \end{dmath*}
\end{dgroup*}

\clearpage

\begin{dgroup*}
   \begin{dmath*} \cdb*{Gamma.200} \end{dmath*}
\end{dgroup*}

\clearpage

\begin{cadabra}
   cdblib.create ('connection.json')

   cdblib.put ('Gamma',Gamma,'connection.json')

   cdblib.put ('GammaRterm0',term0,'connection.json')
   cdblib.put ('GammaRterm1',term1,'connection.json')
   cdblib.put ('GammaRterm2',term2,'connection.json')
   cdblib.put ('GammaRterm3',term3,'connection.json')
   cdblib.put ('GammaRterm4',term4,'connection.json')
   cdblib.put ('GammaRterm5',term5,'connection.json')

   checkpoint.append (term0)
   checkpoint.append (term1)
   checkpoint.append (term2)
   checkpoint.append (term3)
   checkpoint.append (term4)
   checkpoint.append (term5)

\end{cadabra}

% =================================================================================================
% the remaining code is just for pretty printing

\clearpage

\begin{cadabra}
   # note: keeping numbering as is (out of order) to ensure R appears before \nabla R etc.
   def product_sort (obj):
       substitute (obj,$ A^{a}                            -> A001^{a}               $)
       substitute (obj,$ x^{a}                            -> A002^{a}               $)
       substitute (obj,$ g^{a b}                          -> A003^{a b}             $)
       substitute (obj,$ \nabla_{e f g h}{R_{a b c d}}    -> A008_{a b c d e f g h} $)
       substitute (obj,$ \nabla_{e f g}{R_{a b c d}}      -> A007_{a b c d e f g}   $)
       substitute (obj,$ \nabla_{e f}{R_{a b c d}}        -> A006_{a b c d e f}     $)
       substitute (obj,$ \nabla_{e}{R_{a b c d}}          -> A005_{a b c d e}       $)
       substitute (obj,$ R_{a b c d}                      -> A004_{a b c d}         $)
       sort_product   (obj)
       rename_dummies (obj)
       substitute (obj,$ A001^{a}                  -> A^{a}                         $)
       substitute (obj,$ A002^{a}                  -> x^{a}                         $)
       substitute (obj,$ A003^{a b}                -> g^{a b}                       $)
       substitute (obj,$ A008_{a b c d e f g h}    -> \nabla_{e f g h}{R_{a b c d}} $)
       substitute (obj,$ A007_{a b c d e f g}      -> \nabla_{e f g}{R_{a b c d}}   $)
       substitute (obj,$ A006_{a b c d e f}        -> \nabla_{e f}{R_{a b c d}}     $)
       substitute (obj,$ A005_{a b c d e}          -> \nabla_{e}{R_{a b c d}}       $)
       substitute (obj,$ A004_{a b c d}            -> R_{a b c d}                   $)

       return obj

   def reformat (obj,scale):
      foo  = Ex(str(scale))
      bah := @(foo) @(obj).
      distribute     (bah)
      bah = product_sort (bah)
      rename_dummies (bah)
      canonicalise   (bah)
      factor_out     (bah,$A^{a?},x^{b?}$)
      ans := @(bah) / @(foo).
      return ans

   def rescale (obj,scale):
      foo  = Ex(str(scale))
      bah := @(foo) @(obj).
      distribute  (bah)
      factor_out  (bah,$A^{a?},x^{b?}$)
      return bah

   Rterm2 := @(term2) A^{a} A^{b}.
   Rterm3 := @(term3) A^{a} A^{b}.
   Rterm4 := @(term4) A^{a} A^{b}.
   Rterm5 := @(term5) A^{a} A^{b}.

   Rterm2 = reformat (Rterm2,  3)    # cdb(Rterm2.301,Rterm2)
   Rterm3 = reformat (Rterm3, 12)    # cdb(Rterm3.301,Rterm3)
   Rterm4 = reformat (Rterm4,360)    # cdb(Rterm4.301,Rterm4)
   Rterm5 = reformat (Rterm5,180)    # cdb(Rterm5.301,Rterm5)

   Gamma  := @(Rterm2) + @(Rterm3) + @(Rterm4) + @(Rterm5).  # cdb (Gamma.301,Gamma)
   Scaled := 360 @(Gamma).                                   # cdb (Scaled.301,Scaled)

   scaled2 = rescale (Rterm2,   3)   # cdb (scaled2.301,scaled2)
   scaled3 = rescale (Rterm3,  12)   # cdb (scaled3.301,scaled3)
   scaled4 = rescale (Rterm4, 360)   # cdb (scaled4.301,scaled4)
   scaled5 = rescale (Rterm5, 180)   # cdb (scaled5.301,scaled5)
\end{cadabra}

\clearpage

% =================================================================================================
\section*{The connection in Riemann normal coordinates}

\begin{dgroup*}
   \begin{dmath*} A^a A^b \Gamma^{d}_{a b} = \cdb{Gamma.301} \end{dmath*}
\end{dgroup*}

\begin{dgroup*}
   \begin{dmath*} 360 A^a A^b \Gamma^{d}_{a b} = \cdb{Scaled.301} \end{dmath*}
\end{dgroup*}

\clearpage

% =================================================================================================
\section*{Curvature expansion of the connection}

\begin{align*}
     A^a A^b \Gamma^{d}_{a b} =
     A^a A^b \nGamma{2}^{d}{}_{a b}
   + A^a A^b \nGamma{3}^{d}{}_{a b}
   + A^a A^b \nGamma{4}^{d}{}_{a b}
   + A^a A^b \nGamma{5}^{d}{}_{a b}+\BigO{\eps^6}
\end{align*}
\begin{dgroup*}
   \begin{dmath*}   3 A^a A^b \nGamma{2}^{d}_{a b} = \cdb{scaled2.301} \end{dmath*}
   \begin{dmath*}  12 A^a A^b \nGamma{3}^{d}_{a b} = \cdb{scaled3.301} \end{dmath*}
   \begin{dmath*} 360 A^a A^b \nGamma{4}^{d}_{a b} = \cdb{scaled4.301} \end{dmath*}
   \begin{dmath*} 180 A^a A^b \nGamma{5}^{d}_{a b} = \cdb{scaled5.301} \end{dmath*}
\end{dgroup*}

% =================================================================================================
% export checkpoints in json format

\bgroup
\CdbSetup{action=hide}
\begin{cadabra}
   for i in range( len(checkpoint) ):
      cdblib.put ('check{:03d}'.format(i),checkpoint[i],checkpoint_file)
\end{cadabra}
\egroup

\end{document}


\begin{dgroup*}
   \begin{dmath*} A^a A^b \Gamma^{d}_{a b} = \cdb{Gamma.301} \end{dmath*}
\end{dgroup*}

\begin{dgroup*}
   \begin{dmath*} 360 A^a A^b \Gamma^{d}_{a b} = \cdb{Scaled.301} \end{dmath*}
\end{dgroup*}

\clearpage

% =================================================================================================
\section*{Curvature expansion of the connection}
\begin{align*}
     A^a A^b \Gamma^{d}_{a b} =
     A^a A^b \nGamma{2}^{d}{}_{a b}
   + A^a A^b \nGamma{3}^{d}{}_{a b}
   + A^a A^b \nGamma{4}^{d}{}_{a b}
   + A^a A^b \nGamma{5}^{d}{}_{a b}+\BigO{\eps^6}
\end{align*}
\begin{dgroup*}
   \begin{dmath*}   3 A^a A^b \nGamma{2}^{d}_{a b} = \cdb{scaled2.301} \end{dmath*}
   \begin{dmath*}  12 A^a A^b \nGamma{3}^{d}_{a b} = \cdb{scaled3.301} \end{dmath*}
   \begin{dmath*} 360 A^a A^b \nGamma{4}^{d}_{a b} = \cdb{scaled4.301} \end{dmath*}
   \begin{dmath*} 180 A^a A^b \nGamma{5}^{d}_{a b} = \cdb{scaled5.301} \end{dmath*}
\end{dgroup*}

\clearpage

% =================================================================================================
\section*{Symmetrised partial derivatives of the connection}
\documentclass[12pt]{cdblatex}
\usepackage{fancyhdr}
\usepackage{footer}

\begin{document}

\section*{\jobname}

\CdbSetup{action=hide}

\begin{cadabra}
   import shared

   import cdblib

   term00A = cdblib.get ('check000','expected/dGamma.json')
   term01A = cdblib.get ('check001','expected/dGamma.json')
   term02A = cdblib.get ('check002','expected/dGamma.json')
   term03A = cdblib.get ('check003','expected/dGamma.json')
   term04A = cdblib.get ('check004','expected/dGamma.json')

   term00B = cdblib.get ('check000','output/dGamma.json')
   term01B = cdblib.get ('check001','output/dGamma.json')
   term02B = cdblib.get ('check002','output/dGamma.json')
   term03B = cdblib.get ('check003','output/dGamma.json')
   term04B = cdblib.get ('check004','output/dGamma.json')

   diff000 = shared.check (term00A,term00B)   # cdb (diff000,diff000)
   diff001 = shared.check (term01A,term01B)   # cdb (diff001,diff001)
   diff002 = shared.check (term02A,term02B)   # cdb (diff002,diff002)
   diff003 = shared.check (term03A,term03B)   # cdb (diff003,diff003)
   diff004 = shared.check (term04A,term04B)   # cdb (diff004,diff004)

\end{cadabra}

\begin{dgroup*}
   \Dmath*{ \cdb*{diff000} }
   \Dmath*{ \cdb*{diff001} }
   \Dmath*{ \cdb*{diff002} }
   \Dmath*{ \cdb*{diff003} }
   \Dmath*{ \cdb*{diff004} }
\end{dgroup*}

\end{document}


\begin{dgroup*}
   \begin{dmath*}   3 A^b A^c\Gamma^a{}_{d(b,c)} = \cdb{scaled1.002} \end{dmath*}
   \begin{dmath*}   6 A^b A^c A^e \Gamma^a{}_{d(b,ce)} = \cdb{scaled2.002} \end{dmath*}
   \begin{dmath*}  15 A^b A^c A^e A^f \Gamma^a{}_{d(b,cef)} = \cdb{scaled3.002} \end{dmath*}
   \begin{dmath*}   9 A^b A^c A^e A^f A^g \Gamma^a{}_{d(b,cefg)} = \cdb{scaled4.002} \end{dmath*}
   \begin{dmath*} 252 A^b A^c A^e A^f A^g A^h\Gamma^a{}_{d(b,cefgh)} = \cdb{scaled5.002} \end{dmath*}
\end{dgroup*}

\clearpage

% =================================================================================================
\section*{Symmetrised partial derivatives of $R^a{}_{bcd}$}
\documentclass[12pt]{cdblatex}

\begin{document}

\section*{\jobname}

\CdbSetup{action=hide}

\begin{cadabra}
   import shared

   import cdblib

   term00A = cdblib.get ('check000','expected/dRabcd.json')
   term01A = cdblib.get ('check001','expected/dRabcd.json')
   term02A = cdblib.get ('check002','expected/dRabcd.json')
   term03A = cdblib.get ('check003','expected/dRabcd.json')
   term04A = cdblib.get ('check004','expected/dRabcd.json')

   term00B = cdblib.get ('check000','output/dRabcd.json')
   term01B = cdblib.get ('check001','output/dRabcd.json')
   term02B = cdblib.get ('check002','output/dRabcd.json')
   term03B = cdblib.get ('check003','output/dRabcd.json')
   term04B = cdblib.get ('check004','output/dRabcd.json')

   diff000 = shared.check (term00A,term00B)   # cdb (diff000,diff000)
   diff001 = shared.check (term01A,term01B)   # cdb (diff001,diff001)
   diff002 = shared.check (term02A,term02B)   # cdb (diff002,diff002)
   diff003 = shared.check (term03A,term03B)   # cdb (diff003,diff003)
   diff004 = shared.check (term04A,term04B)   # cdb (diff004,diff004)

\end{cadabra}

\begin{dgroup*}
   \Dmath*{ \cdb*{diff000} }
   \Dmath*{ \cdb*{diff001} }
   \Dmath*{ \cdb*{diff002} }
   \Dmath*{ \cdb*{diff003} }
   \Dmath*{ \cdb*{diff004} }
\end{dgroup*}

\end{document}


\begin{dgroup*}
   \begin{dmath*}    A^c A^d A^e R^a{}_{cdb,e} = \cdb{scaled1.601} \end{dmath*}
   \begin{dmath*}    A^c A^d A^e A^{f} R^a{}_{cdb,ef} = \cdb{scaled2.601} \end{dmath*}
   \begin{dmath*} -2 A^c A^d A^e A^{f} A^{g} R^a{}_{cdb,efg} = \cdb{scaled3.601} \end{dmath*}
   \begin{dmath*} -5 A^c A^d A^e A^{f} A^{g} A^{h} R^a{}_{cdb,efgh} = \cdb{scaled4.601} \end{dmath*}
   \begin{dmath*} -3 A^c A^d A^e A^{f} A^{g} A^{h} A^{i}R^a{}_{cdb,efghi} = \cdb{scaled5.601} \end{dmath*}
\end{dgroup*}

\clearpage

% =================================================================================================
\section*{The generalised connection in RNC}
\def\Date{19 Jan 2024}
% \def\FileID{file:}

\documentclass[12pt]{cdblatex}

\begin{document}

\section*{\jobname}

\CdbSetup{action=hide}

\begin{cadabra}
   import shared

   import cdblib

   term00A = cdblib.get ('check000','expected/genGamma.json')
   term01A = cdblib.get ('check001','expected/genGamma.json')
   term02A = cdblib.get ('check002','expected/genGamma.json')
   term03A = cdblib.get ('check003','expected/genGamma.json')
   term04A = cdblib.get ('check004','expected/genGamma.json')
   term05A = cdblib.get ('check005','expected/genGamma.json')
   term06A = cdblib.get ('check005','expected/genGamma.json')
   term07A = cdblib.get ('check005','expected/genGamma.json')
   term08A = cdblib.get ('check005','expected/genGamma.json')
   term09A = cdblib.get ('check005','expected/genGamma.json')

   term00B = cdblib.get ('check000','output/genGamma.json')
   term01B = cdblib.get ('check001','output/genGamma.json')
   term02B = cdblib.get ('check002','output/genGamma.json')
   term03B = cdblib.get ('check003','output/genGamma.json')
   term04B = cdblib.get ('check004','output/genGamma.json')
   term05B = cdblib.get ('check005','output/genGamma.json')
   term06B = cdblib.get ('check005','output/genGamma.json')
   term07B = cdblib.get ('check005','output/genGamma.json')
   term08B = cdblib.get ('check005','output/genGamma.json')
   term09B = cdblib.get ('check005','output/genGamma.json')

   diff000 = shared.check (term00A,term00B)   # cdb (diff000,diff000)
   diff001 = shared.check (term01A,term01B)   # cdb (diff001,diff001)
   diff002 = shared.check (term02A,term02B)   # cdb (diff002,diff002)
   diff003 = shared.check (term03A,term03B)   # cdb (diff003,diff003)
   diff004 = shared.check (term04A,term04B)   # cdb (diff004,diff004)
   diff005 = shared.check (term05A,term05B)   # cdb (diff005,diff005)
   diff006 = shared.check (term06A,term06B)   # cdb (diff006,diff006)
   diff007 = shared.check (term07A,term07B)   # cdb (diff007,diff007)
   diff008 = shared.check (term08A,term08B)   # cdb (diff008,diff008)
   diff009 = shared.check (term09A,term09B)   # cdb (diff009,diff009)

\end{cadabra}

\begin{dgroup*}
   \Dmath*{ \cdb*{diff000} }
   \Dmath*{ \cdb*{diff001} }
   \Dmath*{ \cdb*{diff002} }
   \Dmath*{ \cdb*{diff003} }
   \Dmath*{ \cdb*{diff004} }
   \Dmath*{ \cdb*{diff005} }
   \Dmath*{ \cdb*{diff006} }
   \Dmath*{ \cdb*{diff007} }
   \Dmath*{ \cdb*{diff008} }
   \Dmath*{ \cdb*{diff009} }
\end{dgroup*}

\end{document}


\begin{dgroup*}
   \begin{dmath*} A^b A^c \Gamma^{a}_{b c} = \cdb{genGamma0.000} \end{dmath*}
   \begin{dmath*} A^b A^c A^d \Gamma^{a}_{b c d} = \cdb{genGamma1.000} \end{dmath*}
   \begin{dmath*} A^b A^c A^d A^e \Gamma^{a}_{b c d e} = \cdb{genGamma2.000} \end{dmath*}
   \begin{dmath*} A^b A^c A^d A^e A^f \Gamma^{a}_{b c d e f} = \cdb{genGamma3.000} \end{dmath*}
\end{dgroup*}

\clearpage

% =================================================================================================
\section*{The generalised connection in RNC}

This is the same as the previous page but with a small change in the format to avoid fractions.

\begin{dgroup*}
   \begin{dmath*} 360 A^b A^c \Gamma^{a}_{b c} = \cdb{scaledGamma0.001} \end{dmath*}
   \begin{dmath*} 360 A^b A^c A^d \Gamma^{a}_{b c d} = \cdb{scaledGamma1.001} \end{dmath*}
   \begin{dmath*}  90 A^b A^c A^d A^e \Gamma^{a}_{b c d e} = \cdb{scaledGamma2.001} \end{dmath*}
   \begin{dmath*}   3 A^b A^c A^d A^e A^f \Gamma^{a}_{b c d e f} = \cdb{scaledGamma3.001} \end{dmath*}
\end{dgroup*}

\clearpage

% =================================================================================================
\section*{Convert from generic (x) to local RNC coords (y)}
\def\Date{19 Jan 2024}
% \def\FileID{file:}

\documentclass[12pt]{cdblatex}

\begin{document}

\section*{\jobname}

\CdbSetup{action=hide}

\begin{cadabra}
   import shared

   import cdblib

   term00A = cdblib.get ('check000','expected/gen2rnc.json')
   term01A = cdblib.get ('check001','expected/gen2rnc.json')
   term02A = cdblib.get ('check002','expected/gen2rnc.json')
   term03A = cdblib.get ('check003','expected/gen2rnc.json')
   term04A = cdblib.get ('check004','expected/gen2rnc.json')

   term00B = cdblib.get ('check000','output/gen2rnc.json')
   term01B = cdblib.get ('check001','output/gen2rnc.json')
   term02B = cdblib.get ('check002','output/gen2rnc.json')
   term03B = cdblib.get ('check003','output/gen2rnc.json')
   term04B = cdblib.get ('check004','output/gen2rnc.json')

   diff000 = shared.check (term00A,term00B)   # cdb (diff000,diff000)
   diff001 = shared.check (term01A,term01B)   # cdb (diff001,diff001)
   diff002 = shared.check (term02A,term02B)   # cdb (diff002,diff002)
   diff003 = shared.check (term03A,term03B)   # cdb (diff003,diff003)
   diff004 = shared.check (term04A,term04B)   # cdb (diff004,diff004)

\end{cadabra}

\begin{dgroup*}
   \Dmath*{ \cdb*{diff000} }
   \Dmath*{ \cdb*{diff001} }
   \Dmath*{ \cdb*{diff002} }
   \Dmath*{ \cdb*{diff003} }
   \Dmath*{ \cdb*{diff004} }
\end{dgroup*}

\end{document}


\begin{align*}
   y^a = \ny{0}^{a} + \ny{1}^{a} + \ny{2}^{a} + \ny{3}^{a} + \ny{4}^{a}
\end{align*}

\begin{dgroup*}
   \begin{dmath*}     \ny{0}^{a} = \cdb{scaled1.002} \end{dmath*}
   \begin{dmath*}   2 \ny{1}^{a} = \cdb{scaled2.002} \end{dmath*}
   \begin{dmath*}   6 \ny{2}^{a} = \cdb{scaled3.002} \end{dmath*}
   \begin{dmath*}  24 \ny{3}^{a} = \cdb{scaled4.002} \end{dmath*}
   \begin{dmath*} 360 \ny{4}^{a} = \cdb{scaled5.002} \end{dmath*}
\end{dgroup*}

\clearpage

% =================================================================================================
\section*{The geodesic ivp}
\documentclass[12pt]{cdblatex}

\lstset{gobble=2}

\begin{document}

% =================================================================================================
% create checkpoint file

\bgroup
\CdbSetup{action=hide}
\begin{cadabra}
   import cdblib
   checkpoint_file = 'tests/semantic/output/geodesic-ivp.json'
   cdblib.create (checkpoint_file)
   checkpoint = []
\end{cadabra}
\egroup

% =================================================================================================
\section*{Geodesic IVP}

Our game here is to find the solution of
\begin{align*}
   0 = \frac{d^2 x^{a}}{ds^2} + \Gamma^{a}_{bc}(x) \frac{dx^b}{ds} \frac{dx^c}{ds}
\end{align*}
subject to the initial conditions $x^{a}(s) = x^a$ and $dx^a(s)/ds={\Dot x}^{a}$ at $s=0$.

% =================================================================================================
\section*{Algorithm}

By successive differentiation of the above equation we can compute
\begin{align*}
   \frac{d^n x^{a}}{ds^n} = -\Gamma^{a}_{\udn}\frac{dx^{\udn}}{ds}
\end{align*}
at $s=0$ for $n=2,3,4,\dotsc$. The $\Gamma^{a}_{\udn}$ are the \emph{generalised connections}.

We can then construct the Taylor series solution for $x^{a}(s)$
\begin{align*}
   x^a(s) = x^a + s {\Dot x}^a - \sum_{k=2}^\infty\>\frac{s^{k}}{k!} \Gamma^{a}_{\udk}{\Dot x}^{\udk}
\end{align*}

\clearpage

\begin{cadabra}
   {a,b,c,d,e,f,g,h,i,j,k,l,m,n,o,p,q,r,s,t,u,v,w#}::Indices(position=independent).

   \nabla{#}::Derivative.

   import cdblib

   # change signs to account for - sign in front of the sum for x^a(s), see above preamble

   def flip_sign (obj):
       return Ex(0) - obj

   sterm21 = flip_sign (cdblib.get ('genGamma01','genGamma.json'))
   sterm22 = flip_sign (cdblib.get ('genGamma02','genGamma.json'))
   sterm23 = flip_sign (cdblib.get ('genGamma03','genGamma.json'))
   sterm24 = flip_sign (cdblib.get ('genGamma04','genGamma.json'))

   sterm31 = flip_sign (cdblib.get ('genGamma11','genGamma.json'))
   sterm32 = flip_sign (cdblib.get ('genGamma12','genGamma.json'))
   sterm33 = flip_sign (cdblib.get ('genGamma13','genGamma.json'))

   sterm41 = flip_sign (cdblib.get ('genGamma21','genGamma.json'))
   sterm42 = flip_sign (cdblib.get ('genGamma22','genGamma.json'))

   sterm51 = flip_sign (cdblib.get ('genGamma31','genGamma.json'))

   sterm2 := @(sterm21) + @(sterm22) + @(sterm23) + @(sterm24).  # cdb (sterm2.000,sterm2)
   sterm3 := @(sterm31) + @(sterm32) + @(sterm33).               # cdb (sterm3.000,sterm3)
   sterm4 := @(sterm41) + @(sterm42).                            # cdb (sterm4.000,sterm4)
   sterm5 := @(sterm51).                                         # cdb (sterm5.000,sterm5)

   factor_out (sterm2,$A^{a?}$)                                  # cdb (sterm2.001,sterm2)
   factor_out (sterm3,$A^{a?}$)                                  # cdb (sterm3.001,sterm3)
   factor_out (sterm4,$A^{a?}$)                                  # cdb (sterm4.001,sterm4)
   factor_out (sterm5,$A^{a?}$)                                  # cdb (sterm5.001,sterm5)

   sterm2 := 360 @(sterm2).
   sterm3 := 360 @(sterm3).
   sterm4 :=  90 @(sterm4).
   sterm5 :=   3 @(sterm5).

   substitute (sterm2,$A^{a}->1$)                                # cdb (sterm2.002,sterm2)
   substitute (sterm3,$A^{a}->1$)                                # cdb (sterm3.002,sterm3)
   substitute (sterm4,$A^{a}->1$)                                # cdb (sterm4.002,sterm4)
   substitute (sterm5,$A^{a}->1$)                                # cdb (sterm5.002,sterm5)

\end{cadabra}

% =================================================================================================
% the remaining code is just for pretty printing

\clearpage

% =================================================================================================
\section*{The geodesic ivp}

\begin{align*}
   x^{a}(s) = x^{a}
            + s {\dot{x}^a}
            + \frac{s^2}{2!} {\dot{x}^b} {\dot{x}^c} A^{a}_{bc}
            + \frac{s^3}{3!} {\dot{x}^b} {\dot{x}^c} {\dot{x}^d} A^{a}_{bcd}
            + \frac{s^4}{4!} {\dot{x}^b} {\dot{x}^c} {\dot{x}^d} {\dot{x}^e} A^{a}_{bcde}
            + \frac{s^5}{5!} {\dot{x}^b} {\dot{x}^c} {\dot{x}^d} {\dot{x}^e} {\dot{x}^f} A^{a}_{bcdef}
            + \dotsb
\end{align*}
\begin{dgroup*}
   \begin{dmath*} 360 A^{a}_{bc} = \cdb{sterm2.002} \end{dmath*}
   \begin{dmath*} 360 A^{a}_{bcd} = \cdb{sterm3.002} \end{dmath*}
   \begin{dmath*}  90 A^{a}_{bcde} = \cdb{sterm4.002} \end{dmath*}
   \begin{dmath*}   3 A^{a}_{bcdef} = \cdb{sterm5.002} \end{dmath*}
\end{dgroup*}

\clearpage

% =================================================================================================
% export selected objects, these will later be imported into a library
% these are the objects that will appear in the paper

\begin{cadabra}
   sterm2short := @(sterm21) + @(sterm22).             # cdb (sterm2.short.001,sterm2short)
   sterm3short := @(sterm31).                          # cdb (sterm3.short.001,sterm3short)
   sterm2shortscaled := 12 @(sterm2short).             # cdb (sterm2.short.scaled.002,sterm2shortscaled)
   sterm3shortscaled :=  2 @(sterm3short).             # cdb (sterm3.short.scaled.002,sterm3shortscaled)

   substitute (sterm2shortscaled,$A^{a}->1$)           # cdb (sterm2.short.scaled.003,sterm2shortscaled)
   substitute (sterm3shortscaled,$A^{a}->1$)           # cdb (sterm3.short.scaled.003,sterm3shortscaled)

   cdblib.create ('geodesic-ivp.export')

   # 4th order ivp terms scaled
   cdblib.put ('ivp42',sterm2shortscaled,'geodesic-ivp.export')
   cdblib.put ('ivp43',sterm3shortscaled,'geodesic-ivp.export')

   # 6th order ivp terms scaled
   cdblib.put ('ivp62',sterm2,'geodesic-ivp.export')
   cdblib.put ('ivp63',sterm3,'geodesic-ivp.export')
   cdblib.put ('ivp64',sterm4,'geodesic-ivp.export')
   cdblib.put ('ivp65',sterm5,'geodesic-ivp.export')

   checkpoint.append (sterm2shortscaled)
   checkpoint.append (sterm3shortscaled)

   checkpoint.append (sterm2)
   checkpoint.append (sterm3)
   checkpoint.append (sterm4)
   checkpoint.append (sterm5)
\end{cadabra}

% just to check that we are exporting the correct 4th order terms

\begin{dgroup*}
   \begin{dmath*} \cdb*{sterm2.short.001} \end{dmath*}
   \begin{dmath*} \cdb*{sterm3.short.001} \end{dmath*}
   \begin{dmath*} \cdb*{sterm2.short.scaled.002} \end{dmath*}
   \begin{dmath*} \cdb*{sterm3.short.scaled.002} \end{dmath*}
   \begin{dmath*} \cdb*{sterm2.short.scaled.003} \end{dmath*}
   \begin{dmath*} \cdb*{sterm3.short.scaled.003} \end{dmath*}
\end{dgroup*}

% =================================================================================================
% export checkpoints in json format

\bgroup
\CdbSetup{action=hide}
\begin{cadabra}
   for i in range( len(checkpoint) ):
      cdblib.put ('check{:03d}'.format(i),checkpoint[i],checkpoint_file)
\end{cadabra}
\egroup

\end{document}


\begin{align*}
   x^{d}(s) = x^{d}
            + s {\dot{x}^d}
            + \frac{s^2}{2!} {\dot{x}^a} {\dot{x}^b} A^{d}_{ab}
            + \frac{s^3}{3!} {\dot{x}^a} {\dot{x}^b} {\dot{x}^c} A^{d}_{abc}
            + \frac{s^4}{4!} {\dot{x}^a} {\dot{x}^b} {\dot{x}^c} {\dot{x}^d} A^{d}_{abce}
            + \frac{s^5}{5!} {\dot{x}^a} {\dot{x}^b} {\dot{x}^c} {\dot{x}^d} {\dot{x}^e} A^{d}_{abcef}
            + \dotsb
\end{align*}
\begin{dgroup*}
   \begin{dmath*} 360 A^{d}_{ab} = \cdb{sterm2.002} \end{dmath*}
   \begin{dmath*} 360 A^{d}_{abc} = \cdb{sterm3.002} \end{dmath*}
   \begin{dmath*}  90 A^{d}_{abce} = \cdb{sterm4.002} \end{dmath*}
   \begin{dmath*}   3 A^{d}_{abcef} = \cdb{sterm5.002} \end{dmath*}
\end{dgroup*}

\clearpage

% =================================================================================================
\section*{Geodesic boundary value problem to terms linear in $R$}
\documentclass[12pt]{cdblatex}

\begin{document}

% =================================================================================================
% create checkpoint file

\bgroup
\CdbSetup{action=hide}
\begin{cadabra}
   import cdblib
   checkpoint_file = 'tests/semantic/output/geodesic-bvp.json'
   cdblib.create (checkpoint_file)
   checkpoint = []
\end{cadabra}
\egroup

% =================================================================================================
\section*{Geodesic BVP}

Consider a geodesic that connects two points $P_i$ and $P_j$ with RNC coordinates
$x^a_i$ and $x^a_j$. Our aim is to construct a solution $x^a(s)$ of the geodesic
equation such that $x^a(0)= x^a_i$ and $x^a(1)=x^a_j$.

We will do this in two stages. First we will solve
\begin{align}
   \label{eq:twoptBVP}
   x^a_j = x^a_i + y^a - \sum_{k=2}^\infty\>\frac{1}{k!}\>\Gamma^{a}_{\ubk}y^{.\ubk}
\end{align}
for $y^a$ as an explicit polynomial in $x^a_i$ and $x^a_j$. The functions $\Gamma^{a}_{\ubk}$ are
the generalised connections for the RNC frame evaluated at $x^a=x^a_i$.

In the second stage, we will substitute our expression for $y^a$ into
\begin{align}
   \label{eq:solBVP}
   x^a(s) = x^a_i + s y^a - \sum_{k=2}^\infty\>\frac{1}{k!}\>\Gamma^{a}_{\ubk}y^{.\ubk} s^k
\end{align}
to obtain the desired solution to the two point boundary value problem.

% =================================================================================================
\section*{Stage 1: The fixed point iteration scheme}

First we rewrite the main equation \eqref{eq:twoptBVP} in the suggestive form
\begin{align*}
   y^a = \Dx^a + \sum_{k=2}^\infty\>\frac{1}{k!}\>\Gamma^{a}_{\ubk}y^{\ubk}
\end{align*}
where $\Dx^a = x^a_j-x^a_i$. Our approximate solution for $y^a$ will be taken to be the partial
sums for the infinite series. Thus we will solve
\begin{align*}
   \ny{n}^a = \Dx^a + \sum_{k=2}^n\>\frac{1}{k!}\>\Gamma^{a}_{\ubk}\ny{n}^{\ubk}
\end{align*}
for $\ny{n}^a$. Note that in the last term of the sum, the $\Gamma^{a}_{\ubn}$ will contain
curvature terms of order $\BigO{\eps^n}$. Thus in truncating the series at this point we will
loose contributions to the curvature terms of order $\BigO{\eps^{n+1}}$ and higher. So to be
consistent we must truncate all terms of the partial sum to order $\BigO{\eps^n}$ (i.e., exclude
any contributions from terms $\BigO{\eps^{n+1}}$ and higher, these are the terms that would
couple with the terms that we excluded when truncating the original infinite series). Let
$\nT{k}$ be the operator that truncates its argument to contain terms no higher than
$\BigO{\eps^n}$. Then we have the following modified version of the equation for $\ny{n}^a$
\begin{align*}
   \ny{n}^a = \Dx^a
            + \sum_{k=2}^n\>\frac{1}{k!}\>\nT{k}\left(\Gamma^{a}_{\ubk}\ny{n}^{\ubk}\right)
\end{align*}
Finally we note that since $\Gamma^{a}_{\ubk} = \BigO{\eps^k}$, we can use lower order estimates
for the $\ny{k}^a$ in the right hand side of the sum. This allows us to compute $\ny{n}^a$ by
successive approximations such as
\begin{align*}
   \ny{0}^a &= \Dx^a\\
   \ny{2}^a &= \ny{0}^a + \frac{1}{2!}\nT{2}\left(\Gamma^a_{bc}\>\ny{0}^b \ny{0}^c\right)\\[5pt]
   \ny{3}^a &= \ny{0}^a + \frac{1}{2!}\nT{3}\left(\Gamma^a_{bc}\>\ny{2}^b \ny{2}^c\right)
                        + \frac{1}{3!}\nT{3}\left(\Gamma^a_{bcd}\>\ny{0}^b \ny{0}^c \ny{0}^d\right)\\[5pt]
   \ny{4}^a &= \ny{0}^a + \frac{1}{2!}\nT{4}\left(\Gamma^a_{bc}\>\ny{3}^b \ny{3}^c\right)
                        + \frac{1}{3!}\nT{4}\left(\Gamma^a_{bcd}\>\ny{2}^b \ny{2}^c \ny{2}^d\right)
                        + \frac{1}{4!}\nT{4}\left(\Gamma^a_{bcde}\>\ny{0}^b \ny{0}^c \ny{0}^d \ny{0}^e\right)\\[5pt]
   \ny{5}^a &= \ny{0}^a + \frac{1}{2!}\nT{5}\left(\Gamma^a_{bc}\>\ny{4}^b \ny{4}^c\right)
                        + \frac{1}{3!}\nT{5}\left(\Gamma^a_{bcd}\>\ny{3}^b \ny{3}^c \ny{3}^d\right)
                        + \frac{1}{4!}\nT{5}\left(\Gamma^a_{bcde}\>\ny{2}^b \ny{2}^c \ny{2}^d \ny{2}^e\right)
                        + \frac{1}{5!}\nT{5}\left(\Gamma^a_{bcdef}\>\ny{0}^b \ny{0}^c \ny{0}^d \ny{0}^e \ny{0}^f\right)
\end{align*}
and so on. Note that there are no $\ny{1}^a$ terms.

% =================================================================================================
\section*{Stage 2: Introduce the generalised connections}

This is the final stage -- it introduces the generalised connecstion after the
completion of the fixed point scheme.

All that needs be done is to substitute our expression for $y^a$ into \eqref{eq:solBVP}
% \begin{align}
%    \label{eq:solBVP}
%    x^a(s) = x^a_i + s y^a - \sum_{k=2}^\infty\>\frac{1}{k!}\>\Gamma^{a}{}_{\ubk} y^{.\ubk} s^k
% \end{align}
to obtain the desired solution to the two point boundary value problem.

The generalised connections $\Gamma^{a}{}_{\ubk}$ are taken from the results of the
{\tts genGamma} code.

\clearpage

% =================================================================================================
\section*{Stage 1: The fixed point iteration scheme}

\begin{cadabra}
   import time

   {a,b,c,d,e,f,g,h,i,j,k,l,m,n,o,p,q,r,s,t,u,v,w#}::Indices(position=independent).

   \nabla{#}::Derivative.

   g_{a b}::Metric.
   g^{a b}::InverseMetric.

   R_{a b c d}::RiemannTensor.
   R_{a b c d}::Depends(\nabla{#}).

   {Gam22^{a}_{b c},Gam23^{a}_{b c},Gam24^{a}_{b c},Gam25^{a}_{b c}}::TableauSymmetry(shape={2}, indices={1,2}).
   {Gam33^{a}_{b c d},Gam34^{a}_{b c d},Gam35^{a}_{b c d}}::TableauSymmetry(shape={3}, indices={1,2,3}).
   {Gam44^{a}_{b c d e},Gam45^{a}_{b c d e}}::TableauSymmetry(shape={4}, indices={1,2,3,4}).
   {Gam55^{a}_{b c d e f}}::TableauSymmetry(shape={5}, indices={1,2,3,4,5}).

   {Gam22^{a}_{b c}}::Weight(label=eps,value=2).
   {Gam23^{a}_{b c},Gam33^{a}_{b c d}}::Weight(label=eps,value=3).
   {Gam24^{a}_{b c},Gam34^{a}_{b c d},Gam44^{a}_{b c d e}}::Weight(label=eps,value=4).
   {Gam25^{a}_{b c},Gam35^{a}_{b c d},Gam45^{a}_{b c d e},Gam55^{a}_{b c d e f}}::Weight(label=eps,value=5).

   {Dx^{a}}::Weight(label=eps,value=0).

   {y00^{a},y20^{a},y30^{a},y40^{a},y50^{a}}::Weight(label=eps,value=0).
   {y22^{a},y32^{a},y42^{a},y52^{a}}::Weight(label=eps,value=2).
   {y33^{a},y43^{a},y53^{a}}::Weight(label=eps,value=3).
   {y44^{a},y54^{a}}::Weight(label=eps,value=4).
   {y55^{a}}::Weight(label=eps,value=5).

   # Dx{#}::LaTeXForm{"{\Dx}"}.  # LCB: currently causes a bug, it kills ::KeepWeight for Dx

   # note: keeping numbering as is (out of order) to ensure R appears before \nabla R etc.
   def product_sort (obj):
       substitute (obj,$ x^{a}                            -> A001^{a}               $)
       substitute (obj,$ Dx^{a}                           -> A002^{a}               $)
       substitute (obj,$ g^{a b}                          -> A003^{a b}             $)
       substitute (obj,$ \nabla_{e f g h}{R_{a b c d}}    -> A008_{a b c d e f g h} $)
       substitute (obj,$ \nabla_{e f g}{R_{a b c d}}      -> A007_{a b c d e f g}   $)
       substitute (obj,$ \nabla_{e f}{R_{a b c d}}        -> A006_{a b c d e f}     $)
       substitute (obj,$ \nabla_{e}{R_{a b c d}}          -> A005_{a b c d e}       $)
       substitute (obj,$ R_{a b c d}                      -> A004_{a b c d}         $)
       sort_product   (obj)
       rename_dummies (obj)
       substitute (obj,$ A001^{a}                  -> x^{a}                         $)
       substitute (obj,$ A002^{a}                  -> Dx^{a}                        $)
       substitute (obj,$ A003^{a b}                -> g^{a b}                       $)
       substitute (obj,$ A004_{a b c d}            -> R_{a b c d}                   $)
       substitute (obj,$ A005_{a b c d e}          -> \nabla_{e}{R_{a b c d}}       $)
       substitute (obj,$ A006_{a b c d e f}        -> \nabla_{e f}{R_{a b c d}}     $)
       substitute (obj,$ A007_{a b c d e f g}      -> \nabla_{e f g}{R_{a b c d}}   $)
       substitute (obj,$ A008_{a b c d e f g h}    -> \nabla_{e f g h}{R_{a b c d}} $)

       return obj

   def get_term (obj,n):

       tmp := @(obj).
       foo = Ex("eps = " + str(n))
       distribute  (tmp)
       keep_weight (tmp, foo)

       return tmp

   def truncate (obj,n):

       ans = Ex(0)

       for i in range (0,n+1):
          foo := @(obj).
          bah = Ex("eps = " + str(i))
          distribute  (foo)
          keep_weight (foo, bah)
          ans = ans + foo

       return ans

   def substitute_eps (obj):
       substitute     (obj,epsy0)
       substitute     (obj,epsy2)
       substitute     (obj,epsy3)
       substitute     (obj,epsy4)
       substitute     (obj,epsy5)
       substitute     (obj,epsGam2)
       substitute     (obj,epsGam3)
       substitute     (obj,epsGam4)
       substitute     (obj,epsGam5)
       distribute     (obj)
       obj = truncate     (obj,5)
       obj = product_sort (obj)
       rename_dummies (obj)
       canonicalise   (obj)

       return obj

   beg_stage_1 = time.time()

   # yn = y expanded to terms upto and including O(eps^n)

   y0 := Dx^{a}.
   y2 := Dx^{a} +   (1/2) Gam^{a}_{b c} y0^{b} y0^{c}.
   y3 := Dx^{a} +   (1/2) Gam^{a}_{b c} y2^{b} y2^{c}
                +   (1/6) Gam^{a}_{b c d} y0^{b} y0^{c} y0^{d}.
   y4 := Dx^{a} +   (1/2) Gam^{a}_{b c} y3^{b} y3^{c}
                +   (1/6) Gam^{a}_{b c d} y2^{b} y2^{c} y2^{d}
                +  (1/24) Gam^{a}_{b c d e} y0^{b} y0^{c} y0^{d} y0^{e}.
   y5 := Dx^{a} +   (1/2) Gam^{a}_{b c} y4^{b} y4^{c}
                +   (1/6) Gam^{a}_{b c d} y3^{b} y3^{c} y3^{d}
                +  (1/24) Gam^{a}_{b c d e} y2^{b} y2^{c} y2^{d} y2^{e}
                + (1/120) Gam^{a}_{b c d e f} y0^{b} y0^{c} y0^{d} y0^{e} y0^{f}.

   # epsyN = y expanded to terms upto and including O(eps^N)
   # yPQ = O(eps^Q) term of epsyP

   # expand to O(eps^5)

   epsy0 := y0^{a} -> y00^{a}.
   epsy2 := y2^{a} -> y20^{a}+y22^{a}.
   epsy3 := y3^{a} -> y30^{a}+y32^{a}+y33^{a}.
   epsy4 := y4^{a} -> y40^{a}+y42^{a}+y43^{a}+y44^{a}.
   epsy5 := y5^{a} -> y50^{a}+y52^{a}+y53^{a}+y54^{a}+y55^{a}.

   # epsGamN = gen. gamma with N lower indices (epsGam2 = the connection)
   # epsGamPQ = O(eps^Q) term of epsGamP

   epsGam2 := Gam^{a}_{b c} -> Gam22^{a}_{b c}+Gam23^{a}_{b c}+Gam24^{a}_{b c}+Gam25^{a}_{b c}.
   epsGam3 := Gam^{a}_{b c d} -> Gam33^{a}_{b c d}+Gam34^{a}_{b c d}+Gam35^{a}_{b c d}.
   epsGam4 := Gam^{a}_{b c d e} -> Gam44^{a}_{b c d e}+Gam45^{a}_{b c d e}.
   epsGam5 := Gam^{a}_{b c d e f} -> Gam55^{a}_{b c d e f}.

   y0 = substitute_eps (y0)   # cdb (y0.001,y0)
   y2 = substitute_eps (y2)   # cdb (y2.001,y2)
   y3 = substitute_eps (y3)   # cdb (y3.001,y3)
   y4 = substitute_eps (y4)   # cdb (y4.001,y4)
   y5 = substitute_eps (y5)   # cdb (y5.001,y5)

   y0 = truncate (y0,1)       # cdb (y0.002,y0)
   y2 = truncate (y2,2)       # cdb (y2.002,y2)
   y3 = truncate (y3,3)       # cdb (y3.002,y3)
   y4 = truncate (y4,4)       # cdb (y4.002,y4)
   y5 = truncate (y5,5)       # cdb (y5.002,y5)

   defy0 := y0^{a} -> @(y0).
   defy2 := y2^{a} -> @(y2).
   defy3 := y3^{a} -> @(y3).
   defy4 := y4^{a} -> @(y4).
   defy5 := y5^{a} -> @(y5).

   # -----------------------------------
   def tidy (obj):
       obj = product_sort (obj)
       rename_dummies     (obj)
       canonicalise       (obj)
       return obj

   # -----------------------------------
   # y0

   y00 := @(y0).           # cdb (y00.101,y00)

   defy00 := y00^{a} -> @(y00).

   # -----------------------------------
   # y2

   substitute (y2,defy00)

   distribute (y2)

   y20 = get_term (y2,0)   # cdb (y20.101,y20)
   y22 = get_term (y2,2)   # cdb (y22.101,y22)

   y20 = tidy (y20)        # cdb (y20.201,y20)
   y22 = tidy (y22)        # cdb (y22.201,y22)

   defy20 := y20^{a} -> @(y20).
   defy22 := y22^{a} -> @(y22).

   # -----------------------------------
   # y3

   substitute (y3,defy00)

   substitute (y3,defy20)
   substitute (y3,defy22)

   distribute (y3)

   y30 = get_term (y3,0)   # cdb (y30.101,y30)
   y32 = get_term (y3,2)   # cdb (y32.101,y32)
   y33 = get_term (y3,3)   # cdb (y33.101,y33)

   y30 = tidy (y30)        # cdb (y30.201,y30)
   y32 = tidy (y32)        # cdb (y32.201,y32)
   y33 = tidy (y33)        # cdb (y33.201,y33)

   defy30 := y30^{a} -> @(y30).
   defy32 := y32^{a} -> @(y32).
   defy33 := y33^{a} -> @(y33).

   # -----------------------------------
   # y4

   substitute (y4,defy00)

   substitute (y4,defy20)
   substitute (y4,defy22)

   substitute (y4,defy30)
   substitute (y4,defy32)
   substitute (y4,defy33)

   distribute (y4)

   y40 = get_term (y4,0)   # cdb (y40.101,y40)
   y42 = get_term (y4,2)   # cdb (y42.101,y42)
   y43 = get_term (y4,3)   # cdb (y43.101,y43)
   y44 = get_term (y4,4)   # cdb (y44.101,y44)

   y40 = tidy (y40)        # cdb (y40.201,y40)
   y42 = tidy (y42)        # cdb (y42.201,y42)
   y43 = tidy (y43)        # cdb (y43.201,y43)
   y44 = tidy (y44)        # cdb (y44.201,y44)

   defy40 := y40^{a} -> @(y40).
   defy42 := y42^{a} -> @(y42).
   defy43 := y43^{a} -> @(y43).
   defy44 := y44^{a} -> @(y44).

   # -----------------------------------
   # y5

   substitute (y5,defy00)

   substitute (y5,defy20)
   substitute (y5,defy22)

   substitute (y5,defy30)
   substitute (y5,defy32)
   substitute (y5,defy33)

   substitute (y5,defy40)
   substitute (y5,defy42)
   substitute (y5,defy43)
   substitute (y5,defy44)

   distribute (y5)

   y50 = get_term (y5,0)   # cdb (y50.101,y50)
   y52 = get_term (y5,2)   # cdb (y52.101,y52)
   y53 = get_term (y5,3)   # cdb (y53.101,y53)
   y54 = get_term (y5,4)   # cdb (y54.101,y54)
   y55 = get_term (y5,5)   # cdb (y55.101,y55)

   y50 = tidy (y50)        # cdb (y50.201,y50)
   y52 = tidy (y52)        # cdb (y52.201,y52)
   y53 = tidy (y53)        # cdb (y53.201,y53)
   y54 = tidy (y54)        # cdb (y54.201,y54)
   y55 = tidy (y55)        # cdb (y55.201,y55)

   defy50 := y50^{a} -> @(y50).
   defy52 := y52^{a} -> @(y52).
   defy53 := y53^{a} -> @(y53).
   defy54 := y54^{a} -> @(y54).
   defy55 := y55^{a} -> @(y55).

   end_stage_1 = time.time()

\end{cadabra}

\clearpage
\begin{dgroup*}
   \begin{dmath*} \cdb*{y0.001} \end{dmath*}
   \begin{dmath*} \cdb*{y2.001} \end{dmath*}
   \begin{dmath*} \cdb*{y3.001} \end{dmath*}
   \begin{dmath*} \cdb*{y4.001} \end{dmath*}
   \begin{dmath*} \cdb*{y5.001} \end{dmath*}
\end{dgroup*}

\clearpage
\begin{dgroup*}
   \begin{dmath*} \cdb*{y0.002} \end{dmath*}
   \begin{dmath*} \cdb*{y2.002} \end{dmath*}
   \begin{dmath*} \cdb*{y3.002} \end{dmath*}
   \begin{dmath*} \cdb*{y4.002} \end{dmath*}
   \begin{dmath*} \cdb*{y5.002} \end{dmath*}
\end{dgroup*}

\clearpage
\begin{dgroup*}
   \begin{dmath*} \cdb*{y00.101} \end{dmath*}
\end{dgroup*}

\begin{dgroup*}
   \begin{dmath*} \cdb*{y20.201} \end{dmath*}
   \begin{dmath*} \cdb*{y22.201} \end{dmath*}
\end{dgroup*}

\begin{dgroup*}
   \begin{dmath*} \cdb*{y30.201} \end{dmath*}
   \begin{dmath*} \cdb*{y32.201} \end{dmath*}
   \begin{dmath*} \cdb*{y33.201} \end{dmath*}
\end{dgroup*}

\begin{dgroup*}
   \begin{dmath*} \cdb*{y40.201} \end{dmath*}
   \begin{dmath*} \cdb*{y42.201} \end{dmath*}
   \begin{dmath*} \cdb*{y43.201} \end{dmath*}
   \begin{dmath*} \cdb*{y44.201} \end{dmath*}
\end{dgroup*}

\clearpage
\begin{dgroup*}
   \begin{dmath*} \cdb*{y50.201} \end{dmath*}
   \begin{dmath*} \cdb*{y52.201} \end{dmath*}
   \begin{dmath*} \cdb*{y53.201} \end{dmath*}
   \begin{dmath*} \cdb*{y54.201} \end{dmath*}
   \begin{dmath*} \cdb*{y55.201} \end{dmath*}
\end{dgroup*}

\clearpage

% =================================================================================================
\section*{Stage 2a: Introduce the generalised connections, build terms of $y^{a}$}

\begin{cadabra}
   def substitute_gam (obj):

       substitute (obj,defGam22)
       substitute (obj,defGam23)
       substitute (obj,defGam24)
       substitute (obj,defGam25)

       substitute (obj,defGam33)
       substitute (obj,defGam34)
       substitute (obj,defGam35)

       substitute (obj,defGam44)
       substitute (obj,defGam45)

       substitute (obj,defGam55)

       distribute (obj)
       return obj

   import cdblib

   beg_stage_2a = time.time()

   Gam22 = cdblib.get ('genGamma01','genGamma.json')
   Gam23 = cdblib.get ('genGamma02','genGamma.json')
   Gam24 = cdblib.get ('genGamma03','genGamma.json')
   Gam25 = cdblib.get ('genGamma04','genGamma.json')

   Gam33 = cdblib.get ('genGamma11','genGamma.json')
   Gam34 = cdblib.get ('genGamma12','genGamma.json')
   Gam35 = cdblib.get ('genGamma13','genGamma.json')

   Gam44 = cdblib.get ('genGamma21','genGamma.json')
   Gam45 = cdblib.get ('genGamma22','genGamma.json')

   Gam55 = cdblib.get ('genGamma31','genGamma.json')

   # peel off the A^{a}, must then symmetrise over revealed indices

   substitute (Gam22,$A^{a}->1$)
   substitute (Gam23,$A^{a}->1$)
   substitute (Gam24,$A^{a}->1$)
   substitute (Gam25,$A^{a}->1$)

   substitute (Gam33,$A^{a}->1$)
   substitute (Gam34,$A^{a}->1$)
   substitute (Gam35,$A^{a}->1$)

   substitute (Gam44,$A^{a}->1$)
   substitute (Gam45,$A^{a}->1$)

   substitute (Gam55,$A^{a}->1$)

   # now symmetrise

   sym (Gam22,$_{b},_{c}$)
   sym (Gam23,$_{b},_{c}$)
   sym (Gam24,$_{b},_{c}$)
   sym (Gam25,$_{b},_{c}$)

   sym (Gam33,$_{b},_{c},_{d}$)
   sym (Gam34,$_{b},_{c},_{d}$)
   sym (Gam35,$_{b},_{c},_{d}$)

   sym (Gam44,$_{b},_{c},_{d},_{e}$)
   sym (Gam45,$_{b},_{c},_{d},_{e}$)

   sym (Gam55,$_{b},_{c},_{d},_{e},_{f}$)

   defGam22 := Gam22^{a}_{b c} -> @(Gam22).
   defGam23 := Gam23^{a}_{b c} -> @(Gam23).
   defGam24 := Gam24^{a}_{b c} -> @(Gam24).
   defGam25 := Gam25^{a}_{b c} -> @(Gam25).

   defGam33 := Gam33^{a}_{b c d} -> @(Gam33).
   defGam34 := Gam34^{a}_{b c d} -> @(Gam34).
   defGam35 := Gam35^{a}_{b c d} -> @(Gam35).

   defGam44 := Gam44^{a}_{b c d e} -> @(Gam44).
   defGam45 := Gam45^{a}_{b c d e} -> @(Gam45).

   defGam55 := Gam55^{a}_{b c d e f} -> @(Gam55).

   # ---------------------------------------------------
   # y2

   y22 = substitute_gam (y22)

   y22 = tidy (y22)                                        # cdb (y22.301,y22)

   y2 := @(y20) + @(y22).                                  # cdb (y2.301,y2)

   # ---------------------------------------------------
   # y3

   y32 = substitute_gam (y32)
   y33 = substitute_gam (y33)

   y32 = tidy (y32)                                        # cdb (y32.301,y32)
   y33 = tidy (y33)                                        # cdb (y33.301,y33)

   y3 := @(y30) + @(y32) + @(y33).                         # cdb (y3.301,y3)

   # ---------------------------------------------------
   # y4

   y42 = substitute_gam (y42)
   y43 = substitute_gam (y43)
   y44 = substitute_gam (y44)

   y42 = tidy (y42)                                        # cdb (y42.301,y42)
   y43 = tidy (y43)                                        # cdb (y43.301,y43)
   y44 = tidy (y44)                                        # cdb (y44.301,y44)

   y4 := @(y40) + @(y42) + @(y43) + @(y44).                # cdb (y4.301,y4)

   # ---------------------------------------------------
   # y5

   y52 = substitute_gam (y52)
   y53 = substitute_gam (y53)
   y54 = substitute_gam (y54)
   y55 = substitute_gam (y55)

   y52 = tidy (y52)                                        # cdb (y52.301,y52)
   y53 = tidy (y53)                                        # cdb (y53.301,y53)
   y54 = tidy (y54)                                        # cdb (y54.301,y54)
   y55 = tidy (y55)                                        # cdb (y55.301,y55)

   y5 := @(y50) + @(y52) + @(y53) + @(y54) + @(y55).       # cdb (y5.301,y5)

   # ---------------------------------------------------
   cdblib.create ('geodesic-bvp.json')

   cdblib.put ('y2',y2,'geodesic-bvp.json')
   cdblib.put ('y3',y3,'geodesic-bvp.json')
   cdblib.put ('y4',y4,'geodesic-bvp.json')
   cdblib.put ('y5',y5,'geodesic-bvp.json')

   cdblib.put ('y20',y20,'geodesic-bvp.json')
   cdblib.put ('y22',y22,'geodesic-bvp.json')

   cdblib.put ('y30',y30,'geodesic-bvp.json')
   cdblib.put ('y32',y32,'geodesic-bvp.json')
   cdblib.put ('y33',y33,'geodesic-bvp.json')

   cdblib.put ('y40',y40,'geodesic-bvp.json')
   cdblib.put ('y42',y42,'geodesic-bvp.json')
   cdblib.put ('y43',y43,'geodesic-bvp.json')
   cdblib.put ('y44',y44,'geodesic-bvp.json')

   cdblib.put ('y50',y50,'geodesic-bvp.json')
   cdblib.put ('y52',y52,'geodesic-bvp.json')
   cdblib.put ('y53',y53,'geodesic-bvp.json')
   cdblib.put ('y54',y54,'geodesic-bvp.json')
   cdblib.put ('y55',y55,'geodesic-bvp.json')

   end_stage_2a = time.time()

\end{cadabra}

% note that:
%   y00 = y20 = y30 = y40 = y50
%   y22 = y32 = y42 = y52
%   y33 = y43 = y53
%   y44 = y54
%   y55

\clearpage

\begin{dgroup*}
   \begin{dmath*} \cdb*{y50.201} \end{dmath*}  % unchanged
   \begin{dmath*} \cdb*{y52.301} \end{dmath*}
   \begin{dmath*} \cdb*{y53.301} \end{dmath*}
   \begin{dmath*} \cdb*{y54.301} \end{dmath*}
   \begin{dmath*} \cdb*{y55.301} \end{dmath*}
\end{dgroup*}

\clearpage

% =================================================================================================
\section*{Stage 2b: Building the terms of $x^a(s)$}

\begin{cadabra}
   def substitute_y (obj):
       substitute (obj,defy00)
       substitute (obj,defy20)
       substitute (obj,defy30)
       substitute (obj,defy32)
       substitute (obj,defy40)
       substitute (obj,defy42)
       substitute (obj,defy43)
       distribute (obj)
       return obj

   beg_stage_2b = time.time()

   term2 := Gam^{a}_{b c} y4^{b} y4^{c}.
   term3 := Gam^{a}_{b c d} y3^{b} y3^{c} y3^{d}.
   term4 := Gam^{a}_{b c d e} y2^{b} y2^{c} y2^{d} y2^{e}.
   term5 := Gam^{a}_{b c d e f} y0^{b} y0^{c} y0^{d} y0^{e} y0^{f}.

   term2 = substitute_eps (term2)   # cdb (term2.401,term2)
   term3 = substitute_eps (term3)   # cdb (term3.401,term3)
   term4 = substitute_eps (term4)   # cdb (term4.401,term4)
   term5 = substitute_eps (term5)   # cdb (term5.401,term5)

   term2 = substitute_y (term2)
   term3 = substitute_y (term3)
   term4 = substitute_y (term4)
   term5 = substitute_y (term5)

   term2 = substitute_gam (term2)
   term3 = substitute_gam (term3)
   term4 = substitute_gam (term4)
   term5 = substitute_gam (term5)

   term2 = tidy (term2)   # cdb (term2.501,term2)
   term3 = tidy (term3)   # cdb (term3.501,term3)
   term4 = tidy (term4)   # cdb (term4.501,term4)
   term5 = tidy (term5)   # cdb (term5.501,term5)

\end{cadabra}

\clearpage
\begin{dgroup*}
   \begin{dmath*} \cdb*{term2.401} \end{dmath*}
   \begin{dmath*} \cdb*{term3.401} \end{dmath*}
   \begin{dmath*} \cdb*{term4.401} \end{dmath*}
   \begin{dmath*} \cdb*{term5.401} \end{dmath*}
\end{dgroup*}

\clearpage
\begin{dgroup*}
   \begin{dmath*} \cdb*{term2.501} \end{dmath*}
   \begin{dmath*} \cdb*{term3.501} \end{dmath*}
   \begin{dmath*} \cdb*{term4.501} \end{dmath*}
   \begin{dmath*} \cdb*{term5.501} \end{dmath*}
\end{dgroup*}

\clearpage

\begin{cadabra}
   # Check:
   #    x^{a} at s=1 should equal x^{a} + Dx^{a}
   #    but x^{a}(s) = x^{a} + s y^{a} - \sum (1/n!) @(termn) s^n
   #    thus foo should equal Dx^{a} and it does (yeah)

   foo := @(y5)
        -   (1/2) @(term2)
        -   (1/6) @(term3)
        -  (1/24) @(term4)
        - (1/120) @(term5).

   distribute         (foo)
   obj = product_sort (foo)
   rename_dummies     (foo)
   canonicalise       (foo)     # cdb (foo.001,foo)

   term2 :=   (1/2) @(term2).   # cdb(term2.502,term2)
   term3 :=   (1/6) @(term3).   # cdb(term3.502,term3)
   term4 :=  (1/24) @(term4).   # cdb(term4.502,term4)
   term5 := (1/120) @(term5).   # cdb(term5.502,term5)

   end_stage_2b = time.time()

\end{cadabra}

\begin{dgroup*}
   \begin{dmath*} \cdb*{foo.001} \end{dmath*}
\end{dgroup*}

\begin{dgroup*}
   \begin{dmath*} \cdb*{y2.301} \end{dmath*}
   \begin{dmath*} \cdb*{y3.301} \end{dmath*}
   \begin{dmath*} \cdb*{y4.301} \end{dmath*}
   % \begin{dmath*} \cdb*{y5.301} \end{dmath*}
\end{dgroup*}

\clearpage

% =================================================================================================
\section*{Stage 3: Reformatting and output}

\begin{cadabra}
   def get_Rterm (obj,n):

   # I would like to assign different weights to \nabla_{a}, \nabla_{a b}, \nabla_{a b c} etc. but no matter
   # what I do it appears that Cadabra assigns the same weight to all of these regardless of the number of subscripts.
   # It seems that the weight is assigned to the symbol \nabla alone. So I'm forced to use the following substitution trick.

       Q_{a b c d}::Weight(label=numR,value=2).
       Q_{a b c d e}::Weight(label=numR,value=3).
       Q_{a b c d e f}::Weight(label=numR,value=4).
       Q_{a b c d e f g}::Weight(label=numR,value=5).

       tmp := @(obj).

       distribute (tmp)

       substitute (tmp, $\nabla_{e f g}{R_{a b c d}} -> Q_{a b c d e f g}$)
       substitute (tmp, $\nabla_{e f}{R_{a b c d}} -> Q_{a b c d e f}$)
       substitute (tmp, $\nabla_{e}{R_{a b c d}} -> Q_{a b c d e}$)
       substitute (tmp, $R_{a b c d} -> Q_{a b c d}$)

       foo := @(tmp).
       bah = Ex("numR = " + str(n))
       keep_weight (foo, bah)

       substitute (foo, $Q_{a b c d e f g} -> \nabla_{e f g}{R_{a b c d}}$)
       substitute (foo, $Q_{a b c d e f} -> \nabla_{e f}{R_{a b c d}}$)
       substitute (foo, $Q_{a b c d e} -> \nabla_{e}{R_{a b c d}}$)
       substitute (foo, $Q_{a b c d} -> R_{a b c d}$)

       return foo

   def reformat (obj,scale):
       foo  = Ex(str(scale))
       bah := @(foo) @(obj).
       distribute     (bah)
       bah = product_sort (bah)
       rename_dummies (bah)
       canonicalise   (bah)
       substitute     (bah,$Dx^{b}->zzz^{b}$)
       factor_out     (bah,$x^{a?},zzz^{b?}$)
       substitute     (bah,$zzz^{b}->Dx^{b}$)
       ans := @(bah) / @(foo).
       return ans

   def rescale (obj,scale):
       foo  = Ex(str(scale))
       bah := @(foo) @(obj).
       distribute  (bah)
       substitute  (bah,$Dx^{b}->zzz^{b}$)
       factor_out  (bah,$x^{a?},zzz^{b?}$)
       substitute  (bah,$zzz^{b}->Dx^{b}$)
       return bah

   beg_stage_3 = time.time()

   Rterm22 = get_Rterm (term2,2)                           # cdb(Rterm22.101,Rterm22)
   Rterm23 = get_Rterm (term2,3)                           # cdb(Rterm23.101,Rterm23)
   Rterm24 = get_Rterm (term2,4)                           # cdb(Rterm24.101,Rterm24)
   Rterm25 = get_Rterm (term2,5)                           # cdb(Rterm25.101,Rterm25)

   Rterm32 = get_Rterm (term3,2)                           # cdb(Rterm32.101,Rterm32)  # zero
   Rterm33 = get_Rterm (term3,3)                           # cdb(Rterm33.101,Rterm33)
   Rterm34 = get_Rterm (term3,4)                           # cdb(Rterm34.101,Rterm34)
   Rterm35 = get_Rterm (term3,5)                           # cdb(Rterm35.101,Rterm35)

   Rterm42 = get_Rterm (term4,2)                           # cdb(Rterm42.101,Rterm42)  # zero
   Rterm43 = get_Rterm (term4,3)                           # cdb(Rterm43.101,Rterm43)  # zero
   Rterm44 = get_Rterm (term4,4)                           # cdb(Rterm44.101,Rterm44)
   Rterm45 = get_Rterm (term4,5)                           # cdb(Rterm45.101,Rterm45)

   Rterm52 = get_Rterm (term5,2)                           # cdb(Rterm52.101,Rterm52)  # zero
   Rterm53 = get_Rterm (term5,3)                           # cdb(Rterm53.101,Rterm53)  # zero
   Rterm54 = get_Rterm (term5,4)                           # cdb(Rterm54.101,Rterm54)  # zero
   Rterm55 = get_Rterm (term5,5)                           # cdb(Rterm55.101,Rterm55)

   Rterm22 = rescale ( reformat (Rterm22,   -3),    -3 )   # cdb(Rterm22.102,Rterm22)
   Rterm23 = rescale ( reformat (Rterm23,  -24),   -24 )   # cdb(Rterm23.102,Rterm23)
   Rterm24 = rescale ( reformat (Rterm24, -720),  -720 )   # cdb(Rterm24.102,Rterm24)
   Rterm25 = rescale ( reformat (Rterm25, -360),  -360 )   # cdb(Rterm25.102,Rterm25)

   Rterm33 = rescale ( reformat (Rterm33,  -12),   -12 )   # cdb(Rterm33.102,Rterm33)
   Rterm34 = rescale ( reformat (Rterm34, -720),  -720 )   # cdb(Rterm34.102,Rterm34)
   Rterm35 = rescale ( reformat (Rterm35,-1080), -1080 )   # cdb(Rterm35.102,Rterm35)

   Rterm44 = rescale ( reformat (Rterm44, -180),  -180 )   # cdb(Rterm44.102,Rterm44)
   Rterm45 = rescale ( reformat (Rterm45,-2160), -2160 )   # cdb(Rterm45.102,Rterm45)

   Rterm55 = rescale ( reformat (Rterm55, -360),  -360 )   # cdb(Rterm55.102,Rterm55)
\end{cadabra}

\clearpage

\begin{cadabra}
   # ----------------------------------------------------------------
   # bvp to terms linear in R

   tmp2 := -(1/3) @(Rterm22).

   bvp2 := x^{a}
        + s Dx^{a}
        + (s-s**2) @(tmp2).                                  # cdb(bvp.601,bvp2)

   cdblib.put ('bvp2',bvp2,'geodesic-bvp.json')
   cdblib.put ('bvp22',tmp2,'geodesic-bvp.json')

   y2 := Dx^{a} + @(tmp2).                                   # cdb(y2.600,y2)

   # ----------------------------------------------------------------
   # bvp to terms linear in dR

   tmp2 :=  -(1/3) @(Rterm22) - (1/24) @(Rterm23).
   tmp3 := -(1/12) @(Rterm33).

   bvp3 := x^{a}
        + s Dx^{a}
        + (s-s**2) @(tmp2)
        + (s-s**3) @(tmp3).                                  # cdb(bvp.602,bvp3)

   cdblib.put ('bvp3',bvp3,'geodesic-bvp.json')
   cdblib.put ('bvp32',tmp2,'geodesic-bvp.json')
   cdblib.put ('bvp33',tmp3,'geodesic-bvp.json')

   y3 := Dx^{a} + @(tmp2) + @(tmp3).                         # cdb(y3.600,y3)

   # ----------------------------------------------------------------
   # bvp to terms linear in d^2 R

   tmp2 :=   -(1/3) @(Rterm22) -  (1/24) @(Rterm23) - (1/720) @(Rterm24).
   tmp3 :=  -(1/12) @(Rterm33) - (1/720) @(Rterm34).
   tmp4 := -(1/180) @(Rterm44).

   bvp4 := x^{a}
        + s Dx^{a}
        + (s-s**2) @(tmp2)
        + (s-s**3) @(tmp3)
        + (s-s**4) @(tmp4).                                  # cdb(bvp.603,bvp4)

   cdblib.put ('bvp4',bvp4,'geodesic-bvp.json')
   cdblib.put ('bvp42',tmp2,'geodesic-bvp.json')
   cdblib.put ('bvp43',tmp3,'geodesic-bvp.json')
   cdblib.put ('bvp44',tmp4,'geodesic-bvp.json')

   y4 := Dx^{a} + @(tmp2) + @(tmp3) + @(tmp4).               # cdb(y4.600,y4)

   # ----------------------------------------------------------------
   # bvp to terms linear in d^3 R

   tmp2 := @(term2).
   tmp3 := @(term3).
   tmp4 := @(term4).
   tmp5 := @(term5).

   bvp5 := x^{a}
        + s Dx^{a}
        + (s-s**2) @(tmp2)
        + (s-s**3) @(tmp3)
        + (s-s**4) @(tmp4)
        + (s-s**5) @(tmp5).                                  # cdb(bvp.604,bvp5)

   cdblib.put ('bvp5',bvp5,'geodesic-bvp.json')
   cdblib.put ('bvp52',term2,'geodesic-bvp.json')
   cdblib.put ('bvp53',term3,'geodesic-bvp.json')
   cdblib.put ('bvp54',term4,'geodesic-bvp.json')
   cdblib.put ('bvp55',term5,'geodesic-bvp.json')

   y5 := Dx^{a} + @(tmp2) + @(tmp3) + @(tmp4) + @(tmp5).     # cdb(y5.600,y5)

   end_stage_3 = time.time()

   # cdbBeg (timing)
   print ("Stage 1:  {:7.1f} secs\\hfill\\break".format(end_stage_1-beg_stage_1))
   print ("Stage 2a: {:7.1f} secs\\hfill\\break".format(end_stage_2a-beg_stage_2a))
   print ("Stage 2b: {:7.1f} secs\\hfill\\break".format(end_stage_2b-beg_stage_2b))
   print ("Stage 3:  {:7.1f} secs\\hfill\\break".format(end_stage_3-beg_stage_3))
   # cdbEnd (timing)

\end{cadabra}

\clearpage

% -------------------------------------------------------------------------------------------------
\subsection*{Non-unit tangent vectors at $P$}

These are not unit vectors, their length is the geodesic distance from $P$ to $Q$

\begin{dgroup*}
   \begin{dmath*} \cdb*{y2.600} \end{dmath*}
   \begin{dmath*} \cdb*{y3.600} \end{dmath*}
   \begin{dmath*} \cdb*{y4.600} \end{dmath*}
   % \begin{dmath*} \cdb*{y5.600} \end{dmath*}
\end{dgroup*}

\clearpage

% =================================================================================================
\section*{Geodesic boundary value problem to terms linear in $R$}

\begin{dgroup*}
   \begin{dmath*} x^{a}(s) = \cdb{bvp.601} + \BigO{s^3,\eps^3} \end{dmath*}
\end{dgroup*}

\begin{align*}
   x^{a}(s) &= x^{a} + s Dx^{a}
                     + (s-s^2) x^{a}_2
                     + \BigO{s^3,\eps^3}
\end{align*}

\begin{dgroup*}
   \begin{dmath*} x^{a}_2 = \nx{2}^{a}_2 + \BigO{\eps^3} \end{dmath*}
   \begin{dmath*} -3 \nx{2}^{a}_2 = \cdb{Rterm22.102} \end{dmath*}
\end{dgroup*}

\clearpage

% =================================================================================================
\section*{Geodesic boundary value problem to terms linear in $\nabla R$}

\begin{dgroup*}
   \begin{dmath*} x^{a}(s) = \cdb{bvp.602} + \BigO{s^4,\eps^4} \end{dmath*}
\end{dgroup*}

\begin{align*}
   x^{a}(s) &= x^{a} + s Dx^{a}
                     + (s-s^2) x^{a}_2
                     + (s-s^3) x^{a}_3
                     + \BigO{s^4,\eps^4}
\end{align*}

\begin{dgroup*}
   \begin{dmath*} x^{a}_2 = \nx{2}^{a}_2 + \nx{3}^{a}_2 + \BigO{\eps^4} \end{dmath*}
   \begin{dmath*}   -3 \nx{2}^{a}_2 = \cdb{Rterm22.102} \end{dmath*}
   \begin{dmath*}  -24 \nx{3}^{a}_2 = \cdb{Rterm23.102} \end{dmath*}
\end{dgroup*}

\begin{dgroup*}
   \begin{dmath*} x^{a}_3 = \nx{3}^{a}_3 + \BigO{\eps^4} \end{dmath*}
   \begin{dmath*}   -12 \nx{3}^{a}_3 = \cdb{Rterm33.102} \end{dmath*}
\end{dgroup*}

% =================================================================================================
\section*{Geodesic boundary value problem to terms linear in $\nabla^2 R$}

\begin{dgroup*}
   \begin{dmath*} x^{a}(s) = \cdb{bvp.603} + \BigO{s^5,\eps^5} \end{dmath*}
\end{dgroup*}

\begin{align*}
   x^{a}(s) &= x^{a} + s Dx^{a}
                     + (s-s^2) x^{a}_2
                     + (s-s^3) x^{a}_3
                     + (s-s^4) x^{a}_4
                     + \BigO{s^5,\eps^5}
\end{align*}

\begin{dgroup*}
   \begin{dmath*} x^{a}_2 = \nx{2}^{a}_2 + \nx{3}^{a}_2 + \nx{4}^{a}_2 + \BigO{\eps^5} \end{dmath*}
   \begin{dmath*}   -3 \nx{2}^{a}_2 = \cdb{Rterm22.102} \end{dmath*}
   \begin{dmath*}  -24 \nx{3}^{a}_2 = \cdb{Rterm23.102} \end{dmath*}
   \begin{dmath*} -720 \nx{4}^{a}_2 = \cdb{Rterm24.102} \end{dmath*}
\end{dgroup*}

\begin{dgroup*}
   \begin{dmath*} x^{a}_3 = \nx{3}^{a}_3 + \nx{4}^{a}_3 + \BigO{\eps^5} \end{dmath*}
   \begin{dmath*}   -12 \nx{3}^{a}_3 = \cdb{Rterm33.102} \end{dmath*}
   \begin{dmath*}  -720 \nx{4}^{a}_3 = \cdb{Rterm34.102} \end{dmath*}
\end{dgroup*}

\begin{dgroup*}
   \begin{dmath*} x^{a}_4 = \nx{4}^{a}_4 + \BigO{\eps^5} \end{dmath*}
   \begin{dmath*}  -180 \nx{4}^{a}_4 = \cdb{Rterm44.102} \end{dmath*}
\end{dgroup*}

\clearpage

% =================================================================================================
\section*{Geodesic boundary value problem to terms linear in $\nabla^3 R$}

The geodesic that connects the points with RNC coordinates $x^a$ and $x^a+Dx^a$ is described, for $0\le s\le 1$, by
%
% \begin{dgroup*}
%    \begin{dmath*} x^{a}(s) = \cdb{bvp.604} + \BigO{s^6,\eps^6} \end{dmath*} % too big for pdfLaTeX
% \end{dgroup*}
%
\begin{align*}
   x^{a}(s) &= x^{a} + s Dx^{a}
                     + (s-s^2) x^{a}_2
                     + (s-s^3) x^{a}_3
                     + (s-s^4) x^{a}_4
                     + (s-s^5) x^{a}_5
                     + \BigO{s^6,\eps^6}
\end{align*}

\begin{dgroup*}
   \begin{dmath*} x^{a}_2 = \nx{2}^{a}_2 + \nx{3}^{a}_2 + \nx{4}^{a}_2 + \nx{5}^{a}_2 + \BigO{\eps^6} \end{dmath*}
   \begin{dmath*}   -3 \nx{2}^{a}_2 = \cdb{Rterm22.102} \end{dmath*}
   \begin{dmath*}  -24 \nx{3}^{a}_2 = \cdb{Rterm23.102} \end{dmath*}
   \begin{dmath*} -720 \nx{4}^{a}_2 = \cdb{Rterm24.102} \end{dmath*}
   \begin{dmath*} -360 \nx{5}^{a}_2 = \cdb{Rterm25.102} \end{dmath*}
\end{dgroup*}

\clearpage

\begin{dgroup*}
   \begin{dmath*} x^{a}_3 = \nx{3}^{a}_3 + \nx{4}^{a}_3 + \nx{5}^{a}_3 + \BigO{\eps^6} \end{dmath*}
   \begin{dmath*}   -12 \nx{3}^{a}_3 = \cdb{Rterm33.102} \end{dmath*}
   \begin{dmath*}  -720 \nx{4}^{a}_3 = \cdb{Rterm34.102} \end{dmath*}
   \begin{dmath*} -1080 \nx{5}^{a}_3 = \cdb{Rterm35.102} \end{dmath*}
\end{dgroup*}

\begin{dgroup*}
   \begin{dmath*} x^{a}_4 = \nx{4}^{a}_4 + \nx{5}^{a}_4 + \BigO{\eps^6} \end{dmath*}
   \begin{dmath*}  -180 \nx{4}^{a}_4 = \cdb{Rterm44.102} \end{dmath*}
   \begin{dmath*} -2160 \nx{5}^{a}_4 = \cdb{Rterm45.102} \end{dmath*}
\end{dgroup*}

\begin{dgroup*}
   \begin{dmath*} x^{a}_5 = \nx{5}^{a}_5 + \BigO{\eps^6} \end{dmath*}
   \begin{dmath*} -360 \nx{5}^{a}_5 = \cdb{Rterm55.102} \end{dmath*}
\end{dgroup*}

\clearpage

% =================================================================================================
% export selected objects, these will later be imported into a library
% these are the objects that will appear in the paper

\begin{cadabra}
   tmp2 := 8 @(Rterm22) + @(Rterm23).
   tmp3 := @(Rterm33).

   factor_out     (tmp2,$Dx^{a?}$) # cdb(tmp2.001,tmp2)
   rename_dummies (tmp2)
   factor_out     (tmp2,$Dx^{a?}$) # cdb(tmp2.002,tmp2)

   bvp4 := x^{a}
        + \lam Dx^{a}
        - (1/24) (\lam-\lam**2) @(tmp2)
        - (1/12) (\lam-\lam**3) @(tmp3).     # cdb(bvp4,bvp4)

   cdblib.create ('geodesic-bvp.export')

   # 4th order bvp
   cdblib.put ('bvp4',bvp4,'geodesic-bvp.export')

   # 6th order bvp terms, scaled
   cdblib.put ('bvp622',Rterm22,'geodesic-bvp.export')
   cdblib.put ('bvp623',Rterm23,'geodesic-bvp.export')
   cdblib.put ('bvp624',Rterm24,'geodesic-bvp.export')
   cdblib.put ('bvp625',Rterm25,'geodesic-bvp.export')

   cdblib.put ('bvp633',Rterm33,'geodesic-bvp.export')
   cdblib.put ('bvp634',Rterm34,'geodesic-bvp.export')
   cdblib.put ('bvp635',Rterm35,'geodesic-bvp.export')

   cdblib.put ('bvp644',Rterm44,'geodesic-bvp.export')
   cdblib.put ('bvp645',Rterm45,'geodesic-bvp.export')

   cdblib.put ('bvp655',Rterm55,'geodesic-bvp.export')

   checkpoint.append (bvp4)

   checkpoint.append (Rterm22)
   checkpoint.append (Rterm23)
   checkpoint.append (Rterm24)
   checkpoint.append (Rterm25)

   checkpoint.append (Rterm33)
   checkpoint.append (Rterm34)
   checkpoint.append (Rterm35)

   checkpoint.append (Rterm44)
   checkpoint.append (Rterm45)

   checkpoint.append (Rterm55)
\end{cadabra}

\clearpage

% =================================================================================================
\section*{Timing}

\cdb{timing}

% =================================================================================================
% export checkpoints in json format

\bgroup
\CdbSetup{action=hide}
\begin{cadabra}
   for i in range( len(checkpoint) ):
      cdblib.put ('check{:03d}'.format(i),checkpoint[i],checkpoint_file)
\end{cadabra}
\egroup

\end{document}


\begin{dgroup*}
   \begin{dmath*} x^{a}(s) = \cdb{bvp.601} + \BigO{s^3,\eps^3} \end{dmath*}
\end{dgroup*}

\begin{align*}
   x^{a}(s) &= x^{a} + s Dx^{a}
                     + (s-s^2) x^{a}_2
                     + \BigO{s^3,\eps^3}
\end{align*}

\begin{dgroup*}
   \begin{dmath*} x^{a}_2 = \nx{2}^{a}_2 + \BigO{\eps^3} \end{dmath*}
   \begin{dmath*} -3 \nx{2}^{a}_2 = \cdb{Rterm22.102} \end{dmath*}
\end{dgroup*}

\clearpage

% =================================================================================================
\section*{Geodesic boundary value problem to terms linear in $\nabla R$}

\begin{dgroup*}
   \begin{dmath*} x^{a}(s) = \cdb{bvp.602} + \BigO{s^4,\eps^4} \end{dmath*}
\end{dgroup*}

\begin{align*}
   x^{a}(s) &= x^{a} + s Dx^{a}
                     + (s-s^2) x^{a}_2
                     + (s-s^3) x^{a}_3
                     + \BigO{s^4,\eps^4}
\end{align*}

\begin{dgroup*}
   \begin{dmath*} x^{a}_2 = \nx{2}^{a}_2 + \nx{3}^{a}_2 + \BigO{\eps^4} \end{dmath*}
   \begin{dmath*}   -3 \nx{2}^{a}_2 = \cdb{Rterm22.102} \end{dmath*}
   \begin{dmath*}  -24 \nx{3}^{a}_2 = \cdb{Rterm23.102} \end{dmath*}
\end{dgroup*}

\begin{dgroup*}
   \begin{dmath*} x^{a}_3 = \nx{3}^{a}_3 + \BigO{\eps^4} \end{dmath*}
   \begin{dmath*}   -12 \nx{3}^{a}_3 = \cdb{Rterm33.102} \end{dmath*}
\end{dgroup*}

% =================================================================================================
\section*{Geodesic boundary value problem to terms linear in $\nabla^2 R$}

\begin{dgroup*}
   \begin{dmath*} x^{a}(s) = \cdb{bvp.603} + \BigO{s^5,\eps^5} \end{dmath*}
\end{dgroup*}

\begin{align*}
   x^{a}(s) &= x^{a} + s Dx^{a}
                     + (s-s^2) x^{a}_2
                     + (s-s^3) x^{a}_3
                     + (s-s^4) x^{a}_4
                     + \BigO{s^5,\eps^5}
\end{align*}

\begin{dgroup*}
   \begin{dmath*} x^{a}_2 = \nx{2}^{a}_2 + \nx{3}^{a}_2 + \nx{4}^{a}_2 + \BigO{\eps^5} \end{dmath*}
   \begin{dmath*}   -3 \nx{2}^{a}_2 = \cdb{Rterm22.102} \end{dmath*}
   \begin{dmath*}  -24 \nx{3}^{a}_2 = \cdb{Rterm23.102} \end{dmath*}
   \begin{dmath*} -720 \nx{4}^{a}_2 = \cdb{Rterm24.102} \end{dmath*}
\end{dgroup*}

\begin{dgroup*}
   \begin{dmath*} x^{a}_3 = \nx{3}^{a}_3 + \nx{4}^{a}_3 + \BigO{\eps^5} \end{dmath*}
   \begin{dmath*}   -12 \nx{3}^{a}_3 = \cdb{Rterm33.102} \end{dmath*}
   \begin{dmath*}  -720 \nx{4}^{a}_3 = \cdb{Rterm34.102} \end{dmath*}
\end{dgroup*}

\begin{dgroup*}
   \begin{dmath*} x^{a}_4 = \nx{4}^{a}_4 + \BigO{\eps^5} \end{dmath*}
   \begin{dmath*}  -180 \nx{4}^{a}_4 = \cdb{Rterm44.102} \end{dmath*}
\end{dgroup*}

\clearpage

% =================================================================================================
\section*{Geodesic boundary value problem to terms linear in $\nabla^3 R$}

The geodesic that connects the points with RNC coordinates $x^a$ and $x^a+Dx^a$ is described, for $0\le s\le 1$, by
%
% \begin{dgroup*}
%    \begin{dmath*} x^{a}(s) = \cdb{bvp.604} + \BigO{s^6,\eps^6} \end{dmath*} % too big for pdfLaTeX
% \end{dgroup*}
%
\begin{align*}
   x^{a}(s) &= x^{a} + s Dx^{a}
                     + (s-s^2) x^{a}_2
                     + (s-s^3) x^{a}_3
                     + (s-s^4) x^{a}_4
                     + (s-s^5) x^{a}_5
                     + \BigO{s^6,\eps^6}
\end{align*}

\begin{dgroup*}
   \begin{dmath*} x^{a}_2 = \nx{2}^{a}_2 + \nx{3}^{a}_2 + \nx{4}^{a}_2 + \nx{5}^{a}_2 + \BigO{\eps^6} \end{dmath*}
   \begin{dmath*}   -3 \nx{2}^{a}_2 = \cdb{Rterm22.102} \end{dmath*}
   \begin{dmath*}  -24 \nx{3}^{a}_2 = \cdb{Rterm23.102} \end{dmath*}
   \begin{dmath*} -720 \nx{4}^{a}_2 = \cdb{Rterm24.102} \end{dmath*}
   \begin{dmath*} -360 \nx{5}^{a}_2 = \cdb{Rterm25.102} \end{dmath*}
\end{dgroup*}

\clearpage

\begin{dgroup*}
   \begin{dmath*} x^{a}_3 = \nx{3}^{a}_3 + \nx{4}^{a}_3 + \nx{5}^{a}_3 + \BigO{\eps^6} \end{dmath*}
   \begin{dmath*}   -12 \nx{3}^{a}_3 = \cdb{Rterm33.102} \end{dmath*}
   \begin{dmath*}  -720 \nx{4}^{a}_3 = \cdb{Rterm34.102} \end{dmath*}
   \begin{dmath*} -1080 \nx{5}^{a}_3 = \cdb{Rterm35.102} \end{dmath*}
\end{dgroup*}

\begin{dgroup*}
   \begin{dmath*} x^{a}_4 = \nx{4}^{a}_4 + \nx{5}^{a}_4 + \BigO{\eps^6} \end{dmath*}
   \begin{dmath*}  -180 \nx{4}^{a}_4 = \cdb{Rterm44.102} \end{dmath*}
   \begin{dmath*} -2160 \nx{5}^{a}_4 = \cdb{Rterm45.102} \end{dmath*}
\end{dgroup*}

\begin{dgroup*}
   \begin{dmath*} x^{a}_5 = \nx{5}^{a}_5 + \BigO{\eps^6} \end{dmath*}
   \begin{dmath*} -360 \nx{5}^{a}_5 = \cdb{Rterm55.102} \end{dmath*}
\end{dgroup*}

\clearpage

% =================================================================================================
\section*{Geodesic arc-length}
\documentclass[12pt]{cdblatex}

\begin{document}

\section*{\jobname}

\CdbSetup{action=hide}

\begin{cadabra}
   import shared

   import cdblib

   term00A = cdblib.get ('check000','expected/geodesic-lsq.json')
   term01A = cdblib.get ('check001','expected/geodesic-lsq.json')
   term02A = cdblib.get ('check002','expected/geodesic-lsq.json')
   term03A = cdblib.get ('check003','expected/geodesic-lsq.json')
   term04A = cdblib.get ('check004','expected/geodesic-lsq.json')
   term05A = cdblib.get ('check005','expected/geodesic-lsq.json')

   term00B = cdblib.get ('check000','output/geodesic-lsq.json')
   term01B = cdblib.get ('check001','output/geodesic-lsq.json')
   term02B = cdblib.get ('check002','output/geodesic-lsq.json')
   term03B = cdblib.get ('check003','output/geodesic-lsq.json')
   term04B = cdblib.get ('check004','output/geodesic-lsq.json')
   term05B = cdblib.get ('check005','output/geodesic-lsq.json')

   diff000 = shared.check (term00A,term00B)   # cdb (diff000,diff000)
   diff001 = shared.check (term01A,term01B)   # cdb (diff001,diff001)
   diff002 = shared.check (term02A,term02B)   # cdb (diff002,diff002)
   diff003 = shared.check (term03A,term03B)   # cdb (diff003,diff003)
   diff004 = shared.check (term04A,term04B)   # cdb (diff004,diff004)
   diff005 = shared.check (term05A,term05B)   # cdb (diff005,diff005)

\end{cadabra}

\begin{dgroup*}
   \Dmath*{ \cdb*{diff000} }
   \Dmath*{ \cdb*{diff001} }
   \Dmath*{ \cdb*{diff002} }
   \Dmath*{ \cdb*{diff003} }
   \Dmath*{ \cdb*{diff004} }
   \Dmath*{ \cdb*{diff005} }
\end{dgroup*}

\end{document}


\begin{dgroup*}[spread=5pt]
   % LCB: which of these is correct?
   % \begin{dmath*} \left(\Delta s\right)^2 = \cdb{lsq.301} + \BigO{\eps^6,Dx^6} \end{dmath*}
   % \begin{dmath*} \left(\Delta s\right)^2 = \cdb{lsq.301} + \BigO{\eps^6,Dx^7} \end{dmath*}
   \begin{dmath*} \left(\Delta s\right)^2 = \cdb{lsq.301} + \BigO{\eps^6} \end{dmath*}
\end{dgroup*}

\clearpage

% =================================================================================================
\section*{Geodesic arc-length curvature expansion}

\begin{align*}
   % LCB: which of these is correct?
   % \left(\Delta s\right)^2 = \nD{0} + \nD{2} + \nD{3} + \nD{4} + \nD{5} + \BigO{\eps^6,Dx^6}
   % \left(\Delta s\right)^2 = \nD{0} + \nD{2} + \nD{3} + \nD{4} + \nD{5} + \BigO{\eps^6,Dx^7}
   \left(\Delta s\right)^2 = \nD{0} + \nD{2} + \nD{3} + \nD{4} + \nD{5} + \BigO{\eps^6}
\end{align*}

\begin{dgroup*}[spread=5pt]
   \begin{dmath*}      \nD{0} = \cdb{scaled0.301} \end{dmath*}
   \begin{dmath*}    3 \nD{2} = \cdb{scaled2.301} \end{dmath*}
   \begin{dmath*}   12 \nD{3} = \cdb{scaled3.301} \end{dmath*}
   \begin{dmath*}  360 \nD{4} = \cdb{scaled4.301} \end{dmath*}
   \begin{dmath*} 1080 \nD{5} = \cdb{scaled5.301} \end{dmath*}
\end{dgroup*}

\clearpage

% =================================================================================================
\section*{Tranformation between two RNC frames}
\def\Date{19 Jan 2024}
% \def\FileID{file:}

\documentclass[12pt]{cdblatex}

\begin{document}

% =================================================================================================
% create checkpoint file

\bgroup
\CdbSetup{action=hide}
\begin{cadabra}
   import cdblib
   checkpoint_file = 'tests/semantic/output/rnc2rnc.json'
   cdblib.create (checkpoint_file)
   checkpoint = []
\end{cadabra}
\egroup

% =================================================================================================
\section*{From one RNC to another}

Consider an RNC frame with RNC cooridnates $x^{a}$.

In the {\tts geodesic-bvp} code the two point boundary value problem (for the geodesic connecting
two points) was solved. There is a bonus in that calculation -- it can be trivaly adapted to the
case of transforming form one RNC into another.

The starting point is the basic equation for the geodesic connecting $P$ (with coordinaties
$x^{a}$) to Q (with coordinates $x^{a} + Dx^{a}$)
\begin{equation*}
   x^a(s) = x^a_i + s y^a - \sum_{k=2}^\infty\>\frac{1}{k!}\>\Gamma^{a}{}_{\ubk}y^{.\ubk} s^k
\end{equation*}
The affine parameter $s$ varies form 0 (at $P$) to 1 (at $Q$).

A new RNC frame, with origin at $P$, can be defined via the $y^{a}$ with the coordinates of $Q$ in
the new RNC frame defined by $y^{a}$ (since $s=1$ at $Q$). Recall that in an RNC all geodesics
through the origin are described by $y^{a}(s) = s y^{a}$. Thus the transformation from $x^a$ to
$y^a$ satisfies
\begin{equation*}
   x^a = x^a_i + y^a - \sum_{k=2}^\infty\>\frac{1}{k!}\>\Gamma^{a}{}_{\ubk}y^{.\ubk}
\end{equation*}
where the $\Gamma^{a}{}_{\ubk}$ are the generalised connections of the $x^a$ frame evaluated at
$x^a=0$. This equation can be inverted to express $y^a$ in terms of $x^a$. This computation is
done in the {\tts geodesic-bvp} code -- we only quote the results here (at the end).

The new $y^a$ frame has origin at $P$. Its coordinate axes are aligned with those (at $P$) of the
origianl RNC frame. To see this just note that $\partial x^a/\partial y^b = \delta^a_b$ at $P$.
Thus the metric at $P$ in the new frame has values $g_{ab}(x)$ (i.e., exactly those of the
original RNC frame). Note that this means that the coordinate axes of the new frame are not
ncessarily orthogonal.

The calculations in this code are trivial. It uses the $y^{a}$ found in {\tts geodesic-bvp} as
the basis of the transformation from $x^{a}$ to $y^{a}$. Most of the code involves reformatting
the $y^{a}$.

\clearpage

\begin{cadabra}
   {a,b,c,d,e,f,g,h,i,j,k,l,m,n,o,p,q,r,s,t,u,v,w#}::Indices(position=independent).

   \nabla{#}::Derivative.

   g_{a b}::Metric.
   g^{a b}::InverseMetric.

   R_{a b c d}::RiemannTensor.
   R^{a}_{b c d}::RiemannTensor.

   # Dx{#}::LaTeXForm{"{\Dx}"}.  # LCB: currently causes a bug, it kills ::KeepWeight for Dx

   import cdblib

   Y5 = cdblib.get ('y5','geodesic-bvp.json')

   Y50 = cdblib.get ('y50','geodesic-bvp.json')
   Y52 = cdblib.get ('y52','geodesic-bvp.json')
   Y53 = cdblib.get ('y53','geodesic-bvp.json')
   Y54 = cdblib.get ('y54','geodesic-bvp.json')
   Y55 = cdblib.get ('y55','geodesic-bvp.json')

   # this copies y5* from geodesic-bvp.json to rnc2rnc.json

   cdblib.create ('rnc2rnc.json')

   cdblib.put ('rnc2rnc',Y5,'rnc2rnc.json')

   cdblib.put ('rnc2rnc0',Y50,'rnc2rnc.json')
   cdblib.put ('rnc2rnc2',Y52,'rnc2rnc.json')
   cdblib.put ('rnc2rnc3',Y53,'rnc2rnc.json')
   cdblib.put ('rnc2rnc4',Y54,'rnc2rnc.json')
   cdblib.put ('rnc2rnc5',Y55,'rnc2rnc.json')

\end{cadabra}

% =================================================================================================
% the remaining code is just for pretty printing

\clearpage

\begin{cadabra}
   # note: keeping numbering as is (out of order) to ensure R appears before \nabla R etc.
   def product_sort (obj):
       substitute (obj,$ x^{a}                            -> A001^{a}               $)
       substitute (obj,$ Dx^{a}                           -> A002^{a}               $)
       substitute (obj,$ g^{a b}                          -> A003^{a b}             $)
       substitute (obj,$ \nabla_{e f g h}{R_{a b c d}}    -> A008_{a b c d e f g h} $)
       substitute (obj,$ \nabla_{e f g}{R_{a b c d}}      -> A007_{a b c d e f g}   $)
       substitute (obj,$ \nabla_{e f}{R_{a b c d}}        -> A006_{a b c d e f}     $)
       substitute (obj,$ \nabla_{e}{R_{a b c d}}          -> A005_{a b c d e}       $)
       substitute (obj,$ R_{a b c d}                      -> A004_{a b c d}         $)
       sort_product   (obj)
       rename_dummies (obj)
       substitute (obj,$ A001^{a}                  -> x^{a}                         $)
       substitute (obj,$ A002^{a}                  -> Dx^{a}                        $)
       substitute (obj,$ A003^{a b}                -> g^{a b}                       $)
       substitute (obj,$ A004_{a b c d}            -> R_{a b c d}                   $)
       substitute (obj,$ A005_{a b c d e}          -> \nabla_{e}{R_{a b c d}}       $)
       substitute (obj,$ A006_{a b c d e f}        -> \nabla_{e f}{R_{a b c d}}     $)
       substitute (obj,$ A007_{a b c d e f g}      -> \nabla_{e f g}{R_{a b c d}}   $)
       substitute (obj,$ A008_{a b c d e f g h}    -> \nabla_{e f g h}{R_{a b c d}} $)

       return obj

   def get_xDxterm (obj,n,m):

       x^{a}::Weight(label=numx,value=1).
       Dx^{a}::Weight(label=numDx,value=1).

       tmp := @(obj).
       distribute  (tmp)

       foo = Ex("numx = " + str(n))
       bah = Ex("numDx = " + str(m))
       keep_weight (tmp, foo)
       keep_weight (tmp, bah)

       return tmp

   def reformat (obj,scale):
       foo  = Ex(str(scale))
       bah := @(foo) @(obj).
       distribute     (bah)
       bah = product_sort (bah)
       rename_dummies (bah)
       canonicalise   (bah)
       substitute     (bah,$Dx^{b}->zzz^{b}$)
       factor_out     (bah,$x^{a?},zzz^{b?}$)
       substitute     (bah,$zzz^{b}->Dx^{b}$)
       ans := @(bah) / @(foo).
       return ans

   def rescale (obj,scale):
       foo  = Ex(str(scale))
       bah := @(foo) @(obj).
       distribute  (bah)
       substitute  (bah,$Dx^{b}->zzz^{b}$)
       factor_out  (bah,$x^{a?},zzz^{b?}$)
       substitute  (bah,$zzz^{b}->Dx^{b}$)
       return bah

   term0 := @(Y50).  # cdb (term0.101,term0)
   term2 := @(Y52).  # cdb (term2.101,term2)
   term3 := @(Y53).  # cdb (term3.101,term3)
   term4 := @(Y54).  # cdb (term4.101,term4)
   term5 := @(Y55).  # cdb (term5.101,term5)

   term0 = reformat (term0,1)  # cdb (term0.102,term0)
   term2 = reformat (term2,1)  # cdb (term2.102,term2)
   term3 = reformat (term3,1)  # cdb (term3.102,term3)
   term4 = reformat (term4,1)  # cdb (term4.102,term4)
   term5 = reformat (term5,1)  # cdb (term5.102,term5)

   xDxterm12 = get_xDxterm (term2,1,2)   # cdb(xDxterm12.101,xDxterm12)

   xDxterm13 = get_xDxterm (term3,1,3)   # cdb(xDxterm13.101,xDxterm13)
   xDxterm22 = get_xDxterm (term3,2,2)   # cdb(xDxterm22.101,xDxterm22)

   xDxterm14 = get_xDxterm (term4,1,4)   # cdb(xDxterm14.101,xDxterm14)
   xDxterm23 = get_xDxterm (term4,2,3)   # cdb(xDxterm23.101,xDxterm23)
   xDxterm32 = get_xDxterm (term4,3,2)   # cdb(xDxterm32.101,xDxterm32)

   xDxterm15 = get_xDxterm (term5,1,5)   # cdb(xDxterm15.101,xDxterm15)
   xDxterm24 = get_xDxterm (term5,2,4)   # cdb(xDxterm24.101,xDxterm24)
   xDxterm33 = get_xDxterm (term5,3,3)   # cdb(xDxterm33.101,xDxterm33)
   xDxterm42 = get_xDxterm (term5,4,2)   # cdb(xDxterm42.101,xDxterm42)


   xDxterm12 = rescale ( reformat (xDxterm12,    3),     3 )   # cdb(xDxterm12.102,xDxterm12)

   xDxterm13 = rescale ( reformat (xDxterm13,   12),   -12 )   # cdb(xDxterm13.102,xDxterm13)
   xDxterm22 = rescale ( reformat (xDxterm22,   24),   -24 )   # cdb(xDxterm22.102,xDxterm22)

   xDxterm14 = rescale ( reformat (xDxterm14,  180),  -180 )   # cdb(xDxterm14.102,xDxterm14)
   xDxterm23 = rescale ( reformat (xDxterm23,  720),  -720 )   # cdb(xDxterm23.102,xDxterm23)
   xDxterm32 = rescale ( reformat (xDxterm32,  720),  -720 )   # cdb(xDxterm32.102,xDxterm32)

   xDxterm15 = rescale ( reformat (xDxterm15,  360),  -360 )   # cdb(xDxterm15.102,xDxterm15)
   xDxterm24 = rescale ( reformat (xDxterm24, 2160), -2160 )   # cdb(xDxterm24.102,xDxterm24)
   xDxterm33 = rescale ( reformat (xDxterm33, 1080), -1080 )   # cdb(xDxterm33.102,xDxterm33)
   xDxterm42 = rescale ( reformat (xDxterm42,  360),  -360 )   # cdb(xDxterm42.102,xDxterm42)

   checkpoint.append (term0)
   checkpoint.append (term2)
   checkpoint.append (term3)
   checkpoint.append (term4)
   checkpoint.append (term5)

\end{cadabra}

\clearpage

% =================================================================================================
\section*{Tranformation between two RNC frames}

\begin{align*}
     y^{a} = \ny{0}^{a} + \ny{2}^{a} + \ny{3}^{a} + \ny{4}^{a} + \ny{5}^{a} + \BigO{\eps^6}
\end{align*}

\begin{dgroup*}
   \begin{dmath*} \ny{0}^{a} = \cdb{term0.102} \end{dmath*}
   \begin{dmath*} \ny{2}^{a} = \cdb{term2.102} \end{dmath*}
   \begin{dmath*} \ny{3}^{a} = \cdb{term3.102} \end{dmath*}
   \begin{dmath*} \ny{4}^{a} = \cdb{term4.102} \end{dmath*}
   \begin{dmath*} \ny{5}^{a} = \cdb{term5.102} \end{dmath*}
\end{dgroup*}

\clearpage

% =================================================================================================
\section*{Tranformation between two RNC frames}

Same as before but with an improved format (maybe) for the expressions.

\begin{align}
   y^{a} = \ny{0}^{a} + \ny{2}^{a} + \ny{3}^{a} + \ny{4}^{a} + \ny{5}^{a} + \BigO{\eps^6}
\end{align}

\begin{dgroup}
   \begin{dmath} \ny{0}^{a} = Dx^{a} \end{dmath}
\end{dgroup}

\begin{dgroup}
   \begin{dmath} \ny{2}^{a} = \ny{2}^{a}_1 \end{dmath}
   \begin{dmath}   3 \ny{2}^{a}_1 = \cdb{xDxterm12.102} \end{dmath}
\end{dgroup}

\begin{dgroup}
   \begin{dmath} \ny{3}^{a} = \ny{3}^{a}_1 + \ny{3}^{a}_2 \end{dmath}
   \begin{dmath} -12 \ny{3}^{a}_1 = \cdb{xDxterm13.102} \end{dmath}
   \begin{dmath} -24 \ny{3}^{a}_2 = \cdb{xDxterm22.102} \end{dmath}
\end{dgroup}

\begin{dgroup}
   \begin{dmath} \ny{4}^{a} = \ny{4}^{a}_1 + \ny{4}^{a}_2 + \ny{4}^{a}_3 \end{dmath}
   \begin{dmath} -180 \ny{4}^{a}_1 = \cdb{xDxterm14.102} \end{dmath}
   \begin{dmath} -720 \ny{4}^{a}_2 = \cdb{xDxterm23.102} \end{dmath}
   \begin{dmath} -720 \ny{4}^{a}_3 = \cdb{xDxterm32.102} \end{dmath}
\end{dgroup}

\begin{dgroup}
   \begin{dmath} \ny{5}^{a} = \ny{5}^{a}_1 + \ny{5}^{a}_2 + \ny{5}^{a}_3 + \ny{5}^{a}_4 \end{dmath}
   \begin{dmath}  -360 \ny{5}^{a}_1 = \cdb{xDxterm15.102} \end{dmath}
   \begin{dmath} -2160 \ny{5}^{a}_2 = \cdb{xDxterm24.102} \end{dmath}
   \begin{dmath} -1080 \ny{5}^{a}_3 = \cdb{xDxterm33.102} \end{dmath}
   \begin{dmath}  -360 \ny{5}^{a}_4 = \cdb{xDxterm42.102} \end{dmath}
\end{dgroup}

% =================================================================================================
% export checkpoints in json format

\bgroup
\CdbSetup{action=hide}
\begin{cadabra}
   for i in range( len(checkpoint) ):
      cdblib.put ('check{:03d}'.format(i),checkpoint[i],checkpoint_file)
\end{cadabra}
\egroup

\end{document}


\begin{align*}
   y^{a} = \ny{0}^{a} + \ny{2}^{a} + \ny{3}^{a} + \ny{4}^{a} + \ny{5}^{a} + \BigO{\eps^6}
\end{align*}

\begin{dgroup*}
   \begin{dmath*} \ny{0}^{a} = Dx^{a} \end{dmath*}
\end{dgroup*}

\begin{dgroup*}
   \begin{dmath*} \ny{2}^{a} = \ny{2}^{a}_1 \end{dmath*}
   \begin{dmath*}   3 \ny{2}^{a}_1 = \cdb{xDxterm12.102} \end{dmath*}
\end{dgroup*}

\begin{dgroup*}
   \begin{dmath*} \ny{3}^{a} = \ny{3}^{a}_1 + \ny{3}^{a}_2 \end{dmath*}
   \begin{dmath*} -12 \ny{3}^{a}_1 = \cdb{xDxterm13.102} \end{dmath*}
   \begin{dmath*} -24 \ny{3}^{a}_2 = \cdb{xDxterm22.102} \end{dmath*}
\end{dgroup*}

\begin{dgroup*}
   \begin{dmath*} \ny{4}^{a} = \ny{4}^{a}_1 + \ny{4}^{a}_2 + \ny{4}^{a}_3 \end{dmath*}
   \begin{dmath*} -180 \ny{4}^{a}_1 = \cdb{xDxterm14.102} \end{dmath*}
   \begin{dmath*} -720 \ny{4}^{a}_2 = \cdb{xDxterm23.102} \end{dmath*}
   \begin{dmath*} -720 \ny{4}^{a}_3 = \cdb{xDxterm32.102} \end{dmath*}
\end{dgroup*}

\begin{dgroup*}
   \begin{dmath*} \ny{5}^{a} = \ny{5}^{a}_1 + \ny{5}^{a}_2 + \ny{5}^{a}_3 + \ny{5}^{a}_4 \end{dmath*}
   \begin{dmath*}  -360 \ny{5}^{a}_1 = \cdb{xDxterm15.102} \end{dmath*}
   \begin{dmath*} -2160 \ny{5}^{a}_2 = \cdb{xDxterm24.102} \end{dmath*}
   \begin{dmath*} -1080 \ny{5}^{a}_3 = \cdb{xDxterm33.102} \end{dmath*}
   \begin{dmath*}  -360 \ny{5}^{a}_4 = \cdb{xDxterm42.102} \end{dmath*}
\end{dgroup*}

\end{document}
