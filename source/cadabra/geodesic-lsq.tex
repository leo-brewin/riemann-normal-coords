\def\Date{19 Jan 2024}
% \def\FileID{file:}

\documentclass[12pt]{cdblatex}

\begin{document}

% =================================================================================================
% create checkpoint file

\bgroup
\CdbSetup{action=hide}
\begin{cadabra}
   import cdblib
   checkpoint_file = 'tests/semantic/output/geodesic-lsq.json'
   cdblib.create (checkpoint_file)
   checkpoint = []
\end{cadabra}
\egroup

% =================================================================================================
\section*{Geodesic arc-length}

Give a pair of points $P$ and $Q$ the geodesic arc-length can be computed using
\begin{align}
   L_{PQ} = \int_P^Q\>\left(g_{ab}(x)\frac{dx^a}{ds}\frac{dx^b}{ds}\right)^{1/2}\>ds
\end{align}
Since the path is a geodesic the integrand is constant and thus
\begin{align}
   L^2_{PQ} = \left.g_{ab}(x)\frac{dx^a}{ds}\frac{dx^b}{ds}\right\vert_{P}
\end{align}
where $s$ is a re-scaled parameter (0 at $P$ and 1 at $Q$). The point $P$ has
RNC coordinates $x^{a}$ while the point $Q$ has coordinates $x^{a} + Dx^{a}$.

The vector $dx^a/ds$ at $P$ is given by the solution of the geodesic boundary value
problem. This was found in the previous code ({\tts geodesic-bvp}). That is
\begin{align}
   \left.\frac{dx^b}{ds}\right\vert_{P} = y^{a}
\end{align}
and thus
\begin{align}
   \label{eq:lsq}
   L^2_{PQ} = g_{ab}(x) y^{a} y^{b}
\end{align}

It is possible to directly evaluate the right hand side of (\ref{eq:lsq}) using the results from
the {\tts geodesic-bvp} and {\tts metric} codes. The result would need to be truncated (to an
order consistent with the results form those codes). But doing so would be computationaly
expensive as at least half of the terms will be thrown away. A better approach is compute just
the terms that will survive the truncation. This is done by expanding $g_{ab}(x)$ and $y^{a}$ as
a truncated series in the curvatures and its derivatives.

The $g_{ab}(x)$ and $y^{a}$ are written in a (truncated) formal power series in the curvature and
its derivatives
\begin{align}
   y^{a} &= \ny{0}^a + \ny{2}^a + \ny{3}^a + \ny{4}^a + \ny{5}^a + \BigO{\eps^6}\\
   g_{a b}(x) &=   \ngab{0}_{a b}
                 + \ngab{2}_{a b}
                 + \ngab{3}_{a b}
                 + \ngab{4}_{a b}
                 + \ngab{5}_{a b}
                 + \BigO{\eps^6}
\end{align}
Note that this use of $\ny{i}$ differs from that used in {\tts geodesic-bvp}. Here the
index above $y^{a}$ denotes a particular term in the curvature expansion while in
{\tts geodesic-bvp} the index denoted the iteration number (in the fixed point scheme
used to solve the BVP for $y^{a}$).

% \clarepage

% =================================================================================================
\section*{Stage 1}
The formal curvature expansions are substituted into equation (\ref{eq:lsq}), expanded and
truncated to retain terms of order $\BigO{\eps^5}$ or less. The expansion to 4th order terms is
as follows.

\begin{dgroup*}
   \begin{dmath*} L^2_{PQ} = \cdb{lsq4.002} \end{dmath*}
\end{dgroup*}

\documentclass[12pt]{cdblatex}
\usepackage{fancyhdr}
\usepackage{footer}

\begin{document}

% =================================================================================================
% create checkpoint file

\bgroup
\CdbSetup{action=hide}
\begin{cadabra}
   import cdblib
   checkpoint_file = 'tests/semantic/output/rnc2rnc.json'
   cdblib.create (checkpoint_file)
   checkpoint = []
\end{cadabra}
\egroup

% =================================================================================================
\section*{From one RNC to another}

Consider an RNC frame with RNC cooridnates $x^{a}$.

In the {\tts geodesic-bvp} code the two point boundary value problem (for the geodesic connecting
two points) was solved. There is a bonus in that calculation -- it can be trivaly adapted to the
case of transforming form one RNC into another.

The starting point is the basic equation for the geodesic connecting $P$ (with coordinaties
$x^{a}$) to Q (with coordinates $x^{a} + Dx^{a}$)
\begin{equation*}
   x^a(s) = x^a_i + s y^a - \sum_{k=2}^\infty\>\frac{1}{k!}\>\Gamma^{a}{}_{\ubk}y^{.\ubk} s^k
\end{equation*}
The affine parameter $s$ varies form 0 (at $P$) to 1 (at $Q$).

A new RNC frame, with origin at $P$, can be defined via the $y^{a}$ with the coordinates of $Q$ in
the new RNC frame defined by $y^{a}$ (since $s=1$ at $Q$). Recall that in an RNC all geodesics
through the origin are described by $y^{a}(s) = s y^{a}$. Thus the transformation from $x^a$ to
$y^a$ satisfies
\begin{equation*}
   x^a = x^a_i + y^a - \sum_{k=2}^\infty\>\frac{1}{k!}\>\Gamma^{a}{}_{\ubk}y^{.\ubk}
\end{equation*}
where the $\Gamma^{a}{}_{\ubk}$ are the generalised connections of the $x^a$ frame evaluated at
$x^a=0$. This equation can be inverted to express $y^a$ in terms of $x^a$. This computation is
done in the {\tts geodesic-bvp} code -- we only quote the results here (at the end).

The new $y^a$ frame has origin at $P$. Its coordinate axes are aligned with those (at $P$) of the
origianl RNC frame. To see this just note that $\partial x^a/\partial y^b = \delta^a_b$ at $P$.
Thus the metric at $P$ in the new frame has values $g_{ab}(x)$ (i.e., exactly those of the
original RNC frame). Note that this means that the coordinate axes of the new frame are not
ncessarily orthogonal.

The calculations in this code are trivial. It uses the $y^{a}$ found in {\tts geodesic-bvp} as
the basis of the transformation from $x^{a}$ to $y^{a}$. Most of the code involves reformatting
the $y^{a}$.

\clearpage

\begin{cadabra}
   {a,b,c,d,e,f,g,h,i,j,k,l,m,n,o,p,q,r,s,t,u,v,w#}::Indices(position=independent).

   \nabla{#}::Derivative.

   g_{a b}::Metric.
   g^{a b}::InverseMetric.

   R_{a b c d}::RiemannTensor.
   R^{a}_{b c d}::RiemannTensor.

   # Dx{#}::LaTeXForm{"{\Dx}"}.  # LCB: currently causes a bug, it kills ::KeepWeight for Dx

   import cdblib

   Y5 = cdblib.get ('y5','geodesic-bvp.json')

   Y50 = cdblib.get ('y50','geodesic-bvp.json')
   Y52 = cdblib.get ('y52','geodesic-bvp.json')
   Y53 = cdblib.get ('y53','geodesic-bvp.json')
   Y54 = cdblib.get ('y54','geodesic-bvp.json')
   Y55 = cdblib.get ('y55','geodesic-bvp.json')

   # this copies y5* from geodesic-bvp.json to rnc2rnc.json

   cdblib.create ('rnc2rnc.json')

   cdblib.put ('rnc2rnc',Y5,'rnc2rnc.json')

   cdblib.put ('rnc2rnc0',Y50,'rnc2rnc.json')
   cdblib.put ('rnc2rnc2',Y52,'rnc2rnc.json')
   cdblib.put ('rnc2rnc3',Y53,'rnc2rnc.json')
   cdblib.put ('rnc2rnc4',Y54,'rnc2rnc.json')
   cdblib.put ('rnc2rnc5',Y55,'rnc2rnc.json')

\end{cadabra}

% =================================================================================================
% the remaining code is just for pretty printing

\clearpage

\begin{cadabra}
   # note: keeping numbering as is (out of order) to ensure R appears before \nabla R etc.
   def product_sort (obj):
       substitute (obj,$ x^{a}                            -> A001^{a}               $)
       substitute (obj,$ Dx^{a}                           -> A002^{a}               $)
       substitute (obj,$ g^{a b}                          -> A003^{a b}             $)
       substitute (obj,$ \nabla_{e f g h}{R_{a b c d}}    -> A008_{a b c d e f g h} $)
       substitute (obj,$ \nabla_{e f g}{R_{a b c d}}      -> A007_{a b c d e f g}   $)
       substitute (obj,$ \nabla_{e f}{R_{a b c d}}        -> A006_{a b c d e f}     $)
       substitute (obj,$ \nabla_{e}{R_{a b c d}}          -> A005_{a b c d e}       $)
       substitute (obj,$ R_{a b c d}                      -> A004_{a b c d}         $)
       sort_product   (obj)
       rename_dummies (obj)
       substitute (obj,$ A001^{a}                  -> x^{a}                         $)
       substitute (obj,$ A002^{a}                  -> Dx^{a}                        $)
       substitute (obj,$ A003^{a b}                -> g^{a b}                       $)
       substitute (obj,$ A004_{a b c d}            -> R_{a b c d}                   $)
       substitute (obj,$ A005_{a b c d e}          -> \nabla_{e}{R_{a b c d}}       $)
       substitute (obj,$ A006_{a b c d e f}        -> \nabla_{e f}{R_{a b c d}}     $)
       substitute (obj,$ A007_{a b c d e f g}      -> \nabla_{e f g}{R_{a b c d}}   $)
       substitute (obj,$ A008_{a b c d e f g h}    -> \nabla_{e f g h}{R_{a b c d}} $)

       return obj

   def get_xDxterm (obj,n,m):

       x^{a}::Weight(label=numx,value=1).
       Dx^{a}::Weight(label=numDx,value=1).

       tmp := @(obj).
       distribute  (tmp)

       foo = Ex("numx = " + str(n))
       bah = Ex("numDx = " + str(m))
       keep_weight (tmp, foo)
       keep_weight (tmp, bah)

       return tmp

   def reformat (obj,scale):
       foo  = Ex(str(scale))
       bah := @(foo) @(obj).
       distribute     (bah)
       bah = product_sort (bah)
       rename_dummies (bah)
       canonicalise   (bah)
       substitute     (bah,$Dx^{b}->zzz^{b}$)
       factor_out     (bah,$x^{a?},zzz^{b?}$)
       substitute     (bah,$zzz^{b}->Dx^{b}$)
       ans := @(bah) / @(foo).
       return ans

   def rescale (obj,scale):
       foo  = Ex(str(scale))
       bah := @(foo) @(obj).
       distribute  (bah)
       substitute  (bah,$Dx^{b}->zzz^{b}$)
       factor_out  (bah,$x^{a?},zzz^{b?}$)
       substitute  (bah,$zzz^{b}->Dx^{b}$)
       return bah

   term0 := @(Y50).  # cdb (term0.101,term0)
   term2 := @(Y52).  # cdb (term2.101,term2)
   term3 := @(Y53).  # cdb (term3.101,term3)
   term4 := @(Y54).  # cdb (term4.101,term4)
   term5 := @(Y55).  # cdb (term5.101,term5)

   term0 = reformat (term0,1)  # cdb (term0.102,term0)
   term2 = reformat (term2,1)  # cdb (term2.102,term2)
   term3 = reformat (term3,1)  # cdb (term3.102,term3)
   term4 = reformat (term4,1)  # cdb (term4.102,term4)
   term5 = reformat (term5,1)  # cdb (term5.102,term5)

   xDxterm12 = get_xDxterm (term2,1,2)   # cdb(xDxterm12.101,xDxterm12)

   xDxterm13 = get_xDxterm (term3,1,3)   # cdb(xDxterm13.101,xDxterm13)
   xDxterm22 = get_xDxterm (term3,2,2)   # cdb(xDxterm22.101,xDxterm22)

   xDxterm14 = get_xDxterm (term4,1,4)   # cdb(xDxterm14.101,xDxterm14)
   xDxterm23 = get_xDxterm (term4,2,3)   # cdb(xDxterm23.101,xDxterm23)
   xDxterm32 = get_xDxterm (term4,3,2)   # cdb(xDxterm32.101,xDxterm32)

   xDxterm15 = get_xDxterm (term5,1,5)   # cdb(xDxterm15.101,xDxterm15)
   xDxterm24 = get_xDxterm (term5,2,4)   # cdb(xDxterm24.101,xDxterm24)
   xDxterm33 = get_xDxterm (term5,3,3)   # cdb(xDxterm33.101,xDxterm33)
   xDxterm42 = get_xDxterm (term5,4,2)   # cdb(xDxterm42.101,xDxterm42)


   xDxterm12 = rescale ( reformat (xDxterm12,    3),     3 )   # cdb(xDxterm12.102,xDxterm12)

   xDxterm13 = rescale ( reformat (xDxterm13,   12),   -12 )   # cdb(xDxterm13.102,xDxterm13)
   xDxterm22 = rescale ( reformat (xDxterm22,   24),   -24 )   # cdb(xDxterm22.102,xDxterm22)

   xDxterm14 = rescale ( reformat (xDxterm14,  180),  -180 )   # cdb(xDxterm14.102,xDxterm14)
   xDxterm23 = rescale ( reformat (xDxterm23,  720),  -720 )   # cdb(xDxterm23.102,xDxterm23)
   xDxterm32 = rescale ( reformat (xDxterm32,  720),  -720 )   # cdb(xDxterm32.102,xDxterm32)

   xDxterm15 = rescale ( reformat (xDxterm15,  360),  -360 )   # cdb(xDxterm15.102,xDxterm15)
   xDxterm24 = rescale ( reformat (xDxterm24, 2160), -2160 )   # cdb(xDxterm24.102,xDxterm24)
   xDxterm33 = rescale ( reformat (xDxterm33, 1080), -1080 )   # cdb(xDxterm33.102,xDxterm33)
   xDxterm42 = rescale ( reformat (xDxterm42,  360),  -360 )   # cdb(xDxterm42.102,xDxterm42)

   checkpoint.append (term0)
   checkpoint.append (term2)
   checkpoint.append (term3)
   checkpoint.append (term4)
   checkpoint.append (term5)

\end{cadabra}

\clearpage

% =================================================================================================
\section*{Tranformation between two RNC frames}

\begin{align*}
     y^{a} = \ny{0}^{a} + \ny{2}^{a} + \ny{3}^{a} + \ny{4}^{a} + \ny{5}^{a} + \BigO{\eps^6}
\end{align*}

\begin{dgroup*}
   \begin{dmath*} \ny{0}^{a} = \cdb{term0.102} \end{dmath*}
   \begin{dmath*} \ny{2}^{a} = \cdb{term2.102} \end{dmath*}
   \begin{dmath*} \ny{3}^{a} = \cdb{term3.102} \end{dmath*}
   \begin{dmath*} \ny{4}^{a} = \cdb{term4.102} \end{dmath*}
   \begin{dmath*} \ny{5}^{a} = \cdb{term5.102} \end{dmath*}
\end{dgroup*}

\clearpage

% =================================================================================================
\section*{Tranformation between two RNC frames}

Same as before but with an improved format (maybe) for the expressions.

\begin{align}
   y^{a} = \ny{0}^{a} + \ny{2}^{a} + \ny{3}^{a} + \ny{4}^{a} + \ny{5}^{a} + \BigO{\eps^6}
\end{align}

\begin{dgroup}
   \begin{dmath} \ny{0}^{a} = Dx^{a} \end{dmath}
\end{dgroup}

\begin{dgroup}
   \begin{dmath} \ny{2}^{a} = \ny{2}^{a}_1 \end{dmath}
   \begin{dmath}   3 \ny{2}^{a}_1 = \cdb{xDxterm12.102} \end{dmath}
\end{dgroup}

\begin{dgroup}
   \begin{dmath} \ny{3}^{a} = \ny{3}^{a}_1 + \ny{3}^{a}_2 \end{dmath}
   \begin{dmath} -12 \ny{3}^{a}_1 = \cdb{xDxterm13.102} \end{dmath}
   \begin{dmath} -24 \ny{3}^{a}_2 = \cdb{xDxterm22.102} \end{dmath}
\end{dgroup}

\begin{dgroup}
   \begin{dmath} \ny{4}^{a} = \ny{4}^{a}_1 + \ny{4}^{a}_2 + \ny{4}^{a}_3 \end{dmath}
   \begin{dmath} -180 \ny{4}^{a}_1 = \cdb{xDxterm14.102} \end{dmath}
   \begin{dmath} -720 \ny{4}^{a}_2 = \cdb{xDxterm23.102} \end{dmath}
   \begin{dmath} -720 \ny{4}^{a}_3 = \cdb{xDxterm32.102} \end{dmath}
\end{dgroup}

\begin{dgroup}
   \begin{dmath} \ny{5}^{a} = \ny{5}^{a}_1 + \ny{5}^{a}_2 + \ny{5}^{a}_3 + \ny{5}^{a}_4 \end{dmath}
   \begin{dmath}  -360 \ny{5}^{a}_1 = \cdb{xDxterm15.102} \end{dmath}
   \begin{dmath} -2160 \ny{5}^{a}_2 = \cdb{xDxterm24.102} \end{dmath}
   \begin{dmath} -1080 \ny{5}^{a}_3 = \cdb{xDxterm33.102} \end{dmath}
   \begin{dmath}  -360 \ny{5}^{a}_4 = \cdb{xDxterm42.102} \end{dmath}
\end{dgroup}

% =================================================================================================
% export checkpoints in json format

\bgroup
\CdbSetup{action=hide}
\begin{cadabra}
   for i in range( len(checkpoint) ):
      cdblib.put ('check{:03d}'.format(i),checkpoint[i],checkpoint_file)
\end{cadabra}
\egroup

\end{document}

\documentclass[12pt]{cdblatex}

\begin{document}

\section*{\jobname}

\CdbSetup{action=hide}

\begin{cadabra}
   import shared

   import cdblib

   term00A = cdblib.get ('check000','expected/metric.json')
   term01A = cdblib.get ('check001','expected/metric.json')
   term02A = cdblib.get ('check002','expected/metric.json')
   term03A = cdblib.get ('check003','expected/metric.json')
   term04A = cdblib.get ('check004','expected/metric.json')
   term05A = cdblib.get ('check005','expected/metric.json')
   term06A = cdblib.get ('check005','expected/metric.json')
   term07A = cdblib.get ('check005','expected/metric.json')

   term00B = cdblib.get ('check000','output/metric.json')
   term01B = cdblib.get ('check001','output/metric.json')
   term02B = cdblib.get ('check002','output/metric.json')
   term03B = cdblib.get ('check003','output/metric.json')
   term04B = cdblib.get ('check004','output/metric.json')
   term05B = cdblib.get ('check005','output/metric.json')
   term06B = cdblib.get ('check005','output/metric.json')
   term07B = cdblib.get ('check005','output/metric.json')

   # bug: can't push this function into shared.py
   #      no synatx error, but cadabra doesn't cancel equal terms
   # see ~/cadabra/bugs/bug02

   def check (objA,objB):
       tmp := @(objA) - @(objB).
       distribute         (tmp)
       tmp = shared.standard_indices (tmp)
       tmp = shared.product_sort (tmp)
       rename_dummies     (tmp)
       canonicalise       (tmp)

       return tmp

   diff000 = shared.check (term00A,term00B)   # cdb (diff000,diff000)
   diff001 = shared.check (term01A,term01B)   # cdb (diff001,diff001)
   diff002 = shared.check (term02A,term02B)   # cdb (diff002,diff002)
   diff003 = shared.check (term03A,term03B)   # cdb (diff003,diff003)
   diff004 = shared.check (term04A,term04B)   # cdb (diff004,diff004)
   diff005 = shared.check (term05A,term05B)   # cdb (diff005,diff005)
   diff006 = shared.check (term06A,term06B)   # cdb (diff006,diff006)
   diff007 = shared.check (term07A,term07B)   # cdb (diff007,diff007)

\end{cadabra}

\begin{dgroup*}
   \Dmath*{ \cdb*{diff000} }
   \Dmath*{ \cdb*{diff001} }
   \Dmath*{ \cdb*{diff002} }
   \Dmath*{ \cdb*{diff003} }
   \Dmath*{ \cdb*{diff004} }
   \Dmath*{ \cdb*{diff005} }
   \Dmath*{ \cdb*{diff006} }
   \Dmath*{ \cdb*{diff007} }
\end{dgroup*}

\end{document}


From {\tts geodesic-bvp} (actually from {\tts rnc2rnc which reformatted the results nicely}) we have
% \begin{align*}
%      y^{a} = \ny{0}^{a} + \ny{2}^{a} + \ny{3}^{a} + \ny{4}^{a} + \BigO{\eps^5}
% \end{align*}

\begin{dgroup*}
   \begin{dmath*} \ny{0}^{a} = \cdb{term0.102} \end{dmath*}
   \begin{dmath*} \ny{2}^{a} = \cdb{term2.102} \end{dmath*}
   \begin{dmath*} \ny{3}^{a} = \cdb{term3.102} \end{dmath*}
   \begin{dmath*} \ny{4}^{a} = \cdb{term4.102} \end{dmath*}
\end{dgroup*}

and from {\tts metric} we have

% \begin{align*}
%      g_{a b}(x) =
%      \ngab{0}_{a b}
%    + \ngab{2}_{a b}
%    + \ngab{3}_{a b}
%    + \ngab{4}_{a b}
%    + \BigO{\eps^6}
% \end{align*}
\begin{dgroup*}
   \begin{dmath*}     \ngab{0}_{a b} = \cdb{scaled0.601} \end{dmath*}
   \begin{dmath*}   3 \ngab{2}_{a b} = \cdb{scaled2.601} \end{dmath*}
   \begin{dmath*}   6 \ngab{3}_{a b} = \cdb{scaled3.601} \end{dmath*}
   \begin{dmath*} 180 \ngab{4}_{a b} = \cdb{scaled4.601} \end{dmath*}
\end{dgroup*}

% =================================================================================================
\section*{Stage 2}
The results from the {\tts geodesic-bvp} and {\tts metric} codes are read to provide
values for the $\ny{n}^{a}$ and $\ngab{m}_{ab}$. These are substituted into the result from
Stage 1, et volia, the final answer. To 4th-order terms the result is given by

\Dmath*{ L^2_{PQ} = \cdb{lsq5.301} + \BigO{\eps^5}}

\clearpage

% =================================================================================================
\section*{Shared properties}

\begin{cadabra}
   {a,b,c,d,e,f,g,h,i,j,k,l,m,n,o,p,q,r,s,t,u,v,w#}::Indices(position=independent).

   D{#}::Derivative.
   \nabla{#}::Derivative.
   \partial{#}::PartialDerivative.

   g_{a b}::Metric.
   g^{a b}::InverseMetric.
   g_{a}^{b}::KroneckerDelta.
   g^{a}_{b}::KroneckerDelta.
   \delta^{a}_{b}::KroneckerDelta.
   \delta_{a}^{b}::KroneckerDelta.

   R_{a b c d}::RiemannTensor.
   R^{a}_{b c d}::RiemannTensor.
   R_{a b c}^{d}::RiemannTensor.

   \Gamma^{a}_{b c}::TableauSymmetry(shape={2}, indices={1,2}).
   \Gamma^{a}_{b c d}::TableauSymmetry(shape={3}, indices={1,2,3}).
   \Gamma^{a}_{b c d e}::TableauSymmetry(shape={4}, indices={1,2,3,4}).
   \Gamma^{a}_{b c d e f}::TableauSymmetry(shape={5}, indices={1,2,3,4,5}).

   x^{a}::Depends(D{#}).

   g_{a b}::Depends(\partial{#}).
   R_{a b c d}::Depends(\partial{#}).
   R^{a}_{b c d}::Depends(\partial{#}).
   \Gamma^{a}_{b c}::Depends(\partial{#}).

   R_{a b c d}::Depends(\nabla{#}).
   R^{a}_{b c d}::Depends(\nabla{#}).

   g0{#}::LaTeXForm("\ngab{0}").
   g2{#}::LaTeXForm("\ngab{2}").
   g3{#}::LaTeXForm("\ngab{3}").
   g4{#}::LaTeXForm("\ngab{4}").
   g5{#}::LaTeXForm("\ngab{5}").

   y0{#}::LaTeXForm("\ny{0}").
   y2{#}::LaTeXForm("\ny{2}").
   y3{#}::LaTeXForm("\ny{3}").
   y4{#}::LaTeXForm("\ny{4}").
   y5{#}::LaTeXForm("\ny{5}").

\end{cadabra}

\clearpage

% =================================================================================================
\section*{Stage 1: The formal expansion}

\begin{cadabra}
   g0_{a b}::Symmetric.
   g2_{a b}::Symmetric.
   g3_{a b}::Symmetric.
   g4_{a b}::Symmetric.
   g5_{a b}::Symmetric.

   g0_{a b}::Weight(label=num,value=0).
   g2_{a b}::Weight(label=num,value=2).
   g3_{a b}::Weight(label=num,value=3).
   g4_{a b}::Weight(label=num,value=4).
   g5_{a b}::Weight(label=num,value=5).

   y0^{a}::Weight(label=num,value=0).
   y2^{a}::Weight(label=num,value=2).
   y3^{a}::Weight(label=num,value=3).
   y4^{a}::Weight(label=num,value=4).
   y5^{a}::Weight(label=num,value=5).

   # note: keeping numbering as is (out of order) to ensure R appears before \nabla R etc.
   def product_sort (obj):
       substitute (obj,$ A^{a}                            -> A001^{a}               $)
       substitute (obj,$ x^{a}                            -> A002^{a}               $)
       substitute (obj,$ Dx^{a}                           -> A003^{a}               $)
       substitute (obj,$ g_{a b}                          -> A004_{a b}             $)
       substitute (obj,$ g^{a b}                          -> A005^{a b}             $)
       substitute (obj,$ \nabla_{e f g h}{R_{a b c d}}    -> A010_{a b c d e f g h} $)
       substitute (obj,$ \nabla_{e f g}{R_{a b c d}}      -> A009_{a b c d e f g}   $)
       substitute (obj,$ \nabla_{e f}{R_{a b c d}}        -> A008_{a b c d e f}     $)
       substitute (obj,$ \nabla_{e}{R_{a b c d}}          -> A007_{a b c d e}       $)
       substitute (obj,$ R_{a b c d}                      -> A006_{a b c d}         $)
       sort_product   (obj)
       rename_dummies (obj)
       substitute (obj,$ A001^{a}                  -> A^{a}                         $)
       substitute (obj,$ A002^{a}                  -> x^{a}                         $)
       substitute (obj,$ A003^{a}                  -> Dx^{a}                        $)
       substitute (obj,$ A004_{a b}                -> g_{a b}                       $)
       substitute (obj,$ A005^{a b}                -> g^{a b}                       $)
       substitute (obj,$ A006_{a b c d}            -> R_{a b c d}                   $)
       substitute (obj,$ A007_{a b c d e}          -> \nabla_{e}{R_{a b c d}}       $)
       substitute (obj,$ A008_{a b c d e f}        -> \nabla_{e f}{R_{a b c d}}     $)
       substitute (obj,$ A009_{a b c d e f g}      -> \nabla_{e f g}{R_{a b c d}}   $)
       substitute (obj,$ A010_{a b c d e f g h}    -> \nabla_{e f g h}{R_{a b c d}} $)

       return obj

   def truncate (obj,n):
       ans = Ex(0)

       for i in range (0,n+1):
          foo := @(obj).
          bah = Ex("num = " + str(i))
          keep_weight (foo, bah)
          ans = ans + foo

       return ans

   # expansions wrt the curvature

   defgab := g_{a b} -> g0_{a b} + g2_{a b} + g3_{a b} + g4_{a b} + g5_{a b}.
   defy   := y^{a}   -> y0^{a} + y2^{a} + y3^{a} + y4^{a} + y5^{a}.

   lsq    := g_{a b} y^{a} y^{b}.

   substitute (lsq,defgab)
   substitute (lsq,defy)
   distribute (lsq)

   def tidy (obj):
       foo := @(obj).
       sort_product    (foo)
       rename_dummies  (foo)
       canonicalise    (foo)
       return foo

   lsq0 = tidy ( truncate (lsq,0) )  # cdb (lsq0.002,lsq0)
   lsq2 = tidy ( truncate (lsq,2) )  # cdb (lsq2.002,lsq2)
   lsq3 = tidy ( truncate (lsq,3) )  # cdb (lsq3.002,lsq3)
   lsq4 = tidy ( truncate (lsq,4) )  # cdb (lsq4.002,lsq4)
   lsq5 = tidy ( truncate (lsq,5) )  # cdb (lsq5.002,lsq5)

   d20 := @(lsq2) - @(lsq0).         # cdb (d20.001,d20)   # check, should contain only O(2) terms
   d32 := @(lsq3) - @(lsq2).         # cdb (d32.001,d32)   # check, should contain only O(3) terms
   d43 := @(lsq4) - @(lsq3).         # cdb (d43.001,d43)   # check, should contain only O(4) terms
   d54 := @(lsq5) - @(lsq4).         # cdb (d54.001,d54)   # check, should contain only O(5) terms

   d5 := @(lsq5) - @(lsq).           # cdb (d5.001,d5)
   d5  = tidy (d5)                   # cdb (d5.002,d5)  # all higher order terms, should see no O(5) terms

\end{cadabra}

\clearpage

\begin{dgroup*}
   \begin{dmath*} \cdb*{lsq0.002} \end{dmath*}
   \begin{dmath*} \cdb*{lsq2.002} \end{dmath*}
   \begin{dmath*} \cdb*{lsq3.002} \end{dmath*}
   \begin{dmath*} \cdb*{lsq4.002} \end{dmath*}
   \begin{dmath*} \cdb*{lsq5.002} \end{dmath*}
\end{dgroup*}

\begin{dgroup*}
   \begin{dmath*} \cdb*{d20.001} \end{dmath*}
   \begin{dmath*} \cdb*{d32.001} \end{dmath*}
   \begin{dmath*} \cdb*{d43.001} \end{dmath*}
   \begin{dmath*} \cdb*{d54.001} \end{dmath*}
   \begin{dmath*} \cdb*{d5.002} \end{dmath*}
\end{dgroup*}

\clearpage

% =================================================================================================
\section*{Stage 2: Substution of $\ny{n}^{a}$ and $\ngab{m}_{ab}$}

\begin{cadabra}
   import cdblib

   g0ab = cdblib.get('g_ab_0','metric.json')
   g2ab = cdblib.get('g_ab_2','metric.json')
   g3ab = cdblib.get('g_ab_3','metric.json')
   g4ab = cdblib.get('g_ab_4','metric.json')
   g5ab = cdblib.get('g_ab_5','metric.json')

   defg0ab := g0_{a b} -> @(g0ab).
   defg2ab := g2_{a b} -> @(g2ab).
   defg3ab := g3_{a b} -> @(g3ab).
   defg4ab := g4_{a b} -> @(g4ab).
   defg5ab := g5_{a b} -> @(g5ab).

   y0a = cdblib.get('y50','geodesic-bvp.json')
   y2a = cdblib.get('y52','geodesic-bvp.json')
   y3a = cdblib.get('y53','geodesic-bvp.json')
   y4a = cdblib.get('y54','geodesic-bvp.json')
   y5a = cdblib.get('y55','geodesic-bvp.json')

   defy0a := y0^{a} -> @(y0a).
   defy2a := y2^{a} -> @(y2a).
   defy3a := y3^{a} -> @(y3a).
   defy4a := y4^{a} -> @(y4a).
   defy5a := y5^{a} -> @(y5a).

   def substitute_gab_ya (obj):

      foo := @(obj).

      substitute (foo,defg0ab)
      substitute (foo,defg2ab)
      substitute (foo,defg3ab)
      substitute (foo,defg4ab)
      substitute (foo,defg5ab)

      substitute (foo,defy0a)
      substitute (foo,defy2a)
      substitute (foo,defy3a)
      substitute (foo,defy4a)
      substitute (foo,defy5a)

      distribute     (foo)
      sort_product   (foo)
      rename_dummies (foo)
      canonicalise   (foo)

      substitute     (foo,$g_{a b} g^{c b} -> \delta^{c}_{a}$)
      eliminate_kronecker (foo)
      foo = product_sort  (foo)
      rename_dummies      (foo)
      canonicalise        (foo)

      return foo

   def get_Rterm (obj,n):

   # I would like to assign different weights to \nabla_{a}, \nabla_{a b}, \nabla_{a b c} etc. but no matter
   # what I do it appears that Cadabra assigns the same weight to all of these regardless of the number of subscripts.
   # It seems that the weight is assigned to the symbol \nabla alone. So I'm forced to use the following substitution trick.

       Q_{a b c d}::Weight(label=numR,value=2).
       Q_{a b c d e}::Weight(label=numR,value=3).
       Q_{a b c d e f}::Weight(label=numR,value=4).
       Q_{a b c d e f g}::Weight(label=numR,value=5).

       tmp := @(obj).

       distribute (tmp)

       substitute (tmp, $\nabla_{e f g}{R_{a b c d}} -> Q_{a b c d e f g}$)
       substitute (tmp, $\nabla_{e f}{R_{a b c d}} -> Q_{a b c d e f}$)
       substitute (tmp, $\nabla_{e}{R_{a b c d}} -> Q_{a b c d e}$)
       substitute (tmp, $R_{a b c d} -> Q_{a b c d}$)

       foo := @(tmp).
       bah = Ex("numR = " + str(n))
       keep_weight (foo, bah)

       substitute (foo, $Q_{a b c d e f g} -> \nabla_{e f g}{R_{a b c d}}$)
       substitute (foo, $Q_{a b c d e f} -> \nabla_{e f}{R_{a b c d}}$)
       substitute (foo, $Q_{a b c d e} -> \nabla_{e}{R_{a b c d}}$)
       substitute (foo, $Q_{a b c d} -> R_{a b c d}$)

       return foo

   lsq2 = substitute_gab_ya (lsq2)  # cdb (lsq2.101,lsq2)
   lsq3 = substitute_gab_ya (lsq3)  # cdb (lsq3.101,lsq3)
   lsq4 = substitute_gab_ya (lsq4)  # cdb (lsq4.101,lsq4)
   lsq5 = substitute_gab_ya (lsq5)  # cdb (lsq5.101,lsq5)

   lsq50 = get_Rterm (lsq5,0)
   lsq52 = get_Rterm (lsq5,2)
   lsq53 = get_Rterm (lsq5,3)
   lsq54 = get_Rterm (lsq5,4)
   lsq55 = get_Rterm (lsq5,5)

   cdblib.create ('geodesic-lsq.json')

   cdblib.put ('lsq2',lsq2,'geodesic-lsq.json')
   cdblib.put ('lsq3',lsq3,'geodesic-lsq.json')
   cdblib.put ('lsq4',lsq4,'geodesic-lsq.json')
   cdblib.put ('lsq5',lsq5,'geodesic-lsq.json')

   cdblib.put ('lsq50',lsq50,'geodesic-lsq.json')
   cdblib.put ('lsq52',lsq52,'geodesic-lsq.json')
   cdblib.put ('lsq53',lsq53,'geodesic-lsq.json')
   cdblib.put ('lsq54',lsq54,'geodesic-lsq.json')
   cdblib.put ('lsq55',lsq55,'geodesic-lsq.json')

\end{cadabra}

\clearpage
\begin{dgroup*}
   \begin{dmath*} \cdb*{lsq2.101} \end{dmath*}
   \begin{dmath*} \cdb*{lsq3.101} \end{dmath*}
   \begin{dmath*} \cdb*{lsq4.101} \end{dmath*}
   \begin{dmath*} \cdb*{lsq5.101} \end{dmath*}
\end{dgroup*}

% =================================================================================================
% the remaining code is just for pretty printing

\clearpage

% =================================================================================================
\section*{Stage 3: Reformatting}

\begin{cadabra}
   def reformat (obj,scale):
      foo  = Ex(str(scale))
      bah := @(foo) @(obj).
      distribute     (bah)
      bah = product_sort (bah)
      rename_dummies (bah)
      canonicalise   (bah)
      substitute     (bah,$Dx^{b}->zzz^{b}$)
      factor_out     (bah,$x^{a?},zzz^{b?}$)
      substitute     (bah,$zzz^{b}->Dx^{b}$)
      ans := @(bah) / @(foo).
      return ans

   def rescale (obj,scale):
      foo  = Ex(str(scale))
      bah := @(foo) @(obj).
      distribute  (bah)
      substitute  (bah,$Dx^{b}->zzz^{b}$)
      factor_out  (bah,$x^{a?},zzz^{b?}$)
      substitute  (bah,$zzz^{b}->Dx^{b}$)
      return bah

   Rterm0 := @(lsq50).
   Rterm2 := @(lsq52).
   Rterm3 := @(lsq53).
   Rterm4 := @(lsq54).
   Rterm5 := @(lsq55).

   Rterm0 = reformat (Rterm0,   1)    # cdb(Rterm0.301,Rterm0) # LCB: returns Dx before g, not what I want
   Rterm2 = reformat (Rterm2,   3)    # cdb(Rterm2.301,Rterm2)
   Rterm3 = reformat (Rterm3,  12)    # cdb(Rterm3.301,Rterm3)
   Rterm4 = reformat (Rterm4, 360)    # cdb(Rterm4.301,Rterm4)
   Rterm5 = reformat (Rterm5,1080)    # cdb(Rterm5.301,Rterm5)

   Rterm0 := g_{a b} Dx^{a} Dx^{b}.   # LCB: fixes the order of terms, g before Dx,

   lsq3 := @(Rterm0) + @(Rterm2).                                      # cdb (lsq4.301,lsq3)
   lsq4 := @(Rterm0) + @(Rterm2) + @(Rterm3).                          # cdb (lsq4.301,lsq4)
   lsq5 := @(Rterm0) + @(Rterm2) + @(Rterm3) + @(Rterm4).              # cdb (lsq5.301,lsq5)
   lsq6 := @(Rterm0) + @(Rterm2) + @(Rterm3) + @(Rterm4) + @(Rterm5).  # cdb (lsq5.301,lsq6)

   lsq  := @(lsq6).                   # cdb (lsq.301,lsq)

   scaled0 = rescale (Rterm0,    1)   # cdb (scaled0.301,scaled0) # LCB: returns Dx before g, not what I want
   scaled2 = rescale (Rterm2,    3)   # cdb (scaled2.301,scaled2)
   scaled3 = rescale (Rterm3,   12)   # cdb (scaled3.301,scaled3)
   scaled4 = rescale (Rterm4,  360)   # cdb (scaled4.301,scaled4)
   scaled5 = rescale (Rterm5, 1080)   # cdb (scaled5.301,scaled5)

   scaled0 := g_{a b} Dx^{a} Dx^{b}.  # cdb (scaled0.301,scaled0) # LCB: fixes the order of terms, g before Dx, good

\end{cadabra}

\clearpage

% =================================================================================================
\section*{Geodesic arc-length}

\begin{dgroup*}[spread=5pt]
   % LCB: which of these is correct?
   % \begin{dmath*} \left(\Delta s\right)^2 = \cdb{lsq.301} + \BigO{\eps^6,Dx^6} \end{dmath*}
   % \begin{dmath*} \left(\Delta s\right)^2 = \cdb{lsq.301} + \BigO{\eps^6,Dx^7} \end{dmath*}
   \begin{dmath*} \left(\Delta s\right)^2 = \cdb{lsq.301} + \BigO{\eps^6} \end{dmath*}
\end{dgroup*}

\clearpage

% =================================================================================================
\section*{Geodesic arc-length curvature expansion}

\begin{align*}
   % LCB: which of these is correct?
   % \left(\Delta s\right)^2 = \nD{0} + \nD{2} + \nD{3} + \nD{4} + \nD{5} + \BigO{\eps^6,Dx^6}
   % \left(\Delta s\right)^2 = \nD{0} + \nD{2} + \nD{3} + \nD{4} + \nD{5} + \BigO{\eps^6,Dx^7}
   \left(\Delta s\right)^2 = \nD{0} + \nD{2} + \nD{3} + \nD{4} + \nD{5} + \BigO{\eps^6}
\end{align*}

\begin{dgroup*}[spread=5pt]
   \begin{dmath*}      \nD{0} = \cdb{scaled0.301} \end{dmath*}
   \begin{dmath*}    3 \nD{2} = \cdb{scaled2.301} \end{dmath*}
   \begin{dmath*}   12 \nD{3} = \cdb{scaled3.301} \end{dmath*}
   \begin{dmath*}  360 \nD{4} = \cdb{scaled4.301} \end{dmath*}
   \begin{dmath*} 1080 \nD{5} = \cdb{scaled5.301} \end{dmath*}
\end{dgroup*}

\clearpage

% =================================================================================================
% export selected objects, these will later be imported into a library
% these are the objects that will appear in the paper

\begin{cadabra}
   cdblib.create ('geodesic-lsq.export')

   # 3rd to 6th order lsq
   cdblib.put ('lsq3',lsq3,'geodesic-lsq.export')
   cdblib.put ('lsq4',lsq4,'geodesic-lsq.export')
   cdblib.put ('lsq5',lsq5,'geodesic-lsq.export')
   cdblib.put ('lsq6',lsq6,'geodesic-lsq.export')

   # 6th order lsq terms, scaled
   cdblib.put ('lsq60',scaled0,'geodesic-lsq.export')
   cdblib.put ('lsq62',scaled2,'geodesic-lsq.export')
   cdblib.put ('lsq63',scaled3,'geodesic-lsq.export')
   cdblib.put ('lsq64',scaled4,'geodesic-lsq.export')
   cdblib.put ('lsq65',scaled5,'geodesic-lsq.export')

   checkpoint.append (lsq4)

   checkpoint.append (scaled0)
   checkpoint.append (scaled2)
   checkpoint.append (scaled3)
   checkpoint.append (scaled4)
   checkpoint.append (scaled5)
\end{cadabra}

% =================================================================================================
% export checkpoints in json format

\bgroup
\CdbSetup{action=hide}
\begin{cadabra}
   for i in range( len(checkpoint) ):
      cdblib.put ('check{:03d}'.format(i),checkpoint[i],checkpoint_file)
\end{cadabra}
\egroup

\end{document}
