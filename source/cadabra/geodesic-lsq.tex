\def\Date{19 Jan 2024}
% \def\FileID{file:}

\documentclass[12pt]{cdblatex}

\begin{document}

% =================================================================================================
% create checkpoint file

\bgroup
\CdbSetup{action=hide}
\begin{cadabra}
   import cdblib
   checkpoint_file = 'tests/semantic/output/geodesic-lsq.json'
   cdblib.create (checkpoint_file)
   checkpoint = []
\end{cadabra}
\egroup

% =================================================================================================
\section*{Geodesic arc-length}

Give a pair of points $P$ and $Q$ the geodesic arc-length can be computed using
\begin{align}
   L_{PQ} = \int_P^Q\>\left(g_{ab}(x)\frac{dx^a}{ds}\frac{dx^b}{ds}\right)^{1/2}\>ds
\end{align}
Since the path is a geodesic the integrand is constant and thus
\begin{align}
   L^2_{PQ} = \left.g_{ab}(x)\frac{dx^a}{ds}\frac{dx^b}{ds}\right\vert_{P}
\end{align}
where $s$ is a re-scaled parameter (0 at $P$ and 1 at $Q$). The point $P$ has
RNC coordinates $x^{a}$ while the point $Q$ has coordinates $x^{a} + Dx^{a}$.

The vector $dx^a/ds$ at $P$ is given by the solution of the geodesic boundary value
problem. This was found in the previous code ({\tts geodesic-bvp}). That is
\begin{align}
   \left.\frac{dx^b}{ds}\right\vert_{P} = y^{a}
\end{align}
and thus
\begin{align}
   \label{eq:lsq}
   L^2_{PQ} = g_{ab}(x) y^{a} y^{b}
\end{align}

It is possible to directly evaluate the right hand side of (\ref{eq:lsq}) using the results from
the {\tts geodesic-bvp} and {\tts metric} codes. The result would need to be truncated (to an
order consistent with the results form those codes). But doing so would be computationaly
expensive as at least half of the terms will be thrown away. A better approach is compute just
the terms that will survive the truncation. This is done by expanding $g_{ab}(x)$ and $y^{a}$ as
a truncated series in the curvatures and its derivatives.

The $g_{ab}(x)$ and $y^{a}$ are written in a (truncated) formal power series in the curvature and
its derivatives
\begin{align}
   y^{a} &= \ny{0}^a + \ny{2}^a + \ny{3}^a + \ny{4}^a + \ny{5}^a + \BigO{\eps^6}\\
   g_{a b}(x) &=   \ngab{0}_{a b}
                 + \ngab{2}_{a b}
                 + \ngab{3}_{a b}
                 + \ngab{4}_{a b}
                 + \ngab{5}_{a b}
                 + \BigO{\eps^6}
\end{align}
Note that this use of $\ny{i}$ differs from that used in {\tts geodesic-bvp}. Here the
index above $y^{a}$ denotes a particular term in the curvature expansion while in
{\tts geodesic-bvp} the index denoted the iteration number (in the fixed point scheme
used to solve the BVP for $y^{a}$).

% \clarepage

% =================================================================================================
\section*{Stage 1}
The formal curvature expansions are substituted into equation (\ref{eq:lsq}), expanded and
truncated to retain terms of order $\BigO{\eps^5}$ or less. The expansion to 4th order terms is
as follows.

\begin{dgroup*}
   \begin{dmath*} L^2_{PQ} = \cdb{lsq4.002} \end{dmath*}
\end{dgroup*}

\def\Date{19 Jan 2024}
% \def\FileID{file:}

\documentclass[12pt]{cdblatex}

\begin{document}

% =================================================================================================
% create checkpoint file

\bgroup
\CdbSetup{action=hide}
\begin{cadabra}
   import cdblib
   checkpoint_file = 'tests/semantic/output/rnc2rnc.json'
   cdblib.create (checkpoint_file)
   checkpoint = []
\end{cadabra}
\egroup

% =================================================================================================
\section*{From one RNC to another}

Consider an RNC frame with RNC cooridnates $x^{a}$.

In the {\tts geodesic-bvp} code the two point boundary value problem (for the geodesic connecting
two points) was solved. There is a bonus in that calculation -- it can be trivaly adapted to the
case of transforming form one RNC into another.

The starting point is the basic equation for the geodesic connecting $P$ (with coordinaties
$x^{a}$) to Q (with coordinates $x^{a} + Dx^{a}$)
\begin{equation*}
   x^a(s) = x^a_i + s y^a - \sum_{k=2}^\infty\>\frac{1}{k!}\>\Gamma^{a}{}_{\ubk}y^{.\ubk} s^k
\end{equation*}
The affine parameter $s$ varies form 0 (at $P$) to 1 (at $Q$).

A new RNC frame, with origin at $P$, can be defined via the $y^{a}$ with the coordinates of $Q$ in
the new RNC frame defined by $y^{a}$ (since $s=1$ at $Q$). Recall that in an RNC all geodesics
through the origin are described by $y^{a}(s) = s y^{a}$. Thus the transformation from $x^a$ to
$y^a$ satisfies
\begin{equation*}
   x^a = x^a_i + y^a - \sum_{k=2}^\infty\>\frac{1}{k!}\>\Gamma^{a}{}_{\ubk}y^{.\ubk}
\end{equation*}
where the $\Gamma^{a}{}_{\ubk}$ are the generalised connections of the $x^a$ frame evaluated at
$x^a=0$. This equation can be inverted to express $y^a$ in terms of $x^a$. This computation is
done in the {\tts geodesic-bvp} code -- we only quote the results here (at the end).

The new $y^a$ frame has origin at $P$. Its coordinate axes are aligned with those (at $P$) of the
origianl RNC frame. To see this just note that $\partial x^a/\partial y^b = \delta^a_b$ at $P$.
Thus the metric at $P$ in the new frame has values $g_{ab}(x)$ (i.e., exactly those of the
original RNC frame). Note that this means that the coordinate axes of the new frame are not
ncessarily orthogonal.

The calculations in this code are trivial. It uses the $y^{a}$ found in {\tts geodesic-bvp} as
the basis of the transformation from $x^{a}$ to $y^{a}$. Most of the code involves reformatting
the $y^{a}$.

\clearpage

\begin{cadabra}
   {a,b,c,d,e,f,g,h,i,j,k,l,m,n,o,p,q,r,s,t,u,v,w#}::Indices(position=independent).

   \nabla{#}::Derivative.

   g_{a b}::Metric.
   g^{a b}::InverseMetric.

   R_{a b c d}::RiemannTensor.
   R^{a}_{b c d}::RiemannTensor.

   # Dx{#}::LaTeXForm{"{\Dx}"}.  # LCB: currently causes a bug, it kills ::KeepWeight for Dx

   import cdblib

   Y5 = cdblib.get ('y5','geodesic-bvp.json')

   Y50 = cdblib.get ('y50','geodesic-bvp.json')
   Y52 = cdblib.get ('y52','geodesic-bvp.json')
   Y53 = cdblib.get ('y53','geodesic-bvp.json')
   Y54 = cdblib.get ('y54','geodesic-bvp.json')
   Y55 = cdblib.get ('y55','geodesic-bvp.json')

   # this copies y5* from geodesic-bvp.json to rnc2rnc.json

   cdblib.create ('rnc2rnc.json')

   cdblib.put ('rnc2rnc',Y5,'rnc2rnc.json')

   cdblib.put ('rnc2rnc0',Y50,'rnc2rnc.json')
   cdblib.put ('rnc2rnc2',Y52,'rnc2rnc.json')
   cdblib.put ('rnc2rnc3',Y53,'rnc2rnc.json')
   cdblib.put ('rnc2rnc4',Y54,'rnc2rnc.json')
   cdblib.put ('rnc2rnc5',Y55,'rnc2rnc.json')

\end{cadabra}

% =================================================================================================
% the remaining code is just for pretty printing

\clearpage

\begin{cadabra}
   # note: keeping numbering as is (out of order) to ensure R appears before \nabla R etc.
   def product_sort (obj):
       substitute (obj,$ x^{a}                            -> A001^{a}               $)
       substitute (obj,$ Dx^{a}                           -> A002^{a}               $)
       substitute (obj,$ g^{a b}                          -> A003^{a b}             $)
       substitute (obj,$ \nabla_{e f g h}{R_{a b c d}}    -> A008_{a b c d e f g h} $)
       substitute (obj,$ \nabla_{e f g}{R_{a b c d}}      -> A007_{a b c d e f g}   $)
       substitute (obj,$ \nabla_{e f}{R_{a b c d}}        -> A006_{a b c d e f}     $)
       substitute (obj,$ \nabla_{e}{R_{a b c d}}          -> A005_{a b c d e}       $)
       substitute (obj,$ R_{a b c d}                      -> A004_{a b c d}         $)
       sort_product   (obj)
       rename_dummies (obj)
       substitute (obj,$ A001^{a}                  -> x^{a}                         $)
       substitute (obj,$ A002^{a}                  -> Dx^{a}                        $)
       substitute (obj,$ A003^{a b}                -> g^{a b}                       $)
       substitute (obj,$ A004_{a b c d}            -> R_{a b c d}                   $)
       substitute (obj,$ A005_{a b c d e}          -> \nabla_{e}{R_{a b c d}}       $)
       substitute (obj,$ A006_{a b c d e f}        -> \nabla_{e f}{R_{a b c d}}     $)
       substitute (obj,$ A007_{a b c d e f g}      -> \nabla_{e f g}{R_{a b c d}}   $)
       substitute (obj,$ A008_{a b c d e f g h}    -> \nabla_{e f g h}{R_{a b c d}} $)

       return obj

   def get_xDxterm (obj,n,m):

       x^{a}::Weight(label=numx,value=1).
       Dx^{a}::Weight(label=numDx,value=1).

       tmp := @(obj).
       distribute  (tmp)

       foo = Ex("numx = " + str(n))
       bah = Ex("numDx = " + str(m))
       keep_weight (tmp, foo)
       keep_weight (tmp, bah)

       return tmp

   def reformat (obj,scale):
       foo  = Ex(str(scale))
       bah := @(foo) @(obj).
       distribute     (bah)
       bah = product_sort (bah)
       rename_dummies (bah)
       canonicalise   (bah)
       substitute     (bah,$Dx^{b}->zzz^{b}$)
       factor_out     (bah,$x^{a?},zzz^{b?}$)
       substitute     (bah,$zzz^{b}->Dx^{b}$)
       ans := @(bah) / @(foo).
       return ans

   def rescale (obj,scale):
       foo  = Ex(str(scale))
       bah := @(foo) @(obj).
       distribute  (bah)
       substitute  (bah,$Dx^{b}->zzz^{b}$)
       factor_out  (bah,$x^{a?},zzz^{b?}$)
       substitute  (bah,$zzz^{b}->Dx^{b}$)
       return bah

   term0 := @(Y50).  # cdb (term0.101,term0)
   term2 := @(Y52).  # cdb (term2.101,term2)
   term3 := @(Y53).  # cdb (term3.101,term3)
   term4 := @(Y54).  # cdb (term4.101,term4)
   term5 := @(Y55).  # cdb (term5.101,term5)

   term0 = reformat (term0,1)  # cdb (term0.102,term0)
   term2 = reformat (term2,1)  # cdb (term2.102,term2)
   term3 = reformat (term3,1)  # cdb (term3.102,term3)
   term4 = reformat (term4,1)  # cdb (term4.102,term4)
   term5 = reformat (term5,1)  # cdb (term5.102,term5)

   xDxterm12 = get_xDxterm (term2,1,2)   # cdb(xDxterm12.101,xDxterm12)

   xDxterm13 = get_xDxterm (term3,1,3)   # cdb(xDxterm13.101,xDxterm13)
   xDxterm22 = get_xDxterm (term3,2,2)   # cdb(xDxterm22.101,xDxterm22)

   xDxterm14 = get_xDxterm (term4,1,4)   # cdb(xDxterm14.101,xDxterm14)
   xDxterm23 = get_xDxterm (term4,2,3)   # cdb(xDxterm23.101,xDxterm23)
   xDxterm32 = get_xDxterm (term4,3,2)   # cdb(xDxterm32.101,xDxterm32)

   xDxterm15 = get_xDxterm (term5,1,5)   # cdb(xDxterm15.101,xDxterm15)
   xDxterm24 = get_xDxterm (term5,2,4)   # cdb(xDxterm24.101,xDxterm24)
   xDxterm33 = get_xDxterm (term5,3,3)   # cdb(xDxterm33.101,xDxterm33)
   xDxterm42 = get_xDxterm (term5,4,2)   # cdb(xDxterm42.101,xDxterm42)


   xDxterm12 = rescale ( reformat (xDxterm12,    3),     3 )   # cdb(xDxterm12.102,xDxterm12)

   xDxterm13 = rescale ( reformat (xDxterm13,   12),   -12 )   # cdb(xDxterm13.102,xDxterm13)
   xDxterm22 = rescale ( reformat (xDxterm22,   24),   -24 )   # cdb(xDxterm22.102,xDxterm22)

   xDxterm14 = rescale ( reformat (xDxterm14,  180),  -180 )   # cdb(xDxterm14.102,xDxterm14)
   xDxterm23 = rescale ( reformat (xDxterm23,  720),  -720 )   # cdb(xDxterm23.102,xDxterm23)
   xDxterm32 = rescale ( reformat (xDxterm32,  720),  -720 )   # cdb(xDxterm32.102,xDxterm32)

   xDxterm15 = rescale ( reformat (xDxterm15,  360),  -360 )   # cdb(xDxterm15.102,xDxterm15)
   xDxterm24 = rescale ( reformat (xDxterm24, 2160), -2160 )   # cdb(xDxterm24.102,xDxterm24)
   xDxterm33 = rescale ( reformat (xDxterm33, 1080), -1080 )   # cdb(xDxterm33.102,xDxterm33)
   xDxterm42 = rescale ( reformat (xDxterm42,  360),  -360 )   # cdb(xDxterm42.102,xDxterm42)

   checkpoint.append (term0)
   checkpoint.append (term2)
   checkpoint.append (term3)
   checkpoint.append (term4)
   checkpoint.append (term5)

\end{cadabra}

\clearpage

% =================================================================================================
\section*{Tranformation between two RNC frames}

\begin{align*}
     y^{a} = \ny{0}^{a} + \ny{2}^{a} + \ny{3}^{a} + \ny{4}^{a} + \ny{5}^{a} + \BigO{\eps^6}
\end{align*}

\begin{dgroup*}
   \begin{dmath*} \ny{0}^{a} = \cdb{term0.102} \end{dmath*}
   \begin{dmath*} \ny{2}^{a} = \cdb{term2.102} \end{dmath*}
   \begin{dmath*} \ny{3}^{a} = \cdb{term3.102} \end{dmath*}
   \begin{dmath*} \ny{4}^{a} = \cdb{term4.102} \end{dmath*}
   \begin{dmath*} \ny{5}^{a} = \cdb{term5.102} \end{dmath*}
\end{dgroup*}

\clearpage

% =================================================================================================
\section*{Tranformation between two RNC frames}

Same as before but with an improved format (maybe) for the expressions.

\begin{align}
   y^{a} = \ny{0}^{a} + \ny{2}^{a} + \ny{3}^{a} + \ny{4}^{a} + \ny{5}^{a} + \BigO{\eps^6}
\end{align}

\begin{dgroup}
   \begin{dmath} \ny{0}^{a} = Dx^{a} \end{dmath}
\end{dgroup}

\begin{dgroup}
   \begin{dmath} \ny{2}^{a} = \ny{2}^{a}_1 \end{dmath}
   \begin{dmath}   3 \ny{2}^{a}_1 = \cdb{xDxterm12.102} \end{dmath}
\end{dgroup}

\begin{dgroup}
   \begin{dmath} \ny{3}^{a} = \ny{3}^{a}_1 + \ny{3}^{a}_2 \end{dmath}
   \begin{dmath} -12 \ny{3}^{a}_1 = \cdb{xDxterm13.102} \end{dmath}
   \begin{dmath} -24 \ny{3}^{a}_2 = \cdb{xDxterm22.102} \end{dmath}
\end{dgroup}

\begin{dgroup}
   \begin{dmath} \ny{4}^{a} = \ny{4}^{a}_1 + \ny{4}^{a}_2 + \ny{4}^{a}_3 \end{dmath}
   \begin{dmath} -180 \ny{4}^{a}_1 = \cdb{xDxterm14.102} \end{dmath}
   \begin{dmath} -720 \ny{4}^{a}_2 = \cdb{xDxterm23.102} \end{dmath}
   \begin{dmath} -720 \ny{4}^{a}_3 = \cdb{xDxterm32.102} \end{dmath}
\end{dgroup}

\begin{dgroup}
   \begin{dmath} \ny{5}^{a} = \ny{5}^{a}_1 + \ny{5}^{a}_2 + \ny{5}^{a}_3 + \ny{5}^{a}_4 \end{dmath}
   \begin{dmath}  -360 \ny{5}^{a}_1 = \cdb{xDxterm15.102} \end{dmath}
   \begin{dmath} -2160 \ny{5}^{a}_2 = \cdb{xDxterm24.102} \end{dmath}
   \begin{dmath} -1080 \ny{5}^{a}_3 = \cdb{xDxterm33.102} \end{dmath}
   \begin{dmath}  -360 \ny{5}^{a}_4 = \cdb{xDxterm42.102} \end{dmath}
\end{dgroup}

% =================================================================================================
% export checkpoints in json format

\bgroup
\CdbSetup{action=hide}
\begin{cadabra}
   for i in range( len(checkpoint) ):
      cdblib.put ('check{:03d}'.format(i),checkpoint[i],checkpoint_file)
\end{cadabra}
\egroup

\end{document}

\def\Date{19 Jan 2024}
% \def\FileID{file:}

\documentclass[12pt]{cdblatex}

\begin{document}

% =================================================================================================
% create checkpoint file

\bgroup
\CdbSetup{action=hide}
\begin{cadabra}
   import cdblib
   checkpoint_file = 'tests/semantic/output/metric.json'
   cdblib.create (checkpoint_file)
   checkpoint = []
\end{cadabra}
\egroup

% =================================================================================================
\section*{The metric tensor in Riemann normal coordinates}

In this notebook we compute the recursive sequences
\begin{align}
\label{eq:pdgab}
g_{ab,d\ue} &=  \left(g_{cb}\Gamma^{c}{}_{a(d}\right){}_{,\ue)}
               + \left(g_{ac}\Gamma^{c}{}_{b(d}\right){}_{,\ue)}\\[10pt]
\label{eq:pdGamma}
(n+3)\Gamma^a{}_{d(b,c\ue)} &= (n+1)\left(R^a{}_{(bc\Dot d,\ue)}
                               - \left(\Gamma^a{}_{f(c}\Gamma^f{}_{b{\Dot d}}\right){}_{,\ue)}\right)
\end{align}
for $n=1,2,3,\cdots$. Note in these equations that the (extended) index $\ue$ contains $n$
normal indices.

We then construct a Taylor series for the metric using
\begin{dmath*}[spread=5pt]
g_{ab}(x) = g_{ab} + g_{ab,c}x^c + \frac{1}{2!} g_{ab,cd}x^cx^d + \frac{1}{3!} g_{ab,cde}x^cx^dx^e + \cdots
          = g_{ab} + \sum_{n=1}^\infty\> \frac{1}{n!}\>g_{ab,\uc}\>x^{.\uc}
\end{dmath*}

% =================================================================================================
\section*{Stage 1: Symmetrised partial derivatives of $g_{ab}$}

In this stage, equation (\ref{eq:pdgab}) is used to express the symmetrised partial derivatives
of the metric in terms of the symmetrised partial derivatives of the connection.

\begin{dgroup*}
   \begin{dmath*} g_{ab,c} A^{c} = \cdb{term1.200} \end{dmath*}
   \begin{dmath*} g_{ab,cd} A^{c} A^{d} = \cdb{term2.200} \end{dmath*}
   \begin{dmath*} g_{ab,cde} A^{c} A^{d} A^{e} = \cdb{term3.200} \end{dmath*}
\end{dgroup*}

% =================================================================================================
\section*{Stage 2: Replace derivatives of $\Gamma$ with partial derivs of $R$}

Now we use the results from {\verb|dGamma|} to replace derivatives of $\Gamma$ with
partial derivatives of $R$. These were computed in {\verb|dGamma|} using equation
(\ref{eq:pdGamma}) above.

\begin{dgroup*}
   \begin{dmath*} g_{ab,c} A^{c} = \cdb{term1.200} \end{dmath*}
   \begin{dmath*} g_{ab,cd} A^{c} A^{d} = \cdb{term2.303} \end{dmath*}
   \begin{dmath*} g_{ab,cde} A^{c} A^{d} A^{e} = \cdb{term3.305} \end{dmath*}
\end{dgroup*}

% =================================================================================================
\section*{Stage 3: Replace partial derivs of $R$ with covariant derivs of $R$}

Next we use the results from {\verb|dRabcd|} to replace the partial derivatives of $R$ with
covariant deriavtives.

\begin{dgroup*}
   \begin{dmath*} g_{ab,c} A^{c} = \cdb{term1.404} \end{dmath*}
   \begin{dmath*} g_{ab,cd} A^{c} A^{d} = \cdb{term2.404} \end{dmath*}
   \begin{dmath*} g_{ab,cde} A^{c} A^{d} A^{e} = \cdb{term3.403} \end{dmath*}
\end{dgroup*}

% =================================================================================================
\section*{Stage 4: Build the Taylor series for $g_{ab}$, reformatting and output}

Each of the above expressions constitutues one term in the Taylor series for the metric.
We also make the trivial change $A\rightarrow x$. Then we do some trivial reformatting.

\begin{align*}
   g_{ab}(x) &=   g_{ab}
                + g_{ab,c} x^c
                + \frac{1}{2!} g_{ab,cd} x^c x^d
                + \frac{1}{3!} g_{ab,cde} x^c x^d x^e +  \BigO{\eps^4}\\
             &= \cdb{metric4.501} + \BigO{\eps^4}
\end{align*}

\clearpage

% =================================================================================================
\section*{Shared properties}

\begin{cadabra}
   import time

   def flatten_Rabcd (obj):
       substitute (obj,$R^{a}_{b c d}   -> g^{a e} R_{e b c d}$)
       substitute (obj,$R_{a}^{b}_{c d} -> g^{b e} R_{a e c d}$)
       substitute (obj,$R_{a b}^{c}_{b} -> g^{c e} R_{a b e d}$)
       substitute (obj,$R_{a b c}^{d}   -> g^{d e} R_{a b c e}$)
       unwrap     (obj)
       return obj

   def impose_rnc (obj):
       # hide the derivatives of Gamma
       substitute (obj,$\partial_{d}{\Gamma^{a}_{b c}} -> zzz_{d}^{a}_{b c}$,repeat=True)
       substitute (obj,$\partial_{d e}{\Gamma^{a}_{b c}} -> zzz_{d e}^{a}_{b c}$,repeat=True)
       substitute (obj,$\partial_{d e f}{\Gamma^{a}_{b c}} -> zzz_{d e f}^{a}_{b c}$,repeat=True)
       substitute (obj,$\partial_{d e f g}{\Gamma^{a}_{b c}} -> zzz_{d e f g}^{a}_{b c}$,repeat=True)
       substitute (obj,$\partial_{d e f g h}{\Gamma^{a}_{b c}} -> zzz_{d e f g h}^{a}_{b c}$,repeat=True)
       # set Gamma to zero
       substitute (obj,$\Gamma^{a}_{b c} -> 0$,repeat=True)
       # recover the derivatives Gamma
       substitute (obj,$zzz_{d}^{a}_{b c} -> \partial_{d}{\Gamma^{a}_{b c}}$,repeat=True)
       substitute (obj,$zzz_{d e}^{a}_{b c} -> \partial_{d e}{\Gamma^{a}_{b c}}$,repeat=True)
       substitute (obj,$zzz_{d e f}^{a}_{b c} -> \partial_{d e f}{\Gamma^{a}_{b c}}$,repeat=True)
       substitute (obj,$zzz_{d e f g}^{a}_{b c} -> \partial_{d e f g}{\Gamma^{a}_{b c}}$,repeat=True)
       substitute (obj,$zzz_{d e f g h}^{a}_{b c} -> \partial_{d e f g h}{\Gamma^{a}_{b c}}$,repeat=True)
       return obj

   def get_xterm (obj,n):

       x^{a}::Weight(label=numx).

       foo := @(obj).
       bah  = Ex("numx = " + str(n))
       keep_weight (foo,bah)

       return foo

   # note: keeping numbering as is (out of order) to ensure R appears before \nabla R etc.
   def product_sort (obj):
       substitute (obj,$ A^{a}                             -> A001^{a}                  $)
       substitute (obj,$ x^{a}                             -> A002^{a}                  $)
       substitute (obj,$ g_{a b}                           -> A003_{a b}                $)
       substitute (obj,$ g^{a b}                           -> A004^{a b}                $)
       substitute (obj,$ \nabla_{e f g h}{R_{a b c d}}     -> A010_{a b c d e f g h}    $)
       substitute (obj,$ \nabla_{e f g}{R_{a b c d}}       -> A009_{a b c d e f g}      $)
       substitute (obj,$ \nabla_{e f}{R_{a b c d}}         -> A008_{a b c d e f}        $)
       substitute (obj,$ \nabla_{e}{R_{a b c d}}           -> A007_{a b c d e}          $)
       substitute (obj,$ \partial_{e f g h}{R_{a b c d}}   -> A014_{a b c d e f g h}    $)
       substitute (obj,$ \partial_{e f g}{R_{a b c d}}     -> A013_{a b c d e f g}      $)
       substitute (obj,$ \partial_{e f}{R_{a b c d}}       -> A012_{a b c d e f}        $)
       substitute (obj,$ \partial_{e}{R_{a b c d}}         -> A011_{a b c d e}          $)
       substitute (obj,$ \partial_{e f g h}{R^{a}_{b c d}} -> A018^{a}_{b c d e f g h}  $)
       substitute (obj,$ \partial_{e f g}{R^{a}_{b c d}}   -> A017^{a}_{b c d e f g}    $)
       substitute (obj,$ \partial_{e f}{R^{a}_{b c d}}     -> A016^{a}_{b c d e f}      $)
       substitute (obj,$ \partial_{e}{R^{a}_{b c d}}       -> A015^{a}_{b c d e}        $)
       substitute (obj,$ R_{a b c d}                       -> A005_{a b c d}            $)
       substitute (obj,$ R^{a}_{b c d}                     -> A006^{a}_{b c d}          $)
       sort_product   (obj)
       rename_dummies (obj)
       substitute (obj,$ A001^{a}                  -> A^{a}                             $)
       substitute (obj,$ A002^{a}                  -> x^{a}                             $)
       substitute (obj,$ A003_{a b}                -> g_{a b}                           $)
       substitute (obj,$ A004^{a b}                -> g^{a b}                           $)
       substitute (obj,$ A005_{a b c d}            -> R_{a b c d}                       $)
       substitute (obj,$ A006^{a}_{b c d}          -> R^{a}_{b c d}                     $)
       substitute (obj,$ A007_{a b c d e}          -> \nabla_{e}{R_{a b c d}}           $)
       substitute (obj,$ A008_{a b c d e f}        -> \nabla_{e f}{R_{a b c d}}         $)
       substitute (obj,$ A009_{a b c d e f g}      -> \nabla_{e f g}{R_{a b c d}}       $)
       substitute (obj,$ A010_{a b c d e f g h}    -> \nabla_{e f g h}{R_{a b c d}}     $)
       substitute (obj,$ A011_{a b c d e}          -> \partial_{e}{R_{a b c d}}         $)
       substitute (obj,$ A012_{a b c d e f}        -> \partial_{e f}{R_{a b c d}}       $)
       substitute (obj,$ A013_{a b c d e f g}      -> \partial_{e f g}{R_{a b c d}}     $)
       substitute (obj,$ A014_{a b c d e f g h}    -> \partial_{e f g h}{R_{a b c d}}   $)
       substitute (obj,$ A015^{a}_{b c d e}        -> \partial_{e}{R^{a}_{b c d}}       $)
       substitute (obj,$ A016^{a}_{b c d e f}      -> \partial_{e f}{R^{a}_{b c d}}     $)
       substitute (obj,$ A017^{a}_{b c d e f g}    -> \partial_{e f g}{R^{a}_{b c d}}   $)
       substitute (obj,$ A018^{a}_{b c d e f g h}  -> \partial_{e f g h}{R^{a}_{b c d}} $)

       return obj

   def reformat_xterm (obj,scale):
       foo  = Ex(str(scale))
       bah := @(foo) @(obj).
       distribute     (bah)
       bah = product_sort (bah)
       rename_dummies (bah)
       canonicalise   (bah)
       factor_out     (bah,$x^{a?}$)
       ans := @(bah) / @(foo).
       return ans

   def rescale_xterm (obj,scale):
       foo  = Ex(str(scale))
       bah := @(foo) @(obj).
       distribute  (bah)
       factor_out  (bah,$x^{a?}$)
       return bah

   {a,b,c,d,e,f,g,h,i,j,k,l,m,n,o,p,q,r,s,t,u,v,w#}::Indices(position=independent).

   \nabla{#}::Derivative.
   \partial{#}::PartialDerivative.

   g_{a b}::Metric.
   g^{a b}::InverseMetric.
   g_{a}^{b}::KroneckerDelta.
   g^{a}_{b}::KroneckerDelta.

   R_{a b c d}::RiemannTensor.
   R^{a}_{b c d}::RiemannTensor.
   R_{a b c}^{d}::RiemannTensor.

   \Gamma^{a}_{b c}::TableauSymmetry(shape={2}, indices={1,2}).

   g_{a b}::Depends(\partial{#}).
   R_{a b c d}::Depends(\partial{#}).
   R^{a}_{b c d}::Depends(\partial{#}).
   \Gamma^{a}_{b c}::Depends(\partial{#}).

   R_{a b c d}::Depends(\nabla{#}).
   R^{a}_{b c d}::Depends(\nabla{#}).

\end{cadabra}

\clearpage

% =================================================================================================
\section*{Stage 1: Symmetrised partial derivatives of $g_{ab}$}

\begin{cadabra}
   beg_stage_1 = time.time()

   # symmetrised partial derivatives of g_{ab}

   gab00:=g_{a b}.                                              # cdb (gab00.101,gab00)

   gab01:=g_{c b}\Gamma^{c}_{a d} + g_{a c}\Gamma^{c}_{b d}.    # cdb (gab01.101,gab01)

   gab02:=\partial_{e}{ @(gab01) }.                             # cdb (gab02.101,gab02)
   distribute   (gab02)                                         # cdb (gab02.102,gab02)
   product_rule (gab02)                                         # cdb (gab02.103,gab02)
   substitute   (gab02, $\partial_{d}{g_{a b}} -> @(gab01)$)    # cdb (gab02.104,gab02)
   distribute   (gab02)                                         # cdb (gab02.105,gab02)

   gab03:=\partial_{f}{ @(gab02) }.                             # cdb (gab03.101,gab03)
   distribute   (gab03)                                         # cdb (gab03.102,gab03)
   product_rule (gab03)                                         # cdb (gab03.103,gab03)
   substitute   (gab03, $\partial_{d}{g_{a b}} -> @(gab01)$)    # cdb (gab03.104,gab03)
   distribute   (gab03)                                         # cdb (gab03.105,gab03)

   gab04:=\partial_{g}{ @(gab03) }.                             # cdb (gab04.101,gab04)
   distribute   (gab04)                                         # cdb (gab04.102,gab04)
   product_rule (gab04)                                         # cdb (gab04.103,gab04)
   substitute   (gab04, $\partial_{d}{g_{a b}} -> @(gab01)$)    # cdb (gab04.104,gab04)
   distribute   (gab04)                                         # cdb (gab04.105,gab04)

   gab05:=\partial_{h}{ @(gab04) }.                             # cdb (gab05.101,gab05)
   distribute   (gab05)                                         # cdb (gab05.102,gab05)
   product_rule (gab05)                                         # cdb (gab05.103,gab05)
   substitute   (gab05, $\partial_{d}{g_{a b}} -> @(gab01)$)    # cdb (gab05.104,gab05)
   distribute   (gab05)                                         # cdb (gab05.105,gab05)

   gab00 = impose_rnc (gab00)   # cdb (gab00.102,gab00)
   gab01 = impose_rnc (gab01)   # cdb (gab01.102,gab01)
   gab02 = impose_rnc (gab02)   # cdb (gab02.106,gab02)
   gab03 = impose_rnc (gab03)   # cdb (gab03.106,gab03)
   gab04 = impose_rnc (gab04)   # cdb (gab04.106,gab04)
   gab05 = impose_rnc (gab05)   # cdb (gab05.106,gab05)

\end{cadabra}

\clearpage

\begin{dgroup*}
   \begin{dmath*} \cdb*{gab00.101} \end{dmath*}
   \begin{dmath*} \cdb*{gab00.102} \end{dmath*}
   \begin{dmath*} \cdb*{gab01.101} \end{dmath*}
   \begin{dmath*} \cdb*{gab01.102} \end{dmath*}
\end{dgroup*}

\begin{dgroup*}
   \begin{dmath*} \cdb*{gab02.101} \end{dmath*}
   \begin{dmath*} \cdb*{gab02.102} \end{dmath*}
   \begin{dmath*} \cdb*{gab02.103} \end{dmath*}
   \begin{dmath*} \cdb*{gab02.104} \end{dmath*}
   \begin{dmath*} \cdb*{gab02.105} \end{dmath*}
   \begin{dmath*} \cdb*{gab02.106} \end{dmath*}
\end{dgroup*}

\begin{dgroup*}
   \begin{dmath*} \cdb*{gab03.101} \end{dmath*}
   \begin{dmath*} \cdb*{gab03.102} \end{dmath*}
   \begin{dmath*} \cdb*{gab03.103} \end{dmath*}
   \begin{dmath*} \cdb*{gab03.104} \end{dmath*}
   \begin{dmath*} \cdb*{gab03.105} \end{dmath*}
   \begin{dmath*} \cdb*{gab03.106} \end{dmath*}
\end{dgroup*}

\begin{dgroup*}
   \begin{dmath*} \cdb*{gab04.101} \end{dmath*}
   \begin{dmath*} \cdb*{gab04.102} \end{dmath*}
   \begin{dmath*} \cdb*{gab04.103} \end{dmath*}
   \begin{dmath*} \cdb*{gab04.104} \end{dmath*}
   \begin{dmath*} \cdb*{gab04.105} \end{dmath*}
   \begin{dmath*} \cdb*{gab04.106} \end{dmath*}
\end{dgroup*}

\begin{cadabra}
   # prepare first six terms in the Taylor series expansion of g_{ab}(x)

   term0:= @(gab00).
   distribute (term0)                             # cdb(term0.200,term0)

   term1:= @(gab01) A^d.
   distribute (term1)                             # cdb(term1.200,term1)

   term2:= @(gab02) A^d A^e.
   distribute (term2)                             # cdb(term2.200,term2)

   term3:= @(gab03) A^d A^e A^f.
   distribute (term3)                             # cdb(term3.200,term3)

   term4:= @(gab04) A^d A^e A^f A^g.
   distribute (term4)                             # cdb(term4.200,term4)

   term5:= @(gab05) A^d A^e A^f A^g A^h.
   distribute (term5)                             # cdb(term5.200,term5)

   end_stage_1 = time.time()
\end{cadabra}

\begin{dgroup*}
   \begin{dmath*} \cdb*{term0.200} \end{dmath*}
   \begin{dmath*} \cdb*{term1.200} \end{dmath*}
   \begin{dmath*} \cdb*{term2.200} \end{dmath*}
   \begin{dmath*} \cdb*{term3.200} \end{dmath*}
   % \begin{dmath*} \cdb*{term4.200} \end{dmath*}
   % \begin{dmath*} \cdb*{term5.200} \end{dmath*}
\end{dgroup*}

\clearpage

% =================================================================================================
\section*{Stage 2: Replace derivatives of $\Gamma$ with partial derivs of $R$}

\begin{cadabra}
   import cdblib

   beg_stage_2 = time.time()

   dGamma01 = cdblib.get ('dGamma01','dGamma.json')  # cdb(dGamma01.300,dGamma01)
   dGamma02 = cdblib.get ('dGamma02','dGamma.json')  # cdb(dGamma02.300,dGamma02)
   dGamma03 = cdblib.get ('dGamma03','dGamma.json')  # cdb(dGamma03.300,dGamma03)
   dGamma04 = cdblib.get ('dGamma04','dGamma.json')  # cdb(dGamma04.300,dGamma04)
   dGamma05 = cdblib.get ('dGamma05','dGamma.json')  # cdb(dGamma05.300,dGamma05)

   # replace partial derivs of \Gamma with products and derivs of Riemann tensor

   substitute (term2,$\partial_{c}{\Gamma^{a}_{b d}}A^{c}A^{b} -> @(dGamma01)$,repeat=True)                       # cdb(term2.301,term2)
   substitute (term2,$\partial_{c}{\Gamma^{a}_{d b}}A^{c}A^{b} -> @(dGamma01)$,repeat=True)                       # cdb(term2.302,term2)
   distribute (term2)                                                                                             # cdb(term2.303,term2)

   substitute (term3,$\partial_{c e}{\Gamma^{a}_{d b}}A^{c}A^{b}A^{e} -> @(dGamma02)$,repeat=True)                # cdb(term3.301,term3)
   substitute (term3,$\partial_{c e}{\Gamma^{a}_{b d}}A^{c}A^{b}A^{e} -> @(dGamma02)$,repeat=True)                # cdb(term3.302,term3)
   substitute (term3,$\partial_{c}{\Gamma^{a}_{b d}}A^{c}A^{b} -> @(dGamma01)$,repeat=True)                       # cdb(term3.303,term3)
   substitute (term3,$\partial_{c}{\Gamma^{a}_{d b}}A^{c}A^{b} -> @(dGamma01)$,repeat=True)                       # cdb(term3.304,term3)
   distribute (term3)                                                                                             # cdb(term3.305,term3)

   substitute (term4,$\partial_{c e f}{\Gamma^{a}_{d b}}A^{c}A^{b}A^{e}A^{f} -> @(dGamma03)$,repeat=True)         # cdb(term4.301,term4)
   substitute (term4,$\partial_{c e f}{\Gamma^{a}_{b d}}A^{c}A^{b}A^{e}A^{f} -> @(dGamma03)$,repeat=True)         # cdb(term4.302,term4)
   substitute (term4,$\partial_{c e}{\Gamma^{a}_{d b}}A^{c}A^{b}A^{e} -> @(dGamma02)$,repeat=True)                # cdb(term4.303,term4)
   substitute (term4,$\partial_{c e}{\Gamma^{a}_{b d}}A^{c}A^{b}A^{e} -> @(dGamma02)$,repeat=True)                # cdb(term4.304,term4)
   substitute (term4,$\partial_{c}{\Gamma^{a}_{b d}}A^{c}A^{b} -> @(dGamma01)$,repeat=True)                       # cdb(term4.305,term4)
   substitute (term4,$\partial_{c}{\Gamma^{a}_{d b}}A^{c}A^{b} -> @(dGamma01)$,repeat=True)                       # cdb(term4.306,term4)
   distribute (term4)                                                                                             # cdb(term4.307,term4)

   substitute (term5,$\partial_{c e f g}{\Gamma^{a}_{d b}}A^{c}A^{b}A^{e}A^{f}A^{g} -> @(dGamma04)$,repeat=True)  # cdb(term5.301,term5)
   substitute (term5,$\partial_{c e f g}{\Gamma^{a}_{b d}}A^{c}A^{b}A^{e}A^{f}A^{g} -> @(dGamma04)$,repeat=True)  # cdb(term5.302,term5)
   substitute (term5,$\partial_{c e f}{\Gamma^{a}_{d b}}A^{c}A^{b}A^{e}A^{f} -> @(dGamma03)$,repeat=True)         # cdb(term5.303,term5)
   substitute (term5,$\partial_{c e f}{\Gamma^{a}_{b d}}A^{c}A^{b}A^{e}A^{f} -> @(dGamma03)$,repeat=True)         # cdb(term5.304,term5)
   substitute (term5,$\partial_{c e}{\Gamma^{a}_{d b}}A^{c}A^{b}A^{e} -> @(dGamma02)$,repeat=True)                # cdb(term5.305,term5)
   substitute (term5,$\partial_{c e}{\Gamma^{a}_{b d}}A^{c}A^{b}A^{e} -> @(dGamma02)$,repeat=True)                # cdb(term5.306,term5)
   substitute (term5,$\partial_{c}{\Gamma^{a}_{b d}}A^{c}A^{b} -> @(dGamma01)$,repeat=True)                       # cdb(term5.307,term5)
   substitute (term5,$\partial_{c}{\Gamma^{a}_{d b}}A^{c}A^{b} -> @(dGamma01)$,repeat=True)                       # cdb(term5.308,term5)
   distribute (term5)                                                                                             # cdb(term5.309,term5)

   end_stage_2 = time.time()

   # -------------------------------------------------------------------------------------------
   # this block of Xterms only produces formatted output, it's not part of the main computation
   # -------------------------------------------------------------------------------------------

   # the metric in terms of partial derivatives of Rabcd

   metric:=@(term0)
         + (1/1) @(term1)  # zero
         + (1/2) @(term2)
         + (1/6) @(term3)
         + (1/24) @(term4)
         + (1/120) @(term5).  # cdb(metric.301,metric)

   substitute (metric,$A^{a} -> x^{a}$)  # cdb (metric.302,metric)

   # reformat and tidy up

   Xterm0 := @(term0).
   Xterm1 := (1/1) @(term1).
   Xterm2 := (1/2) @(term2).
   Xterm3 := (1/6) @(term3).
   Xterm4 := (1/24) @(term4).
   Xterm5 := (1/120) @(term5).

   substitute (Xterm0,$A^{a} -> x^{a}$)
   substitute (Xterm1,$A^{a} -> x^{a}$)
   substitute (Xterm2,$A^{a} -> x^{a}$)
   substitute (Xterm3,$A^{a} -> x^{a}$)
   substitute (Xterm4,$A^{a} -> x^{a}$)
   substitute (Xterm5,$A^{a} -> x^{a}$)

   substitute (Xterm2,$g_{a b} \partial_{c}{R^{b}_{d e f}} -> \partial_{c}{R_{a d e f}}$)  # cdb(Xterm2.301,Xterm2)
   substitute (Xterm3,$g_{a b} \partial_{c}{R^{b}_{d e f}} -> \partial_{c}{R_{a d e f}}$)  # cdb(Xterm3.301,Xterm3)
   substitute (Xterm4,$g_{a b} \partial_{c}{R^{b}_{d e f}} -> \partial_{c}{R_{a d e f}}$)  # cdb(Xterm4.301,Xterm4)
   substitute (Xterm5,$g_{a b} \partial_{c}{R^{b}_{d e f}} -> \partial_{c}{R_{a d e f}}$)  # cdb(Xterm5.301,Xterm5)

   substitute (Xterm2,$g_{b a} \partial_{c}{R^{b}_{d e f}} -> \partial_{c}{R_{a d e f}}$)  # cdb(Xterm2.301,Xterm2)
   substitute (Xterm3,$g_{b a} \partial_{c}{R^{b}_{d e f}} -> \partial_{c}{R_{a d e f}}$)  # cdb(Xterm3.301,Xterm3)
   substitute (Xterm4,$g_{b a} \partial_{c}{R^{b}_{d e f}} -> \partial_{c}{R_{a d e f}}$)  # cdb(Xterm4.301,Xterm4)
   substitute (Xterm5,$g_{b a} \partial_{c}{R^{b}_{d e f}} -> \partial_{c}{R_{a d e f}}$)  # cdb(Xterm5.301,Xterm5)

   eliminate_metric (Xterm2)  # cdb(Xterm2.302,Xterm2)
   eliminate_metric (Xterm3)  # cdb(Xterm3.302,Xterm3)
   eliminate_metric (Xterm4)  # cdb(Xterm4.302,Xterm4)
   eliminate_metric (Xterm5)  # cdb(Xterm5.302,Xterm5)

   sort_product     (Xterm2)  # cdb(Xterm2.303,Xterm2)
   sort_product     (Xterm3)  # cdb(Xterm3.303,Xterm3)
   sort_product     (Xterm4)  # cdb(Xterm4.303,Xterm4)
   sort_product     (Xterm5)  # cdb(Xterm5.303,Xterm5)

   rename_dummies   (Xterm2)  # cdb(Xterm2.304,Xterm2)
   rename_dummies   (Xterm3)  # cdb(Xterm3.304,Xterm3)
   rename_dummies   (Xterm4)  # cdb(Xterm4.304,Xterm4)
   rename_dummies   (Xterm5)  # cdb(Xterm5.304,Xterm5)

   canonicalise     (Xterm2)  # cdb(Xterm2.305,Xterm2)
   canonicalise     (Xterm3)  # cdb(Xterm3.305,Xterm3)
   canonicalise     (Xterm4)  # cdb(Xterm4.305,Xterm4)
   canonicalise     (Xterm5)  # cdb(Xterm5.305,Xterm5)

   # push upper index to the left
   def tidy_Rabcd (obj):
       substitute (obj,$R_{a b c}^{d} -> - R^{d}_{c a b}$)
       substitute (obj,$R_{a b}^{c}_{d} -> R^{c}_{d a b}$)
       substitute (obj,$R_{a}^{b}_{c d} -> - R^{b}_{a c d}$)
       return obj

   Xterm0 = tidy_Rabcd (Xterm0)  # cdb(Xterm0.666,Xterm0)
   Xterm2 = tidy_Rabcd (Xterm2)  # cdb(Xterm2.666,Xterm2)
   Xterm3 = tidy_Rabcd (Xterm3)  # cdb(Xterm3.666,Xterm3)
   Xterm4 = tidy_Rabcd (Xterm4)  # cdb(Xterm4.666,Xterm4)
   Xterm5 = tidy_Rabcd (Xterm5)  # cdb(Xterm5.666,Xterm5)

   Xterm0 = reformat_xterm (Xterm0,  1)    # cdb(Xterm0.301,Xterm0)
   Xterm2 = reformat_xterm (Xterm2,  3)    # cdb(Xterm2.301,Xterm2)
   Xterm3 = reformat_xterm (Xterm3,  6)    # cdb(Xterm3.301,Xterm3)
   Xterm4 = reformat_xterm (Xterm4,360)    # cdb(Xterm4.301,Xterm4)
   Xterm5 = reformat_xterm (Xterm5,180)    # cdb(Xterm5.301,Xterm5)

   # canonicalise from reformat_xterm will slide upper index from left hand side
   # so now we slide the upper index back to the left

   Xterm0 = tidy_Rabcd (Xterm0)  # cdb(Xterm0.667,Xterm0)
   Xterm2 = tidy_Rabcd (Xterm2)  # cdb(Xterm2.667,Xterm2)
   Xterm3 = tidy_Rabcd (Xterm3)  # cdb(Xterm3.667,Xterm3)
   Xterm4 = tidy_Rabcd (Xterm4)  # cdb(Xterm4.667,Xterm4)
   Xterm5 = tidy_Rabcd (Xterm5)  # cdb(Xterm5.667,Xterm5)

   # metric to 3rd, 4th, 5th and 6th order terms in powers of x^a

   Metric3 := @(Xterm0) + @(Xterm2).                                      # cdb (Metric3.301,Metric3)
   Metric4 := @(Xterm0) + @(Xterm2) + @(Xterm3).                          # cdb (Metric4.301,Metric4)
   Metric5 := @(Xterm0) + @(Xterm2) + @(Xterm3) + @(Xterm4).              # cdb (Metric5.301,Metric5)
   Metric6 := @(Xterm0) + @(Xterm2) + @(Xterm3) + @(Xterm4) + @(Xterm5).  # cdb (Metric6.301,Metric6)

   # ------------------------------------------------------------------------------------
   # end of format block
   # ------------------------------------------------------------------------------------

\end{cadabra}

\clearpage

\begin{dgroup*}
   \begin{dmath*} \cdb*{term2.301} \end{dmath*}
   \begin{dmath*} \cdb*{term2.302} \end{dmath*}
   \begin{dmath*} \cdb*{term2.303} \end{dmath*}
\end{dgroup*}

\begin{dgroup*}
   \begin{dmath*} \cdb*{term3.301} \end{dmath*}
   \begin{dmath*} \cdb*{term3.302} \end{dmath*}
   \begin{dmath*} \cdb*{term3.303} \end{dmath*}
   \begin{dmath*} \cdb*{term3.304} \end{dmath*}
   \begin{dmath*} \cdb*{term3.305} \end{dmath*}
\end{dgroup*}

\begin{dgroup*}
   \begin{dmath*} \cdb*{term4.301} \end{dmath*}
   \begin{dmath*} \cdb*{term4.302} \end{dmath*}
   \begin{dmath*} \cdb*{term4.303} \end{dmath*}
   \begin{dmath*} \cdb*{term4.304} \end{dmath*}
   \begin{dmath*} \cdb*{term4.305} \end{dmath*}
   \begin{dmath*} \cdb*{term4.306} \end{dmath*}
   \begin{dmath*} \cdb*{term4.307} \end{dmath*}
\end{dgroup*}

\clearpage

\begin{dgroup*}
   \begin{dmath*} g_{ab}(x) = \cdb{Metric3.301} \end{dmath*}
   \begin{dmath*} g_{ab}(x) = \cdb{Metric4.301} \end{dmath*}
   \begin{dmath*} g_{ab}(x) = \cdb{Metric5.301} \end{dmath*}
   \begin{dmath*} g_{ab}(x) = \cdb{Metric6.301} \end{dmath*}
\end{dgroup*}

\clearpage

% =================================================================================================
\section*{Stage 3: Replace partial derivs of $R$ with covariant derivs of $R$}

\begin{cadabra}
   beg_stage_3 = time.time()

   # now convert partial derivs of Rabcd to covariant derivs

   dRabcd01 = cdblib.get ('dRabcd01','dRabcd.json')  # cdb(dRabcd01.400,dRabcd01)
   dRabcd02 = cdblib.get ('dRabcd02','dRabcd.json')  # cdb(dRabcd02.400,dRabcd02)
   dRabcd03 = cdblib.get ('dRabcd03','dRabcd.json')  # cdb(dRabcd03.400,dRabcd03)

   # term1 & term2 need no special care, just a bit of tidying

   eliminate_metric (term1)   # cdb(term1.401,term1)
   sort_product     (term1)   # cdb(term1.402,term1)
   rename_dummies   (term1)   # cdb(term1.403,term1)
   canonicalise     (term1)   # cdb(term1.404,term1)

   eliminate_metric (term2)   # cdb(term2.401,term2)
   sort_product     (term2)   # cdb(term2.402,term2)
   rename_dummies   (term2)   # cdb(term2.403,term2)
   canonicalise     (term2)   # cdb(term2.404,term2)

   # replace partial derivatives of Riemann tensor in term3, term4 etc. with covariant derivatives of Rabcd

   tmp01 := @(dRabcd01).      # cdb(tmp01.403,tmp01)
   tmp02 := @(dRabcd02).      # cdb(tmp02.403,tmp02)
   tmp03 := @(dRabcd03).      # cdb(tmp03.403,tmp03)

   substitute (term3,$A^{c}A^{d}A^{e}\partial_{e}{R^{a}_{c d b}} ->   @(tmp01)$,repeat=True)         # cdb(term3.401,term3)
   substitute (term3,$A^{c}A^{d}A^{e}\partial_{e}{R^{a}_{c b d}} -> - @(tmp01)$,repeat=True)         # cdb(term3.402,term3)
   distribute (term3)                                                                                # cdb(term3.403,term3)

   substitute (term4,$A^{c}A^{d}A^{e}A^{f}\partial_{e f}{R^{a}_{c d b}} ->   @(tmp02)$,repeat=True)  # cdb(term4.401,term4)
   substitute (term4,$A^{c}A^{d}A^{e}A^{f}\partial_{e f}{R^{a}_{c b d}} -> - @(tmp02)$,repeat=True)  # cdb(term4.402,term4)
   substitute (term4,$A^{c}A^{d}A^{e}\partial_{e}{R^{a}_{c d b}} ->   @(tmp01)$,repeat=True)         # cdb(term4.403,term4)
   substitute (term4,$A^{c}A^{d}A^{e}\partial_{e}{R^{a}_{c b d}} -> - @(tmp01)$,repeat=True)         # cdb(term4.404,term4)
   distribute (term4)                                                                                # cdb(term4.405,term4)

   substitute (term5,$A^{c}A^{d}A^{e}A^{f}A^{g}\partial_{e f g}{R^{a}_{c d b}} ->   @(tmp03)$,repeat=True)
   substitute (term5,$A^{c}A^{d}A^{e}A^{f}A^{g}\partial_{e f g}{R^{a}_{c b d}} -> - @(tmp03)$,repeat=True)
   substitute (term5,$A^{c}A^{d}A^{e}A^{f}\partial_{e f}{R^{a}_{c d b}} ->   @(tmp02)$,repeat=True)
   substitute (term5,$A^{c}A^{d}A^{e}A^{f}\partial_{e f}{R^{a}_{c b d}} -> - @(tmp02)$,repeat=True)
   substitute (term5,$A^{c}A^{d}A^{e}\partial_{e}{R^{a}_{c d b}} ->   @(tmp01)$,repeat=True)
   substitute (term5,$A^{c}A^{d}A^{e}\partial_{e}{R^{a}_{c b d}} -> - @(tmp01)$,repeat=True)
   distribute (term5)

   end_stage_3 = time.time()
\end{cadabra}

\begin{dgroup*}
   \begin{dmath*} \cdb*{tmp01.403} \end{dmath*}
   \begin{dmath*} \cdb*{tmp02.403} \end{dmath*}
   \begin{dmath*} \cdb*{tmp03.403} \end{dmath*}
\end{dgroup*}

\clearpage

\begin{dgroup*}
   \begin{dmath*} \cdb*{term1.401} \end{dmath*}
   \begin{dmath*} \cdb*{term1.402} \end{dmath*}
   \begin{dmath*} \cdb*{term1.403} \end{dmath*}
   \begin{dmath*} \cdb*{term1.404} \end{dmath*}
\end{dgroup*}

\begin{dgroup*}
   \begin{dmath*} \cdb*{term2.401} \end{dmath*}
   \begin{dmath*} \cdb*{term2.402} \end{dmath*}
   \begin{dmath*} \cdb*{term2.403} \end{dmath*}
   \begin{dmath*} \cdb*{term2.404} \end{dmath*}
\end{dgroup*}

\begin{dgroup*}
   \begin{dmath*} \cdb*{term3.401} \end{dmath*}
   \begin{dmath*} \cdb*{term3.402} \end{dmath*}
   \begin{dmath*} \cdb*{term3.403} \end{dmath*}
\end{dgroup*}

\begin{dgroup*}
   \begin{dmath*} \cdb*{term4.401} \end{dmath*}
   \begin{dmath*} \cdb*{term4.402} \end{dmath*}
   \begin{dmath*} \cdb*{term4.403} \end{dmath*}
   \begin{dmath*} \cdb*{term4.404} \end{dmath*}
   \begin{dmath*} \cdb*{term4.405} \end{dmath*}
\end{dgroup*}

\clearpage

% =================================================================================================
\section*{Stage 4: Build the Taylor series for $g_{ab}$, reformatting and output}

\begin{cadabra}
   beg_stage_4 = time.time()
   # final housekeeping

   term1 = flatten_Rabcd (term1)         # cdb(term1.501,term1)
   term2 = flatten_Rabcd (term2)         # cdb(term2.501,term2)
   term3 = flatten_Rabcd (term3)         # cdb(term3.501,term3)
   term4 = flatten_Rabcd (term4)         # cdb(term4.501,term4)
   term5 = flatten_Rabcd (term5)         # cdb(term5.501,term5)

   eliminate_metric (term1)
   eliminate_metric (term2)
   eliminate_metric (term3)
   eliminate_metric (term4)
   eliminate_metric (term5)

   eliminate_kronecker (term1)
   eliminate_kronecker (term2)
   eliminate_kronecker (term3)
   eliminate_kronecker (term4)
   eliminate_kronecker (term5)

   sort_product (term1)
   sort_product (term2)
   sort_product (term3)
   sort_product (term4)
   sort_product (term5)

   rename_dummies (term1)
   rename_dummies (term2)
   rename_dummies (term3)
   rename_dummies (term4)
   rename_dummies (term5)

   canonicalise (term1)                  # cdb(term1.502,term1)
   canonicalise (term2)                  # cdb(term2.502,term2)
   canonicalise (term3)                  # cdb(term3.502,term3)
   canonicalise (term4)                  # cdb(term4.502,term4)
   canonicalise (term5)                  # cdb(term5.502,term5)

   # this is out final answer

   metric:=@(term0)
         + (1/1) @(term1)
         + (1/2) @(term2)
         + (1/6) @(term3)
         + (1/24) @(term4)
         + (1/120) @(term5).             # cdb(metric.501,metric)

   substitute (metric,$A^{a} -> x^{a}$)  # cdb (metric.502,metric)

   cdblib.create ('metric.json')

   cdblib.put ('g_ab',metric,'metric.json')

   # extract the terms of the metric in powers of x

   term0 = get_xterm (metric,0)          # cdb(term0.503,term0)
   term1 = get_xterm (metric,1)          # cdb(term1.503,term1)
   term2 = get_xterm (metric,2)          # cdb(term2.503,term2)
   term3 = get_xterm (metric,3)          # cdb(term3.503,term3)
   term4 = get_xterm (metric,4)          # cdb(term4.503,term4)
   term5 = get_xterm (metric,5)          # cdb(term5.503,term5)

   cdblib.put ('g_ab_0',term0,'metric.json')
   cdblib.put ('g_ab_1',term1,'metric.json')
   cdblib.put ('g_ab_2',term2,'metric.json')
   cdblib.put ('g_ab_3',term3,'metric.json')
   cdblib.put ('g_ab_4',term4,'metric.json')
   cdblib.put ('g_ab_5',term5,'metric.json')

   # this version of "metric" is used only in the commentary at the start of this notebook

   metric4:=@(term0) + @(term1) + @(term2) + @(term3).  # cdb(metric4.501,metric4)

\end{cadabra}

\clearpage

\begin{dgroup*}
   \begin{dmath*} \cdb*{term2.501} \end{dmath*}
   \begin{dmath*} \cdb*{term2.502} \end{dmath*}
\end{dgroup*}

\begin{dgroup*}
   \begin{dmath*} \cdb*{term3.501} \end{dmath*}
   \begin{dmath*} \cdb*{term3.502} \end{dmath*}
\end{dgroup*}

\begin{dgroup*}
   \begin{dmath*} \cdb*{term4.501} \end{dmath*}
   \begin{dmath*} \cdb*{term4.502} \end{dmath*}
\end{dgroup*}

\begin{dgroup*}
   \begin{dmath*} \cdb*{term5.501} \end{dmath*}
   \begin{dmath*} \cdb*{term5.502} \end{dmath*}
\end{dgroup*}

\clearpage

\begin{dgroup*}
   \begin{dmath*} \cdb*{metric.501} \end{dmath*}
   \begin{dmath*} \cdb*{metric.502} \end{dmath*}
\end{dgroup*}

\clearpage

\begin{dgroup*}
   \begin{dmath*} \cdb*{term0.503} \end{dmath*}
   \begin{dmath*} \cdb*{term1.503} \end{dmath*}
   \begin{dmath*} \cdb*{term2.503} \end{dmath*}
   \begin{dmath*} \cdb*{term3.503} \end{dmath*}
   \begin{dmath*} \cdb*{term4.503} \end{dmath*}
   \begin{dmath*} \cdb*{term5.503} \end{dmath*}
\end{dgroup*}

% =================================================================================================
% the remaining code is just for pretty printing

\clearpage

\begin{cadabra}
   Xterm0 := @(term0).
   Xterm1 := @(term1).  # zero
   Xterm2 := @(term2).
   Xterm3 := @(term3).
   Xterm4 := @(term4).
   Xterm5 := @(term5).

   Xterm0 = reformat_xterm (Xterm0,  1)    # cdb(Xterm0.601,Xterm0)
   Xterm2 = reformat_xterm (Xterm2,  3)    # cdb(Xterm2.601,Xterm2)
   Xterm3 = reformat_xterm (Xterm3,  6)    # cdb(Xterm3.601,Xterm3)
   Xterm4 = reformat_xterm (Xterm4,180)    # cdb(Xterm4.601,Xterm4)
   Xterm5 = reformat_xterm (Xterm5, 90)    # cdb(Xterm5.601,Xterm5)

   gab3   := @(Xterm0) + @(Xterm2).                                      # cdb (gab3.601,gab3)
   gab4   := @(Xterm0) + @(Xterm2) + @(Xterm3).                          # cdb (gab4.601,gab4)
   gab5   := @(Xterm0) + @(Xterm2) + @(Xterm3) + @(Xterm4).              # cdb (gab5.601,gab5)
   gab6   := @(Xterm0) + @(Xterm2) + @(Xterm3) + @(Xterm4) + @(Xterm5).  # cdb (gab6.601,gab6)

   Metric := @(Xterm0) + @(Xterm2) + @(Xterm3) + @(Xterm4) + @(Xterm5).  # cdb (Metric.601,Metric)

   scaled0 = rescale_xterm (Xterm0,  1)    # cdb(scaled0.601,scaled0)
   scaled2 = rescale_xterm (Xterm2,  3)    # cdb(scaled2.601,scaled2)
   scaled3 = rescale_xterm (Xterm3,  6)    # cdb(scaled3.601,scaled3)
   scaled4 = rescale_xterm (Xterm4,180)    # cdb(scaled4.601,scaled4)
   scaled5 = rescale_xterm (Xterm5, 90)    # cdb(scaled5.601,scaled5)

   end_stage_4 = time.time()
\end{cadabra}

\clearpage

% =================================================================================================
\section*{The metric in Riemann normal coordinates}

\begin{dgroup*}
   \begin{dmath*} g_{a b}(x) = \cdb{Metric.601}+\BigO{\eps^6} \end{dmath*}
\end{dgroup*}

\clearpage

% =================================================================================================
\section*{Curvature expansion of the metric}
\begin{align*}
     g_{a b}(x) =
     \ngab{0}_{a b}
   + \ngab{2}_{a b}
   + \ngab{3}_{a b}
   + \ngab{4}_{a b}
   + \ngab{5}_{a b}+\BigO{\eps^6}
\end{align*}
\begin{dgroup*}
   \begin{dmath*}     \ngab{0}_{a b} = \cdb{scaled0.601} \end{dmath*}
   \begin{dmath*}   3 \ngab{2}_{a b} = \cdb{scaled2.601} \end{dmath*}
   \begin{dmath*}   6 \ngab{3}_{a b} = \cdb{scaled3.601} \end{dmath*}
   \begin{dmath*} 180 \ngab{4}_{a b} = \cdb{scaled4.601} \end{dmath*}
   \begin{dmath*}  90 \ngab{5}_{a b} = \cdb{scaled5.601} \end{dmath*}
\end{dgroup*}

\clearpage

% =================================================================================================
% export selected objects, these will later be imported into a library
% these are the objects that will appear in the paper

\begin{cadabra}
   cdblib.create ('metric.export')

   cdblib.put ('g_ab_3',Metric3,'metric.export')  # R and \partial R
   cdblib.put ('g_ab_4',Metric4,'metric.export')
   cdblib.put ('g_ab_5',Metric5,'metric.export')
   cdblib.put ('g_ab_6',Metric6,'metric.export')

   cdblib.put ('g_ab',  Metric, 'metric.export')  # R and \nabla R

   cdblib.put ('g_ab_scaled0',scaled0,'metric.export')
   cdblib.put ('g_ab_scaled2',scaled2,'metric.export')
   cdblib.put ('g_ab_scaled3',scaled3,'metric.export')
   cdblib.put ('g_ab_scaled4',scaled4,'metric.export')
   cdblib.put ('g_ab_scaled5',scaled5,'metric.export')

   checkpoint.append (Metric4)
   checkpoint.append (Metric6)

   checkpoint.append (Metric)

   checkpoint.append (scaled0)
   checkpoint.append (scaled2)
   checkpoint.append (scaled3)
   checkpoint.append (scaled4)
   checkpoint.append (scaled5)

   # cdbBeg (timing)
   print ("Stage 1: {:7.1f} secs\\hfill\\break".format(end_stage_1-beg_stage_1))
   print ("Stage 2: {:7.1f} secs\\hfill\\break".format(end_stage_2-beg_stage_2))
   print ("Stage 3: {:7.1f} secs\\hfill\\break".format(end_stage_3-beg_stage_3))
   print ("Stage 4: {:7.1f} secs".format(end_stage_4-beg_stage_4))
   # cdbEnd (timing)

\end{cadabra}

\clearpage

% =================================================================================================
\section*{Timing}

\cdb{timing}

% =================================================================================================
% export checkpoints in json format

\bgroup
\CdbSetup{action=hide}
\begin{cadabra}
   for i in range( len(checkpoint) ):
      cdblib.put ('check{:03d}'.format(i),checkpoint[i],checkpoint_file)
\end{cadabra}
\egroup

\end{document}


From {\tts geodesic-bvp} (actually from {\tts rnc2rnc which reformatted the results nicely}) we have
% \begin{align*}
%      y^{a} = \ny{0}^{a} + \ny{2}^{a} + \ny{3}^{a} + \ny{4}^{a} + \BigO{\eps^5}
% \end{align*}

\begin{dgroup*}
   \begin{dmath*} \ny{0}^{a} = \cdb{term0.102} \end{dmath*}
   \begin{dmath*} \ny{2}^{a} = \cdb{term2.102} \end{dmath*}
   \begin{dmath*} \ny{3}^{a} = \cdb{term3.102} \end{dmath*}
   \begin{dmath*} \ny{4}^{a} = \cdb{term4.102} \end{dmath*}
\end{dgroup*}

and from {\tts metric} we have

% \begin{align*}
%      g_{a b}(x) =
%      \ngab{0}_{a b}
%    + \ngab{2}_{a b}
%    + \ngab{3}_{a b}
%    + \ngab{4}_{a b}
%    + \BigO{\eps^6}
% \end{align*}
\begin{dgroup*}
   \begin{dmath*}     \ngab{0}_{a b} = \cdb{scaled0.601} \end{dmath*}
   \begin{dmath*}   3 \ngab{2}_{a b} = \cdb{scaled2.601} \end{dmath*}
   \begin{dmath*}   6 \ngab{3}_{a b} = \cdb{scaled3.601} \end{dmath*}
   \begin{dmath*} 180 \ngab{4}_{a b} = \cdb{scaled4.601} \end{dmath*}
\end{dgroup*}

% =================================================================================================
\section*{Stage 2}
The results from the {\tts geodesic-bvp} and {\tts metric} codes are read to provide
values for the $\ny{n}^{a}$ and $\ngab{m}_{ab}$. These are substituted into the result from
Stage 1, et volia, the final answer. To 4th-order terms the result is given by

\Dmath*{ L^2_{PQ} = \cdb{lsq5.301} + \BigO{\eps^5}}

\clearpage

% =================================================================================================
\section*{Shared properties}

\begin{cadabra}
   {a,b,c,d,e,f,g,h,i,j,k,l,m,n,o,p,q,r,s,t,u,v,w#}::Indices(position=independent).

   D{#}::Derivative.
   \nabla{#}::Derivative.
   \partial{#}::PartialDerivative.

   g_{a b}::Metric.
   g^{a b}::InverseMetric.
   g_{a}^{b}::KroneckerDelta.
   g^{a}_{b}::KroneckerDelta.
   \delta^{a}_{b}::KroneckerDelta.
   \delta_{a}^{b}::KroneckerDelta.

   R_{a b c d}::RiemannTensor.
   R^{a}_{b c d}::RiemannTensor.
   R_{a b c}^{d}::RiemannTensor.

   \Gamma^{a}_{b c}::TableauSymmetry(shape={2}, indices={1,2}).
   \Gamma^{a}_{b c d}::TableauSymmetry(shape={3}, indices={1,2,3}).
   \Gamma^{a}_{b c d e}::TableauSymmetry(shape={4}, indices={1,2,3,4}).
   \Gamma^{a}_{b c d e f}::TableauSymmetry(shape={5}, indices={1,2,3,4,5}).

   x^{a}::Depends(D{#}).

   g_{a b}::Depends(\partial{#}).
   R_{a b c d}::Depends(\partial{#}).
   R^{a}_{b c d}::Depends(\partial{#}).
   \Gamma^{a}_{b c}::Depends(\partial{#}).

   R_{a b c d}::Depends(\nabla{#}).
   R^{a}_{b c d}::Depends(\nabla{#}).

   g0{#}::LaTeXForm("\ngab{0}").
   g2{#}::LaTeXForm("\ngab{2}").
   g3{#}::LaTeXForm("\ngab{3}").
   g4{#}::LaTeXForm("\ngab{4}").
   g5{#}::LaTeXForm("\ngab{5}").

   y0{#}::LaTeXForm("\ny{0}").
   y2{#}::LaTeXForm("\ny{2}").
   y3{#}::LaTeXForm("\ny{3}").
   y4{#}::LaTeXForm("\ny{4}").
   y5{#}::LaTeXForm("\ny{5}").

\end{cadabra}

\clearpage

% =================================================================================================
\section*{Stage 1: The formal expansion}

\begin{cadabra}
   g0_{a b}::Symmetric.
   g2_{a b}::Symmetric.
   g3_{a b}::Symmetric.
   g4_{a b}::Symmetric.
   g5_{a b}::Symmetric.

   g0_{a b}::Weight(label=num,value=0).
   g2_{a b}::Weight(label=num,value=2).
   g3_{a b}::Weight(label=num,value=3).
   g4_{a b}::Weight(label=num,value=4).
   g5_{a b}::Weight(label=num,value=5).

   y0^{a}::Weight(label=num,value=0).
   y2^{a}::Weight(label=num,value=2).
   y3^{a}::Weight(label=num,value=3).
   y4^{a}::Weight(label=num,value=4).
   y5^{a}::Weight(label=num,value=5).

   # note: keeping numbering as is (out of order) to ensure R appears before \nabla R etc.
   def product_sort (obj):
       substitute (obj,$ A^{a}                            -> A001^{a}               $)
       substitute (obj,$ x^{a}                            -> A002^{a}               $)
       substitute (obj,$ Dx^{a}                           -> A003^{a}               $)
       substitute (obj,$ g_{a b}                          -> A004_{a b}             $)
       substitute (obj,$ g^{a b}                          -> A005^{a b}             $)
       substitute (obj,$ \nabla_{e f g h}{R_{a b c d}}    -> A010_{a b c d e f g h} $)
       substitute (obj,$ \nabla_{e f g}{R_{a b c d}}      -> A009_{a b c d e f g}   $)
       substitute (obj,$ \nabla_{e f}{R_{a b c d}}        -> A008_{a b c d e f}     $)
       substitute (obj,$ \nabla_{e}{R_{a b c d}}          -> A007_{a b c d e}       $)
       substitute (obj,$ R_{a b c d}                      -> A006_{a b c d}         $)
       sort_product   (obj)
       rename_dummies (obj)
       substitute (obj,$ A001^{a}                  -> A^{a}                         $)
       substitute (obj,$ A002^{a}                  -> x^{a}                         $)
       substitute (obj,$ A003^{a}                  -> Dx^{a}                        $)
       substitute (obj,$ A004_{a b}                -> g_{a b}                       $)
       substitute (obj,$ A005^{a b}                -> g^{a b}                       $)
       substitute (obj,$ A006_{a b c d}            -> R_{a b c d}                   $)
       substitute (obj,$ A007_{a b c d e}          -> \nabla_{e}{R_{a b c d}}       $)
       substitute (obj,$ A008_{a b c d e f}        -> \nabla_{e f}{R_{a b c d}}     $)
       substitute (obj,$ A009_{a b c d e f g}      -> \nabla_{e f g}{R_{a b c d}}   $)
       substitute (obj,$ A010_{a b c d e f g h}    -> \nabla_{e f g h}{R_{a b c d}} $)

       return obj

   def truncate (obj,n):
       ans = Ex(0)

       for i in range (0,n+1):
          foo := @(obj).
          bah = Ex("num = " + str(i))
          keep_weight (foo, bah)
          ans = ans + foo

       return ans

   # expansions wrt the curvature

   defgab := g_{a b} -> g0_{a b} + g2_{a b} + g3_{a b} + g4_{a b} + g5_{a b}.
   defy   := y^{a}   -> y0^{a} + y2^{a} + y3^{a} + y4^{a} + y5^{a}.

   lsq    := g_{a b} y^{a} y^{b}.

   substitute (lsq,defgab)
   substitute (lsq,defy)
   distribute (lsq)

   def tidy (obj):
       foo := @(obj).
       sort_product    (foo)
       rename_dummies  (foo)
       canonicalise    (foo)
       return foo

   lsq0 = tidy ( truncate (lsq,0) )  # cdb (lsq0.002,lsq0)
   lsq2 = tidy ( truncate (lsq,2) )  # cdb (lsq2.002,lsq2)
   lsq3 = tidy ( truncate (lsq,3) )  # cdb (lsq3.002,lsq3)
   lsq4 = tidy ( truncate (lsq,4) )  # cdb (lsq4.002,lsq4)
   lsq5 = tidy ( truncate (lsq,5) )  # cdb (lsq5.002,lsq5)

   d20 := @(lsq2) - @(lsq0).         # cdb (d20.001,d20)   # check, should contain only O(2) terms
   d32 := @(lsq3) - @(lsq2).         # cdb (d32.001,d32)   # check, should contain only O(3) terms
   d43 := @(lsq4) - @(lsq3).         # cdb (d43.001,d43)   # check, should contain only O(4) terms
   d54 := @(lsq5) - @(lsq4).         # cdb (d54.001,d54)   # check, should contain only O(5) terms

   d5 := @(lsq5) - @(lsq).           # cdb (d5.001,d5)
   d5  = tidy (d5)                   # cdb (d5.002,d5)  # all higher order terms, should see no O(5) terms

\end{cadabra}

\clearpage

\begin{dgroup*}
   \begin{dmath*} \cdb*{lsq0.002} \end{dmath*}
   \begin{dmath*} \cdb*{lsq2.002} \end{dmath*}
   \begin{dmath*} \cdb*{lsq3.002} \end{dmath*}
   \begin{dmath*} \cdb*{lsq4.002} \end{dmath*}
   \begin{dmath*} \cdb*{lsq5.002} \end{dmath*}
\end{dgroup*}

\begin{dgroup*}
   \begin{dmath*} \cdb*{d20.001} \end{dmath*}
   \begin{dmath*} \cdb*{d32.001} \end{dmath*}
   \begin{dmath*} \cdb*{d43.001} \end{dmath*}
   \begin{dmath*} \cdb*{d54.001} \end{dmath*}
   \begin{dmath*} \cdb*{d5.002} \end{dmath*}
\end{dgroup*}

\clearpage

% =================================================================================================
\section*{Stage 2: Substution of $\ny{n}^{a}$ and $\ngab{m}_{ab}$}

\begin{cadabra}
   import cdblib

   g0ab = cdblib.get('g_ab_0','metric.json')
   g2ab = cdblib.get('g_ab_2','metric.json')
   g3ab = cdblib.get('g_ab_3','metric.json')
   g4ab = cdblib.get('g_ab_4','metric.json')
   g5ab = cdblib.get('g_ab_5','metric.json')

   defg0ab := g0_{a b} -> @(g0ab).
   defg2ab := g2_{a b} -> @(g2ab).
   defg3ab := g3_{a b} -> @(g3ab).
   defg4ab := g4_{a b} -> @(g4ab).
   defg5ab := g5_{a b} -> @(g5ab).

   y0a = cdblib.get('y50','geodesic-bvp.json')
   y2a = cdblib.get('y52','geodesic-bvp.json')
   y3a = cdblib.get('y53','geodesic-bvp.json')
   y4a = cdblib.get('y54','geodesic-bvp.json')
   y5a = cdblib.get('y55','geodesic-bvp.json')

   defy0a := y0^{a} -> @(y0a).
   defy2a := y2^{a} -> @(y2a).
   defy3a := y3^{a} -> @(y3a).
   defy4a := y4^{a} -> @(y4a).
   defy5a := y5^{a} -> @(y5a).

   def substitute_gab_ya (obj):

      foo := @(obj).

      substitute (foo,defg0ab)
      substitute (foo,defg2ab)
      substitute (foo,defg3ab)
      substitute (foo,defg4ab)
      substitute (foo,defg5ab)

      substitute (foo,defy0a)
      substitute (foo,defy2a)
      substitute (foo,defy3a)
      substitute (foo,defy4a)
      substitute (foo,defy5a)

      distribute     (foo)
      sort_product   (foo)
      rename_dummies (foo)
      canonicalise   (foo)

      substitute     (foo,$g_{a b} g^{c b} -> \delta^{c}_{a}$)
      eliminate_kronecker (foo)
      foo = product_sort  (foo)
      rename_dummies      (foo)
      canonicalise        (foo)

      return foo

   def get_Rterm (obj,n):

   # I would like to assign different weights to \nabla_{a}, \nabla_{a b}, \nabla_{a b c} etc. but no matter
   # what I do it appears that Cadabra assigns the same weight to all of these regardless of the number of subscripts.
   # It seems that the weight is assigned to the symbol \nabla alone. So I'm forced to use the following substitution trick.

       Q_{a b c d}::Weight(label=numR,value=2).
       Q_{a b c d e}::Weight(label=numR,value=3).
       Q_{a b c d e f}::Weight(label=numR,value=4).
       Q_{a b c d e f g}::Weight(label=numR,value=5).

       tmp := @(obj).

       distribute (tmp)

       substitute (tmp, $\nabla_{e f g}{R_{a b c d}} -> Q_{a b c d e f g}$)
       substitute (tmp, $\nabla_{e f}{R_{a b c d}} -> Q_{a b c d e f}$)
       substitute (tmp, $\nabla_{e}{R_{a b c d}} -> Q_{a b c d e}$)
       substitute (tmp, $R_{a b c d} -> Q_{a b c d}$)

       foo := @(tmp).
       bah = Ex("numR = " + str(n))
       keep_weight (foo, bah)

       substitute (foo, $Q_{a b c d e f g} -> \nabla_{e f g}{R_{a b c d}}$)
       substitute (foo, $Q_{a b c d e f} -> \nabla_{e f}{R_{a b c d}}$)
       substitute (foo, $Q_{a b c d e} -> \nabla_{e}{R_{a b c d}}$)
       substitute (foo, $Q_{a b c d} -> R_{a b c d}$)

       return foo

   lsq2 = substitute_gab_ya (lsq2)  # cdb (lsq2.101,lsq2)
   lsq3 = substitute_gab_ya (lsq3)  # cdb (lsq3.101,lsq3)
   lsq4 = substitute_gab_ya (lsq4)  # cdb (lsq4.101,lsq4)
   lsq5 = substitute_gab_ya (lsq5)  # cdb (lsq5.101,lsq5)

   lsq50 = get_Rterm (lsq5,0)
   lsq52 = get_Rterm (lsq5,2)
   lsq53 = get_Rterm (lsq5,3)
   lsq54 = get_Rterm (lsq5,4)
   lsq55 = get_Rterm (lsq5,5)

   cdblib.create ('geodesic-lsq.json')

   cdblib.put ('lsq2',lsq2,'geodesic-lsq.json')
   cdblib.put ('lsq3',lsq3,'geodesic-lsq.json')
   cdblib.put ('lsq4',lsq4,'geodesic-lsq.json')
   cdblib.put ('lsq5',lsq5,'geodesic-lsq.json')

   cdblib.put ('lsq50',lsq50,'geodesic-lsq.json')
   cdblib.put ('lsq52',lsq52,'geodesic-lsq.json')
   cdblib.put ('lsq53',lsq53,'geodesic-lsq.json')
   cdblib.put ('lsq54',lsq54,'geodesic-lsq.json')
   cdblib.put ('lsq55',lsq55,'geodesic-lsq.json')

\end{cadabra}

\clearpage
\begin{dgroup*}
   \begin{dmath*} \cdb*{lsq2.101} \end{dmath*}
   \begin{dmath*} \cdb*{lsq3.101} \end{dmath*}
   \begin{dmath*} \cdb*{lsq4.101} \end{dmath*}
   \begin{dmath*} \cdb*{lsq5.101} \end{dmath*}
\end{dgroup*}

% =================================================================================================
% the remaining code is just for pretty printing

\clearpage

% =================================================================================================
\section*{Stage 3: Reformatting}

\begin{cadabra}
   def reformat (obj,scale):
      foo  = Ex(str(scale))
      bah := @(foo) @(obj).
      distribute     (bah)
      bah = product_sort (bah)
      rename_dummies (bah)
      canonicalise   (bah)
      substitute     (bah,$Dx^{b}->zzz^{b}$)
      factor_out     (bah,$x^{a?},zzz^{b?}$)
      substitute     (bah,$zzz^{b}->Dx^{b}$)
      ans := @(bah) / @(foo).
      return ans

   def rescale (obj,scale):
      foo  = Ex(str(scale))
      bah := @(foo) @(obj).
      distribute  (bah)
      substitute  (bah,$Dx^{b}->zzz^{b}$)
      factor_out  (bah,$x^{a?},zzz^{b?}$)
      substitute  (bah,$zzz^{b}->Dx^{b}$)
      return bah

   Rterm0 := @(lsq50).
   Rterm2 := @(lsq52).
   Rterm3 := @(lsq53).
   Rterm4 := @(lsq54).
   Rterm5 := @(lsq55).

   Rterm0 = reformat (Rterm0,   1)    # cdb(Rterm0.301,Rterm0) # LCB: returns Dx before g, not what I want
   Rterm2 = reformat (Rterm2,   3)    # cdb(Rterm2.301,Rterm2)
   Rterm3 = reformat (Rterm3,  12)    # cdb(Rterm3.301,Rterm3)
   Rterm4 = reformat (Rterm4, 360)    # cdb(Rterm4.301,Rterm4)
   Rterm5 = reformat (Rterm5,1080)    # cdb(Rterm5.301,Rterm5)

   Rterm0 := g_{a b} Dx^{a} Dx^{b}.   # LCB: fixes the order of terms, g before Dx,

   lsq3 := @(Rterm0) + @(Rterm2).                                      # cdb (lsq4.301,lsq3)
   lsq4 := @(Rterm0) + @(Rterm2) + @(Rterm3).                          # cdb (lsq4.301,lsq4)
   lsq5 := @(Rterm0) + @(Rterm2) + @(Rterm3) + @(Rterm4).              # cdb (lsq5.301,lsq5)
   lsq6 := @(Rterm0) + @(Rterm2) + @(Rterm3) + @(Rterm4) + @(Rterm5).  # cdb (lsq5.301,lsq6)

   lsq  := @(lsq6).                   # cdb (lsq.301,lsq)

   scaled0 = rescale (Rterm0,    1)   # cdb (scaled0.301,scaled0) # LCB: returns Dx before g, not what I want
   scaled2 = rescale (Rterm2,    3)   # cdb (scaled2.301,scaled2)
   scaled3 = rescale (Rterm3,   12)   # cdb (scaled3.301,scaled3)
   scaled4 = rescale (Rterm4,  360)   # cdb (scaled4.301,scaled4)
   scaled5 = rescale (Rterm5, 1080)   # cdb (scaled5.301,scaled5)

   scaled0 := g_{a b} Dx^{a} Dx^{b}.  # cdb (scaled0.301,scaled0) # LCB: fixes the order of terms, g before Dx, good

\end{cadabra}

\clearpage

% =================================================================================================
\section*{Geodesic arc-length}

\begin{dgroup*}[spread=5pt]
   % LCB: which of these is correct?
   % \begin{dmath*} \left(\Delta s\right)^2 = \cdb{lsq.301} + \BigO{\eps^6,Dx^6} \end{dmath*}
   % \begin{dmath*} \left(\Delta s\right)^2 = \cdb{lsq.301} + \BigO{\eps^6,Dx^7} \end{dmath*}
   \begin{dmath*} \left(\Delta s\right)^2 = \cdb{lsq.301} + \BigO{\eps^6} \end{dmath*}
\end{dgroup*}

\clearpage

% =================================================================================================
\section*{Geodesic arc-length curvature expansion}

\begin{align*}
   % LCB: which of these is correct?
   % \left(\Delta s\right)^2 = \nD{0} + \nD{2} + \nD{3} + \nD{4} + \nD{5} + \BigO{\eps^6,Dx^6}
   % \left(\Delta s\right)^2 = \nD{0} + \nD{2} + \nD{3} + \nD{4} + \nD{5} + \BigO{\eps^6,Dx^7}
   \left(\Delta s\right)^2 = \nD{0} + \nD{2} + \nD{3} + \nD{4} + \nD{5} + \BigO{\eps^6}
\end{align*}

\begin{dgroup*}[spread=5pt]
   \begin{dmath*}      \nD{0} = \cdb{scaled0.301} \end{dmath*}
   \begin{dmath*}    3 \nD{2} = \cdb{scaled2.301} \end{dmath*}
   \begin{dmath*}   12 \nD{3} = \cdb{scaled3.301} \end{dmath*}
   \begin{dmath*}  360 \nD{4} = \cdb{scaled4.301} \end{dmath*}
   \begin{dmath*} 1080 \nD{5} = \cdb{scaled5.301} \end{dmath*}
\end{dgroup*}

\clearpage

% =================================================================================================
% export selected objects, these will later be imported into a library
% these are the objects that will appear in the paper

\begin{cadabra}
   cdblib.create ('geodesic-lsq.export')

   # 3rd to 6th order lsq
   cdblib.put ('lsq3',lsq3,'geodesic-lsq.export')
   cdblib.put ('lsq4',lsq4,'geodesic-lsq.export')
   cdblib.put ('lsq5',lsq5,'geodesic-lsq.export')
   cdblib.put ('lsq6',lsq6,'geodesic-lsq.export')

   # 6th order lsq terms, scaled
   cdblib.put ('lsq60',scaled0,'geodesic-lsq.export')
   cdblib.put ('lsq62',scaled2,'geodesic-lsq.export')
   cdblib.put ('lsq63',scaled3,'geodesic-lsq.export')
   cdblib.put ('lsq64',scaled4,'geodesic-lsq.export')
   cdblib.put ('lsq65',scaled5,'geodesic-lsq.export')

   checkpoint.append (lsq4)

   checkpoint.append (scaled0)
   checkpoint.append (scaled2)
   checkpoint.append (scaled3)
   checkpoint.append (scaled4)
   checkpoint.append (scaled5)
\end{cadabra}

% =================================================================================================
% export checkpoints in json format

\bgroup
\CdbSetup{action=hide}
\begin{cadabra}
   for i in range( len(checkpoint) ):
      cdblib.put ('check{:03d}'.format(i),checkpoint[i],checkpoint_file)
\end{cadabra}
\egroup

\end{document}
